\documentclass[12pt]{extreport} % Schriftgröße: 8pt, 9pt, 10pt, 11pt, 12pt, 14pt, 17pt oder 20pt

%% Packages
\usepackage{scrextend}
\usepackage{amssymb}
\usepackage{amsthm}
\usepackage{chngcntr}
\usepackage{cmap}
\usepackage{color}
\usepackage{enumitem}
\usepackage{hyperref}
\usepackage{lmodern}
\usepackage{makeidx}
\usepackage{mathtools} 
\usepackage{xpatch}
\usepackage{pgfplots}
\pgfplotsset{compat=1.7}
\usetikzlibrary{calc}	
\usetikzlibrary{matrix}	

% Language Setup (Deutsch)
\usepackage[utf8]{inputenc} 
\usepackage[T1]{fontenc} 
\usepackage[ngerman]{babel}

% Options
\makeatletter%%  
  % Linkfarbe, {0,0.35,0.35} für Türkis, {0,0,0} für Schwarz 
  \definecolor{linkcolor}{rgb}{0,0.35,0.35}
  % Zeilenabstand für bessere Leserlichkeit
  \def\mystretch{1.2} 
  % Publisher definieren
  \newcommand\publishers[1]{\newcommand\@publishers{#1}} 
  % Enumerate im 1. Level: \alph für a), b), ...
  \renewcommand{\labelenumi}{\alph{enumi})} 
  % Enumerate im 2. Level: \roman für (i), (ii), ...
  \renewcommand{\labelenumii}{(\roman{enumii})}
  % Zeileneinrückung am Anfang des Absatzes
  \setlength{\parindent}{0pt} 
  % Verweise auf Enumerate, z.B.: 3.2 a)
  \setlist[enumerate,1]{ref={\thesatz ~ \alph*)}}
  % Für das Proof-Environment: 'Beweis:' anstatt 'Beweis.'
  \xpatchcmd{\proof}{\@addpunct{.}}{\@addpunct{:}}{}{} 
  % Nummerierung der Bilder, z.B.: Abbildung 4.1
  \@ifundefined{thechapter}{}{\def\thefigure{\thechapter.\arabic{figure}}} 
\makeatother%

% Meta Setup (Für Titelblatt und Metadaten im PDF)
\title{Höhere Mathematik I}
\author{G. Herzog, Ch. Schmoeger}
\date{Wintersemester 2016/17}
\publishers{Karlsruher Institut für Technologie}

%% Math. Definitions
\newcommand{\C}{\mathbb{C}}
\newcommand{\N}{\mathbb{N}}
\newcommand{\Q}{\mathbb{Q}}
\newcommand{\R}{\mathbb{R}}
\newcommand{\Z}{\mathbb{Z}}

%% Theorems (unnamedtheorem = Theorem ohne Namen)
\newtheoremstyle{named}{}{}{\normalfont}{}{\bfseries}{:}{0.25em}{#2 \thmnote{#3}}
\newtheoremstyle{itshape}{}{}{\itshape}{}{\bfseries}{:}{ }{}
\newtheoremstyle{normal}{}{}{\normalfont}{}{\bfseries}{:}{ }{}
\renewcommand*{\qed}{\hfill\ensuremath{\square}}

\theoremstyle{named}
\newtheorem{unnamedtheorem}{Theorem} \counterwithin{unnamedtheorem}{chapter}

\theoremstyle{itshape}
\newtheorem{satz}[unnamedtheorem]{Satz} 
\newtheorem*{definition}{Definition}
\newtheorem{hilfssatz}[unnamedtheorem]{Hilfssatz}
\newtheorem*{hilfssatz*}{Hilfssatz}

\theoremstyle{normal}
\newtheorem{beispiel}[unnamedtheorem]{Beispiel}
\newtheorem{folgerung}[unnamedtheorem]{Folgerung}
%\newtheorem{hilfssatz}[unnamedtheorem]{Hilfssatz}
\newtheorem{anwendung}[unnamedtheorem]{Anwendung}
\newtheorem{anwendungen}[unnamedtheorem]{Anwendungen}
\newtheorem*{beispiel*}{Beispiel}
\newtheorem*{beispiele}{Beispiele}
\newtheorem*{bemerkung}{Bemerkung} 
\newtheorem*{bemerkungen}{Bemerkungen}
\newtheorem*{bezeichnung}{Bezeichnung}
\newtheorem*{eigenschaften}{Eigenschaften}
\newtheorem*{folgerung*}{Folgerung}
\newtheorem*{folgerungen}{Folgerungen}
%\newtheorem*{hilfssatz*}{Hilfssatz}
\newtheorem*{regeln}{Regeln}
\newtheorem*{schreibweise}{Schreibweise}
\newtheorem*{schreibweisen}{Schreibweisen}
\newtheorem*{uebung}{Übung}
\newtheorem*{vereinbarung}{Vereinbarung}

%% Template
\makeatletter%
\DeclareUnicodeCharacter{00A0}{ } \pgfplotsset{compat=1.7} \hypersetup{colorlinks,breaklinks, urlcolor=linkcolor, linkcolor=linkcolor, pdftitle=\@title, pdfauthor=\@author, pdfsubject=\@title, pdfcreator=\@publishers}\DeclareOption*{\PassOptionsToClass{\CurrentOption}{report}} \ProcessOptions \def\baselinestretch{\mystretch} \setlength{\oddsidemargin}{0.125in} \setlength{\evensidemargin}{0.125in} \setlength{\topmargin}{0.5in} \setlength{\textwidth}{6.25in} \setlength{\textheight}{8in} \addtolength{\topmargin}{-\headheight} \addtolength{\topmargin}{-\headsep} \def\pulldownheader{ \addtolength{\topmargin}{\headheight} \addtolength{\topmargin}{\headsep} \addtolength{\textheight}{-\headheight} \addtolength{\textheight}{-\headsep} } \def\pullupfooter{ \addtolength{\textheight}{-\footskip} } \def\ps@headings{\let\@mkboth\markboth \def\@oddfoot{} \def\@evenfoot{} \def\@oddhead{\hbox {}\sl \rightmark \hfil \rm\thepage} \def\chaptermark##1{\markright {\uppercase{\ifnum \c@secnumdepth >\m@ne \@chapapp\ \thechapter. \ \fi ##1}}} \pulldownheader } \def\ps@myheadings{\let\@mkboth\@gobbletwo \def\@oddfoot{} \def\@evenfoot{} \def\sectionmark##1{} \def\subsectionmark##1{}  \def\@evenhead{\rm \thepage\hfil\sl\leftmark\hbox {}} \def\@oddhead{\hbox{}\sl\rightmark \hfil \rm\thepage} \pulldownheader }	\def\chapter{\cleardoublepage  \thispagestyle{plain} \global\@topnum\z@ \@afterindentfalse \secdef\@chapter\@schapter} \def\@makeschapterhead#1{ {\parindent \z@ \raggedright \normalfont \interlinepenalty\@M \Huge \bfseries  #1\par\nobreak \vskip 40\p@ }} \newcommand{\indexsection}{chapter} \patchcmd{\@makechapterhead}{\vspace*{50\p@}}{}{}{}
	% Titlepage
	\def\maketitle{ \begin{titlepage} 
			~\vspace{3cm} 
		\begin{center} {\Huge \@title} \end{center} 
	 		\vspace*{1cm} 
	 	\begin{center} {\large \@author} \end{center} 
	 	\begin{center} \@date \end{center} 
	 		\vspace*{7cm} 
	 	\begin{center} \@publishers \end{center} 
	 		\vfill 
	\end{titlepage} }
\makeatother%

% Indexdatei erstellen
\makeindex 

\begin{document}

\pagenumbering{Alph}
\begin{titlepage}
	\maketitle
	\thispagestyle{empty}
\end{titlepage}
\pagenumbering{arabic}
	
% Inhaltsverzeichnis
\tableofcontents
\thispagestyle{empty}
  
% Skript - Anfang
\chapter{Reelle Zahlen}

Die Grundmenge der Analysis ist die Menge $\R$, die Menge der \textbf{reellen Zahlen}. Diese führen wir \textbf{axiomatisch} ein, d.h. wir nehmen $\R$ als gegeben an und 
\textbf{fordern} in den folgenden 15 \textbf{Axiomen} Eigenschaften von $\R$ aus denen sich alle weiteren Rechenregeln herleiten lassen.  

\bigskip
\bigskip

\index{Axiome!Körper-}
\textbf{Körperaxiome:} In $\R$ sind zwei Verknüpfungen "'$+$"' und "'$\cdot$"' gegeben, die jedem Paar $a, b \in \R$ genau ein $a + b \in \R$ und genau ein 
$a b \coloneqq a \cdot b \in \R$ zuordnen. Dabei gilt:
\begin{description} \addtolength{\itemindent}{0.4cm} \label{k.axiom}
	\item[$(A1)$] $\forall a, b, c \in \R: \: a + \left( b + c \right) = \left( a + b \right) + c$  (Assoziativgesetz für "'$+$"') \label{k.axiom-a1}
	\item[$(A5)$] $\forall a, b, c \in \R: \: a \cdot \left( b \cdot c \right) = \left( a \cdot b \right) \cdot c$ (Assoziativgesetz für "'$\cdot$"') \label{k.axiom-a5}
	\item[$(A2)$] $\exists 0 \in \R$ $\forall a \in \R : a + 0 = a$ (Existenz einer Null) \label{k.axiom-a2}
	\item[$(A6)$] $\exists 1 \in \R$ $\forall a \in \R : a \cdot 1 = a$ und $1 \neq 0$ (Existenz einer Eins) \label{k.axiom-a6}
	\item[$(A3)$] $\forall a \in \R ~ \exists -a \in \R : a + (-a) = 0$ (Inverse bzgl. "'$+$"')  \label{k.axiom-a3} 
	\item[$(A7)$] $\forall a \in \R \setminus \{ 0 \} ~ \exists a^{-1} \in \R : a \cdot a^{-1} = 1$ (Inverse bzgl. "'$\cdot$"')  \label{k.axiom-a7}
	\item[$(A4)$] $\forall a, b \in \R : a + b = b + a$ (Kommutativgesetz für "'$+$"') \label{k.axiom-a4}
	\item[$(A8)$] $\forall a, b \in \R : a \cdot b = b \cdot a$ (Kommutativgesetz für "'$\cdot$"') \label{k.axiom-a8}
	\item[$(A9)$] $\forall a, b, c \in \R : a \cdot (b + c) = a \cdot b + a \cdot c$ (Distributivgesetz) \label{k.axiom-a9}
\end{description}


\begin{schreibweisen}
	Für $a, b \in \R$: $a - b \coloneqq a + (-b)$ und für $b \neq 0$: $ \frac{a}{b} \coloneqq a \cdot b^{-1}$.
\end{schreibweisen}


\textbf{Alle} bekannten Regeln der Grundrechenarten lassen sich aus $(A1)-(A9)$ herleiten. Diese Regeln seien von nun an bekannt.


\begin{beispiele} ~\
	\begin{enumerate}
		\item Behauptung: $\exists_{1} 0 \in  \R$ $\forall a \in \R:$ $a + 0 = a$.
		  \begin{proof}
			Sei $\tilde{0} \in \R$ und es gelte $\forall a \in \R: \: a + \tilde{0} = a$. Mit $a = 0$ folgt: $0 + \tilde{0} = 0$. Mit $a = \tilde{0}$ in $(A2)$ folgt: 
			$\tilde{0} + 0 = \tilde{0}$. Damit ist $0 = 0 + \tilde{0} \overset{(A4)}{=} \tilde{0} + 0 = \tilde{0}$.
		  \end{proof}
		 \item Behauptung: $\forall a \in \R:$ $a \cdot 0 = 0$.
		   \begin{proof}
		      Sei $a \in \R$ und $b \coloneqq a \cdot 0$. Es gilt $b \overset{(A2)}{=} a \cdot (0 + 0) \overset{(A9)}{=} a \cdot 0 + a \cdot 0 = b + b$,
		      und damit 
		      $0 \overset{(A3)}{=} b + (-b) = (b + b) + (-b) \overset{(A1)}{=} b + (b + (-b)) = b + 0 \overset{(A2)}{=} b$.
		   \end{proof}
	\end{enumerate}
\end{beispiele}

\index{Axiome!Anordnungs-} 
\textbf{Anordnungsaxiome:} In $\R$ ist eine Relation "'$\leq$"' gegeben. F\"ur diese gilt:
\begin{description} \addtolength{\itemindent}{0.4cm}
	\item[$(A10)$] $\forall a, b \in \R:$ $a \leq b$ oder $b \leq a$ \label{a.axiom-a10}
	\item[$(A11)$] $a \leq b$ und $b \leq a$ $\Rightarrow$ $a = b$ \label{a.axiom-a11}
	\item[$(A12)$] $a \leq b$ und $b \leq c$ $\Rightarrow$ $a \leq c$ \label{a.axiom-a12}
	\item[$(A13)$] $a \leq b$ und $c \in \R$ $\Rightarrow$ $a + c \leq b + c$ \label{a.axiom-a13}
	\item[$(A14)$] $a \leq b$ und $0 \leq c$ $\Rightarrow$ $ac \leq b c$ \label{a.axiom-a14}
\end{description}


\begin{schreibweisen}
$b \geq a :\iff a \leq b$; $a < b :\iff a \leq b$ und $a \neq b$; $b > a :\iff a < b$.
\end{schreibweisen}

Aus $(A1)-(A14)$ lassen sich alle Regeln für Ungleichungen herleiten. Diese Regeln seien von nun an bekannt.


\begin{beispiele}[Übung] ~\
	\begin{enumerate}
		\item $a < b$ und $0 < c$ $\Rightarrow$ $ac < bc$
		\item $a \leq b$ und $c \leq 0$ $\Rightarrow$ $ac \geq bc$
		\item $a \leq b$ und $c \leq d$ $\Rightarrow$ $a + c \leq b + d$
	\end{enumerate}
\end{beispiele}

\index{Intervalle}
\textbf{Intervalle:} Es seien  $a, b \in \R$ und $a < b$. Wir setzen:
\begin{description} \addtolength{\itemindent}{0.4cm}
	\item $[a, b] \coloneqq \{ x \in \R : a \leq x \leq b \}$ (abgeschlossenes Intervall) 
	\item $(a, b) \coloneqq \{ x \in \R : a < x < b \}$ (offenes Intervall)
	\item $(a, b] \coloneqq \{ x \in \R : a < x \leq b \}$ (halboffenes Intervall)
	\item $[a, b) \coloneqq \{ x \in \R : a \leq x < b \}$ (halboffenes Intervall)
	\item $[a, \infty) \coloneqq \{ x \in \R : x \geq a \}$, $(a , \infty) \coloneqq \{ x \in \R : x > a \}$
	\item $(-\infty, a] \coloneqq \{ x \in \R : x \leq a\}$, $(-\infty, a) \coloneqq \{ x \in \R : x < a\}$ 
	\item $(- \infty, \infty) \coloneqq \R$
\end{description}

\index{Betrag}
\subsection*{Der Betrag} 
Für $a \in \R$ hei{\ss}t $|a| \coloneqq \begin{cases} \hspace{0.35cm} a, & \text{falls } a \geq 0 \\ -a, & \text{falls } a < 0\end{cases}$ 
der \textbf{Betrag} von $a$. F\"ur $a,b \in \R$ hei{\ss}t die Zahl $|a-b|$ der \textbf{Abstand} von $a$ und $b$.


\begin{beispiele}
	$|1| = 1$, $~|-7| = -(-7) = 7$. 
	%Anschaulich:  \tikz[baseline=-0.5ex]{  \draw(0,0)--(12,0);
    %\foreach \x/\xtext in {0/$$,2/$$,4/$a$,6/{\small $0$},8/$$,10/{\small $b$},12/$$}
      %\draw(\x,3pt)--(\x,-3pt) node[below] {\xtext};
    %\draw[decorate,decoration={brace},yshift=2ex]  (4,0) -- node[above=0.4ex] {\small $\underset{ \text{"'Abstand"'} \text{ von } a \text{ und } b}{|a - b| =}$}  (10,0);
    %\draw[decorate,decoration={brace},yshift=-4ex] (6,0) -- node[below=0.3ex] {\small $\underset{\text{"'Abstand"'} \text{ von } a \text{ und } 0}{|a| =}$} (4,0);} \\
\end{beispiele}


\begin{regeln} ~\ F\"ur $a,b \in \R$ gilt:
	\begin{enumerate}
	        \item $|-a| = |a|$ und $|a - b| = |b - a|$
		\item $|a| \geq 0$
		\item $|a| = 0 \iff a = 0$
		\item $|ab| = |a||b|$
		\item $\pm a \leq |a|$
		\item $|a + b| \leq |a| + |b|$ (Dreiecksungleichung)
		\item $\left| |a| - |b| \right| \leq |a - b|$
	\end{enumerate}	

	\begin{proof} ~\
	  \begin{enumerate}
		\item[a)]- e) leichte Übung.
		\item[f)] Fall 1: $a +b \geq 0$. Dann gilt: $|a + b| = a + b \overset{e)}{\leq} |a| + |b|$. \\
			Fall 2: $a + b < 0$. Dann gilt: $|a + b| = - (a + b) = - a + (- b) \overset{e)}{\leq} |a| + |b|$.
		\item[g)] Es sei $c \coloneqq |a| - |b|$. Es gilt 
			$$
				|a| = |a - b + b| \overset{f)}{\leq} |a - b | + |b|\Rightarrow c = |a| - |b| \leq |a - b|. 
			$$
			Analog zeigt man
			$$
			         -c = |b| - |a| \leq |b - a| = |a - b|. 
			$$
			Also gilt $\pm c \leq |a - b| \Rightarrow |c| \leq |a-b|$.
	  \end{enumerate}
	\end{proof}
\end{regeln}

\index{beschränkt!Menge} \index{Schranke} \index{Supremum} \index{Infimum}
\begin{definition}
	Es sei $M \subseteq \R$. 
	\begin{enumerate}
		\item $M$ hei{\ss}t \textbf{nach oben beschränkt} $:\iff \exists \gamma \in \R ~ \forall x \in M: \: x \leq \gamma$. \\
			In diesem Fall hei{\ss}t $\gamma$ eine \textbf{obere Schranke} (OS) von $M$.
		\item Ist $\gamma$ eine obere Schranke von $M$ und gilt $\gamma \leq \delta$ für jede weitere obere Schranke $\delta$ von $M$, so hei{\ss}t $\gamma$ das 
		      \textbf{Supremum} (oder \textbf{die kleinste obere Schranke}) von $M$.
		\item $M$ hei{\ss}t \textbf{nach unten beschränkt} $:\iff \exists \gamma \in \R ~ \forall x \in M: \: \gamma \leq x$.\\
			In diesem Fall hei{\ss}t $\gamma$ eine \textbf{untere Schranke} (US) von $M$.
		\item Ist $\gamma$ eine untere Schranke von $M$ und gilt $\gamma \geq \delta$ für jede weitere untere Schranke $\delta$ von $M$, so hei{\ss}t $\gamma$ das 
		      \textbf{Infimum} (oder \textbf{die grö{\ss}te untere Schranke}) von $M$.
	\end{enumerate}
\end{definition}

\textbf{Bezeichnung in diesem Fall}: $\gamma = \sup M$ bzw. $\gamma = \inf M$.

Aus $(A11)$ folgt: Ist $\sup M$ bzw. $\inf M$ vorhanden, so ist $\sup M$ bzw. $\inf M$ eindeutig bestimmt.

Ist $\sup M$ bzw. $\inf M$ vorhanden und gilt $\sup M \in M$ bzw. $\inf M \in M$, so hei{\ss}t $\sup M$ das \textbf{Maximum} bzw. $\inf M$ das \textbf{Minimum} von $M$ 
und wird mit $\max M$ bzw. $\min M$ bezeichnet.


\begin{beispiele} ~\
	\begin{enumerate}
		\item $M = (1, 2)$. $\sup M = 2 \notin M$, $\inf M = 1 \notin M$. $M$ hat kein Maximum und kein Minimum.
		\item $M = (1, 2]$. $\sup M = 2 \in M$, $\max M = 2$.
		\item $M = (3, \infty)$. $M$ ist nicht nach oben beschränkt, $3 = \inf M \notin M$.
		\item $M = (-\infty, 0]$. $M$ ist nach unten unbeschränkt, $0 = \sup M = \max M$.
		\item $M= \emptyset$. Jedes $\gamma \in \R$ ist eine obere Schranke und eine untere Schranke von $M$.
	\end{enumerate}
\end{beispiele}


\index{Axiome!Vollständigkeits-}
\textbf{Vollständigkeitsaxiom:}
\begin{description} \addtolength{\itemindent}{0.4cm}
	\item[$(A15)$]Ist $\emptyset \neq M \subseteq \R$ und ist $M$ nach oben beschränkt, so ist $\sup M$ vorhanden. \label{v.axiom-a15}
\end{description}

\begin{satz} \label{1.1:satz}
	Ist $\emptyset \neq M \subseteq \R$ und ist $M$ nach unten beschränkt, so ist $\inf M$ vorhanden.
\end{satz} 

\begin{proof}
	In den Übungen.
\end{proof}

\index{beschränkt} 
\begin{definition}
	Es sei $M \subseteq \R$. $M$ hei{\ss}t beschränkt $:\iff$ $M$ ist nach oben und nach unten beschränkt. Äquivalent ist:
	$$
	\exists c \geq 0 ~\forall x \in M: \: |x| \leq c. 
	$$
\end{definition}


\begin{satz} \label{1.2:satz}
	Es sei $\emptyset \neq B \subseteq A \subseteq \R$. 
	\begin{enumerate}
		\item Ist $A$ beschränkt, so ist $\inf A \leq \sup A$.
		\item Ist $A$ nach oben bzw. unten beschränkt, so ist $B$ nach oben beschränkt und $\sup B \leq \sup A$ bzw. nach unten beschränkt und $\inf B \geq \inf A$.
		\item $A$ sei nach oben beschränkt und $\gamma$ eine obere Schranke von $A$. Dann gilt:
			$$
				\gamma = \sup A \iff \forall \varepsilon > 0 ~\exists x = x(\varepsilon) \in A : x > \gamma - \varepsilon
			$$
		\item $A$ sei nach unten beschränkt und $\gamma$ eine untere Schranke von $A$. Dann gilt:
			$$
				\gamma = \inf A \iff \forall \varepsilon > 0 ~\exists x = x(\varepsilon) \in A : x < \gamma + \varepsilon
			$$	
	\end{enumerate}

	\begin{proof} ~\ 
		\begin{enumerate}
			\item $A \neq \emptyset \Rightarrow \exists x \in \R : x \in A$. Es gilt: $\inf A \leq x$ und $x \leq \sup A$ $ \Rightarrow \inf A \leq \sup A $.
			\item Es sei $x \in B$. Dann: $x \in A$, also $x \leq \sup A$. Also ist $B$ oben beschränkt und $\sup A$ ist eine obere Schranke von $B$. Somit ist
			      $\sup B \leq \sup A $. Analog falls $A$ nach unten beschränkt ist.
			\item "'$\Rightarrow$"': Es sei $\gamma := \sup A$ und $\varepsilon > 0$. Dann ist $\gamma - \varepsilon < \gamma$. Also ist $\gamma - \varepsilon$ keine obere 
			        Schranke von $A$. Es folgt: $\exists x \in A : x > \gamma - \varepsilon$. \\
				"'$\Leftarrow$"': Es sei $\tilde{\gamma} := \sup A$. Dann ist $\tilde{\gamma} \leq \gamma$. Annahme: $\gamma \neq \tilde{\gamma}$. Dann ist 
				$\tilde{\gamma} < \gamma$, also	$\varepsilon \coloneqq \gamma - \tilde{\gamma} > 0$. Nach Voraussetzung gilt: 
				$\exists x \in A: x > \gamma - \varepsilon = \gamma- (\gamma - \tilde{\gamma}) = \tilde{\gamma}$. Widerspruch zu $x \leq \tilde{\gamma}$.
			\item Analog zu c).
		\end{enumerate}
	\end{proof}
\end{satz}

\section*{Natürliche Zahlen} 

\index{Natürliche Zahlen} \index{Induktionsmenge}
\begin{definition} ~\
	\begin{enumerate}
		\item Eine Menge $A \subseteq \R$ hei{\ss}t \textbf{Induktionsmenge} (IM)
		$$ :\iff \begin{cases}(i) & 1 \in A; \\ (ii) & \text{aus } x \in A \text{ folgt stets } x + 1 \in A. \end{cases}$$ \\
		Beispiele: $\R$, $[1, \infty)$, $\{ 1 \} \cup [2, \infty)$ sind Induktionsmengen. 
		\item $\N \coloneqq \{ x \in \R : x$ gehört zu jeder IM $\}$ = Durchschnitt aller Induktionsmengen. \\
			Also: $\N \subseteq A$ für jede Induktionsmenge $A$. \\
			Beispiele: $1,2,3,4,17 \in \N$; $\frac{3}{2} \notin \N$.
	\end{enumerate}	
\end{definition}

\begin{satz} ~\ \label{1.3:satz}
	\begin{enumerate}
		\item $\N$ ist eine Induktionsmenge.
		\item $\N$ ist nicht nach oben beschränkt.\label{1.3.b:satz}
		\item Ist $x \in \R$, so existiert ein $n \in \N$ mit $n > x$. \label{1.3.c:satz}
	\end{enumerate}
\end{satz}

\begin{proof} ~\
	\begin{enumerate}
	        \item Es gilt $1 \in A$ für jede IM $A$, also $1 \in \N$. Sei $x \in \N$. Dann ist $x \in A$ für jede IM $A$, somit
	               $x+1 \in A$ für jede IM $A$. Also gilt $x+1 \in \N$.
	        \item Annahme: $\N$ ist beschränkt. Nach $(A15)$ existiert $s:=\sup \N$. Mit \ref{1.2:satz} folgt:\\
	               $\exists n \in \N:$ $n>s-1$. Nun ist $n+1 > s$. Wegen $n+1 \in \N$ ist aber $n+1 \le s$, ein Widerspruch.
	        \item Folgt aus \ref{1.3.b:satz}.
	\end{enumerate}
\end{proof}

\index{vollständige Induktion}
\begin{satz}[Prinzip der vollständigen Induktion] \label{1.4:prop} ~\\
	Ist $A \subseteq \N$ und ist $A$ eine Induktionsmenge, so ist $A = \N$.
\end{satz}

\begin{proof}
	Es gilt $A \subseteq \N$ (nach Voraussetzung) und $\N \subseteq A$ (nach Definition), also ist $A = \N$.
\end{proof}



\subsection*{Beweisverfahren durch vollständige Induktion}
Es sei $A(n)$ eine Aussage, die für jedes $n \in \N$ definiert ist. Für $A(n)$ gelte:
$$\begin{cases}
	(i) & A(1) \text{ ist wahr;} \\ (ii) & \text{ist } n \in \N \text{ und } A(n) \text{ wahr, so ist auch } A(n + 1) \text{ wahr.}
\end{cases}$$
Dann ist $A(n)$ wahr für \textbf{jedes} $n \in \N$.

\begin{proof}
	Sei $A \coloneqq \{ n \in \N : A(n)$ ist wahr $\}$. Dann ist $A \subseteq \N$ und wegen $(i)$, $(ii)$ ist $A$ eine Induktionsmenge.
	Nach {\ref{1.4:prop}} ist $A = \N$.
\end{proof}

\bigskip

\begin{beispiel*}
	Behauptung: $\forall n \in \N: ~ \underbrace{1 + 2 + \dotsc + n = \frac{n (n + 1)}{2}}_{A(n)}$.	
	\begin{proof}(induktiv) \\
		Induktionsanfang (I.A.): Es gilt $1 = \frac{1 (1 + 1)}{2}$, $A(1)$ ist also wahr. \\
		Induktionsvoraussetzung (I.V.): Für ein $n \in \N$ sei $A(n)$ wahr, es gelte also 
		$$1 + 2 + \dotsc + n = \frac{n (n + 1)}{2}.$$
		Induktionsschlu{\ss} ($n \curvearrowright n + 1$): Es gilt:
		$$
		1 + 2 + \dotsc + n + (n + 1) \overset{I.V.}{=}  \frac{n (n + 1)}{2} + (n + 1)
		$$	
		$$
		= (n + 1) \left( \frac{n}{2} + 1 \right)  = \frac{(n + 1)(n + 2)}{2}.
		$$
		Also ist $A(n + 1)$ wahr.
	\end{proof}
\end{beispiel*}

\index{ganze Zahlen} \index{rationale Zahlen} 
\begin{definition} Wir setzen:
	\begin{enumerate}
		\item $\N_{0} \coloneqq \N \cup \{ 0 \}$.
		\item $\Z \coloneqq \N_{0} \cup \{ - n : n \in \N \}$ (Menge der ganzen Zahlen).
		\item $\Q \coloneqq \{ \frac{p}{q} : p \in \Z, q \in \N \}$ (Menge der rationalen Zahlen).
	\end{enumerate}
\end{definition}


\begin{satz} \label{1.5:satz}
	Sind $x, y \in \R$ und $x < y$, so gilt: $\exists r \in \Q$: $x < r < y$.	

	\begin{proof}
		In den Übungen.
	\end{proof}
\end{satz}

\index{Fakultät} \index{Binomialkoeffizient} \index{Binomischer Satz} \index{Bernoullische Ungleichung}
\subsection*{Einige Definitionen und Formeln} 
\begin{enumerate}
	\item \textbf{Ganzzahlige Potenzen}.\\ Für $a \in \R$, $n \in \N: a^{n} \coloneqq \underbrace{a \cdot \dotsc \cdot a}_{n \text{ Faktoren}}$, $a^{0} \coloneqq 1$. \\ 
	        Für $a \in \R\setminus \{0\}$, $n \in \N: a^{-n} \coloneqq \frac{1}{a^{n}}$. \\
		Es gelten die bekannten Rechenregeln.
	\item \textbf{Fakultäten}. 
	        $$ n! \coloneqq 1 \cdot 2 \cdot 3 \cdot \dotsc \cdot n ~ (n \in \N), \quad 0! \coloneqq 1.$$
	\item \textbf{Binomialkoeffizienten}. Für $n \in \N_{0}, k \in \N_{0}$ und $k \leq n$:
		$$ \binom{n}{k} \coloneqq \frac{n!}{k!(n - k)!}. $$
		Es gilt (nachrechnen):
		$$ \binom{n}{k} + \binom{n}{k - 1} = \binom{n + 1}{k} \quad \text{für } 1 \leq k \leq n. $$
	\item Für $a, b \in \R$ und $n \in \N_0$ gilt: 
		\begin{align*}
			a^{n + 1} - b^{n + 1} & = (a - b) \left(a^{n} + a^{n-1}b + a^{n-2}b^{2} + \dotsc + a b^{n-1} + b^{n} \right) \\
				& = (a - b) \sum_{k = 0}^{n} a^{n -k}b^{k} = (a - b) \sum_{k = 0}^{n} a^{k}b^{n-k}.
		\end{align*}
	\item \textbf{Binomischer Satz}. Für $a, b \in \R$ und $n \in \N_0$ gilt: $$(a + b)^{n} = \sum_{k = 0}^{n} \binom{n}{k} a^{n-k}b^{k}.$$
		\begin{proof}
			In den Übungen.
		\end{proof}
	\item \textbf{Bernoullische Ungleichung}. Es sei $x \in \R$ und $x \geq -1$. Dann gilt:
		$$\forall n \in \N: ~  (1 + x)^{n} \geq 1 + n x.$$
		\begin{proof}(induktiv) \\
			I.A.: $n = 1$: $1 + x \geq 1 + x$ ist wahr.\\
			I.V.: Für ein $n \in \N$ gelte $(1 + x)^{n} \geq 1 + nx$. \\
			I.S.: $n \curvearrowright n + 1$: Wegen $1 + x \geq 0$ folgt aus der I.V.:
			$$
			(1 + x)^{n + 1}  \geq (1 + nx)(1 + x)  = 1 + nx + x + \underbrace{nx^{2}}_{\geq 0} 
			$$
			$$
			 \geq 1 + nx + x  = 1 + (n + 1)x.
			$$
		\end{proof}
\end{enumerate}


\begin{hilfssatz} \label{HS1}
	Für $x, y \geq 0$ und $n \in \N$ gilt: $x \leq y \iff x^{n} \leq y^{n}$.

	\begin{proof}
		In den Übungen.
	\end{proof}
\end{hilfssatz}

\index{Wurzeln}
\begin{satz} \label{1.6:satz}
	Es sei $a \geq 0$ und $n \in \N$. Dann gibt es genau ein $x \geq 0$ mit $x^{n} = a$. \\
	Dieses $x$ hei{\ss}t \textbf{die $n$-te Wurzel aus $a$}. Bezeichnung: $x = \sqrt[n]{a}$ ($\sqrt[2]{a} \eqqcolon \sqrt{a}$, $\sqrt[1]{a} = a$).
	
	\begin{proof}
		Existenz: Später in \S 7. Eindeutigkeit: Es seien $x, y \geq 0$ und $x^{n} = a = y^{n}$. Mit \ref{HS1} folgt $x = y$.
	\end{proof}
\end{satz}


\begin{bemerkungen} \
	\begin{enumerate}
		\item Bekannt (Schule): $\sqrt{2} \notin \Q$.
		\item Für $a \geq 0$ ist $\sqrt[n]{a} \geq 0$. Bsp.: $\sqrt{4} = 2$, $\sqrt{4} \neq - 2$. Die Gleichung $x^{2} = 4$ hat zwei Lösungen: $x = 2$ und $x=-2$.
		\item $$\forall x \in \R: ~ \sqrt{x^{2}} = |x|.$$
	\end{enumerate}
\end{bemerkungen}


\subsection*{Rationale Exponenten}
\begin{enumerate}
	\item Es sei zunächst $a \ge 0$ und $r \in \Q$, $r > 0$. Dann existieren $m, n \in \N$ mit $r = \frac{m}{n}$. Wir wollen definieren:
		\begin{align*}
		(\ast) \quad \quad a^{r} \coloneqq \left( \sqrt[n]{a} \right)^{m}. 			
		\end{align*}
		Problem: Gilt auch noch $r = \frac{p}{q}$ mit $p, q \in \N$, gilt dann $\left( \sqrt[n]{a} \right)^{m} = \left( \sqrt[q]{a} \right)^{p}$? \\
		Antwort: Ja (d.h. obige Definition $(*)$ ist sinnvoll).
		\begin{proof}
			Setze $x \coloneqq \left( \sqrt[n]{a} \right)^{m}$, $y \coloneqq \left( \sqrt[q]{a} \right)^{p}$. Dann gilt $x, y \geq 0$ und $mq = np$, also
			\begin{align*}
				x^{q} & = \left( \sqrt[n]{a} \right)^{mq} = \left( \sqrt[n]{a} \right)^{np} = \left(  \left( \sqrt[n]{a} \right)^{n}\right)^{p} = a^{p} \\
					  & = \left( \left( \sqrt[q]{a} \right)^{q}\right)^{p} = \left( \left( \sqrt[q]{a} \right)^{p}\right)^{q} = y^{q}.
			\end{align*}
			Mit \ref{HS1} folgt $x = y$.  
		\end{proof}
	\item Es seien $a > 0$, $r \in \Q$ und $r < 0$. Wir definieren: $$a^{r} \coloneqq \frac{1}{a^{-r}}.$$
	          Es gelten die bekannten Rechenregeln: $a^{r} a^{s} = a^{r + s}, \left( a^{r} \right)^{s} = a^{rs}$.
\end{enumerate}


\newpage


\chapter{Folgen und Konvergenz}

\index{Folge}
\begin{definition}
	Es sei $X$ eine Menge, $X \neq \emptyset$. Eine Funktion $a \colon \N \to X$ hei{\ss}t eine \textbf{Folge in} $X$. Ist $X = \R$, so hei{\ss}t $a$ eine 
	\textbf{reelle Folge}.
\end{definition}


\begin{schreibweisen}
$a_{n}$ statt $a(n)$ ($n$-tes Folgenglied) \\
$(a_{n})$ oder $(a_{n})_{n = 1}^{\infty}$ oder $(a_{1}, a_{2}, \dotsc)$ statt $a$.
\end{schreibweisen}


\begin{beispiele} ~\
	\begin{enumerate}
		\item $a_{n} \coloneqq \frac{1}{n}$ $~(n \in \N)$, also $(a_{n}) = (1, \frac{1}{2}, \frac{1}{3}, \dotsc)$.
		\item $a_{2n} \coloneqq 0$, $a_{2n-1} \coloneqq 1$ $~(n \in \N)$, also $(a_{n}) = (1, 0, 1, 0, \dotsc)$.
	\end{enumerate}
\end{beispiele}


\begin{bemerkung}
	Ist $p \in \Z$ und $a \colon \{ p, p + 1, p+2, \dotsc \} \to X$ eine Funktion, so spricht man ebenfalls von einer Folge in $X$. Bezeichnung: $(a_{n})_{n = p}^{\infty}$. 
	Meistens ist $p = 0$ oder $p = 1$.
\end{bemerkung}

\index{abzählbar} \index{uberabzahlbar@überabzählbar}
\begin{definition}
	Es sei $X$ eine Menge, $X \neq \emptyset$.
	\begin{enumerate}
		\item $X$ hei{\ss}t \textbf{abzählbar} $:\iff$ Es gibt eine Folge $(a_{n})$ in $X$ mit $X =\{ a_{1}, a_{2}, a_{3}, \dotsc \}$.
		\item $X$ hei{\ss}t \textbf{überabzählbar} $:\iff X$ ist nicht abzählbar.
	\end{enumerate}
\end{definition}


\begin{beispiele} ~\
	\begin{enumerate}
		\item Ist $X$ endlich, so ist $X$ abzählbar.
		\item $\N$ ist abzählbar, denn $\N = \{ a_{1}, a_{2}, a_{3}, \dotsc \}$ mit $a_{n} \coloneqq n$ $(n \in \N)$.
		\item $\Z$ ist abzählbar, denn $\Z = \{ a_{1}, a_{2}, a_{3}, \dotsc \}$ mit 
		      $$a_{1} \coloneqq 0, ~ a_{2} \coloneqq 1, ~ a_{3} \coloneqq -1, ~ a_{4} \coloneqq 2, ~ a_{5} \coloneqq -2, \dotsc$$ 
		      also 
		      $$a_1:=0, \quad a_{2n} \coloneqq n, \quad a_{2n + 1} \coloneqq -n \quad (n \in \N). $$
		\item $\Q$ ist abzählbar.
			\begin{figure*}[!ht] \centering
				\begin{tikzpicture}
					\matrix(m)[matrix of math nodes,column sep=1cm,row sep=1cm]{1 & 2 & 3 & 4 & 5 & 6 & \cdots \\
    					\frac{1}{2} & \frac{2}{2} & \frac{3}{2} & \frac{4}{2} & \frac{5}{2} & \cdots & \cdots \\
    					\frac{1}{3} & \frac{2}{3} & \frac{3}{3} & \frac{4}{3} & \frac{5}{3} & \cdots \\
    					\frac{1}{4} & \frac{2}{4} & \frac{3}{4} & \frac{4}{4} & \cdots \\
    					\frac{1}{5} & \frac{2}{5} & \cdots & \cdots \\
    					\cdots & \cdots &  \\
					};
					\draw[->]
					(m-1-1)edge(m-1-2) (m-1-2)edge(m-2-1) (m-2-1)edge(m-3-1) (m-3-1)edge(m-2-2) (m-2-2)edge(m-1-3) (m-1-3)edge(m-1-4) 
					(m-1-4)edge(m-2-3) (m-2-3)edge(m-3-2) (m-3-2)edge(m-4-1) (m-4-1)edge(m-5-1) (m-5-1)edge(m-4-2) (m-4-2)edge(m-3-3) 
					(m-3-3)edge(m-2-4) (m-2-4)edge(m-1-5) (m-1-5)edge(m-1-6); 
				\end{tikzpicture}
    			\caption{Zum Beweis der Abzählbarkeit von $\Q$.}
			\end{figure*} \\
			Durchnummerieren in Pfeilrichtung liefert:
				$$ \{ x \in \Q : x > 0 \} = \{ a_{1}, a_{2}, a_{3}, \dotsc \}. $$
			Setze $b_{1} \coloneqq 0, b_{2n} \coloneqq a_{n}, b_{2n + 1} \coloneqq - a_{n}$ $(n \in \N)$. Dann gilt:
			$$ \Q = \{ b_{1}, b_{2}, b_{3}, \dotsc \}. $$
		\item $\R$ ist überabzählbar (Beweis in \S 5).
	\end{enumerate}	
\end{beispiele}


\begin{vereinbarung}
	Solange nichts anderes gesagt wird, seien alle vorkommenden Folgen stets Folgen in $\R$. 	                                                                                               
	Die folgenden Sätze und Definitionen formulieren wir nur für Folgen der Form $(a_{n})_{n=1}^{\infty}$. Sie gelten sinngemä{\ss} für Folgen der Form 
	$(a_{n})_{n=p}^{\infty}$ $(p \in \Z)$.
\end{vereinbarung}

\index{beschränkt!Folge}
\begin{definition}
	Es sei $(a_{n})$ eine Folge und $M \coloneqq \{ a_{1}, a_{2}, \dotsc \}$.
	\begin{enumerate}
		\item$(a_{n})$ hei{\ss}t \textbf{nach oben beschränkt} $:\iff M$ ist nach oben beschränkt.\\ 
		In diesem Fall: $$\sup_{n \in \N} a_{n} \coloneqq \sup_{n = 1}^{\infty} a_{n} \coloneqq \sup M.$$
		\item$(a_{n})$ hei{\ss}t \textbf{nach unten beschränkt} $:\iff M$ ist nach unten beschränkt. \\ 
		In diesem Fall: $$\inf_{n \in \N} a_{n} \coloneqq \inf_{n = 1}^{\infty} a_{n} \coloneqq \inf M.$$
		\item$(a_{n})$ hei{\ss}t \textbf{beschränkt} $~:\iff M$ ist beschränkt. Äquivalent ist: 
			$$\exists c \geq 0 ~ \forall n \in \N: ~ |a_{n}| \leq c$$
	\end{enumerate}
\end{definition}

\index{für fast alle}
\begin{definition}
	Es sei $A(n)$ eine für jedes $n \in \N$ definierte Aussage. \\
	$A(n)$ gilt \textbf{für fast alle} (ffa) $n \in \N$ $:\iff \exists n_{0} \in \N$ $\forall n \geq n_{0}: A(n)$ ist wahr.
\end{definition}

\index{Umgebung}
\begin{definition}
	Es sei $a \in \R$ und $\varepsilon > 0$. Das Intervall
		$$ U_{\varepsilon}(a) \coloneqq (a - \varepsilon, a + \varepsilon) = \{ x \in \R : | x - a| < \varepsilon \} $$
	hei{\ss}t $\varepsilon$\textbf{-Umgebung von $a$}.
\end{definition}

\index{konvergent} \index{Grenzwert} \index{Limes} \index{divergent}
\begin{definition}
	Eine Folge $(a_{n})$ hei{\ss}t \textbf{konvergent}
	$$ :\iff \exists a \in \R : \begin{cases} \text{Zu jedem } \varepsilon > 0 \text{ existiert ein } n_{0} = n_{0}(\varepsilon) \in \N \text{ so,} \\
		\text{da{\ss} für jedes } n \geq n_{0} \text{ gilt }: |a_{n} - a| < \varepsilon.
	\end{cases} $$
	In diesem Fall hei{\ss}t $a$ \textbf{Grenzwert} (GW) oder \textbf{Limes} von $(a_{n})$ und man schreibt
	$$ 
		a_{n} \rightarrow a ~(n \rightarrow \infty) \text{ oder } a_{n} \rightarrow a \text{ oder } \lim_{n \rightarrow \infty} a_{n} = a.
	$$
	Ist $(a_{n})$ nicht konvergent, so hei{\ss}t $(a_{n})$ \textbf{divergent}. Beachte:
	\begin{align*}
		a_{n} \rightarrow a ~(n \rightarrow \infty) & \iff \forall \varepsilon > 0 ~\exists n_{0} \in \N ~\forall n \geq n_{0}: ~a_{n} \in U_{\varepsilon}(a) \\
				& \iff \forall \varepsilon > 0 \text{ gilt: } a_{n} \in U_{\varepsilon}(a) \text{ ffa } n \in \N \\
				& \iff \forall \varepsilon > 0 \text{ gilt: } a_{n} \notin U_{\varepsilon}(a) \text{ für höchstens endlich viele } n \in \N
	\end{align*}
\end{definition}


\begin{satz} \label{2.1:satz}
	Es sei $(a_{n})$ konvergent und $a = \lim_{n \rightarrow \infty} a_{n}$. Dann gilt:
	\begin{enumerate}
		\item Gilt auch noch $a_{n} \rightarrow b$, so ist $a = b$.
		\item $(a_{n})$ ist beschränkt. \label{2.1.b:satz}
	\end{enumerate}
	
	\begin{proof}\
	  \begin{enumerate}
		\item Annahme $a \neq b$. Dann ist $\varepsilon \coloneqq \frac{|a - b|}{2} > 0$. Nun gilt:
			$$
			\exists n_{0} \in \N ~ \forall n \geq n_{0}: |a_n - a| < \varepsilon \text{ und } 
			\exists n_{1} \in \N ~ \forall n \geq n_{1}: |a_n - b| < \varepsilon. 
			$$
			Es sei $N \coloneqq \max \{ n_{0}, n_{1} \}$. Dann gilt:
			$$
				2 \varepsilon = |a - b| = | a - a_{N} + a_{N} - b| \leq |a_{N} - a| + |a_{N} - b| < 2 \varepsilon.
			$$
			Widerspruch. Also ist $ a = b$.
		\item   Es sei $\varepsilon = 1$. Es gilt: $\exists n_{0} \in \N ~\forall n \geq n_{0}: |a_{n} - a| < 1$. Damit folgt:
			$$
				\forall n \geq n_{0}: ~ |a_{n}| = |a_{n} - a + a| \leq |a_{n} - a| + |a| \leq 1 + |a|.
			$$
			Setze $c \coloneqq \max \{ 1 + |a|, |a_{1}|, \dotsc, |a_{n_{0} - 1}| \}$. Dann: $\forall n \in \N: |a_{n}| \leq c$.
	  \end{enumerate}
	\end{proof}	
\end{satz}


\begin{beispiele}\
	\begin{enumerate}
		\item Es sei $c \in \R$ und $a_{n} \coloneqq c$ $(n \in \N)$. Dann gilt:
			$$
				\forall n \in \N: ~ | a_{n} - c | = 0.
			$$
			Also: $a_{n} \rightarrow c$ $(n \to \infty)$.
		\item $a_{n} \coloneqq \frac{1}{n} ~(n \in \N)$. Behauptung: $a_{n} \rightarrow 0 ~(n \rightarrow \infty)$.
			\begin{proof}
				Es sei $\varepsilon > 0$. Es gilt: $|a_{n} - 0 | = |a_{n}| = \frac{1}{n} < \varepsilon \iff n > \frac{1}{\varepsilon}$. Mit
				\ref{1.3.c:satz} erhalten wir:
				$$
						\exists n_{0} \in \N: ~ n_{0} > \frac{1}{\varepsilon}.
				$$
				Für $n \geq n_{0}$ ist damit $n > \frac{1}{\varepsilon}$, also $\frac{1}{n} < \varepsilon$. Somit ist
				$|a_{n} - 0| < \varepsilon$ $(n \geq n_{0})$.
			\end{proof}
		\item $a_{n} \coloneqq (-1)^{n} ~(n \in \N)$. Es gilt $|a_{n}| = 1$ $(n \in \N)$, also ist $(a_{n})$ beschränkt. \\
		      Behauptung: $(a_{n})$ ist divergent.
			\begin{proof}
				Für jedes $n \in \N$ gilt: 
				$$ |a_{n} - a_{n+1}| = |(-1)^{n} - (-1)^{n+1}| = |(-1)^{n}|| 1 - (-1) | = 2. $$
				Annahme: $(a_{n})$ konvergiert. Definiere $a \coloneqq \lim_{n \to \infty} a_{n}$. Es gilt:
				$$
					 \exists n_{0} \in \N ~ \forall n \geq n_{0}: ~ |a_{n} - a| < \frac{1}{2}. 
				$$
				Für $n \geq n_{0}$ folgt dann aber:
				$$
					2 = |a_{n} - a_{n+1}| = |a_{n} - a + a - a_{n + 1}| \leq |a_{n} - a| + |a_{n+1} - a| < \frac{1}{2} + \frac{1}{2} = 1,
				$$
				ein Widerspruch.
			\end{proof}
		\item $a_{n} \coloneqq n$ $(n \in \N)$. $(a_{n})$ ist nicht beschränkt. Nach \ref{2.1.b:satz} ist $(a_{n})$ also divergent.
		\item $a_{n} \coloneqq  \frac{1}{\sqrt{n}}$ $(n \in \N)$. Behauptung: $a_{n} \rightarrow 0$.
			\begin{proof}
				Es sei $\varepsilon > 0$. Es gilt:
				$$
					|a_{n} - 0| = \frac{1}{\sqrt{n}} < \varepsilon \iff \sqrt{n} > \frac{1}{\varepsilon} \iff n > \frac{1}{\varepsilon^{2}}.
				$$
				Mit {\ref{1.3.c:satz}} erhalten wir: $$\exists n_{0} \in \N: n_{0} > \frac{1}{\varepsilon^{2}}.$$
				Für $n \geq n_{0}$ gilt damit: $n > \frac{1}{\varepsilon^{2}} \Rightarrow \frac{1}{\sqrt{n}} < \varepsilon$, also $|a_{n} - 0 | < \varepsilon$. 
			\end{proof}
		\item $a_{n} \coloneqq \sqrt{n + 1} - \sqrt{n} ~(n \in \N)$. Behauptung: $a_{n} \rightarrow 0$.
			\begin{proof}
                                Es gilt			
				$$
					0 \le a_{n} = \frac{(\sqrt{n + 1} - \sqrt{n})(\sqrt{n + 1} + \sqrt{n})}{\sqrt{n + 1} + \sqrt{n}} 
					= \frac{1}{\sqrt{n + 1} + \sqrt{n}} \leq \frac{1}{\sqrt{n}},
				$$
				also $|a_{n} - 0| =a_n \leq \frac{1}{\sqrt{n}} ~ (n \in \N)$. Es sei $\varepsilon > 0$. Nach Beispiel e) folgt:
				$$
					\exists n_{0} \in \N ~\forall n \geq n_{0}: ~ \frac{1}{\sqrt{n}} < \varepsilon, \text{ somit gilt } 
					\forall n \geq n_{0}: ~ |a_{n} - 0| < \varepsilon. 
				$$
				Also gilt: $a_{n} \rightarrow 0$.
			\end{proof}
	\end{enumerate}
\end{beispiele}


\begin{definition}
	Es seien $(a_{n})$ und $(b_{n})$ Folgen und $\alpha \in \R$.
	$$
		(a_{n}) \pm (b_{n}) \coloneqq (a_{n} \pm b_{n}); ~
		\alpha (a_{n}) \coloneqq (\alpha a_{n}); ~
		(a_{n}) (b_{n}) \coloneqq (a_{n} b_{n}). 		
	$$	
	Gilt $b_{n} \neq 0$ $(n \ge m)$, so ist die Folge $\left( \frac{a_{n}}{b_{n}} \right)_{n = m}^{\infty}$ definiert.
\end{definition}


\begin{satz} \label{2.2:satz}
	Es seien $(a_{n}), (b_{n}), (c_{n})$ und $(\alpha_{n})$ Folgen und $a, b, \alpha \in \R$. Dann gilt:

	\begin{enumerate}
		\item $a_{n} \rightarrow a \iff |a_{n} - a| \rightarrow 0$.
		\item Gilt $|a_{n} - a| \leq \alpha_{n}$ ffa $n \in \N$ und $\alpha_{n} \rightarrow 0$, so gilt $a_{n} \rightarrow a$.
		\item Es gelte $a_{n} \rightarrow a$ und $b_{n} \rightarrow b$. Dann gilt:
			\begin{enumerate}
				\item $|a_{n}| \rightarrow |a|$; 
				\item $a_{n} + b_{n} \rightarrow a + b$;
				\item $\alpha a_{n} \rightarrow \alpha a$;
				\item $a_{n} b_{n} \rightarrow a b$;
				\item ist $a \neq 0$, so existiert ein $m \in \N$ mit:
					$$
						a_{n} \neq 0 ~ (n \geq m) \text{ und für die Folge } 
						\left( \frac{1}{a_{n}} \right)_{n = m}^{\infty} \text{ gilt: } \frac{1}{a_{n}} \rightarrow \frac{1}{a}.
					$$
			\end{enumerate}
		\item Es gelte $a_{n} \rightarrow a$, $b_{n} \rightarrow b$ und $a_{n} \leq b_{n}$ ffa $n \in \N$. Dann ist $a \leq b$.
		\item Es gelte $a_{n} \rightarrow a$, $b_{n} \rightarrow a$ und $a_{n} \leq c_{n} \leq b_{n}$ ffa $n \in \N$. Dann gilt $c_{n} \rightarrow a$. \label{2.2.e:satz}
	\end{enumerate}
\end{satz}

\begin{beispiele} \
	\begin{enumerate}
		\item Es sei $p \in \N$ und $a_{n} \coloneqq \frac{1}{n^{p}}$ $(n \in \N)$. Es gilt $n \leq n^{p}$ $(n \in \N)$. Also: 
			$$ 0 \leq a_{n} \leq \frac{1}{n} ~ (n \in \N) ~  \xRightarrow[]{\ref{2.2.e:satz}}   ~ a_{n} \rightarrow 0. $$
		\item Es sei $a_{n} \coloneqq \frac{5n^{2} + 3n + 1}{4n^{2} - n + 2}$ $(n \in \N)$. Es gilt: 
		      $a_n = \frac{5 + \frac{3}{n} + \frac{1}{n^{2}}}{4 - \frac{1}{n} + \frac{2}{n^{2}}} \xrightarrow[]{\ref{2.2:satz}} \frac{5}{4}$.
	\end{enumerate}
	
	\begin{proof}(von 2.2) ~\
		\begin{enumerate}
			\item Folgt aus der Definition der Konvergenz.
			\item   Es gilt: $\exists m \in \N ~ \forall n \geq m: ~ |a_{n} - a | \leq \alpha_{n}$. Sei $\varepsilon > 0$. Wegen $\alpha_n \to 0$ gilt:
				$$
		 		\exists n_{1} \in \N ~ \forall n \geq n_{1}:  ~ \alpha_{n} < \varepsilon.
		 		$$
		 		Setze $n_{0} \coloneqq \max \{ m , n_{1} \}$. Für $n \geq n_{0}$ gilt nun: $|a_{n} - a| \leq \alpha_{n} < \varepsilon$.
			\item \begin{enumerate}
				\item $\forall n \in \N: ~ | |a_{n}| - |a|| \overset{\S 1}{\leq} |a_{n} - a| ~ \xRightarrow[]{a), b)} ~ |a_{n}| \rightarrow |a|.$
				\item Es sei $\varepsilon > 0$. Es gilt: $\exists n_{1}, n_{2} \in \N$ mit 
				        $$
				        \forall n \geq n_{1}: ~ |a_{n} - a| < \frac{\varepsilon}{2}  \text{ und }  \forall n \geq n_{2}: ~ |b_{n} - b| < \frac{\varepsilon}{2}.
				        $$
					Setze $n_{0} \coloneqq \max \{ n_{1}, n_{2} \}$. Für $n \geq n_{0}$ erhalten wir:
					$$
						|a_{n} + b_{n} - (a + b)| = |a_{n} - a + b_{n} - b| \leq |a_{n} - a| + |b_{n} - b| 
						< \frac{\varepsilon}{2} + \frac{\varepsilon}{2} = \varepsilon.
					$$
				\item Übung.
				\item Es sei $c_{n} \coloneqq |a_{n} b_{n} - ab|$ $(n \in \N)$. Wir zeigen: $c_{n} \rightarrow 0$. Es gilt:
					\begin{align*}
						c_{n} & = |a_{n}b_{n} - a_{n}b + a_{n}b - ab| = |a_{n}(b_{n} - b)+ (a_{n} - a)b| \\
							  & \leq |a_{n}||b_{n} - b| + |b||a_{n}-a|.
					\end{align*}
					Mit \ref{2.1.b:satz} folgt: $\exists c \geq 0 ~ \forall n \in \N: ~ |a_{n}| \leq c$. Damit erhalten wir:
					$$
					\forall n \in \N: ~ c_{n} \leq c|b_{n}-b| + |b||a_{n}-a| \eqqcolon \alpha_{n}.
					$$	
					Mit c) (ii), c) (iii) und a) folgt: $\alpha_{n} \rightarrow 0$. \\
					Also: $|c_{n} - 0| = c_{n} \leq \alpha_{n}$ $(n \in \N)$ und $\alpha_{n} \rightarrow 0$. Mit b) folgt nun $c_{n} \rightarrow 0$.
				\item   Setze $\varepsilon \coloneqq \frac{|a|}{2}$. Aus (i) folgt: $|a_{n}| \rightarrow |a|$. Damit gilt: 
					$$  
					\exists m \in \N ~ \forall n \geq m: ~ |a_{n}| \in U_{\varepsilon}(|a|) = (|a| - \varepsilon, |a| + \varepsilon) 
					= (\frac{|a|}{2}, \frac{3}{2} |a|). 
					$$
					Insbesondere ist $|a_{n}| > \frac{|a|}{2} > 0$ $(n \geq m)$, also $a_{n} \neq 0$ $(n \geq m)$. Für $n \geq m$ gilt nun:
					$$ \left| \frac{1}{a_{n}} - \frac{1}{a} \right| = \frac{|a_{n} - a|}{|a_{n}||a|} \leq \frac{2|a_{n} - a|}{|a|^{2}} \eqqcolon \alpha_{n}. $$
					Es gilt $\alpha_{n} \rightarrow 0$. Mit b) folgt $\frac{1}{a_{n}} \rightarrow \frac{1}{a}$.
			  \end{enumerate}
			\item Annahme: $b < a$. Setze $\varepsilon \coloneqq \frac{a-b}{2} > 0$.  
					Dann gilt: $$\forall x \in U_{\varepsilon}(b) ~\forall y \in U_{\varepsilon}(a): ~x < y.$$ Weiter gilt: 
					$$ \exists n_{0} \in \N~ \forall n \geq n_{0}: ~ b_{n} \in U_{\varepsilon}(b), $$
					$$ \exists m \in \N ~\forall n \geq m: ~ a_{n} \leq b_{n}. $$
				Setze $m_{0} \coloneqq \max \{ n_{0}, m \}$. Für $n \geq m_{0}$ ist $a_{n} \leq b_{n} < b + \varepsilon$, also $a_{n} \notin U_{\varepsilon}(a)$.
				Widerspruch.
			\item   Es gilt: $\exists m \in \N ~\forall n \geq m: ~ a_{n} \leq c_{n} \leq b_{n}$. Sei $\varepsilon > 0$. Es existieren $n_{1}, n_{2} \in \N$ mit: 
				\begin{align*}
					\forall n \geq n_{1}: ~ a - \varepsilon < a_{n} < a + \varepsilon,  \\
					\forall n \geq n_{2}: ~ a - \varepsilon < b_{n} < a + \varepsilon.
				\end{align*}
				Setze $n_{0} \coloneqq \max \{ n_{1}, n_{2}, m \}$. Für $n \geq n_{0}$ gilt nun:
				$$
					a - \varepsilon < a_{n} \leq c_{n} \leq b_{n} < a + \varepsilon.
				$$
				Also: $|a_{n} - a| < \varepsilon$ $(n \geq n_{0})$.
		\end{enumerate}	
	\end{proof}	
\end{beispiele}

\index{monoton}  \index{monoton!wachsend}   \index{monoton!streng wachsend} \index{monoton!streng fallend} \index{monoton!fallend}
\begin{definition}\ 
	\begin{enumerate}
		\item $(a_{n})$ hei{\ss}t \textbf{monoton wachsend} $:\iff \forall n \in \N: ~ a_n \leq a_{n+1}$.
		\item $(a_{n})$ hei{\ss}t \textbf{streng monoton wachsend} $:\iff \forall n \in \N: ~ a_n < a_{n+1}$.
		\item Entsprechend definiert man \textbf{monoton fallend} und \textbf{streng monoton fallend}.
		\item $(a_{n})$ hei{\ss}t \textbf{[streng] monoton} $:\iff (a_{n})$ ist [streng] monoton wachsend oder [streng] monoton fallend.
	\end{enumerate}
\end{definition}

\index{Monotoniekriterium}
\begin{satz}[Monotoniekriterium] ~\ \label{2.3:prop}
	\begin{enumerate}
		\item Die Folge $(a_{n})$ sei monoton wachsend und nach oben beschränkt. Dann ist $(a_{n})$ konvergent und 
			$$
				\lim_{n \rightarrow \infty} a_{n} = \sup_{n \in \N} a_{n}.
			$$
		\item Die Folge $(a_{n})$ sei monoton fallend und nach unten beschränkt. Dann ist $(a_{n})$ konvergent und 
			$$
				\lim_{n \rightarrow \infty} a_{n} = \inf_{n \in \N} a_{n}.
			$$
	\end{enumerate}
\end{satz}

\begin{proof} ~\
\begin{enumerate}		
	\item	Setze $a \coloneqq \sup_{n \in \N} a_{n}$. Es sei $\varepsilon > 0$. Dann ist $a - \varepsilon$ keine obere Schranke von $\{ a_{n}: n \in \N\}$. 
	                Also existiert ein $n_{0} \in \N$ mit $a_{n_{0}} > a - \varepsilon$. Für $n \geq n_{0}$ gilt:
			$$
				a - \varepsilon < a_{n_{0}} \leq a_{n} \leq a < a + \varepsilon,
			$$
			also $|a_{n} - a| < \varepsilon$ $(n \geq n_{0})$.
	\item   Zeigt man analog.
\end{enumerate}
\end{proof}

%\begin{figure*}[!ht] \centering
%	\begin{tikzpicture}
%      \draw[->] (-0.5,0) -- (6,0) node[right] {$x$};
%      \draw[->] (0,-0.5) --  (0,4) node[above] {$y$};
%      \draw[-] (-0.5,3.25) --  (6,3.25) node[above] {$a$};
%      \draw[dotted] (-0.5,2.75) --  (6,2.75) node[below] {$a - \varepsilon$};
%      \draw[scale=1,domain=0:6,loosely dotted,variable=\x,thick] plot ({\x},{4*(5*\x/(5*\x+10))+0.1});
%    \end{tikzpicture}
%    \caption{Zum Beweis des Monotonie-Kriteriums.}
%\end{figure*}


\begin{beispiel*} $a_{1} \coloneqq \sqrt[3]{6}$, $a_{n + 1} \coloneqq \sqrt[3]{6 + a_{n}}$ $(n \geq 1)$.
	
	Behauptung: $\forall n \in \N:$ $0 < a_{n} < 2$ und $a_{n + 1} > a_{n}$.

	\begin{proof}(induktiv) \\
		I.A.: $n=1$. 
		\begin{description}
		\item $0 < a_{1} = \sqrt[3]{6} < \sqrt[3]{8} = 2$;
		\item $a_{2} = \sqrt[3]{6 + a_{1}} > \sqrt[3]{6} = a_{1}$.
	\end{description}
		
		I.V.: Es sei $n \in \N$ und $0 < a_{n} < 2$ und $a_{n+1} > a_{n}$. \\
		I.S. $n \curvearrowright n + 1$: Es gilt $a_{n + 1} = \sqrt[3]{6 + a_{n}} >_{I.V.} 0$. Weiter ist
		$$
			a_{n +1} = \sqrt[3]{6 + a_{n}} <_{I.V.} \sqrt[3]{6 + 2} = 2; \quad a_{n + 2} = \sqrt[3]{6 + a_{n+1}} >_{I.V.} \sqrt[3]{6 + a_{n}} = a_{n + 1}.
		$$
	\end{proof}
		Also ist  $(a_{n})$ nach oben beschränkt und monoton wachsend. Nach {\ref{2.3:prop}} ist $(a_{n})$ konvergent. 
		Setze $a \coloneqq \lim_{n \to \infty} a_{n}$. Es gilt $a_{n} \geq 0$ $(n \in \N)$, also $a \geq 0$. Weiter ist
		$$
			a_{n+1}^{3} = 6 + a_{n} \quad (n \in \N).
		$$
		Mit {\ref{2.2:satz}} folgt $a^{3} = 6 + a \Rightarrow 0 = a^{3} - a - 6 = (a-2)(\underbrace{a^{2}+2a+3}_{\geq 3})$. Also ist $a = 2$.
\end{beispiel*}


\textbf{Wichtige Beispiele:} 


Vorbemerkung: Es seien $x, y \geq 0$ und $p \in \N$: Es ist (vgl. \S 1)
	$$ x^{p} - y^{p} = (x - y) \sum_{k = 0}^{p-1} x^{p-1-k}y^{k} $$
$$ \Rightarrow |x^{p} - y^{p}| = |x-y| \sum_{k=0}^{p-1} x^{p-1-k}y^{k} \geq y^{p-1} |x - y|.$$

\begin{beispiel} \label{2.4:bsp}
	Es sei $(a_{n})$ eine konvergente Folge in $[0, \infty)$ mit Grenzwert $a$ (bea. $a \ge 0$) und $p \in \N$. Dann gilt $\sqrt[p]{a_{n}} \rightarrow \sqrt[p]{a}$.
	
	\begin{proof} ~\\
		Fall 1: $a = 0$. Es sei $\varepsilon > 0$. Dann gilt: $\exists n_{0} \in \N ~\forall n \geq n_{0}: ~ 0 \le a_{n} < \varepsilon^{p}$. Daraus folgt:
		$$  \forall n \geq n_{0}: ~ 0 \le \sqrt[p]{a_{n}} < \varepsilon.$$
		Also gilt: $\sqrt[p]{a_{n}} \rightarrow 0 = \sqrt[p]{a}$.
		
		Fall 2: $a \neq 0$. Dann gilt:
		\begin{align*}
			|a_{n} - a| & = | (\underbrace{\sqrt[p]{a_{n}}}_{\eqqcolon x})^{p} - (\underbrace{\sqrt[p]{a}}_{\eqqcolon y})^{p} | =  |x^{p} - y^{p}| \\
					& \geq_{s.o.} \underbrace{y^{p-1}}_{\coloneqq c} |x - y| = c | \sqrt[p]{a_{n}} - \sqrt[p]{a} |, \quad c > 0.
		\end{align*}
		$\Rightarrow |\sqrt[p]{a_{n}} - \sqrt[p]{a}| \leq \frac{1}{c} |a_{n} - a| \eqqcolon \alpha_{n}$. Es gilt $\alpha_{n} \rightarrow 0$, also
		$\sqrt[p]{a_{n}} \rightarrow \sqrt[p]{a}$.
	\end{proof} 
\end{beispiel}


\begin{beispiel} \label{2.5:bsp}
	Für $x \in \R$ gilt: $(x^{n})$ ist konvergent $\iff x \in (-1,1]$. In diesem Fall:
	$$
		\lim_{n \rightarrow \infty} x^{n} = \begin{cases} 1, & \text{falls } x = 1 \\ 0, & \text{falls } x \in (-1 , 1) \end{cases}
	$$
	
	\begin{proof} ~\\
		Fall 1: $x = 0$. Dann gilt $x^{n} \rightarrow 0$. Fall 2: $x = 1$. Dann gilt $x^{n} \rightarrow 1$. \\
		Fall 3: $x = -1$. Dann ist $(x^{n}) = ((-1)^{n})$ divergent. \\
		Fall 4: $|x| > 1$. Dann gibt es ein $\delta > 0$ mit $|x| = 1 + \delta$.  Damit gilt: 
		$$|x^{n}| = |x|^{n} = (1 + \delta)^{n} \geq 1 + n \delta \geq n \delta \quad (n \in \N).$$
		Also ist $(x^{n})$ nicht beschränkt und somit divergent. \\
		Fall 5: $0 < |x| < 1$. Dann ist $\frac{1}{|x|} > 1$ und es gibt ein $\eta > 0$ mit $\frac{1}{|x|} = 1 + \eta$. Damit gilt:
		$$
			\left|\frac{1}{x^{n}}\right| = \left( \frac{1}{|x|} \right)^{n} = (1 + \eta)^{n} \geq 1 + n \eta \geq n \eta \quad (n \in \N).
		$$
		Also ist $$|x^{n}| \leq \frac{1}{n \eta}  \quad (n \in \N).$$
		Damit folgt $x^{n} \rightarrow 0$.
	\end{proof}	
\end{beispiel}


\begin{beispiel} \label{2.6:bsp}
	Es sei $x \in \R$ und 
	$$s_{n} \coloneqq 1 + x + x^{n} + \dotsc x^{n} = \sum_{k = 0}^{n} x^{k} \quad (n \in \N_0).$$
	Fall 1: $x = 1$. Dann ist $s_{n} = n + 1$ $(n \in \N_0)$, $(s_{n})$ ist also divergent. \\
	Fall 2: $x \neq 1$. Dann ist  $$s_{n} = \frac{1 - x^{n+1}}{1 - x} \quad (n \in \N_0).$$
	Aus \ref{2.5:bsp} folgt:
	$$
		(s_{n}) \text{ ist konvergent} \quad \iff \quad |x| < 1.
	$$
	In diesem Fall gilt: $\lim_{n \to \infty} s_{n} = \frac{1}{1 - x}$.
\end{beispiel}


\begin{beispiel} \label{2.7:bsp}
	Behauptung: Es gilt $\sqrt[n]{n} \rightarrow 1$.
	
	\begin{proof}
		Es ist $\sqrt[n]{n} \geq 1$ $(n \in \N)$, also $a_{n} \coloneqq \sqrt[n]{n} - 1 \geq 0$ $(n \in \N)$. Wir zeigen: $a_{n} \rightarrow 0$. \\
		Für jedes $n \geq 2$ gilt:
		$$
			n = \left( \sqrt[n]{n} \right)^{n} = \left( a_{n} + 1 \right)^{n} 
			\overset{\S 1}{=} \sum_{k=0}^{n} \binom{n}{k} a_{n}^{k} \geq \binom{n}{2} a_{n}^{2} = \frac{n(n-1)}{2} a_{n}^{2}.
		$$
		Es folgt 
		$$\forall n\ge 2: ~ 0 \leq a_{n} \leq \frac{\sqrt{2}}{\sqrt{n-1}}.$$ Wegen $\sqrt{2}/\sqrt{n-1} \to 0$ folgt $a_{n} \rightarrow 0$.
	\end{proof}
\end{beispiel}


\begin{beispiel} \label{2.8:bsp}
	Es sei $c > 0$. Behauptung: Es gilt $\sqrt[n]{c} \rightarrow 1$.
	
	\begin{proof}
		Fall 1: $c \geq 1$. Dann gilt: $\exists m \in \N:$ $1 \leq c \leq m$. Daraus folgt:
		$$ 1 \leq c \leq n ~ (n \geq m) ~ \Rightarrow ~ 1 \leq \sqrt[n]{c} \leq \sqrt[n]{n} ~ (n\geq m). $$
		Mit 2.7 folgt die Behauptung. \\
		Fall 2: $0 < c < 1$. Dann ist $\frac{1}{c} > 1$. Also gilt
		$$\sqrt[n]{c} = \frac{1}{\sqrt[n]{\frac{1}{c}}} \xrightarrow[Fall 1]{} 1  \quad (n \rightarrow \infty).$$ 
	\end{proof}
\end{beispiel}


\begin{beispiel} \label{2.9:bsp}
	Es sei 
	$$a_{n} \coloneqq \left( 1 + \frac{1}{n} \right)^{n}, ~ b_{n} \coloneqq \sum_{k = 0}^{n} \frac{1}{k!} = 1 + 1 + \frac{1}{2!} + \dotsc + \frac{1}{n!} \quad (n \in \N).$$
	Behauptung: $(a_{n})$ und $(b_{n})$ sind konvergent und $\lim_{n \to \infty} a_{n} = \lim_{n \to \infty}  b_{n}$.
	
	\begin{proof}
		In der großen Übungen wird gezeigt: $\forall n \in \N: ~ 2 \leq a_{n} < a_{n+1} < 3$. Nach \ref{2.3:prop} ist $(a_n)$ also konvergent; 
		$a \coloneqq \lim_{n \to \infty} a_{n}$. \\
		Weiter ist $b_{n} > 0$ und $b_{n+1} = b_{n} + \frac{1}{(n+1)!} > b_{n}$ $(n \in \N)$. Also ist $(b_{n})$ monoton wachsend. Für jedes $n > 3$ gilt:
		\begin{align*} 
			b_{n} & = 1 + 1 + \frac{1}{2} + \underbrace{\frac{1}{2 \cdot 3}}_{< \left(\frac{1}{2}\right)^{2}} + \underbrace{\frac{1}{2 \cdot 3 \cdot 4}}_{< 
			\left(\frac{1}{2}\right)^{3}} + \dotsc + \underbrace{\frac{1}{2 \cdot \dotsc \cdot n}}_{< \left(\frac{1}{2}\right)^{n-1}} \\
			& < 1 + \left( 1 + \frac{1}{2} + \left(\frac{1}{2}\right)^{2} + \dotsc + \left(\frac{1}{2}\right)^{n-1} \right) 
			= 1 + \frac{1 - \left( \frac{1}{2} \right)^{n}}{1 - \frac{1}{2}} \\
			& < 1 + \frac{1}{1 - \frac{1}{2}} = 3.
		\end{align*} 
		Nach \ref{2.3:prop} ist $(b_{n})$ konvergent; $b \coloneqq \lim_{n \to \infty} b_{n}$. \\
		Weiter gilt für jedes $n \geq 2$:
		\begin{align*}
			a_{n} & = \left( 1 + \frac{1}{n} \right)^{n} \overset{\S 1}{=} \sum_{k=0}^{n} \binom{n}{k} \frac{1}{n^{k}} \\
				  & = 1 + 1 + \sum_{k = 2}^{n} \frac{1}{k!} \frac{n!}{(n-k)!} \frac{1}{n^{k}} 
				  = 1 + 1 + \sum_{k=2}^{n} \frac{1}{k!} \frac{n(n-1) \cdot \dotsc \cdot (n-(k-1))}{n \cdot n \cdot \dotsc \cdot n} \\
				  & = 1 + 1 + \sum_{k=2}^{n} \frac{1}{k!} \underbrace{(1 - \frac{1}{n})}_{< 1} \underbrace{(1 - \frac{2}{n})}_{< 1} \cdot \dotsc \cdot 
				  \underbrace{(1 - \frac{k-1}{n})}_{< 1} \\
				  & \leq 1 + 1 + \sum_{k=2}^{n} \frac{1}{k!} = b_{n}.
		\end{align*}
		Also gilt $a_{n} \leq b_{n}$ $(n \geq 2)$ und damit folgt $a \leq b$. \\
		Weiter sei $j \in \N$, $j \geq 2$ (zunächst fest). Für jedes $n \in \N$ mit $n \geq j$ gilt:
		\begin{align*}
			a_{n} & \overset{s.o.}{=} 1 + 1 + \sum_{k=2}^{n} \frac{1}{k!} (1-\frac{1}{n})(1-\frac{2}{n}) \cdot \dotsc \cdot (1-\frac{k-1}{n}) \\
				  & \geq 1 + 1 + \sum_{k = 2}^{j} \frac{1}{k!} \underbrace{(1-\frac{1}{n})}_{\rightarrow 1} 
				  \underbrace{(1-\frac{2}{n})}_{\rightarrow 1} \cdot \dotsc \cdot \underbrace{(1-\frac{k-1}{n})}_{\rightarrow 1} \\
				  & \rightarrow 1 + 1 + \sum_{k=2}^{j} \frac{1}{k!} = b_{j} \quad (n \rightarrow \infty).
		\end{align*}
		Also gilt $a \geq b_{j}$ für jedes $j \geq 2$. Wegen $b_j \to b$ $(j \rightarrow \infty)$ folgt $a \geq b$.
	\end{proof}
\end{beispiel}

\index{Eulersche Zahl}
\begin{definition} Die gemeinsame Grenzwert der Folgen in \ref{2.9:bsp}
	$$
		e \coloneqq \lim_{n \rightarrow \infty} \left( 1 + \frac{1}{n} \right)^{n} = \lim_{n \rightarrow \infty} \sum_{k = 0}^{n} \frac{1}{k!} 
	$$
	hei{\ss}t \textbf{Eulersche Zahl}. ($e \approx 2,718\dotsc$).
\end{definition}

Übung: Es gilt: $2 < e < 3$.

\index{Teilfolge}
\begin{definition} 
	Es sei $(a_{n})$ eine Folge und $(n_{1}, n_{2}, n_{3}, \dotsc)$ eine Folge in $\N$ mit \\
	$n_{1} < n_{2} < n_{3} < \dotsc$. Für $k \in \N$ setze
	$$
		b_{k} \coloneqq a_{n_{k}},
	$$
	also $b_{1} = a_{n_{1}}, b_{2} = a_{n_{2}}, b_{3} = a_{n_{3}},\dotsc$. \\
	Dann hei{\ss}t $(b_{k}) = (a_{n_{k}})$ eine \textbf{Teilfolge} (TF) von $(a_{n})$.
\end{definition}


\begin{beispiele}\
	\begin{enumerate}
		\item $(a_{2}, a_{4}, a_{6}, \dotsc)$ ist eine Teilfolge von $(a_{n})$; hier: $n_{k} = 2k$.
		\item $(a_{1}, a_{4}, a_{9}, \dotsc)$ ist eine Teilfolge von $(a_{n})$; hier: $n_{k} = k^2$.
		\item $(a_{2}, a_{6}, a_{4}, a_{10}, a_{8}, a_{14}, \dotsc)$ ist keine Teilfolge von $(a_{n})$.
	\end{enumerate}
\end{beispiele}


\begin{definition}
	Es sei $(a_{n})$ eine Folge. Eine Zahl $\alpha \in \R$ hei{\ss}t ein \textbf{Häufungswert} (HW) von $(a_{n})$,
	wenn eine Teilfolge  $(a_{n_{k}})$ von $(a_{n})$ existiert mit $a_{n_{k}} \rightarrow \alpha ~(k \rightarrow \infty)$. Weiter sei	
	$$
	H(a_{n}) \coloneqq \{ \alpha \in \R: \alpha \text{ ist ein Häufungswert von } (a_{n}) \}.
	$$
	
\end{definition}


\begin{satz} \label{2.10:satz} Es gilt:
	$$
	 \alpha \in H(a_{n})  \iff \forall \varepsilon > 0: ~ a_{n} \in U_{\varepsilon}(\alpha) \text{ für unendlich viele }  n \in \N. 
	$$
\end{satz}

\begin{proof} ~\\
	"'$\Rightarrow$"': Es sei $(a_{n_{k}})$ eine Teilfolge mit $a_{n_{k}} \rightarrow \alpha$ und es sei $\varepsilon > 0$. Dann existiert ein $k_{0} \in \N$ mit
	$a_{n_{k}} \in U_{\varepsilon}(\alpha)$ für $k \geq k_{0}$. \\
	"'$\Leftarrow$"': Es gilt: \\
	$\exists n_{1} \in \N: a_{n_{1}} \in U_{1}(\alpha)$, \\
	$\exists n_{2} \in \N: a_{n_{2}} \in U_{\frac{1}{2}}(\alpha)$ und $n_{2} > n_{1}$, \\
	$\exists n_{3} \in \N: a_{n_{3}} \in U_{\frac{1}{3}}(\alpha)$ und $n_{3} > n_{2}$,  etc... \\
	So entsteht eine Teilfolge $(a_{n_{k}})$ von $(a_{n})$ mit $a_{n_{k}} \in U_{\frac{1}{k}}(\alpha)$ $(k \in \N)$.
	Also gilt: $a_{n_{k}} \rightarrow \alpha$. 
\end{proof}


\begin{beispiele}\
	\begin{enumerate}
		\item $a_{n} = (-1)^{n}$ $(n \in \N)$. Es gilt: $a_{2k} \rightarrow 1, a_{2k+1} \rightarrow -1$, also $1, -1 \in H(a_{n})$. 
		Es sei $\alpha \in \R\setminus \{-1,1\}$. Wähle $\varepsilon>0$ so, da{\ss} $1, -1 \notin U_{\varepsilon}(\alpha)$. Dann gilt $a_{n} \in U_{\varepsilon}(\alpha)$ für kein 
		$n \in \N$. Nach \ref{2.10:satz} ist $\alpha \notin H(a_{n})$. Fazit: $H(a_{n}) = \{ 1, -1 \}$.
		\item $a_{n} = n$ $(n \in \N)$. Ist $\alpha \in \R$ und $\varepsilon > 0$, so gilt: $a_{n} \in U_{\varepsilon}(\alpha)$ für höchstens endlich viele $n$, 
		also $\alpha \notin H(a_{n})$. Fazit: $H(a_{n}) = \emptyset$.
		\item $\Q$ ist abzählbar. Es sei $(a_{n})$ eine Folge mit $\Q = \{a_{n}: n \in \N\}$. Es sei $\alpha \in \R$ und $\varepsilon > 0$. Nach {\ref{1.5:satz}} 
		enthält $U_{\varepsilon}(\alpha) = (\alpha - \varepsilon, \alpha + \varepsilon)$ unendlich viele verschiedene rationale Zahlen. Nach {\ref{2.10:satz}} folgt 
		$\alpha \in H(a_{n})$. Fazit: $H(a_{n}) = \R$.
	\end{enumerate}	
\end{beispiele}

\begin{folgerung*}
Ist $x \in \R$, so existieren Folgen $(r_{n})$ in $\Q$ mit $r_{n} \rightarrow x$.	
\end{folgerung*}


\begin{satz} \label{2.11:satz} 
	Die Folge $(a_{n})$ sei konvergent, $a \coloneqq \lim_{n \to \infty} a_{n}$ und $(a_{n_{k}})$ eine Teilfolge von $(a_{n})$. Dann gilt:
	$$ a_{n_{k}} \rightarrow a \quad (k \rightarrow \infty). $$
	Insbesondere gilt: $H(a_{n}) = \{ \lim_{n \to \infty} a_{n} \}$.
\end{satz}

\begin{proof}
	Es sei $\varepsilon > 0$. Dann ist $a_{n} \in U_{\varepsilon}(a)$ ffa $n \in \N$, also auch $a_{n_{k}} \in U_{\varepsilon}(a)$ ffa $k \in \N$. Somit gilt 
	$a_{n_{k}} \rightarrow a$.
\end{proof}

\index{niedrig}
\begin{definition} Es sei $(a_{n})$ eine Folge und $m \in \N$. \\
$m$ hei{\ss}t \textbf{niedrig} (für $(a_{n})$) $:\iff ~\forall n \geq m: ~  a_{n} \geq a_{m}$.		
\end{definition}

\begin{bemerkung}
Es gilt also: \\
$m \in \N$ ist nicht niedrig $\iff \exists n \geq m: ~ a_{n} < a_{m}$ $\Rightarrow \exists n > m: ~ a_{n} < a_{m}$.
\end{bemerkung}


\begin{hilfssatz} \label{HS2}
	Es sei $(a_{n})$ eine Folge. Dann enthält $(a_{n})$ eine monotone Teilfolge.	
\end{hilfssatz}

\begin{proof} ~\\
	Fall 1: Es existieren höchstens endlich viele niedrige Indizes. Also existiert $n_{1} \in \N$ so, da{\ss} jedes $n \geq n_{1}$ nicht niedrig ist.
	\begin{description}
		\item $n_{1}$ nicht niedrig $\Rightarrow \exists n_{2} > n_{1} : a_{n_{2}} < a_{n_{1}}$,
		\item $n_{2}$ nicht niedrig $\Rightarrow \exists n_{3} > n_{2} : a_{n_{3}} < a_{n_{2}}$,
		\item etc$\dotsc$
	\end{description}
	Wir erhalten so eine streng monoton fallende Teilfolge $(a_{n_{k}})$ von $(a_n)$. \\
	Fall 2: Es existieren unendlich viele niedrige Indizes $n_{1}, n_{2},n_3 \dotsc$; o.B.d.A. sei $$n_{1} < n_{2} < n_3 < \dotsc.$$
	\begin{description}
		\item $n_{1}$ ist niedrig und $n_{2} > n_{1} \Rightarrow a_{n_{2}} \geq a_{n_{1}}$,
		\item $n_{2}$ ist niedrig und $n_{3} > n_{2} \Rightarrow a_{n_{3}} \geq a_{n_{2}}$,
		\item etc$\dotsc$
	\end{description}
	Wir erhalten so eine monoton wachsende Teilfolge $(a_{n_{k}})$ von $(a_n)$.
\end{proof}

\index{Satz!Bolzano-Weierstra{\ss}}
\begin{satz}[Bolzano-Weierstra{\ss}] \label{2.12:satz-BolzanoWeierstrass}  ~\\
	Die Folge $(a_{n})$ sei beschränkt. Dann gilt: $H(a_{n}) \neq \emptyset$, d.h. $(a_{n})$ enthält eine konvergente Teilfolge.
\end{satz}

\begin{proof}
	Es gilt: $\exists c \geq 0 ~ \forall n \in \N: ~ |a_{n}| \leq c$. Nach \ref{HS2} enthält $(a_{n})$ eine monotone Teilfolge $(a_{n_{k}})$. 
	Wegen $|a_{n_{k}}| \leq c$ $(k \in \N)$ ist $(a_{n_{k}})$ auch beschränkt. \\
	Nach \ref{2.3:prop} ist $(a_{n_{k}})$ konvergent. Damit ist $\lim_{k \rightarrow \infty} a_{n_{k}} \in H(a_{n})$.
\end{proof}

\begin{satz} \label{2.13:satz}
	Die Folge $(a_{n})$ sei beschränkt (nach \ref{2.12:satz-BolzanoWeierstrass} gilt damit $H(a_{n}) \neq \emptyset$). Es gilt:
	\begin{enumerate}
		\item $H(a_{n})$ ist beschränkt.
		\item $\sup H(a_{n}), \inf H(a_{n}) \in H(a_{n})$; es existieren also $\max H(a_{n})$ und $\min H(a_{n})$.
	\end{enumerate}
\end{satz}



\begin{proof}\
	\begin{enumerate}
		\item Es gilt: $\exists c \geq 0 ~ \forall n \in \N: ~ |a_{n}| \leq c$. Es sei $\alpha \in H(a_{n})$. Dann existiert eine Teilfolge $(a_{n_{k}})$ von $(a_n)$
		        mit $a_{n_{k}} \rightarrow \alpha$ $(k \rightarrow \infty)$. Es ist $|a_{n_{k}}| \leq c$ $(k \in \N)$, also $|\alpha| \leq c$. Somit gilt
		        $$
		        \forall \alpha \in H(a_{n}): ~ |\alpha| \leq c.
		        $$
		\item ohne Beweis.
	\end{enumerate}
\end{proof}


\index{Limes superior} \index{oberer Limes} \index{Limes inferior} \index{unterer Limes}
\begin{definition} 
	Die Folge $(a_{n})$ sei beschränkt. 
	\begin{enumerate}
		\item Die Zahl
		$$
		\limsup_{n \rightarrow \infty} a_{n} \coloneqq \overline{\lim}_{n \rightarrow \infty} a_{n} \coloneqq \max H(a_{n})
		$$
	        hei{\ss}t \textbf{Limes superior} oder \textbf{oberer Limes} von $(a_{n})$.
		\item Die Zahl 
		$$
		\liminf_{n \rightarrow \infty} a_{n} \coloneqq \underline{\lim}_{n \rightarrow \infty} a_{n} \coloneqq \min H(a_{n})
		$$
		hei{\ss}t \textbf{Limes inferior} oder \textbf{unterer Limes} von $(a_{n})$.
	\end{enumerate}
\end{definition}


\begin{satz} \label{2.14:satz}
	Die Folge $(a_{n})$ sei beschränkt. Dann gilt:
	\begin{enumerate}
		\item $\forall \alpha \in H(a_{n}): ~ \liminf_{n \rightarrow \infty} a_{n} \leq \alpha \leq \limsup_{n \rightarrow \infty} a_{n}$.
		\item Ist $(a_{n})$ konvergent, so ist $\limsup_{n \rightarrow \infty} a_{n} = \liminf_{n \rightarrow \infty} a_{n} = \lim_{n \rightarrow \infty} a_{n}$.
		\item $\forall \alpha \geq 0: ~ \limsup_{n \rightarrow \infty}(\alpha a_{n}) = \alpha \limsup_{n \rightarrow \infty} a_{n}$.
		\item $\limsup_{n \rightarrow \infty} (-a_{n}) = - \liminf_{n \rightarrow \infty} a_{n}$.
	\end{enumerate}
\end{satz}

\begin{proof}
	a) ist klar, b) folgt aus \ref{2.11:satz}, c) und d) Übung.
\end{proof}


\textbf{Vorbemerkung:} Die Folge $(a_{n})$ sei konvergent und $\lim_{n \rightarrow \infty}  a_{n} \eqqcolon a$. Es sei $\varepsilon > 0$. Dann gilt:
	$$ \exists n_{0} \in \N ~ \forall n \geq n_{0}: ~ |a_{n} - a| < \frac{\varepsilon}{2}.$$
Für $n, m \geq n_{0}$ gilt damit:
	$$ |a_{n} - a_{m}| = |a_{n} - a + a - a_{m} | \leq |a_{n} - a| + |a_{m} - a| < \frac{\varepsilon}{2} + \frac{\varepsilon}{2} = \varepsilon. $$
Die Folge $(a_{n})$ hat also die folgende Eigenschaft:
	\begin{align*}
	(c) \quad \quad	\forall \varepsilon > 0 ~  \exists n_{0} \in \N ~ \forall n,m \geq n_{0}: ~ |a_{n} - a_{m}| < \varepsilon.
	\end{align*}
Äquivalent ist:	
$$\forall \varepsilon > 0 ~\exists n_{0} \in \N ~ \forall n \ge n_0 ~ \forall k \in \N: |a_{n} - a_{n+k}| < \varepsilon.$$

\index{Cauchyfolge}
\begin{definition} 
	Eine Folge $(a_{n})$ hei{\ss}t eine \textbf{Cauchyfolge} (CF)
	$$ :\iff (a_{n}) \text{ hat die Eigenschaft } (c). $$	
\end{definition}


Konvergente Folgen sind also Cauchyfolgen!

\index{Cauchykriterium}
\begin{satz}[Cauchykriterium] $(a_{n}) \text{ ist konvergent} \iff (a_{n}) \text{ ist eine Cauchyfolge}$. \label{2.15:prop} 
\end{satz}

\begin{proof}
	"'$\Rightarrow$"': wurde in obiger Vorbemerkung bewiesen. \\
	"'$\Leftarrow$"': Es gilt:
	$$
	\exists N \in \N ~ \forall n,m \ge N: ~ |a_n-a_m| < 1.
	$$
	Für $n \ge N$ ist somit
	$$
	|a_n| = |a_n-a_N+a_N| \le |a_n-a_N|+|a_N| < 1+|a_N| =: c.
	$$
	Also gilt:
	$$
	\forall n \in \N: ~ |a_n| \le \max\{c,|a_1|, \dots |a_{N-1}|\}.
	$$
	Damit ist $(a_n)$ beschränkt und nach \ref{2.12:satz-BolzanoWeierstrass} hat $(a_n)$ eine konvergente Teilfolge $(a_{n_{k}})$.
	Es sei $a:= \lim_{k \to \infty} a_{n_{k}}$. \\
	Es sei $\varepsilon > 0$. Dann gilt:
	$$
	\exists n_0 \in \N ~ \forall n,m \ge n_0: ~ |a_n-a_m| < \frac{\varepsilon}{2},
	$$
	und
	$$
	\exists k_0 \in \N: ~ |a_{n_{k_0}}-a| < \frac{\varepsilon}{2} \text{ und } n_{k_0} \ge n_0.
	$$
	Für jedes $n \ge n_0$ gilt nun
	$$
	|a_n-a| \le|a_n- a_{n_{k_0}}| + |a_{n_{k_0}}-a| < \frac{\varepsilon}{2}+\frac{\varepsilon}{2}= \varepsilon.
	$$
	Also gilt $a_n \to a$ $(n \to \infty)$.
\end{proof}


\begin{beispiel*}
	$a_{1} \coloneqq 1, a_{n+1} \coloneqq \frac{1}{1 + a_{n}}$ $(n \in \N)$. Mit Induktion folgt $0 < a_{n} \leq 1 ~(n \in \N)$ und damit $a_{n} \geq \frac{1}{2} ~(n \in \N)$.	
	Für $n \geq 2$ und $k \in \N$ gilt daher:
	\begin{align*}
		|a_{n+k} - a_{n} | & = \left| \frac{1}{1+a_{n+k-1}} - \frac{1}{1 - a_{n - 1}} \right| = \frac{|a_{n-1} - a_{n +k-1}|}{(1+a_{n+k-1})(1+a_{n-1})} \\
			& \leq \frac{1}{(1+\frac{1}{2})^{2}} |a_{n+k-1} - a_{n-1}| = \frac{4}{9} |a_{n+k-1} - a_{n-1}| \\
			& \leq \left(\frac{4}{9} \right)^{2} |a_{n-k-2} - a_{n-2}| \leq \dotsc \leq \left( \frac{4}{9} \right)^{n-1} |a_{k+1} - a_{1}| \\
			& \leq \left( \frac{4}{9} \right)^{n-1} \left( |a_{k+1}| + |a_{1}|\right) \leq 2 \left( \frac{4}{9} \right)^{n-1} .
	\end{align*}
	Es sei $\varepsilon > 0$. Wegen $2 \left( \frac{4}{9} \right)^{n-1} \to 0$ $(n \to \infty)$ gilt: 
	$$\exists n_{0} \in \N\setminus \{ 1 \} ~ \forall n \geq n_{0}: ~ 2\left(\frac{4}{9}\right)^{n-1} < \varepsilon.$$
	Wir erhalten:
		$$\forall n \geq n_{0} ~ \forall k \in \N: ~ |a_{n+k} - a_{n}| < \varepsilon. $$
	Also ist $(a_{n})$ eine Cauchyfolge und somit konvergent; $a \coloneqq \lim_{n \rightarrow \infty} a_{n}$. Klar ist: 
	$$ a \geq \frac{1}{2} \text{ und } a = \frac{1}{1 + a}.$$
	Also ist 
	$$a^{2} + a - 1 = 0 \Rightarrow a = - \frac{1}{2} + \frac{\sqrt{5}}{2} \text{ oder } a = - \frac{1}{2} - \frac{\sqrt{5}}{2}.$$ 
	Wegen $a \geq \frac{1}{2}$ folgt $a = \frac{\sqrt{5} - 1}{2}$.
\end{beispiel*}


\newpage


\chapter{Unendliche Reihen}


\index{Reihe} \index{Reihe!unendliche} \index{Teilsumme} \index{konvergent} \index{divergent} \index{Reihenwert}
\begin{definition} Es sei $(a_{n})$ sei eine Folge.
	\begin{enumerate}
		\item Wir setzen
		$$ s_{n} \coloneqq a_{1} + a_{2} + \dotsc + a_{n} \quad (n \in \N), $$
		also $s_{1} = a_{1}, s_{2} = a_{1} + a_{2},  s_{3} = a_{1} + a_{2} + a_3, \dotsc$. Die Folge $(s_{n})$ hei{\ss}t \textbf{(unendliche) Reihe} und wird mit 
		$\sum_{n = 1}^{\infty} a_{n}$ bezeichnet. Es gilt also: \\
		$\sum_{n=1}^{\infty} a_{n}$ ist konvergent bzw. divergent $\iff$ $(s_{n})$ ist konvergent bzw. divergent.
		\item $s_{n}$ hei{\ss}t $n$\textbf{-te Teilsumme} von $\sum_{n=1}^{\infty} a_{n}$.	
		\item Ist $\sum_{n = 1}^{\infty} a_{n}$ konvergent, so hei{\ss}t $\lim_{n \to \infty} s_{n}$ der \textbf{Reihenwert} und wird ebenfalls mit 
		$\sum_{n=1}^{\infty} a_{n}$ bezeichnet. ({\bf Vorsicht:} Doppelbedeutung von $\sum_{n=1}^{\infty} a_{n}$.)
	\end{enumerate} 	
\end{definition}


\begin{bemerkung}
	Ist $p \in \Z$ und $(a_{n})_{n=p}^{\infty}$ eine Folge, so definiert man entsprechend
		$$ s_{n} = a_{p} + a_{p+1} + \dotsc + a_{n} \quad (n \geq p) $$
	und $\sum_{n=p}^{\infty} a_{n}$ (meist: $p = 1$ oder $p = 0$).
\end{bemerkung}


Die folgenden Sätze und Definitionen formulieren wir nun für Reihen der Form $\sum_{n=1}^{\infty} a_{n}$. Diese Sätze und Definitionen gelten entsprechend für Reihen der Form 
$\sum_{n=p}^{\infty} a_{n}$ $(p \in \Z)$.

\index{Reihe!geometrische} \index{Reihe!harmonische}
\begin{beispiele} ~\
	\begin{enumerate}
		\item Es sei $x \in \R$. Die Reihe
		$$\sum_{n=0}^{\infty} x^{n} = 1 + x + x^{2} + x^3 + \dotsc$$ 
		hei{\ss}t \textbf{geometrische Reihe}. \\
			Hier ist $s_{n} = 1 + x + \dotsc + x^{n}$ $(n \in \N_0)$. Nach \ref{2.6:bsp} gilt: $(s_{n})$ konvergiert $\iff |x| < 1$ und 
			$\lim_{n \to \infty} s_{n} = \frac{1}{1 - x}$ für $|x| < 1$. Also: $\sum_{n=0}^{\infty} x^{n}$ konvergiert $\iff |x| < 1$ und 
			$$\sum_{n=0}^{\infty} x^{n} = \frac{1}{1 - x} \quad (|x| < 1).$$
		\item $$\sum_{n=1}^{\infty} \frac{1}{n(n+1)}; \quad a_{n} = \frac{1}{n(n+1)} = \frac{1}{n} - \frac{1}{n+1}.$$ Es gilt:
			\begin{align*}
				s_{n} & = a_{1} + \dotsc + a_{n} \\
						& = (1 - \frac{1}{2}) + (\frac{1}{2} - \frac{1}{3}) + \dotsc + (\frac{1}{n-1} - \frac{1}{n}) + (\frac{1}{n} - \frac{1}{n+1}) \\
						& = 1 - \frac{1}{n+1} \rightarrow 1.
			\end{align*}
			Also: $\sum_{n=1}^{\infty} \frac{1}{n(n+1)}$ ist konvergent und $\sum_{n=1}^{\infty} \frac{1}{n(n+1)} = 1$.
		\item   $$\sum_{n=0}^{\infty} \frac{1}{n!} = 1 + 1 + \frac{1}{2!} + \frac{1}{3!} + \dotsc.$$ Nach \ref{2.9:bsp} gilt:
		        $$s_{n} = 1 + 1 + \frac{1}{2!} + \dotsc + \frac{1}{n!} \rightarrow e \quad (n \to \infty).$$
			Also: $\sum_{n = 0}^{\infty} \frac{1}{n!}$ konvergiert und $\sum_{n=0}^{\infty} \frac{1}{n!} = e$.
		\item Die Reihe $$\sum_{n=1}^{\infty} \frac{1}{n}$$ hei{\ss}t \textbf{harmonische Reihe}. Hier ist $s_{n} = 1 + \frac{1}{2} + \dotsc + \frac{1}{n}$ $(n \in \N)$. 
		        Es gilt:
			$$ s_{2n} = 1 + \frac{1}{2} + \dotsc + \frac{1}{n} + \frac{1}{n+1} + \dotsc + \frac{1}{2n} 
			= s_{n} + \underbrace{\frac{1}{n+1}}_{\geq \frac{1}{2n}} + \dotsc + \underbrace{\frac{1}{2n}}_{\geq \frac{1}{2n}} \geq s_{n} + \frac{1}{2}. $$
			Annahme: $(s_{n})$ ist konvergent; $s \coloneqq \lim_{n \to\infty} s_{n}$. Mit \ref{2.11:satz} folgt $s_{2n} \rightarrow s$ $(n \to \infty)$. Somit gilt
			$$s \geq s + \frac{1}{2} \Rightarrow 0 \geq \frac{1}{2}. $$
			Widerspruch. Also: $\sum_{n=1}^{\infty} \frac{1}{n}$ ist divergent.
	\end{enumerate}	
\end{beispiele}

\index{Monotoniekriterium} \index{Cauchykriterium}
\begin{satz} \label{3.1:satz} 
	Es sei $(a_{n})$ eine Folge und $s_{n} = a_{1} + \dotsc + a_{n}$ $(n \in \N)$.
	\begin{enumerate}
		\item \textbf{Monotoniekriterium:} Sind alle $a_{n} \geq 0$ und ist $(s_{n})$ beschränkt, so ist $\sum_{n = 1}^{\infty} a_{n}$ konvergent.
		\item \textbf{Cauchykriterium:} $\sum_{n = 1}^{\infty} a_{n}$ ist konvergent $\iff$
			$$\forall \varepsilon > 0 ~ \exists n_{0} \in \N ~ \forall m > n \geq n_{0}: \left| \sum_{k = n+1}^{m} a_{k} \right| < \varepsilon. $$ \label{3.1.b:satz} 
		\item Ist $\sum_{n=1}^{\infty} a_{n}$ konvergent, so gilt $a_{n} \rightarrow 0$ $(n \to \infty)$. \label{3.1.c:satz} 
		\item Die Reihe $\sum_{n=1}^{\infty} a_{n}$ sei konvergent. Dann ist für jedes $m \in \N$ die Reihe $\sum_{n=m+1}^{\infty} a_{n}$ konvergent und für 
		      $r_{m} \coloneqq \sum_{n = m+1}^{\infty} a_{n}$ gilt: $r_{m} \rightarrow 0$ $(m \to \infty)$.
	\end{enumerate}
\end{satz}

\begin{proof} ~\
	\begin{enumerate}
		\item Es gilt: $s_{n+1} = a_{1} + \dotsc + a_{n} + a_{n+1} = s_{n} + a_{n+1} \geq s_{n}$ $(n \in \N)$. Also ist $(s_{n})$ wachsend und beschränkt.
		 Nach \ref{2.3:prop} ist $(s_{n})$ konvergent.
		\item Für $m > n$ gilt: 
		$$
		|s_{m} - s_{n}| = | a_{1} + \dotsc + a_{n} + a_{n+1} + \dotsc + a_{m} - (a_{1} + \dotsc a_{n})| 
		$$
		$$
		= |a_{n+1} + \dotsc + a_{m}| = |\sum_{k=n+1}^{m} a_{k}|.
		$$
		Die Behauptung folgt damit aus \ref{2.15:prop}.
		\item Es gilt: $s_{n+1} - s_{n} = a_{n+1}$ $(n \in \N)$. Ist $(s_{n})$ konvergent, so folgt $a_{n+1} \rightarrow 0$.
		\item Ohne Beweis.
	\end{enumerate}	
\end{proof}


\begin{bemerkung}
	Ist $(a_{n})$ eine Folge und gilt $a_{n} \not\rightarrow 0$, so ist $\sum_{n=1}^{\infty} a_{n}$ divergent.
\end{bemerkung}


\begin{satz} \label{3.2:satz}
	Die Reihen $\sum_{n=1}^{\infty} a_{n}$ und $\sum_{n=1}^{\infty} b_{n}$ seien konvergent und es seien $\alpha, \beta \in \R$. Dann konvergiert
		$$ \sum_{n=1}^{\infty} ( \alpha a_{n} + \beta b_{n}) $$
	und es gilt $$\sum_{n=1}^{\infty} ( \alpha a_{n} + \beta b_{n}) = \alpha \sum_{n=1}^{\infty} a_{n} + \beta \sum_{n=1}^{\infty} b_{n}.$$
\end{satz}

\begin{proof}
	Folgt aus \ref{2.2:satz}.
\end{proof}

\index{Konvergenzkriterium!Reihen!Leibniz}
\begin{satz}[Leibnizkriterium] \label{3.3:prop-LeibnizKriterium}
	Es sei $(b_{n})$ eine Folge mit:
	\begin{enumerate}
	 \item $(b_{n})$ ist monoton fallend,
	 \item $b_{n} \rightarrow 0$ $(n \to \infty)$.
	\end{enumerate}
	 Dann ist $\sum_{n=1}^{\infty} (-1)^{n+1}b_{n}$ konvergent.
\end{satz}

\index{Reihe!alternierende harmonische Reihe}
\begin{beispiel*}
	Aus \ref{3.3:prop-LeibnizKriterium} folgt: \\
	Die \textbf{alternierende harmonische Reihe} $\sum_{n=1}^{\infty} \frac{(-1)^{n+1}}{n}$ ist konvergent.
\end{beispiel*}

\begin{proof}(von \ref{3.3:prop-LeibnizKriterium}) Da $(b_n)$ eine fallende Nullfolge ist gilt: $b_n \ge 0$ $(n \in \N)$. Wir setzen $a_{n} \coloneqq (-1)^{n+1} b_{n}$ und
        $s_{n} \coloneqq a_{1} + \dotsc + a_{n}$ $(n \in \N)$. Es gilt:
	$$
	s_{2n+2} = s_{2n} + a_{2n+1} + a_{2n+2} = s_{2n} + \underbrace{b_{2n+1}-b_{2n+2}}_{\geq 0} \geq s_{2n} \quad (n \in \N).
	$$
	Also ist $(s_{2n})$ monoton wachsend. Analog zeigt man: $(s_{2n-1})$ ist monoton fallend. \\
	Weiter gilt:
	\begin{align*}
	(\ast)	\quad \quad s_{2n} = s_{2n-1} + a_{2n} = s_{2n-1} - b_{2n} \leq s_{2n-1} \quad (n \in \N).
	\end{align*}
	Also:
	$$ \forall n \in \N: ~ s_{2} \leq s_{4} \leq \dotsc \leq s_{2n} \overset{(*)}{\leq} s_{2n-1} \leq \dotsc \leq s_{3} \leq s_{1} $$
	Somit sind $(s_{2n})$ und $(s_{2n-1})$ beschränkt. Nach \ref{2.3:prop} sind $(s_{2n})$ und $(s_{2n-1})$ konvergent; $s \coloneqq \lim_{n \to \infty} s_{2n}$. Mit $(*)$ 
	folgt $s = \lim_{n \to \infty}  s_{2n-1}$. \\
	Es sei $\varepsilon > 0$. Es gilt:
	$$
		\begin{rcases*}
	 		s_{2n} \in U_{\varepsilon}(s) \text{ ffa } n \in \N \\
	 		s_{2n-1} \in U_{\varepsilon}(s) \text{ ffa } n \in \N  	
		\end{rcases*} \Rightarrow s_{n} \in U_{\varepsilon}(s) \text{ ffa } n \in \N
	$$
	Also gilt: $s_{n} \rightarrow s$ $(n \to \infty)$.
\end{proof}

\index{konvergent!absolut}
\begin{definition}
	$\sum_{n=1}^{\infty} a_{n}$ hei{\ss}t \textbf{absolut konvergent} $:\iff$ $\sum_{n=1}^{\infty} |a_{n}|$ ist konvergent.
\end{definition}


\begin{beispiel*}
	$\sum_{n=1}^{\infty} \frac{(-1)^{n+1}}{n}$ ist konvergent, aber nicht absolut konvergent.
\end{beispiel*}


\begin{satz} \label{3.4:satz}
	$\sum_{n=1}^{\infty} a_{n}$ sei absolut konvergent. Dann gilt:
	\begin{enumerate}
		\item $\sum_{n=1}^{\infty} a_{n}$ ist konvergent,
		\item $|\sum_{n=1}^{\infty} a_{n}| \leq \sum_{n=1}^{\infty} |a_{n}|$ ($\triangle$-Ungleichung für Reihen).
	\end{enumerate}
\end{satz}

\begin{proof} ~\
	\begin{enumerate}
		\item Für $m,n \in \N$, $m > n$ gilt:
			\begin{align*}
			(\ast)	 \quad \quad \underbrace{| \sum_{k = n+1}^{m} a_{k} |}_{\eqqcolon \sigma_{m, n}} \leq \underbrace{\sum_{k = n+1}^{m}|a_{k}|}_{\eqqcolon \tau_{m, n}}.
			\end{align*}
			Es sei $\varepsilon > 0$. Nach Voraussetzung und \ref{3.1.b:satz} gilt: 
			$$
			\exists n_{0} \in \N ~ \forall m > n > n_{0}: ~ \tau_{m, n} < \varepsilon,
			$$
			also mit $(\ast)$  
			$$
			\exists n_{0} \in \N ~ \forall m > n > n_{0}: ~ \sigma_{m, n} < \varepsilon.
			$$
			Nach \ref{3.1.b:satz} ist $\sum_{n=1}^{\infty} a_{n}$ konvergent.
		\item Es sei $s_{n} \coloneqq a_{1} + \dotsc + a_{n}$, $\sigma_{n} \coloneqq |a_{1}| + \dotsc |a_{n}|$ $(n \in \N)$,		
		        $s \coloneqq \lim_{n \to \infty} s_{n}$ und $\sigma \coloneqq \lim_{n \to \infty} \sigma_{n}$. Es gilt: $|s_{n}| \rightarrow |s|$ $(n \to \infty)$ und 
			$|s_n| \leq \sigma_n$ $(n \in \N)$. Damit folgt $|s| \leq \sigma$.
	\end{enumerate}
\end{proof}

\index{Konvergenzkriterium!Reihen!Majoranten} \index{Konvergenzkriterium!Reihen!Minoranten}
\begin{satz} ~\ \label{3.5:satz}
	\begin{enumerate}
		\item \textbf{Majorantenkriterium}: Gilt $|a_{n}| \leq b_{n}$ ffa $n \in \N$ und ist $\sum_{n=1}^{\infty} b_{n}$ konvergent, so ist $\sum_{n=1}^{\infty} a_{n}$ 
		absolut konvergent. \label{3.5.a:satz}
		\item \textbf{Minorantenkriterium}: Gilt $a_{n} \geq b_{n} \geq 0$ ffa $n \in \N$ und ist $\sum_{n=1}^{\infty}b_{n}$ divergent, so ist $\sum_{n=1}^{\infty} a_{n}$ 
		divergent. \label{3.5.b:satz}
	\end{enumerate}
\end{satz}

\begin{proof} ~\
	\begin{enumerate}
		\item Es gilt: $\exists j \in \N~ \forall n \geq j$: $|a_{n}| \leq b_{n}$. Nun sei $m > n \geq j$. Dann ist
			$$ \underbrace{\sum_{k=n+1}^{m}|a_{k}|}_{\eqqcolon \sigma_{m, n}} \leq \underbrace{\sum_{k=n+1}^{m} b_{k}}_{\eqqcolon \tau_{m,n}}. $$
			Es sei $\varepsilon > 0$. Nach Voraussetzung und \ref{3.1.b:satz} gilt: 
			$$
			\exists n_{0} \geq j ~ \forall m > n \geq n_{0}: \tau_{m,n} < \varepsilon,
			$$
			also
			$$
			\exists n_{0} \geq j ~ \forall m > n \geq n_{0}: \sigma_{m,n} < \varepsilon.
			$$
			Nach \ref{3.1.b:satz} ist $\sum_{n=1}^{\infty} |a_{n}|$ konvergiert.
		\item Annahme: $\sum_{n=1}^{\infty} a_{n}$ ist konvergent. Nach a) ist dann $\sum_{n=1}^{\infty} b_{n}$ konvergent. Widerspruch.
	\end{enumerate}	
\end{proof}


\begin{beispiele} ~\
	\begin{enumerate}
		\item $\sum_{n=1}^{\infty} \frac{1}{(n+1)^{2}}$, $a_n:= \frac{1}{(n+1)^{2}}$ $(n \in \N)$. F\"ur jedes $n \in \N$ gilt:
			$$ |a_{n}| = a_n = \frac{1}{(n+1)^{2}} = \frac{1}{n^{2} + 2n +1} \leq \frac{1}{n^{2} + 2n} \leq \frac{1}{n(n+1)} \eqqcolon b_{n}. $$
			Bekannt: $\sum_{n=1}^{\infty} b_{n}$ ist konvergent. Nach \ref{3.5.a:satz} ist auch $\sum_{n=1}^{\infty} \frac{1}{(n+1)^{2}}$ konvergent.
		\item Aus Beispiel a) folgt: $\sum_{n=1}^{\infty} \frac{1}{n^{2}}$ ist konvergent.
		\item Sei $\alpha > 0$ und $\alpha \in \Q$. Wir betrachten $\sum_{n=1}^{\infty} \frac{1}{n^{\alpha}}$. \\
			Fall 1: $\alpha \in (0, 1]$. 
				$$ \forall n \in \N: ~ \frac{1}{n^{\alpha}} \geq \frac{1}{n} \geq 0 \xRightarrow[]{\ref{3.5.b:satz}} 
				\sum_{n=1}^{\infty} \frac{1}{n^{\alpha}} \text{ divergiert.} $$
			Fall 2: $\alpha \geq 2$.
				$$ \forall n \in \N: ~ 0 \leq \frac{1}{n^{\alpha}} \leq \frac{1}{n^{2}} \xRightarrow[]{\ref{3.5.a:satz}} 
				\sum_{n=1}^{\infty} \frac{1}{n^{\alpha}} \text{ konvergiert.} $$
			Fall 3: $\alpha \in (1, 2)$. 
			        $$
			        \sum_{n=1}^{\infty} \frac{1}{n^{\alpha}} \text{ konvergiert}.
			        $$
			        Beweis in den Übungen. 
			
			\textbf{Fazit}: Ist $\alpha > 0$ und $\alpha \in \Q$, so gilt: 
			$$
			\sum_{n=1}^{\infty} \frac{1}{n^{\alpha}} \text{ konvergiert } \Leftrightarrow ~ \alpha > 1.
			$$
			\begin{bemerkung}
	                Ist später (in \S 7) die allgemeine Potenz $a^{x}$ ($a > 0, x \in \R)$ eingeführt, so zeigt man analog:
	                Ist $\alpha > 0$, so gilt:
	                $$
			\sum_{n=1}^{\infty} \frac{1}{n^{\alpha}} \text{ konvergiert } \Leftrightarrow ~ \alpha > 1.
			$$	                
                        \end{bemerkung}
		\item $\sum_{n=1}^{\infty} (-1)^{n} \frac{n + 2}{n^{3} + 1}$. Es gilt:		
		$$
		\left|(-1)^{n} \frac{n + 2}{n^{3} + 1} \right| = \frac{n+2}{n^{3} + 1} \leq \frac{n+2}{n^{3}} \leq \frac{2n}{n^{3}} = \frac{2}{n^{2}} \quad (n \ge 2).
		$$
		Die Reihe $\sum_{n=1}^{\infty} \frac{2}{n^{2}}$ ist konvergent. Nach \ref{3.5.a:satz} ist $\sum_{n=1}^{\infty} (-1)^{n} \frac{n + 2}{n^{3} + 1}$ absolut konvergent. 
		\item $\sum_{n=1}^{\infty} \frac{\sqrt{n}}{n+1}$. Es gilt
		$$
		\frac{\sqrt{n}}{n+1} \geq \frac{\sqrt{n}}{2n} = \frac{1}{2\sqrt{n}} \geq 0 \quad (n \in \N).
		$$
		Die Reihe $\sum_{n=1}^{\infty} \frac{1}{2\sqrt{n}}$ divergiert. Nach \ref{3.5.b:satz} ist auch $\sum_{n=1}^{\infty} \frac{\sqrt{n}}{n+1}$ divergent.
	\end{enumerate}		
\end{beispiele}




\begin{hilfssatz} \label{HS3}
Die Folge $(c_{n})$ sei beschränkt. Dann gilt:
	\begin{enumerate}
		\item Ist $\alpha \coloneqq \limsup_{n \to \infty} c_{n}$ und $x > \alpha$, so ist $c_{n} < x$ ffa $n \in \N$.
				
		\item Ist $\alpha \coloneqq \liminf_{n \to \infty}  c_{n}$ und $x < \alpha$, so ist $c_{n} > x$ ffa $n \in \N$.
		\item Ist $c_{n} \geq 0$ $(n \in \N)$ und $\limsup_{n \to \infty}  c_{n} = 0$, so gilt $c_{n} \rightarrow 0$ $(n \to \infty)$.
	\end{enumerate}
\end{hilfssatz}

\begin{proof} ~\
	\begin{enumerate}
		\item[c)] Es sei $\varepsilon > 0$. Mit a) (für $x=\varepsilon$) folgt: $-\varepsilon < 0 \leq c_{n} < \varepsilon$ ffa $n \in \N$. Also gilt 
		 $c_{n} \in U_{\varepsilon}(0)$ ffa $n \in \N$.
		\item[a)] Annahme: $c_{n} \geq x$ für unendlich viele $n$, etwa für $n_{1}, n_{2}, n_{3}, \dotsc$ mit $$n_{1} < n_{2} < n_{3} < \dotsc$$ \\
		 Die Teilfolge $(c_{n_{k}})$ ist beschränkt. Nach \ref{2.11:satz} enthält $(c_{n_{k}})$ eine konvergente Teilfolge $(c_{n_{k_{j}}})$. Definiere
		 $$ \beta \coloneqq \lim_{j\rightarrow \infty} c_{n_{k_{j}}}. $$
		 Es gilt $c_{n_{k_{j}}} \geq x$ $(j \in \N)$, also ist $\beta \geq x > \alpha$. Auch $(c_{n_{k_{j}}})$ ist eine Teilfolge von $(c_{n})$, also ist 
		 $\beta \in H(a_{n})$ und somit $\beta \leq \alpha$, Widerspruch.
		 \item[b)] Analog wie a).
	\end{enumerate}	
\end{proof}

\index{Konvergenzkriterium!Reihen!Wurzel}
\begin{satz}[Wurzelkriterium (WK)] \label{3.6:prop-Wurzelkriterium}
	Es sei $(a_{n})$ eine Folge, $c_{n} \coloneqq \sqrt[n]{|a_{n}|}$ $(n \in \N)$.
	\begin{enumerate}
		\item Ist $(c_{n})$ unbeschränkt, so ist $\sum_{n=1}^{\infty} a_{n}$ divergent.
		\item Es sei $(c_{n})$ beschränkt und $\alpha \coloneqq \limsup_{n \rightarrow \infty} c_{n}$. Dann gilt:
			\begin{enumerate}
				\item Ist $\alpha < 1$, so ist $\sum_{n=1}^{\infty} a_{n}$ absolut konvergent.
				\item Ist $\alpha > 1$, so ist $\sum_{n=1}^{\infty} a_{n}$ divergent.
			\end{enumerate}
			Im Falle $\alpha = 1$ ist keine allgemeine Aussage möglich.
	\end{enumerate}
\end{satz}

\begin{proof} ~\
	\begin{enumerate}
		\item $(c_{n})$ ist unbeschränkt $\Rightarrow$ $c_{n} \geq 1$ für unendlich viele $n\in \N$ $\Rightarrow$ $|a_{n}| \geq 1$ für unendlich viele 
		      $n\in \N$ $\Rightarrow$ $a_{n} \not\rightarrow 0$. Mit \ref{3.1.c:satz} folgt die Behauptung.
		\item  
			\begin{enumerate}
				\item Es sei $\alpha < 1$. Wähle ein $x \in (\alpha, 1)$. Nach \ref{HS3} gilt: $c_{n} \leq x$ ffa $n \in \N$, also 
				$|a_{n}| \leq x^{n}$ ffa $n \in \N$. Die Reihe $\sum_{n=1}^{\infty} x^{n}$ konvergiert. Nach \ref{3.5.a:satz} konvergiert
				$\sum_{n=1}^{\infty} a_{n}$ absolut. 
				\item Es sei $\alpha > 1$.  Wähle $\varepsilon > 0$ so, da{\ss} $\alpha - \varepsilon > 1$. Es gilt $c_{n} \in U_{\varepsilon}(\alpha)$ 
				für unendlich viele $n \in \N$. Damit ist $c_{n} > \alpha - \varepsilon > 1$ für unendlich viele $n$. Wie bei a) folgt:
				$\sum_{n=1}^{\infty} a_{n}$ divergiert. 
			\end{enumerate}
	\end{enumerate}
\end{proof}


\begin{beispiele} ~\
	\begin{enumerate}
		\item $a_{n} \coloneqq \frac{1}{n}$ $(n \in \N)$; $c_{n} = \sqrt[n]{|a_{n}|} = \frac{1}{\sqrt[n]{n}} \rightarrow 1$, also $\alpha = 1$ und $\sum_{n=1}^{\infty} a_{n}$ 
		      divergiert.
		\item $a_{n} \coloneqq \frac{1}{n^{2}}$ $(n \in \N)$; $c_{n} = \sqrt[n]{|a_{n}|} = \frac{1}{(\sqrt[n]{n})^{2}} \rightarrow 1$, also $\alpha = 1$ und 
		      $\sum_{n=1}^{\infty} a_{n}$ konvergiert.
		\item Es sei $x \in \R$ und $a_{n} \coloneqq \begin{cases} \frac{1}{2^{n}}, & \text{falls } n = 2k \\ n x^{n}, & \text{falls } n = 2k - 1 \end{cases}$ \\
			Frage: Für welche $x$ ist $\sum_{n=1}^{\infty} a_{n}$ (absolut) konvergent? Es ist
				$$ c_{n} = \sqrt[n]{|a_{n}|} = \begin{cases} \frac{1}{2}, & \text{falls } n = 2k \\ \sqrt[n]{n}|x|, & \text{falls } n = 2k - 1 \end{cases} $$
			$(c_{n})$ ist also beschränkt und $H(c_{n}) = \left\{ \frac{1}{2}, |x| \right\}$.  
			
			Fall 1: $|x| < 1$. Dann ist $\alpha = \limsup_{n \to \infty} c_{n} < 1$, also ist $\sum_{n=1}^{\infty} a_{n}$ absolut konvergent. \\
			Fall 2: $|x| > 1$. Dann ist $\alpha = \limsup_{n \to \infty}  c_{n} > 1$, also ist $\sum_{n=1}^{\infty} a_{n}$ divergent. \\
			Fall 3: $|x| = 1$. Dann ist $\alpha = \limsup_{n \to \infty}  c_{n} = 1$ und das Wurzelkriterium liefert keine Entscheidung. 
			Es ist $|a_{n}| = n$ falls $n = 2k - 1$. Also gilt $a_{n} \not\rightarrow 0$. Damit ist $\sum_{n=1}^{\infty} a_{n}$ also divergent.			
	\end{enumerate}	
\end{beispiele}

\index{Konvergenzkriterium!Reihen!Quotienten}
\begin{satz}[Quotientenkriterium (QK)] \label{3.7:prop-Quotientenkriterium}
	Es sei $a_{n} \neq 0$ ffa $n \in \N$ und $c_{n} \coloneqq \left| \frac{a_{n+1}}{a_{n}} \right|$.
	\begin{enumerate}
		\item Ist $c_{n} \geq 1$ ffa $n \in \N$, so ist $\sum_{n=1}^{\infty} a_{n}$ divergent.
		\item Es sei $(c_{n})$ beschränkt, $\alpha \coloneqq \limsup_{n \to\infty} c_{n}$ und $\beta \coloneqq \liminf_{n \to\infty} c_{n}$. Dann gilt:
			\begin{enumerate}
				\item Ist $\alpha < 1$, so ist $\sum_{n=1}^{\infty} a_{n}$ absolut konvergent.
				\item Ist $\beta > 1$, so ist $\sum_{n=1}^{\infty} a_{n}$ divergent.
			\end{enumerate}
	\end{enumerate}
\end{satz}

Ohne Beweis.

\begin{folgerung} \label{3.8:folg}
	$(a_{n})$ und $(c_{n})$ seien wie in \ref{3.7:prop-Quotientenkriterium}, $(c_{n})$ sei konvergent und $\alpha \coloneqq \lim_{n \to\infty} c_{n}$.
	$$ \sum_{n=1}^{\infty} a_{n} \text{ ist } \begin{cases} \text{absolut konvergent}, & \text{falls } \alpha < 1 \\ \text{divergent}, & \text{falls } \alpha > 1 \end{cases}. $$
	Im Falle $\alpha = 1$ ist keine allgemeine Aussage möglich.
\end{folgerung}


\begin{beispiele} ~\
	\begin{enumerate}
		\item $a_{n} = \frac{1}{n}$ $(n \in \N)$, $\left| \frac{a_{n+1}}{a_{n}} \right| = \frac{n}{n+1} \rightarrow 1$, $\sum_{n=1}^{\infty} a_{n}$ divergiert.
		\item $a_{n} = \frac{1}{n^{2}}$ $(n \in \N)$, $\left| \frac{a_{n+1}}{a_{n}} \right| = \frac{n^{2}}{(n+1)^{2}} \rightarrow 1$, $\sum_{n=1}^{\infty} a_{n}$ konvergiert.
	\end{enumerate}	
\end{beispiele}

\index{Exponentialreihe} \index{Exponentialfunktion}
\begin{unnamedtheorem}[Die Exponentialreihe] \label{3.9:prop-Exponentialreihe}
	Für $x \in \R$ betrachte die Reihe 
	$$ \sum_{n=0}^{\infty} \frac{x^{n}}{n!} = 1 + x + \frac{x^{2}}{2!} + \frac{x^{3}}{3!} + \dotsc $$
	Frage: Für welche $x \in \R$ konvergiert diese Reihe (absolut)? \\
	Klar: Die Reihe konvergiert absolut für $x = 0$. Sei nun $x \ne 0$ und $a_{n} \coloneqq \frac{x^{n}}{n!}$ $(n \in \N_0)$. Es gilt:
		$$ \left| \frac{a_{n+1}}{a_{n}} \right| = \left| \frac{x^{n+1}}{(n+1)!} \cdot \frac{n!}{x^{n}} \right| = \frac{|x|}{n+1} \rightarrow 0 \quad (n \rightarrow \infty). $$
	Mit \ref{3.8:folg} folgt:
		$$ \sum_{n=0}^{\infty} \frac{x^{n}}{n!} \text{ konvergiert absolut für jedes } x \in \R. $$
	Damit ist eine Funktion $E \colon \R \rightarrow \R$ definiert:
		$$ E(x) \coloneqq \sum_{n=0}^{\infty} \frac{x^{n}}{n!}. $$
	Sie hei{\ss}t \textbf{Exponentialfunktion}. Es gilt: $E(0) = 1$, $E(1) \overset{\S 2}{=} e$. \\
	Später zeigen wir $\forall r \in \Q:$ $E(r) = e^{r}$ und definieren dann $e^{x} \coloneqq E(x)$ für alle $x \in \R \setminus \Q$. 
	Dann ist also $e^{x} = E(x) ~ (x \in \R)$.
\end{unnamedtheorem}

\index{Umordnung}
\begin{definition}
	Sei $(a_{n})$ eine Folge und $\varphi \colon \N \rightarrow \N$ eine Bijektion. Setze $b_{n} \coloneqq a_{\varphi(n)}$ $(n \in \N)$. Also 
		$$ b_{1} = a_{\varphi(1)}, ~~ b_{2} = a_{\varphi(2)}, ~~ b_{3} = a_{\varphi(3)}, \dotsc $$
	Dann hei{\ss}t $(b_{n})$ eine \textbf{Umordnung} von $(a_{n})$.
\end{definition}

\begin{beispiel*}
$(a_{2}, a_{4}, a_{1}, a_{3}, a_{6}, a_{8}, a_{5}, a_{7}, \dotsc)$ ist eine Umordnung von $(a_{n})$.	
\end{beispiel*}


\begin{satz} \label{3.10:satz}
	Es sei $(b_{n})$ eine Umordnung von $(a_{n})$. Dann gilt:
	\begin{enumerate}
		\item Ist $(a_{n})$ konvergent, so ist $(b_{n})$ konvergent und $\lim_{n \to \infty} b_{n} = \lim_{n \to \infty} a_{n}$.
		\item Ist $\sum_{n=1}^{\infty} a_{n}$ absolut konvergent, so ist $\sum_{n=1}^{\infty} b_{n}$ absolut konvergent und 
			$$ \sum_{n=1}^{\infty} a_{n} = \sum_{n=1}^{\infty} b_{n}. $$
	\end{enumerate}	
\end{satz}

\begin{proof} ~\
	\begin{enumerate}
		\item Setze $a \coloneqq \lim_{n \to \infty} a_{n}$. Es sei $\varepsilon > 0$. Dann gilt: 
		$$
		\exists n_{0} \in \N ~\forall n \geq n_{0}: ~ |a_{n} - a| < \varepsilon.
		$$
		Da $\varphi$ injektiv ist, ist die Menge $\{n \in \N: \varphi(n)< n_0\}$ endlich.
		Also gilt: $$|b_n-a|=|a_{\varphi(n)} - a| < \varepsilon \text{ ffa } n \in \N.$$
		\item Ohne Beweis.
	\end{enumerate}	
\end{proof}


\begin{bemerkung}[ohne Beweis]
	Es sei $\sum_{n=1}^{\infty} a_{n}$ konvergent, aber nicht absolut konvergent. Dann gilt:
	\begin{enumerate}
		\item Ist $s \in \R$, so existiert eine Umordnung $(b_{n})$ von $(a_{n})$ mit: 
		$$
		\sum_{n=1}^{\infty} b_{n} \text{ ist konvergent und }  \sum_{n=1}^{\infty} b_{n} = s.
		$$
		\item Es existiert eine Umordnung $(c_{n})$ von $(a_{n})$ mit: $\sum_{n=1}^{\infty}c_{n}$ ist divergent.
	\end{enumerate}
\end{bemerkung}

\index{Cauchyprodukt}
\begin{definition}
	Gegeben seien die Reihen $\sum_{n=0}^{\infty} a_{n}$ und $\sum_{n=0}^{\infty} b_{n}$. \\
	Für $n \in \N_0$ sei
	\begin{align*}
		c_{n} & \coloneqq \sum_{k=0}^{n} a_{k} b_{n-k} = \sum_{k=0}^{n} a_{n-k} b_{k}, \text{ also} \\
		c_{n} & = a_{0} b_{n} + a_{1} b_{n-1} + \dotsc + a_{n} b_{0}.
	\end{align*} 
	Die Reihe $\sum_{n=0}^{\infty} c_{n}$ hei{\ss}t das \textbf{Cauchyprodukt} (CP) von $\sum_{n=0}^{\infty} a_{n}$ und $\sum_{n=0}^{\infty} b_{n}$.
\end{definition}


\begin{satz} \label{3.11:satz}
Es seien $\sum_{n=0}^{\infty} a_{n}$ und $\sum_{n=0}^{\infty} b_{n}$ absolut konvergent. Für ihr Cauchyprodukt $\sum_{n=0}^{\infty} c_{n}$ gilt dann:
	$$ \sum_{n=0}^{\infty} c_{n} \text{ ist absolut konvergent und } \sum_{n=0}^{\infty} c_{n} = (\sum_{n=0}^{\infty} a_{n}) (\sum_{n=0}^{\infty} b_{n}). $$
\end{satz}

Ohne Beweis.

\begin{beispiel*}
	Es sei $x \in \R$ und $|x| < 1$. \\
	Bekannt: $\sum_{n=0}^{\infty} x^{n}$ konvergiert absolut und	 $\sum_{n=0}^{\infty} x^{n} = \frac{1}{1-x}$. Also ist
	$$ \frac{1}{(1-x)^{2}} = (\sum_{n=0}^{\infty} x^{n})(\sum_{n=0}^{\infty} x^{n}) \overset{\ref{3.11:satz}}{=} \sum_{n=0}^{\infty} c_{n}, $$
	mit $c_{n} = \sum_{k=0}^{n} x^{k} x^{n-k} = (n+1)x^{n}$ $(n \in \N_0)$. Somit gilt:
	$$ \frac{1}{(1-x)^{2}} = \sum_{n=0}^{\infty} (n+1) x^{n} \quad (|x| < 1). $$
	z.B. $(x = \frac{1}{2}): 4 = \sum_{n=0}^{\infty} \frac{(n+1)}{2^{n}}$. \\
	Weiter gilt:
	$$ \frac{x}{(1-x)^{2}} = \sum_{n=0}^{\infty} (n+1) x^{n+1} = \sum_{n=1}^{\infty} n x^{n}. $$
	z.B. $(x = \frac{1}{2}): 2 = \sum_{n=1}^{\infty} \frac{n}{2^{n}}$, also $1 = \sum_{n=1}^{\infty} \frac{n}{2^{n+1}}$.
\end{beispiel*}

\bigskip

\begin{unnamedtheorem}[Eigenschaften der Exponentialfunktion] \label{3.12:prop-Exponentialfunktion}
	$E(x) = \sum_{n=0}^{\infty} \frac{x^{n}}{n!}$ $(x \in \R)$. Es gilt:
	\begin{enumerate}
		\item $E(0) = 1, E(1) = e$;
		\item $\forall x, y \in \R:$ $E(x + y) = E(x) E(y)$;
		\item $\forall x_{1}, \dotsc, x_{m} \in \R:$ $E(x_{1} + \dotsc + x_{m}) = E(x_{1}) \cdot \dotsc \cdot E(x_{m})$;
		\item $E(x) > 1$ $(x > 0)$; $E(x) > 0$ $(x \in \R)$; $E(-x) = E(x)^{-1}$ $(x \in \R)$;
		\item $\forall x \in \R ~ \forall r \in \Q:$ $E(rx) = E(x)^{r}$;
		\item $\forall r \in \Q:$ $E(r) = e^{r}$;
		\item $E$ ist auf $\R$ streng monoton wachsend, d.h. aus $x < y$ folgt stets $E(x) < E(y)$.
	\end{enumerate}	
\end{unnamedtheorem}

\begin{proof} ~\
	\begin{enumerate}
		\item Ist bekannt.
		\item Es gilt
		        $$ E(x)E(y) = (\sum_{n=0}^{\infty} \frac{x^{n}}{n!})(\sum_{n=0}^{\infty} \frac{y^{n}}{n!}) \overset{\ref{3.11:satz}}{=} \sum_{n=0}^{\infty} c_{n},$$
		        mit
			$$ c_{n} = \sum_{k=0}^{n} \frac{x^{k}}{k!} \cdot \frac{y^{n-k}}{(n-k)!} 
			= \frac{1}{n!} \sum_{k=0}^{n} \underbrace{\frac{n!}{k! (n-k)!}}_{=\binom{n}{k}} x^{k} y^{n-k} \overset{\S1}{=} \frac{1}{n!} (x+y)^{n} \quad (n \in \N_0).$$
			Also: $E(x)E(y) = \sum_{n=0}^{\infty} \frac{(x+y)^{n}}{n!} = E(x+y)$.
		\item Folgt aus b).
		\item Für $x > 0$ gilt $E(x) = 1 + \underbrace{x + \frac{x^{2}}{2!} + \frac{x^{3}}{3!} + \dotsc}_{> 0} > 1$. Weiter ist
			$$ 1 = E\left(x + (-x)\right) \overset{b)}{=} E(x) E(-x)  \quad (x \in \R). $$
			Insbesondere gilt: $E(x) > 0$ $(x < 0)$ und $E(-x) = E(x)^{-1}$ $(x \in \R)$.
		\item   Für $x \in \R$ und $n \in \N$ gilt:
			$$ E(nx) = E(x + \dotsc + x) \overset{c)}{=} E(x)^{n}. $$
			Also ist
			$$E(x) = E (n \frac{x}{n}) = (E(\frac{x}{n}))^{n} ~ \Rightarrow ~ E(\frac{1}{n} x) = E(x)^{\frac{1}{n}}.$$
			Für $m, n \in \N$ folgt damit:
			$$ E(\frac{m}{n} x) = E(m \frac{x}{n}) = E(\frac{x}{n})^{m} = (E(x)^{\frac{1}{n}})^{m} = E(x)^{\frac{m}{n}}. $$
			Somit gilt $E(rx) = E(x)^{r}$ für jedes $r \in \Q$ mit $r > 0$. Sei $r \in \Q$ und $r < 0$. Dann ist $-r > 0$, also
			$$ \frac{1}{E(rx)} = E(-rx)=E(x)^{-r} = \frac{1}{E(x)^{r}} ~ \Rightarrow ~ E(rx) = E(x)^{r}.$$
		\item Folgt aus e) mit $x = 1$.
		\item Es sei $x < y$. Dann gilt $y - x > 0$, also 
			$$ \Rightarrow 1 \overset{d)}{<} E(y - x) \overset{b)}{=} E(y)E(-x) \overset{d)}{=} \frac{E(y)}{E(x)} ~ \xRightarrow[]{d)} ~ E(x) < E(y). $$
	\end{enumerate}
\end{proof}


\newpage


\chapter{Potenzreihen}

\index{Potenzreihe}
\begin{definition}
	Es sei $(a_{n})_{n=0}^{\infty}$ eine Folge in $\R$ und $x_{0} \in \R$. Eine Reihe der Form
		$$ \sum_{n=0}^{\infty} a_{n} (x - x_{0})^{n} = a_{0} + a_{1} (x - x_{0}) + a_{2} (x - x_{0})^{2} + \dotsc $$
		hei{\ss}t \textbf{Potenzreihe} (PR). 
\end{definition}

Frage: Für welche $x \in \R$ konvergiert eine Potenzreihe (absolut)? \\
Klar: Eine Potenzreihe konvergiert absolut für $x = x_{0}$.

\begin{beispiele} ~\
	\begin{enumerate}
		\item $\sum_{n=0}^{\infty} \frac{x^{n}}{n!}$. Hier: $a_{n} = \frac{1}{n!}$ $(n \in \N_0)$, $x_{0} = 0$. \\
		Bekannt: Die Potenzreihe konvergiert absolut für jedes $x \in \R$.
		\item $\sum_{n=0}^{\infty} (x - x_{0})^{n}$. Hier: $a_{n} = 1$ $(n \in \N_0)$. \\
		Bekannt: Die Potenzreihe konvergiert absolut $\iff |x - x_{0} | < 1$ (geometrische Reihe).
			
		\item $\sum_{n=0}^{\infty} n^{n} (x - x_{0})^{n}$. Hier: $a_{n} = n^{n}$ $(n \in \N_0)$. \\
		Es sei $x \neq x_{0}$ und $b_{n} \coloneqq n^{n} (x - x_{0})^{n}$. Es gilt: $\sqrt[n]{|b_{n}|} = n |x - x_{0}|$. Wegen $x \neq x_{0}$ ist 
		$\left( \sqrt[n]{|b_{n}|} \right)$ unbeschränkt. Nach \ref{3.6:prop-Wurzelkriterium} ist $\sum_{n=0}^{\infty} n^{n} (x - x_{0})^{n}$ divergent. \\
		Also: $\sum_{n=0}^{\infty} n^{n} (x - x_{0})^{n}$ konvergiert nur für $x = x_{0}$.
	\end{enumerate}	
\end{beispiele}

\index{Konvergenzradius}
\begin{definition}
	Es sei $\sum_{n=0}^{\infty} a_{n} (x - x_{0})^{n}$ eine Potenzreihe. Setze
		$$ \rho \coloneqq \begin{cases}
			\infty, & \text{falls } \left( \sqrt[n]{|a_{n}|} \right) \text{ unbeschränkt} \\
			\limsup_{n \to \infty} \sqrt[n]{|a_{n}|}, & \text{falls } \left( \sqrt[n]{|a_{n}|} \right) \text{ beschränkt}
		\end{cases} $$
	und
		$$ r \coloneqq \begin{cases}
			0, & \text{falls } \rho = \infty \\
			\infty, & \text{falls } \rho = 0 \\
			\frac{1}{\rho}, & \text{falls } \rho \in (0, \infty)
		\end{cases} $$
	(kurz: "'$r = \frac{1}{\rho}$"'). $r$ hei{\ss}t der \textbf{Konvergenzradius} (KR) der Potenzreihe.
\end{definition}


\begin{satz} \label{4.1:satz}
	Es sei $\sum_{n=0}^{\infty} a_{n} (x - x_{0})^{n}$ eine Potenzreihe und $\rho$ und $r$ seien wie in obiger Definition. Dann gilt:
	\begin{enumerate}
		\item Ist $r = 0$, so konvergiert die Potenzreihe nur für $x = x_{0}$.
		\item Ist $r = \infty$, so konvergiert die Potenzreihe absolut für jedes $x \in \R$.
		\item Ist $r \in (0, \infty)$, so konvergiert die Potenzreihe absolut für jedes $x \in \R$ mit $|x - x_{0}| < r$ und sie divergiert für jedes $x \in \R$ mit 
		 $|x - x_{0}| > r$. Für $x = x_{0} \pm r$ ist keine allgemeine Aussage möglich.
	\end{enumerate}
\end{satz}

\begin{proof}
	Für $x \in \R$ sei $b_{n}(x) \coloneqq a_{n} (x - x_{0})^{n} ~(n \in \N_{0})$, also $\sqrt[n]{|b_{n}(x)|} = \sqrt[n]{|a_{n}|} |x - x_{0}|$ $(n \in \N)$.
	\begin{enumerate}
		\item Es sei $x \neq x_{0}$. Es gilt $r = 0$ also $\rho = \infty$. Somit ist $\left( \sqrt[n]{|b_{n}(x)|} \right)$ unbeschränkt. 
		      Nach \ref{3.6:prop-Wurzelkriterium} ist $\sum_{n=0}^{\infty} b_{n}(x)$ divergent.
		\item Es gilt $r = \infty$ also $\rho = 0$. Somit ist $\limsup_{n \to \infty} \sqrt[n]{|b_{n}(x)|} = 0$ $(x \in \R)$. Mit \ref{3.6:prop-Wurzelkriterium}
		      folgt die Behauptung.
		\item Es gilt:
		        $$\limsup_{n \to \infty} \sqrt[n]{|b_{n}(x)|} = \limsup_{n \to \infty} \sqrt[n]{|a_{n}|} |x - x_{0}| = \rho |x - x_{0}| = \frac{1}{r} |x - x_{0}|.$$
		        Also gilt:
			$$\limsup_{n \to \infty} \sqrt[n]{|b_{n}(x)|} < 1 \iff |x - x_0| < r,$$
			$$\limsup_{n \to \infty} \sqrt[n]{|b_{n}(x)|} > 1 \iff |x - x_0| > r.$$
			Die Behauptung folgt aus \ref{3.6:prop-Wurzelkriterium}.
	\end{enumerate}	
\end{proof}


\begin{folgerung*}
        Es gilt:
	$$\lim_{n \rightarrow \infty} \frac{1}{\sqrt[n]{n!}} = 0.$$	
\end{folgerung*}

\begin{proof}
	Bekannt: $\sum_{n=0}^{\infty} \frac{x^{n}}{n!}$ hat den Konvergenzradius $r = \infty$. Nach \ref{4.1:satz} ist $\rho = 0$, d.h. 
	$$ \limsup_{n \to \infty} \frac{1}{\sqrt[n]{n!}} = 0. $$
	Mit \ref{HS3} folgt die Behauptung.
\end{proof}


\begin{beispiele} ~\
	\begin{enumerate}
		\item $\sum_{n=0}^{\infty} x^{n}$; $a_{n} = 1 ~(n \in \N_{0})$, $x_{0} = 0$; $\rho = 1$, $r = 1$. \\
		Die Potenzreihe konvergiert für $|x| < 1$ absolut und sie divergiert für $|x| > 1$. Für $|x| = 1$ ist die Potenzreihe divergent.
		\item $\sum_{n=1}^{\infty} \frac{x^{n}}{n}$; $a_{0} = 0$, $a_{n} = \frac{1}{n}$ $(n \geq 1)$, $x_{0} = 0$. \\
		 Es gilt: $\sqrt[n]{|a_{n}|} = \frac{1}{\sqrt[n]{n}} \rightarrow 1$. Also ist $\rho = 1$ und damit $r = 1$. Die Potenzreihe konvergiert absolut für $|x| < 1$ und 
		 sie divergiert für $|x| > 1$. \\
		 Für $x = 1$: $\sum_{n=1}^{\infty} \frac{1}{n}$ divergiert. \\
		 Für $x = -1$: $\sum_{n=1}^{\infty} \frac{(-1)^{n}}{n}$ konvergiert (nicht absolut).
		\item $\sum_{n=1}^{\infty} \frac{x^{n}}{n^{2}}$; $a_{0} = 0$, $a_{n} = \frac{1}{n^{2}}$ $(n \geq 1)$, $x_{0} = 0$. \\
		 Es gilt: $\sqrt[n]{|a_{n}|} \rightarrow 1$. Also ist $\rho = 1$ und damit $r = 1$. Die Potenzreihe konvergiert absolut für $|x|< 1$ und sie divergiert für $|x| > 1$. \\
		 Für $x = 1$: $\sum_{n=1}^{\infty} \frac{1}{n^{2}}$ konvergiert absolut. \\
		 Für $x = -1$: $\sum_{n=1}^{\infty} \frac{(-1)^{n}}{n^{2}}$ konvergiert absolut.
	\end{enumerate}	
\end{beispiele}

\bigskip

In vielen Fällen lä{\ss}t sich auch über das Quotientenkriterium \ref{3.7:prop-Quotientenkriterium} der Konvergenzradius einer Potenzreihe bestimmen:
        
\begin{satz} \label{4.4:satz}
	Es sei $a_{n} \neq 0$ ffa $n \in \N_0$, die Folge $\left( \left| \frac{a_{n}}{a_{n+1}} \right| \right)$ sei konvergent und 
	$L \coloneqq \lim_{n\to \infty} \left| \frac{a_{n}}{a_{n+1}} \right|$. Dann hat die Potenzreihe $\sum_{n=0}^{\infty} a_{n} (x - x_{0})^{n}$ den Konvergenzradius $L$.
\end{satz}

Ohne Beweis.

\index{Cosinus}
\begin{unnamedtheorem}[Cosinus] \label{4.3:prop-Cosinus}
	Wir betrachten die Reihe
	$$ \sum_{n=0}^{\infty} (-1)^{n} \frac{x^{2n}}{(2n)!} = 1 - \frac{x^{2}}{2!} + \frac{x^{4}}{4!} - \frac{x^{6}}{6!} + \dotsc $$
	Hier: $x_{0} = 0$, $a_{2n + 1} = 0$, $a_{2n} = \frac{(-1)^{n}}{(2n)!}$ $(n \in \N_0)$. \\
	Wegen $0 \le \sqrt[n]{|a_{n}|} \leq \frac{1}{\sqrt[n]{n!}}$ $(n \in \N)$ und
	$\frac{1}{\sqrt[n]{n!}} \to 0$ $(n \to \infty)$ folgt
	$$ \sqrt[n]{|a_{n}|}  \rightarrow 0 \quad (n \to \infty). $$ 
	Nach \ref{4.1:satz} hat obige Potenzreihe den Konvergenzradius $r = \infty$, konvergiert also absolut für jedes $x \in \R$.
	$$ \textbf{Cosinus: } \begin{cases} \cos \colon \R \rightarrow \R \\ \cos x \coloneqq \sum_{n=0}^{\infty} (-1)^{n} \frac{x^{2n}}{(2n)!} \end{cases} $$
\end{unnamedtheorem}

\index{Sinus}
\begin{unnamedtheorem}[Sinus] \label{4.4:prop-Sinus}
	Analog wie bei \ref{4.3:prop-Cosinus} sieht man: Die Potenzreihe 
	$$ \sum_{n=0}^{\infty} (-1)^{n} \frac{x^{2n+1}}{(2n+1)!} = x - \frac{x^{3}}{3!} + \frac{x^{5}}{5!} - \frac{x^{7}}{7!} + \dotsc $$
	konvergiert absolut für jedes $x \in \R$.
	$$ \textbf{Sinus: } \begin{cases} \sin \colon \R \rightarrow \R \\ \sin x \coloneqq \sum_{n=0}^{\infty} (-1)^{n} \frac{x^{2n+1}}{(2n+1)!} \end{cases} $$	
\end{unnamedtheorem}

Offensichtlich gilt: $\sin 0 = 0, ~ \cos 0 = 1$, sowie
$$~\forall x \in \R: ~ \sin (-x) = - \sin(x), ~ \cos(-x) = \cos(x).$$

\index{Additionstheoreme}
Ähnlich wie in \ref{3.12:prop-Exponentialfunktion} zeigt man (mit dem Cauchyprodukt) die folgenden Additionstheoreme:
\begin{align*}
	\forall x,y \in \R: ~ \sin(x+y) & = \sin x \cos y + \cos x \sin y, \\
	\forall x,y \in \R: ~ \cos(x+y) & = \cos x \cos y - \sin x \sin y.
\end{align*}

Für $x \in \R$ erhalten wir
	$$ 1 = \cos(0) = \cos( x + (-x) )= \cos x \cos(-x) - \sin x \sin(-x) = \cos^{2} x + \sin^{2} x, $$
        $$\cos^{2} x \leq \cos^{2} x + \sin^{2} x = 1, ~ \sin^{2} x \leq \cos^{2} x + \sin^{2} x = 1,$$
        und damit $|\cos x | \leq 1$ und $|\sin x | \leq 1$.




\newpage


\chapter{q-adische Entwicklung}

\begin{definition}
	Es sei $x \in \R$. Dann existiert genau eine grö{\ss}te ganze Zahl $\le x$, also ein $k \in \Z$ mit $k \leq x < k + 1$; 
		$$ [x] \coloneqq k. $$
	
\end{definition}


\begin{vereinbarung}
	In diesem $\S$en sei stets $a \geq 0$ und $q \in \N\setminus \{1\}$.
\end{vereinbarung}


Setze $z_{0} \coloneqq [a]$, dann gilt: $z_{0} \leq a < z_{0} + 1$. \\
Setze $z_{1} \coloneqq [(a-z_{0})q]$, dann gilt: $z_{1} \leq aq - z_{0}q < z_{1} + 1$. \\
Also: 
       $$ z_{0} + \frac{z_{1}}{q} \leq a < z_{0} + \frac{z_{1}}{q} + \frac{1}{q}. $$
Es gilt $z_{1} \in \N_{0}$.  Annahme: $z_{1} \geq q$. Dann gilt  $\frac{z_{1}}{q} \geq 1$, also
	$$ z_{0} + 1 \leq z_{0} + \frac{z_{1}}{q} \leq a < z_{0} +1. $$
Widerspruch. Also ist $z_{1} \in \{ 0, 1, \dotsc, q - 1 \}$. \\
Setze  $z_{2} \coloneqq [(a-z_{0}-\frac{z_{1}}{q})q^{2}]$. Wie oben folgt
	$$ z_{0} + \frac{z_{1}}{q} + \frac{z_{2}}{q^{2}} \leq a < z_{0} + \frac{z_{1}}{q} + \frac{z_{2}}{q^{2}} + \frac{1}{q^{2}}. $$
und $z_{2} \in \{ 0, 1, \dotsc, q - 1 \}$. \\
Allgemein (induktiv): Sind $z_{0}, \dotsc, z_{n}$ schon definiert, so setze
	$$ z_{n+1} \coloneqq [(a - z_{0} - \frac{z_{1}}{q} - \dotsc - \frac{z_{n}}{q^{n}}) q^{n+1}]. $$
Wir erhalten so eine Folge $(z_{n})_{n=0}^{\infty}$ mit:
	$$ (*) \begin{cases} ~ z_{0} \in \N_{0},  ~ z_{n} \in \{ 0, 1, \dotsc, q - 1 \} ~ (n \geq 1) \\ 
	\text{ und} \\ ~\underbrace{z_{0} + \frac{z_{1}}{q} + \dotsc + \frac{z_{n}}{q^{n}}}_{\eqqcolon s_{n}} \leq a 
	< \underbrace{ z_{0} + \frac{z_{1}}{q} + \dotsc + \frac{z_{n}}{q^{n}} + \frac{1}{q^{n}}}_{= s_{n} + \frac{1}{q^{n}}} \end{cases} $$


In den gro{\ss}en Übungen wird gezeigt:

\begin{satz} \label{5.1:satz}
	Ist $(\tilde{z}_{n})_{n=0}^{\infty}$ eine weitere Folge mit den Eigenschaften in $(*)$, so gilt: 
	$$\forall n \in \N_{0}: ~  z_{n} = \tilde{z}_{n}.$$
\end{satz}

\bigskip

\index{q-adische Entwicklung}
Es gilt:
	$$\forall n \in \N: ~   0 \leq \frac{z_{n}}{q^{n}} \leq \frac{q - 1}{q^n} \quad \text{und} \quad \sum_{n=1}^{\infty} \frac{q - 1}{q^{n}} \text{ konvergiert}. $$
Nach \ref{3.5.a:satz} ist $\sum_{n=0}^{\infty}\frac{z_{n}}{q^{n}}$ konvergent. Also ist $(s_{n})$ konvergent und mit $(\ast)$ folgt
	$$ a = \lim_{n \to \infty} s_{n} = \sum_{n=0}^{\infty} \frac{z_{n}}{q^{n}}. $$
	
\begin{definition}
Ist $(y_n)_{n=0}^\infty$ eine Folge mit $y_0 \in \N_0$ und $y_n \in \{0,1 \dots, q-1\}$, so schreibt man 
$$
y_0,y_1 y_2 y_3 y_4 \dotsc := \sum_{n=0}^{\infty} \frac{y_{n}}{q^{n}}.
$$
\end{definition}

\begin{bemerkungen} \
\begin{enumerate}
\item Die Darstellung einer reellen Zahl als ein solcher Reihenwert ist nicht eindeutig. Z.B. ist $(q=10)$:
$$
1,0000000 \dotsc = 1 = 0,99999999 \dotsc
$$
\item Gilt mit einem $m \in \N:$ $y_{n} = 0$ $(n > m)$, so schreibt man auch: 
$$
y_{0}, y_{1} \dotsc y_{m}.
$$
\item Obige Konstruktion der Folge $(z_n)$ zeigt, da{\ss} jede reelle Zahl $a \ge 0$ als ein solcher Reihenwert geschrieben werden kann:
$$
a = z_{0}, z_{1} z_{2} z_{3} z_4 \dotsc
$$
Die so erhaltene Darstellung von $a$ hei{\ss}t \text{\bf die q-adische Entwicklung von} $a$. Sie ist nach \ref{5.1:satz} durch $(\ast)$ eindeutig
bestimmt.
\item Sprechweisen: $q = 10$: Dezimalentwicklung; $q = 2$: Dualentwicklung.
\end{enumerate}
\end{bemerkungen}

\begin{beispiele} ~\
	\begin{enumerate}
		\item $q = 10$, $a = 1$. Dann gilt: 
		$$z_{0} = 1, ~ z_{1} = [(a - z_{0})q] = 0, ~ z_{2} = [(a - z_{0} - \frac{z_{1}}{q})q^{2}] = 0, \dotsc.$$
		Induktiv folgt: $z_{n} = 0$ $(n \geq 1)$, also $1 = 1,000\dotsc$.
		\item $q = 10$, $a = \frac{1}{2}$. Dann gilt:
		$$z_{0} = 0, ~ z_{1} = [(a - z_{0})q] = [\frac{10}{2}] = 5, ~ z_{2} = [(a - z_{0} - \frac{z_{1}}{q})q^{2}] = [(\frac{1}{2} - \frac{5}{10}) 100] = 0, \dotsc. $$
		Induktiv folgt: $z_{n} = 0$ $(n \geq 2)$, also $\frac{1}{2} = 0,5000\dotsc = 0,5$.
	\end{enumerate}
\end{beispiele}


\begin{definition}
	Es sei $b \in \R$ und $b < 0$. Weiter sei $z_{0}, z_{1} z_{2} z_3 \dotsc $
	die q-adische Entwicklung von $-b$. Dann hei{\ss}t $- z_{0}, z_{1} z_{2} z_3\dotsc$ die q-adische Entwicklung von $b$.
\end{definition}


\begin{satz} \label{5.2:satz} ~\
\begin{enumerate}
\item Es sei $z_{0}, z_{1} z_{2} z_{3} \dotsc$ die q-adische Entwicklung von $a$. Dann ist $z_{n} = q-1$ ffa $n \in \N$ nicht möglich.
\item Ist $(y_n)_{n=0}^\infty$ eine Folge mit $y_0 \in \N_0$, $y_n \in \{0,1 \dots, q-1\}$, $a = y_{0}, y_{1} y_{2} y_{3} \dotsc$ und 
$y_{n} = q-1$ nicht ffa $n \in \N$, so ist $y_{0}, y_{1} y_{2} y_{3} \dotsc$ die q-adische Entwicklung von $a$.
\end{enumerate}	
\end{satz}

\begin{proof}
	a) Annahme: $\exists m \in \N$ $\forall n \geq m$: $z_{n} = q - 1$. Dann gilt:
		$$ a = \sum_{n=0}^{\infty} \frac{z_{n}}{q^{n}} = \underbrace{\sum_{n=0}^{m-1} \frac{z_{n}}{q^{n}}}_{= s_{m-1}} + \sum_{n=m}^{\infty} \frac{q-1}{q^{n}} $$
	und 
	\begin{align*}
		\sum_{n=m}^{\infty} \frac{q-1}{q^{m}} & = (q-1) \left( \frac{1}{q^{m}} + \frac{1}{q^{m+1}} + \dotsc \right) \\
			& = \frac{q - 1}{q^{m}} \left(1 + \frac{1}{q} + \frac{1}{q^{2}} + \dotsc\right) \\
			& = \frac{q - 1}{q^{m}} \frac{1}{1 - \frac{1}{q}} = \frac{1}{q^{m-1}}.
	\end{align*} 
	Also ist $a = s_{m-1} + \frac{1}{q^{m-1}} \overset{(*)}{>} a$. Widerspruch. \\
	b) Übung (mit \ref{5.1:satz}).
\end{proof}


\begin{satz} \label{5.3:satz}
	$\R$ ist überabzählbar.
\end{satz}

\begin{proof}
	Es genügt zu zeigen, da{\ss} $[0, 1)$ überabzählbar ist. Annahme: $[0, 1)$ ist abzählbar, also $[0, 1) = \{ a_{1}, a_{2}, a_3, \dotsc \}$. Für $j \in \N$ sei
		$$ a_{j} = 0, z_{1}^{(j)} z_{2}^{(j)} z_{3}^{(j)} \dotsc $$
	die 3-adische Entwicklung von $a_{j}$, also $z_{n}^{(j)} \in \{ 0, 1, 2 \}$. Setze
		$$ z_{n} \coloneqq \begin{cases} 1, & \text{falls } z_{n}^{(n)} = 0 \text{ oder } z_{n}^{(n)} = 2 \\ 0, & \text{falls } z_{n}^{(n)} = 1 \end{cases} $$
	Dann gilt $z_{n} \neq z_{n}^{(n)}$ $(n \in \N)$ $(**)$. Setze $a \coloneqq \sum_{n=1}^{\infty} \frac{z_{n}}{3^{n}}$. Es gilt:
		$$ 0 \leq a \leq \sum_{n=1}^{\infty} \frac{1}{3^{n}} = \frac{1}{2}, \text{ also } a \in [0, 1). $$
	Nach  \ref{5.2:satz} b) ist $0, z_{1} z_{2} z_{3} \dotsc$ ist die 3-adische Entwicklung von $a$. Wegen $a \in [0, 1)$ existiert ein $m \in \N$ mit $a = a_{m}$, also
		$$ 0, z_{1} z_{2} z_{3} \dotsc = 0, z_{1}^{(m)} z_{2}^{(m)} z_{3}^{(m)}\dotsc. $$
	Es folgt $z_{j} = z_{j}^{(m)}$ $(j \in \N)$, also für $j = m$: $z_{m} = z_{m}^{(m)}$. Widerspruch zu $(**)$.
\end{proof}


\newpage


\chapter{Grenzwerte bei Funktionen}

\index{Häufungspunkt}
\begin{definition}
	Es sei $D \subseteq \R$ und $x_{0} \in \R$. $x_{0}$ hei{\ss}t ein \textbf{Häufungspunkt} (HP) von $D$ $:\iff$ Es gibt eine Folge $(x_{n})$ in $D \setminus \{ x_{0} \}$ 
	mit $x_{n} \rightarrow x_{0}$.
\end{definition}


\begin{beispiele} \
	\begin{enumerate}
		\item $D = (0, 1]$. Es gilt:  
			$x_{0}$ ist Häufungspunkt von $D \iff x_{0} \in [0, 1]$.
		\item $D = \{ \frac{1}{n} : n \in \N \}$. Es gilt:  
		      $D$ hat genau einen Häufungspunkt: $x_{0} = 0$.
		\item $D=\Q$. Es gilt: 
		      Jedes $x \in \R$ ist Häufungspunkt von $D$. 
		\item Ist $D$ endlich, so hat $D$ keine Häufungspunkte.
		
	\end{enumerate}	
\end{beispiele}


\begin{hilfssatz} \label{6.1:hsatz}
	Sei $D \subseteq \R$ und $x_{0} \in \R$. Dann gilt: 
	$$
	x_{0} \text{ ist Häufungspunkt von } D \iff \forall \varepsilon > 0: U_{\varepsilon}(x_{0}) \cap (D \setminus \{ x_{0} \}) \neq \emptyset.
	$$
\end{hilfssatz}

\begin{proof} ~\\
	"'$\Rightarrow$"': Es gibt eine Folge $(x_{n})$ in $D \setminus \{ x_{0} \}$ mit $x_{n} \rightarrow x_{0}$. Es sei $\varepsilon > 0$. Dann gilt:
	$$ \exists n_{0} \in \N ~ \forall n \geq n_{0}: ~ x_{n} \in U_{\varepsilon}(x_{0}) \cap (D \setminus \{ x_{0} \}). $$
	"'$\Leftarrow$"': Nach Voraussetzung gilt:\\
	$\exists x_{1} \in U_{1}(x_{0}) \cap (D \setminus \{ x_{0} \})$, also $|x_{1} - x_{0}| < 1$; \\
	$\exists x_{2} \in U_{\frac{1}{2}}(x_{0}) \cap (D \setminus \{ x_{0} \})$, also $|x_{2} - x_{0}| < \frac{1}{2}$; etc. \\
	Wir erhalten eine Folge $(x_{n})$ in $D \setminus \{ x_{0} \}$ mit 
	$$ |x_{n} - x_{0}| < \frac{1}{n} \quad (n \in \N), $$
	also $x_{n} \rightarrow x_{0}$.
\end{proof}


\begin{vereinbarung}
	Ab jetzt sei stets in diesem $\S$en
	$\emptyset \neq D \subseteq \R$, $x_{0}$ ein Häufungspunkt von $D$ und $f : D \rightarrow \R$ eine Funktion.
\end{vereinbarung}


\begin{bezeichnung} ~\
	\begin{enumerate}
		\item $D_{\delta}(x_{0}) \coloneqq U_{\delta}(x_{0}) \cap (D \setminus \{ x_{0} \})$.
		\item Sei $M \subseteq D$ und $g \colon D \rightarrow \R$ eine weitere Funktion. Wir schreiben ``$f \leq g$ auf $M$'' für
		``$f(x) \leq g(x)$ $(x \in M)$''.
	\end{enumerate}
\end{bezeichnung}


\begin{definition}
	$\lim_{x \rightarrow x_{0}} f(x)$ existiert $:\iff$ Es gibt ein $a \in \R$ so, da{\ss} für jede Folge $(x_{n})$ in $D \setminus \{ x_{0} \}$ mit $x_{n} \rightarrow x_{0}$ gilt: 
	$f(x_{n}) \rightarrow a$. \\
	In diesem Fall ist $a$ eindeutig bestimmt und wir schreiben:
		$$ \lim_{x \rightarrow x_{0}} f(x) = a \text{ oder } f(x) \rightarrow a ~(x \rightarrow x_{0}). $$
\end{definition}


\begin{bemerkung}
	Sollte $x_{0} \in D$ sein, so ist der Wert $f(x_{0})$ in obiger Definition nicht relevant. Relevant ist allein das Verhalten von $f$ in das "'Nähe"' von $x_{0}$.	
\end{bemerkung}

\index{Grenzwert!linksseitiger} \index{Grenzwert!rechtsseitiger}
\begin{beispiele} ~\
	\begin{enumerate}
		\item $D \coloneqq [0, \infty)$, $p \in \N$, $f(x) \coloneqq \sqrt[p]{x}$. \\
		        Es sei $x_{0} \in D$ (dann ist $x_{0}$ eine Häufungspunkt von $D$). Es sei $(x_{n})$ eine Folge in $D$ mit 
		        $x_{n} \rightarrow x_{0}$. Nach \ref{2.4:bsp} gilt dann: $\sqrt[p]{x_{n}} \rightarrow \sqrt[p]{x_{0}}$ $(n \to \infty)$. Also gilt:
			$$ \lim_{x \rightarrow x_{0}} \sqrt[p]{x} = \sqrt[p]{x_{0}}. $$
		\item $D = (0, 1]$,
			$$ f(x) \coloneqq \begin{cases} x^{2}, & 0 < x < \frac{1}{2} \\ \frac{1}{2}, & x = \frac{1}{2} \\ 1, & \frac{1}{2} < x < 1 \\ 0, & x = 1 \end{cases} $$
			Klar: $\lim_{x \rightarrow 0} f(x)=0$, $\lim_{x \rightarrow 1} f(x) = 1$. Weiter sei
			$$ x_{n} \coloneqq \frac{1}{2} - \frac{1}{n},  ~  y_{n} \coloneqq \frac{1}{2} + \frac{1}{n} \quad (n \geq 3). $$
			Es gilt $x_{n} \rightarrow \frac{1}{2}$, $y_{n} \rightarrow \frac{1}{2}$, aber 
			$f(x_n) = \left( \frac{1}{2} - \frac{1}{n} \right)^{2} \rightarrow \frac{1}{4} \neq 1 \leftarrow f(y_{n})$. \\
			D.h. $\lim_{x \rightarrow \frac{1}{2}} f(x)$ existiert nicht. Schränkt man aber $f$ auf $D\cap (-\infty,\frac{1}{2})=(0,\frac{1}{2})$ ein, so gilt 
			$$\lim_{\underset{x \in (0, \frac{1}{2})}{x \rightarrow \frac{1}{2}}} f(x) = \frac{1}{4}.$$
			Dafür schreibt man 
			$$ \lim_{x \rightarrow \frac{1}{2}-} f(x) = \frac{1}{4} \text{ (linksseitiger Grenzwert)}. $$ \\
			Analog: Schränkt man $f$ auf $D\cap (\frac{1}{2}, \infty)=(\frac{1}{2},1]$ ein, so ist
			$$ \lim_{x \rightarrow \frac{1}{2}+} f(x) := \lim_{\underset{x \in (\frac{1}{2}, 1]}{x \rightarrow \frac{1}{2}}} f(x) = 1 
			\text{ (rechtsseitiger Grenzwert)}. $$ 
		\item $D = \R$, $f = E$, also $f(x) = \sum_{n=0}^{\infty} \frac{x^{n}}{n!}$. \\
		        Für $|x| \leq 1$ gilt:
			\begin{align*}
				|E(x) - E(0)| & = |E(x) - 1| = |x + \frac{x^{2}}{2!} + \frac{x^{3}}{3!} + \dotsc | \\
				& = |x| \left| 1 + \frac{x}{2!} + \frac{x^{2}}{3!} + \dotsc \right| \\
				& \leq |x| \left( 1 + \frac{|x|}{2!} + \frac{|x|^{2}}{3!} + \dotsc \right) \\
				& \leq |x| \left( 1 + \frac{1}{2!} + \frac{1}{3!} + \dotsc \right) \\
				& = |x| ( e - 1 ).
			\end{align*}
			Es sei $(x_{n})$ Folge in $\R$ mit $x_{n} \rightarrow 0$. Dann gilt:
			$$
			\exists n_{0} \in \N~ \forall n \geq n_{0}: |x_{n}| \leq 1,
			$$
			und somit
			$$
			\forall n \geq n_{0}: ~ |E(x_{n}) - 1| \leq |x_{n}|(e-1).
			$$
			Damit folgt $E(x_{n}) \rightarrow 1$. Somit ist $\lim_{x \rightarrow 0} E(x) = 1 = E(0)$. Es gilt also:		
			$$
			\lim_{x \rightarrow 0} \sum_{n=0}^{\infty} \frac{x^{n}}{n!} = \sum_{n=0}^{\infty} \left( \lim_{x \rightarrow 0} \frac{x^{n}}{n!} \right).
			$$
	\end{enumerate}	
\end{beispiele}

\index{Konvergenzkriterium!Cauchy}
\begin{satz} \label{6.2:satz} Es gilt:
        \begin{enumerate}
		\item 
		$$
		\lim_{x \rightarrow x_{0}} f(x) = a \iff \forall \varepsilon > 0 ~\exists \delta > 0 ~ \forall x \in D_{\delta}(x_{0}): ~|f(x)- a| < \varepsilon. 
		$$
		\label{6.2.a:satz}
		\item $\lim_{x \rightarrow x_{0}} f(x)$ existiert
			$$ \iff \text{ Für jede Folge } (x_{n}) \text{ in } D \setminus \{ x_{0} \} \text{ mit } x_{n} \rightarrow x_{0} \text{ ist } (f(x_{n})) \text{ konvergent}. $$ \label{6.2.b:satz} 
		\item \textbf{Cauchykriterium}: 
			$$ \lim_{x \rightarrow x_{0}} f(x) \text{ existiert } \iff \forall \varepsilon > 0 ~\exists \delta  > 0 ~\forall x_{1}, x_{2} \in D_{\delta}(x_{0}): ~ |f(x_{1}) - f(x_{2})| < \varepsilon. $$ \label{6.2.c:satz}
	\end{enumerate}
\end{satz}

\begin{proof} ~\\
b) und c) ohne Beweis. \\
a) ``$\Rightarrow$'': Es sei $\varepsilon > 0$. Annahme: Für kein $\delta > 0$ gilt $|f(x)- a| < \varepsilon$ ($x \in D_{\delta}(x_{0})$).
Dann existiert zu jedem $n \in \N$ ein $x_n \in D_{1/n}(x_{0})$ mit $|f(x_n)- a| \ge \varepsilon$. Damit ist $(x_n)$ eine Folge in $D \setminus \{x_0\}$ mit
$x_n \to x_0$ und $f(x_n) \not\to a$ Widerspruch. \\
``$\Leftarrow$'': Es sei $(x_n)$ eine Folge in $D \setminus \{x_0\}$ mit $x_n \to x_0$. Es sei $\varepsilon > 0$. Wähle $\delta > 0$ so, da{\ss} 
$|f(x)- a| < \varepsilon$ ($x \in D_{\delta}(x_{0})$). Es gibt ein $n_0 \in \N$ mit $|x_n-x_0| < \delta$ $(n \ge n_0)$. Für $n \ge n_0$ gilt damit
$|f(x_n)- a| < \varepsilon$. Also gilt: $f(x_n) \to a$.
\end{proof}


\begin{satz} \label{6.3:satz}
	Es seien $f, g, h \colon D \rightarrow \R$ Funktionen. 
	Weiter seien $a, b \in \R$ und es gelte $f(x) \rightarrow a$, $g(x) \rightarrow b$ $(x \rightarrow x_{0})$. Dann gilt:
	\begin{enumerate}
		\item $$\alpha f(x) + \beta g(x) \rightarrow \alpha a + \beta b; ~~ f(x) g(x) \rightarrow a b; ~~ |f(x)| \rightarrow |a| \quad (x \rightarrow x_{0}). $$ 
		\label{6.3.a:satz}
		\item Ist $a \neq 0$, so existiert ein $\delta > 0$ mit $f(x) \neq 0$ $(x \in D_{\delta}(x_{0}))$. Für 
		$\frac{1}{f} \colon D_{\delta}(x_{0}) \rightarrow \R$ gilt: 
		$$\frac{1}{f(x)} \rightarrow \frac{1}{a} \quad (x \rightarrow x_{0}).$$ \label{6.3.b:satz}
		\item Für ein $\delta > 0$ gelte $f \leq g$ auf $D_{\delta}(x_{0})$. Dann ist $a \leq b$. \label{6.3.c:satz}
		\item Für ein $\delta > 0$ gelte $f \leq h \leq g$ auf $D_{\delta}(x_{0})$. Ist $a = b$, so gilt $h(x) \rightarrow a$ $(x \rightarrow x_{0})$. \label{6.3.d:satz}
	\end{enumerate}
\end{satz}

\begin{proof}
	z. B.: c) Es sei $(x_{n})$ eine Folge in $D \setminus \{ x_{0} \}$ mit $x_{n} \rightarrow x_{0}$. Dann gilt: 
		$$\exists n_{0} \in \N ~\forall n \geq n_{0}: ~ x_{n} \in D_{\delta}(x_{0}). $$ 
	Also ist $f(x_{n}) \leq g(x_{n})$ $(n \geq n_{0})$. Nach \ref{2.2:satz} folgt 
	$$a = \lim_{n \to \infty} f(x_{n}) \leq \lim_{n \to \infty} g(x_{n}) = b.$$	\\
	Die anderen Aussagen beweist man analog durch Zurückführen auf \ref{2.2:satz}.
\end{proof}


\begin{definition} ~\
	\begin{enumerate}
		\item Es sei $(x_{n})$ eine Folge in $\R$.
			\begin{align*}
				x_{n} \rightarrow \infty & :\iff \forall c > 0 ~ \exists n_{0}  \in \N ~ \forall n \geq n_{0}: ~ x_{n} > c, \\
				x_{n} \rightarrow - \infty & :\iff \forall c < 0 ~ \exists n_{0} \in \N ~ \forall n \geq n_{0}: ~ x_{n} < c.
			\end{align*}
			Übung:  Es gilt: \\
			$x_{n} \rightarrow \infty \iff x_{n} > 0 \text{ ffa } n \in \N \text{ und } \frac{1}{x_{n}} \rightarrow 0$, \\
			$x_{n} \rightarrow -\infty \iff x_{n} < 0 \text{ ffa } n \in \N \text{ und } \frac{1}{x_{n}} \rightarrow 0$.
		\item Es sei $D \subseteq \R$, $x_{0}$ sei ein Häufungspunkt von $D$ und $g \colon D \rightarrow \R$ eine Funktion.
			\begin{align*}
				\lim_{x \rightarrow x_{0}} g(x) = \infty :\iff & \text{Für jede Folge } (x_{n}) \text{ in } D \setminus \{ x_{0} \} \text{ mit } 
				x_{n} \rightarrow x_{0} \text{ gilt: } g(x_{n}) \rightarrow \infty, \\
				\lim_{x \rightarrow x_{0}} g(x) = -\infty :\iff & \text{Für jede Folge } (x_{n}) \text{ in } D \setminus \{ x_{0} \} \text{ mit } 
				x_{n} \rightarrow x_{0} \text{ gilt: } g(x_{n}) \rightarrow -\infty.
			\end{align*}
		\item Es sei $D$ nicht nach oben beschränkt, $g \colon D \rightarrow \R$ sei eine Funktion und es sei $a \in \R \cup \{ \infty, - \infty \}$.
			$$ \lim_{x \rightarrow \infty} g(x) = a :\iff \text{ Für jede Folge } (x_{n}) \text{ in } D \text{ mit } x_{n} \rightarrow \infty \text{ gilt: } 
			g(x_{n}) \rightarrow a. $$
		\item Es sei $D$ sei nicht nach unten beschränkt, $g \colon D \rightarrow \R$ sei eine Funktion und es sei $a \in \R \cup \{ \infty, - \infty \}$.
			$$ \lim_{x \rightarrow -\infty} g(x) = a :\iff \text{ Für jede Folge } (x_{n}) \text{ in } D \text{ mit } x_{n} \rightarrow -\infty \text{ gilt: } 
			g(x_{n}) \rightarrow a. $$		
	\end{enumerate}
\end{definition}


\begin{beispiel}
	$\frac{1}{x} \rightarrow \infty ~(x \rightarrow 0+)$, $\frac{1}{x} \rightarrow -\infty ~(x \rightarrow 0-)$, $\frac{1}{x} \rightarrow 0$ $(x \rightarrow \pm \infty)$.
\end{beispiel}


\begin{unnamedtheorem}[Exponentialfunktion] \label{6.4:prop-Exponentialfunktion}
	$E(x) = \sum_{n=0}^{\infty} \frac{x^{n}}{n!} = 1 + x + \frac{x^{2}}{2!} + \frac{x^{3}}{3!} + \dotsc$ \\
	Es sei $p \in \N_{0}$. Für jedes $x \geq 0$ gilt
		$$ E(x) = 1 + x + \frac{x^{2}}{2!} + \dotsc + \frac{x^{p+1}}{(p+1)!} + \dotsc \geq \frac{x^{p+1}}{(p+1)!}, $$
		also $$\frac{E(x)}{x^{p}} \geq \frac{x}{(p+1)!} \quad (x>0). $$
	Somit folgt:
		$$ \frac{E(x)}{x^{p}} \rightarrow \infty \quad (x \rightarrow \infty). $$
	Insbesondere gilt ($p = 0$): 
	$$
	E(x) \rightarrow \infty \quad (x \rightarrow \infty),
	$$
	also
	$$ E(-x) = \frac{1}{E(x)} \rightarrow 0 \quad (x \rightarrow \infty),$$
	und damit
	$$E(x) \rightarrow  0 \quad (x \rightarrow -\infty).$$
\end{unnamedtheorem}


\newpage


\chapter{Stetigkeit}

\index{stetig}
\begin{definition}
	Es sei $D \subseteq \R$, $f \colon D \rightarrow \R$ eine Funktion und $x_{0} \in D$. 
	\begin{enumerate}
		\item $f$ hei{\ss}t $\textbf{in }  x_{0} \textbf{ stetig}$ $:\iff$ Für jede Folge $(x_{n})$ in $D$ mit $x_{n} \rightarrow x_{0}$ gilt: 
		$f(x_{n}) \rightarrow f(x_{0})$.
		\item $f$ hei{\ss}t \textbf{auf $D$ stetig} $:\iff f$ ist in jedem $x \in D$ stetig.
		\item Wir setzen
		$$
		C(D):=C(D,\R):= \{g:D \to \R: ~ g \text{ ist stetig auf } D \}.
		$$
	\end{enumerate}
\end{definition}


\begin{beispiele} \
	\begin{enumerate}
		\item $D = [0, \infty)$, $p \in \N$, $f(x) = \sqrt[p]{x}$. \\
		Bekannt: Ist $(x_{n})$ eine Folge in $D$ mit $x_{n} \rightarrow x_{0} \in D$, so gilt $f(x_{n}) \rightarrow f(x_{0})$. Also gilt $f \in C \left([0, \infty)\right)$.
		\item $D = [0, 1] \cup \{ 2 \}$, $ f(x) = \begin{cases}
				x^{2}, & 0 \leq x < 1 \\ 0, & x = 1 \\ 1, & x = 2
			\end{cases}$ \\
			Offensichtlich gilt: $f$ ist stetig in jedem $x \in [0, 1)$. 
			\begin{enumerate}
				\item Es sei $x_{0} = 1$, $x_{n} =1 - \frac{1}{n}$ $(n \in \N)$. Dann ist $(x_{n})$ eine Folge in $D$ mit $x_{n} \rightarrow 1$, aber 
				$f(x_{n}) = x_{n}^{2} \rightarrow 1 \neq 0 = f(1)$. Also ist $f$ in $x_{0} = 1$ nicht stetig.
				\item Es sei $x_{0} =2$, und $(x_{n})$ eine Folge in $D$ mit $x_{n} \rightarrow 2$. Dann ist $x_{n} = 2$ ffa $n \in \N$, 
				also $f(x_{n}) = 1$ ffa $n \in \N$. Somit gilt $f(x_{n}) \rightarrow 1 = f(2)$. Also ist $f$ in $x_{0} = 2$ stetig.
			\end{enumerate}	
	\end{enumerate}	
\end{beispiele}


\begin{satz} \label{7.1:satz}
	Es sei $D \subseteq \R$, $f \colon D \rightarrow \R$ eine Funktion und $x_{0} \in D$. Dann gilt:
	\begin{enumerate}
		\item 	$$ f \text{ ist in } x_{0} \text{ stetig }  \iff  \forall \varepsilon > 0 ~ \exists \delta > 0 ~ 
			\forall x \in U_{\delta}(x_{0}) \cap D: ~ |f(x) - f(x_{0})| < \varepsilon.$$
		\item Ist $x_{0}$ Häufungspunkt von $D$, so gilt:
			$$ f \text{ ist in } x_{0} \text{ stetig } \iff \lim_{x \rightarrow x_{0}} f(x) = f(x_{0}). $$
	\end{enumerate}
\end{satz}

\begin{proof} \
	\begin{enumerate}
		\item Fast wörtlich wie bei \ref{6.2:satz}.
		\item Übung.
	\end{enumerate}
\end{proof}


\begin{satz} ~\ \label{7.2:satz} 
	\begin{enumerate}
		\item Es seien $f, g \colon D \rightarrow \R$ stetig in $x_{0} \in D$ und es seien $\alpha, \beta \in \R$. Dann sind 
			$$ \alpha f + \beta g, ~ fg \text{ und }  ~ |f| ~ \text{ stetig in } x_0. $$
			Ist $x_{0} \in \tilde{D} \coloneqq \{ x \in D : f(x) \neq 0 \}$, so ist $\frac{1}{f} \colon \tilde{D} \rightarrow \R$ stetig in $x_{0}$.
		\item Sind $f, g \in C(D)$ und $\alpha, \beta \in \R$, so gilt:
			$$ \alpha f + \beta g, ~ fg, ~ |f| \in C(D).$$
	\end{enumerate}
\end{satz}



\begin{proof} a) Folgt aus \ref{2.2:satz}; b) folgt aus a). \end{proof}


\begin{bemerkung}
Satz \ref {7.2:satz} b) zeigt insbesondere: $C(D)$ ist ein reeller Vektorraum.
\end{bemerkung}


\begin{satz} \label{7.3:satz}
	Es seien $D, D_{0} \subseteq \R$, $f \colon D \rightarrow \R$, $g \colon D_{0} \rightarrow \R$ Funktionen, $f(D) \subseteq D_{0}$, $x_{0} \in D$ und 
	                $y_{0} \coloneqq f(x_{0})$. Ist $f$ in $x_{0}$ stetig und ist $g$ in $y_{0}$ stetig, so ist
			$$ g \circ f \colon D \rightarrow \R, ~ (g \circ f)(x) \coloneqq g(f(x)) $$
		        stetig in $x_{0}$.
\end{satz}

\begin{proof}
	Es sei $(x_{n})$ eine Folge in $D$ mit $x_{n} \rightarrow x_{0}$. \\
	Da $f$ stetig in $x_{0}$ ist gilt: $f(x_{n}) \rightarrow f(x_{0}) = y_{0}$. Da $g$ stetig in $y_{0}$ ist folgt 
	$$
	(g \circ f)(x_{n}) = g(f(x_{n}))  \rightarrow g(y_{0}) = g(f(x_{0})) = (g \circ f)(x_{0}).
	$$
\end{proof}


\begin{satz} \label{7.4:satz}
	Es sei $\sum_{n=0}^{\infty} a_{n} (x - x_{0})^{n}$ eine Potenzreihe mit Konvergenzradius $r > 0$. Es sei $D \coloneqq (x_{0} - r, x_{0} + r)$ falls $r < \infty$ und 
	$D \coloneqq \R$ falls $r = \infty$. Weiter sei
		$$ f(x) \coloneqq \sum_{n=0}^{\infty} a_{n} (x - x_{0})^{n} \quad (x \in D). $$
	Dann gilt: $f \in C(D)$.	
\end{satz}


Wir beweisen \ref{7.4:satz} später, nach \ref{8.3:satz}.


\begin{beispiele}
	Nach \ref{7.4:satz} sind die Exponentialfunktion, und die Funktionen Sinus und Cosinus auf $\R$ stetig.	
\end{beispiele}


\begin{beispiel} \label{7.5:bsp}
	Behauptung: $$\lim_{x \rightarrow 0} \frac{\sin x}{x} = 1.$$
\end{beispiel}

\begin{proof}
	Für $x \neq 0$ gilt: 
	$$ \frac{\sin x}{x} = \frac{1}{x} \left( x - \frac{x^{3}}{3!} + \frac{x^{5}}{5!} - \frac{x^{7}}{7!} + \dotsc \right) 
	= \underbrace{1 - \frac{x^{2}}{3!} + \frac{x^{4}}{5!} - \frac{x^{6}}{7!} + \dotsc}_{\text{PR mit KR } r = \infty} \xrightarrow[]{\ref{7.4:satz}} 1 ~(x \rightarrow 0). $$
\end{proof}


\begin{beispiel} \label{7.6:bsp}
	Behauptung: $$\lim_{x \rightarrow 0} \frac{E(x) - 1}{x} = 1.$$
\end{beispiel}

\begin{proof}
	Für $x \neq 0$ gilt:
	$$ \frac{E(x) - 1}{x} = \frac{1}{x} \left( ( 1+ x + \frac{x^{2}}{2!} + \dotsc) - 1 \right) = \underbrace{1 + \frac{x}{2!} + \frac{x^{2}}{3!} + \dotsc}_{\text{PR mit KR } r = \infty} \xrightarrow[]{\ref{7.4:satz}} 1 ~(x \rightarrow 0). $$
\end{proof}


\begin{folgerung*}
	Für jedes $x_{0} \in \R$ gilt:
	$$\lim_{h \rightarrow 0} \frac{E(x_{0} + h) - E(x_{0})}{h} = E(x_{0}).$$
\end{folgerung*}

\begin{proof}
        Es gilt: 
	$$\frac{E(x_{0} + h) - E(x_{0})}{h} =  \frac{E(x_{0}) E(h) - E(x_{0})}{h} =  E(x_{0}) \frac{E(h) - 1}{h} \xrightarrow[]{\ref{7.6:bsp}} E(x_{0}) ~(h \rightarrow 0).$$
\end{proof}

\index{Zwischenwertsatz}
\begin{satz}[Zwischenwertsatz]  \label{7.7:prop-Zwischenwertsatz} 
	Es seien $a, b \in \R$, $a < b$, $f \in C\left([a,b]\right)$ und 
	$$
	y_0  \in [\min\{f(a),f(b)\}, \max\{f(a),f(b)\}],
	$$
	also $y_{0}$ zwischen $f(a)$ und $f(b)$. Dann existiert ein $x_{0} \in [a, b]$ mit $f(x_{0}) = y_{0}$.
\end{satz}

\begin{proof}
	Fall 1: Ist $f(a) = y_{0}$ oder $f(b) = y_{0}$ so gilt die Behauptung. \\
	Fall 2: Es sei $f(a) \neq y_{0} \neq f(b)$. O.B.d.A. sei $f(a) < f(b)$, also $f(a) < y_{0} < f(b)$. Wir setzen
	$$ M \coloneqq \{ x \in [a, b]: f(x) \leq y_{0} \}. $$
	Es gilt $a \in M$, also $M \neq \emptyset$. Wegen $M \subseteq [a, b]$ ist $M$ beschränkt. Damit existiert $x_{0} \coloneqq \sup M$ und es gilt $x_0 \in [a, b]$. 
	Ist $n \in \N$, so ist $x_{0} - \frac{1}{n}$ keine obere Schranke von $M$, also existiert ein $x_{n} \in M$ mit
		$$ x_{n} > x_{0} - \frac{1}{n}. $$
	Also: $\forall n \in \N$: $x_{0} - \frac{1}{n} < x_{n} \leq x_{0}$. Somit gilt $x_{n} \rightarrow x_{0}$. Da $f$ stetig in $x_{0}$ ist folgt $f(x_{n}) \rightarrow f(x_{0})$.
	Nach Definition von $M$ ist $f(x_{n}) \leq y_{0}$ $(n \in \N)$, also $f(x_{0}) \leq y_{0}$. \\
	Weiter gilt $x_{0} < b$ (andernfalls: $x_{0} = b \Rightarrow f(b) = f(x_{0}) \leq y_{0} < f(b)$, Widerspruch). \\
	Es sei $z_{n} \coloneqq x_{0} + \frac{1}{n}$. Es gilt $z_{n} \in [a, b]$ ffa $n \in \N$, und für diese $n$ gilt:
	$$
	z_{n} > x_{0} ~ \Rightarrow ~ z_{n} \notin M ~ \Rightarrow ~ f(z_{n}) > y_{0}.
	$$
	Wegen $z_{n} \rightarrow x_{0}$ folgt ($f$ ist stetig): $f(z_{n}) \rightarrow f(x_{0})$. Damit ist $f(x_{0}) \geq y_{0}$.
\end{proof}


\begin{folgerung*}[vgl. \ref{1.6:satz}]
	Ist $a > 0$ und $n \in \N$, so existiert ein $x_{0} > 0$ mit $x_{0}^{n} = a$.	
\end{folgerung*}

\begin{proof}
	Es sei $b \coloneqq 1 + a$ und $f(x) \coloneqq x^{n}$ $(x \in [0, b])$. \\
	Dann gilt: 
	$$
	f \in C([0, b]), ~ f(0) = 0 < a, ~  f(b) = (1 + a)^{n} \geq 1 + n a > a.
	$$
	Mit \ref{7.7:prop-Zwischenwertsatz} folgt: $\exists x_{0} \in [0, b]: f(x_{0}) = a$, also $x_{0}^{n} = a$. Wegen $a > 0$ ist $x_0 > 0$.
\end{proof}

\begin{bemerkung}
	Erst jetzt ist \ref{1.6:satz} vollständig bewiesen!
\end{bemerkung}


Aus \ref{7.7:prop-Zwischenwertsatz} folgt mit $y_{0} = 0$:

\index{Nullstellensatz}
\begin{satz}[Nullstellensatz von Bolzano] \label{7.8:prop-NullstellensatzVonBolzano}
	Ist $f \in C\left([a, b]\right)$ und $f(a)f(b) \leq 0$, so existiert ein $x_{0} \in [a, b]$ mit $f(x_{0}) = 0$.	
\end{satz}


\begin{unnamedtheorem}[Exponentialfunktion]
	$E(x) = \sum_{n=0}^{\infty} \frac{x^{n}}{n!}$ $(x \in \R)$. \\
	Behauptung: $E(\R) = (0, \infty)$.	
\end{unnamedtheorem}

\begin{proof}
	Nach \ref{3.12:prop-Exponentialfunktion} gilt $E(x) > 0$ $(x \in \R)$, also $E(\R) \subseteq (0, \infty)$. \\
	Es sei $y_{0} \in (0, \infty)$. Nach \ref{6.4:prop-Exponentialfunktion} gilt:
	$$ E(x) \rightarrow \infty ~ (x \rightarrow \infty) ~ \Rightarrow ~ \exists b > 0: ~ E(b) > y_{0}, $$	
	$$ E(x) \rightarrow 0 ~ (x \rightarrow -\infty) ~ \Rightarrow  ~ \exists a < 0: ~ E(a) < y_{0}. $$
	Mit \ref{7.7:prop-Zwischenwertsatz} folgt: $\exists x_{0} \in [a, b]:$ $E(x_{0}) = y_{0}$, also $y_{0} \in E(\R)$.
	Somit ist $(0, \infty) \subseteq E(\R)$.
\end{proof}

\index{abgeschlossen} \index{kompakt}
\begin{definition}
	Es sei $D \subseteq \R$.
	\begin{enumerate}
		\item $D$ hei{\ss}t \textbf{abgeschlossen} $:\iff$ Für jede konvergente Folge $(x_{n})$ in $D$ gilt $$\lim_{n \to \infty} x_{n} \in D.$$
		\item $D$ hei{\ss}t \textbf{kompakt} $:\iff$ Jede Folge $(x_{n})$ in $D$ enthält eine konvergente Teilfolge $(x_{n_{k}})$ mit 
		$$\lim_{k \rightarrow \infty} x_{n_{k}} \in D.$$
	\end{enumerate}
\end{definition}


\begin{satz} \label{7.10:satz}
	Es sei $D \subseteq \R$. Dann gilt:
	\begin{enumerate}
		\item $D$ ist abgeschlossen $\iff$ Jeder Häufungspunkt von $D$ gehört zu $D$.
		\item $D$ ist kompakt $\iff$ $D$ ist beschränkt und abgeschlossen.
		\item Ist $D$ kompakt und $D \neq \emptyset$, so existieren $\max D$ und $\min D$.
	\end{enumerate}
\end{satz}

\begin{beispiele} ~\
	\begin{enumerate}
		\item $[a, b]$ ist kompakt, also auch abgeschlossen.	
		\item Endliche Mengen sind kompakt.
		\item $[a, \infty)$, $(-\infty, a]$ und $\R$ sind abgeschlossen, aber nicht kompakt.
		\item $\emptyset$ ist kompakt.
		\item $(a, b]$, $[a, b)$, $(a, b)$ sind nicht abgeschlossen.
	\end{enumerate}
\end{beispiele}


\begin{proof}(von 7.10):
	\begin{enumerate}
		\item Übung.
		\item "'$\Leftarrow$"' Folgt direkt aus \ref{2.12:satz-BolzanoWeierstrass}, "'$\Rightarrow$"' Übung.
		\item Es sei $s \coloneqq \sup D$. Dann gilt: 
			$$ \forall n \in \N ~\exists x_{n} \in D: ~ s - \frac{1}{n} < x_{n} \leq s. $$
			 Somit gilt $x_{n} \rightarrow s$. Da $D$ abgeschlossen ist folgt $s \in D$. Also ist $s = \max D$. \\
			 Analog zeigt man: $\inf D \in D$.
	\end{enumerate}	
\end{proof}

\index{beschränkt}
\begin{definition}
	$f \colon D \rightarrow \R$ hei{\ss}t beschränkt $:\iff f(D)$ ist \textbf{beschränkt}. Äquivalent ist 
	$$ \exists c \geq 0 ~\forall x \in D: ~ |f(x)| \leq c. $$
\end{definition}


\begin{satz} \label{7.11:satz}
	Es sei $\emptyset \not= D \subseteq \R$ kompakt und $f \in C(D)$. Dann ist $f(D)$ kompakt. Insbesondere ist $f$ beschränkt und es existieren $x_{1}, x_{2} \in D$ mit 
	$f(x_1)=\min f(D)$ und $f(x_2)=\max f(D)$, d.h.
		$$\forall x \in D: ~  f(x_{1}) \leq f(x) \leq f(x_{2}). $$
\end{satz}

\begin{proof}
	Es sei $(y_{n})$ eine Folge in $f(D)$. Dann existiert eine Folge $(x_{n})$ in $D$ mit $f(x_{n}) = y_{n}$ $(n \in \N)$. Da $D$ kompakt ist enthält $(x_n)$ eine konvergente 
	Teilfolge $(x_{n_{k}})$ mit $x_{0} \coloneqq \lim_{k \rightarrow \infty} x_{n_{k}} \in D$. Da $f$ stetig ist folgt
	$$ y_{n_{k}} = f(x_{n_{k}}) \rightarrow f(x_{0}) \in f(D). $$
\end{proof}


\begin{satz} ~\ \label{7.12:satz}
	\begin{enumerate}
		\item Ist $I \subseteq \R$ ein Intervall und ist $f \in C(I)$, so ist $f(I)$ ein Intervall.
		\item Sei $f \in C([a, b])$, $A \coloneqq \min f([a, b])$ und $B \coloneqq \max f([a, b])$, so ist $f([a, b]) = [A, B]$.
	\end{enumerate}
\end{satz}

\begin{proof}
	a) Übung: Eine Teilmenge $M\subseteq \R$ ist genau dann ein Intervall, wenn gilt:
	$$
	x,y \in M, ~ x<z<y ~ \Rightarrow z \in M.
	$$
	Damit folgt die Behauptung aus \ref{7.7:prop-Zwischenwertsatz}.\\
	b) folgt aus a).
\end{proof}

 \index{monoton! wachsend}   \index{monoton! streng wachsend} \index{monoton! streng fallend} \index{monoton! fallend}
\begin{definition} ~\
	\begin{enumerate}
		\item $f \colon D \rightarrow \R$ hei{\ss}t \textbf{monoton wachsend} $:\iff$ Aus $x_{1}, x_{2} \in D$ und $x_{1} < x_{2}$ folgt stets $f(x_{1}) \leq f(x_{2})$. \\
			$f \colon D \rightarrow \R$ hei{\ss}t \textbf{streng monoton wachsend} $:\iff$ Aus $x_{1}, x_{2} \in D$ und $x_{1} < x_{2}$ folgt stets $f(x_{1}) < f(x_{2})$.
		\item Entsprechend definiert man \textbf{(streng) monoton fallend}.
		\item $f$ hei{\ss}t \textbf{(streng) monoton} $:\iff f$ ist (streng) monoton wachsend oder (streng) monoton fallend.
	\end{enumerate}
\end{definition}


Es sei $I \subseteq \R$ ein Intervall und $f \colon I \to \R$ sei streng monoton wachsend (bzw. streng monoton fallend). Dann ist $f$ auf $I$ injektiv, es existiert also die Umkehrfunktion 
$f^{-1} \colon f(I) \to I$ und $f^{-1}$ ist streng monoton wachsend (bzw. streng monoton fallend).

Es gilt: 
$$
\forall x \in I: ~ f^{-1}(f(x)) = x, \quad  \forall y \in f(I): ~ f(f^{-1}(y)) =y.
$$

\begin{bemerkung}
	 $f(I)$ ist im allgemeinen kein Intervall. 
\end{bemerkung}



\begin{satz} \label{7.13:satz}
	Es sei $I \subseteq \R$ ein Intervall, $f \in C(I)$ und $f$ sei auf $I$ streng monoton. Dann ist $f(I)$ ein Intervall (vgl. \ref{7.12:satz}) und
	$$ f^{-1} \in C \left( f(I) \right). $$	
\end{satz}

Ohne Beweis.

\index{Logarithmus}
\begin{unnamedtheorem}[Der Logarithmus] \label{7.14:prop-Logarithmus}
	Bekannt: $E$ ist auf $\R$ streng monoton wachsend und $E(\R) = (0, \infty)$. Es existiert also $E^{-1} \colon (0, \infty) \rightarrow \R$. Die Funktion
		$$ \log x \coloneqq \ln x \coloneqq E^{-1}(x) \quad (x \in (0, \infty)) $$
	hei{\ss}t \textbf{Logarithmus}.
\end{unnamedtheorem}


\begin{eigenschaften} Es gilt:
	\begin{enumerate}
		\item $\log 1 = 0$, $\log e = 1$;
		\item $\log \colon (0, \infty) \rightarrow \R$ ist stetig und streng monoton wachsend;
		\item $\log \left( (0, \infty) \right) = \R$;
		\item $\log x \rightarrow \infty ~(x \rightarrow \infty)$, $\log x \rightarrow -\infty ~(x \rightarrow 0)$;
		\item $\forall x, y > 0: ~ \log(xy) = \log x + \log y$;
		\item $\forall x, y > 0: ~ \log\left(\frac{x}{y}\right) = \log x - \log y$.
	\end{enumerate}	
\end{eigenschaften}

\begin{proof} ~\
	\begin{enumerate}
		\item Folgt aus $E(0)=1$ und $E(1)=e$.
		\item Folgt aus \ref{7.13:satz}.
		\item Folgt aus $E(\R) = (0, \infty)$.
		\item Folgt aus $E(x) \rightarrow \infty ~(x \rightarrow \infty)$ bzw. $E(x) \rightarrow 0 ~(x \rightarrow -\infty)$.
		\item Es sei $z \coloneqq \log x + \log y$. Dann gilt: 
		$$E(z) =E(\log x + \log y) = E(\log x) E(\log y) = x y,$$ also 
		$$\log(xy) = \log E(z) = z.$$ 
		\item Übung. (Ähnlich wie e)).
	\end{enumerate}
\end{proof}


\textbf{Erinnerung}: Nach {\ref{3.12:prop-Exponentialfunktion} gilt: $\forall x \in \R ~\forall r \in \Q: ~ E(rx) = E(x)^{r}$. \\
	Es sei $a > 0$. Mit $x \coloneqq \log a$ erhalten wir:
		$$ \forall r \in \Q: ~ E(r \log a) = E(\log a)^{r} = a^{r}. $$

\begin{unnamedtheorem}[Die allgemeine Potenz] \label{7.15:prop-AllgPotenz}
	Es sei $a > 0$. Wir definieren:
	$$ a^{x} \coloneqq E(x \log a) \quad (x \in \R\setminus \Q). $$
	Ist speziell $a = e$, so ist $e^{x} = E(x \log e) = E(x)$ $(x \in \R)$. Somit gilt
	$$ a^{x} = e^{x \log a} ~(x \in \R, a > 0). $$ 
\end{unnamedtheorem}


\begin{eigenschaften}
	Es sei $a > 0$ und $x, y \in \R$. Dann gilt:
	\begin{enumerate}
	        \item $a^{x} > 0$;
		\item Die Funktion $x \mapsto a^{x}$ ist auf $\R$ stetig;
		\item $a^{x+y} = e^{(x+y)\log a} = e^{x\log a + y \log a} = e^{x \log a} e^{y \log a} = a^{x} a^{y}$;
		\item $a^{-x} = e^{-x\log a} = \frac{1}{e^{x} \log a} = \frac{1}{a^{x}}$;
		\item $\log(a^{x}) = \log ( e^{x \log a}) = x \log a$;
		\item $(a^{x})^{y} = e^{y \log a^{x}} \overset{e)}{=} e^{xy \log a} = a^{xy}$;
		\item Ist auch $x > 0$, so ist $a^{x^{y}} \coloneqq a^{\left(x^{y}\right)}$. Im allgemeinen ist $a^{x^{y}} \not= (a^{x})^{y}$.
	\end{enumerate}	
\end{eigenschaften}




\index{stetig!gleichmä{\ss}ig}
\begin{definition}
	$f \colon D \rightarrow \R$ hei{\ss}t auf $D$ \textbf{gleichmä{\ss}ig stetig} $:\iff$ Sind $(x_n)$, $(y_n)$ Folgen in $D$ mit $x_n-y_n \to 0$,
	so gilt $f(x_n)-f(y_n) \to 0$.
\end{definition}	
	
\textbf{Erinnerung an} \ref{7.1:satz}: Es sei $f \in C(D)$, $x_{0} \in D$ und $\varepsilon > 0$. Dann existiert ein $\delta = \delta(\varepsilon, x_{0}) > 0$ mit:
	$$ \forall x \in D: ~ |x - x_{0}| < \delta ~ \Rightarrow ~ |f(x) - f(x_{0})| < \varepsilon. $$
	Die Zahl $\delta$ hängt also im Allgemeinen von $\varepsilon$ und $x_{0}$ ab!	
	
\begin{bemerkung} \
Ähnlich wie in \ref{7.1:satz} kann man eine $\varepsilon$-$\delta$-Bedingung für gleichmä{\ss}ige Stetigkeit beweisen. Es gilt (ohne Beweis): \\
$f \colon D \rightarrow \R$ ist gleichmä{\ss}ig stetig $\iff$ 
$$
\forall \varepsilon > 0 ~ \exists \delta=\delta(\varepsilon) > 0 ~ \forall x, y \in D: |x - y| < \delta ~ \Rightarrow ~ |f(x) - f(y)| < \varepsilon.
$$
\end{bemerkung}


Offensichtlich gilt: Ist $f$ gleichmä{\ss}ig stetig auf $D$, so ist $f$ stetig auf $D$.


\begin{satz}[Satz von Heine]  \label{7.16:satz} ~\\
	Ist $D \subseteq \R$ kompakt und ist $f \in C(D)$, so ist $f$ auf $D$ gleichmä{\ss}ig stetig.
\end{satz}

\begin{proof}
Annahme: $f$ ist nicht gleichmäßig stetig. Dann existieren ein $\varepsilon > 0$ und Folgen $(x_n)$, $(y_n)$ in $D$ mit $x_n-y_n \to 0$, aber $|f(x_n)-f(y_n)| \ge \varepsilon$ $(n \in \N)$.
Da $D$ kompakt ist enthält $(x_n)$ eine konvergente Teilfolge $(x_{n_k})$ mit $x_0:=\lim_{k \to \infty} x_{n_k} \in D$. Nun gilt 
$$
y_{n_k}= y_{n_k} - x_{n_k} +  x_{n_k} \to 0 + x_0 = x_0 \quad (k \to \infty),
$$
also $f(x_{n_k})-f(y_{n_k}) \to f(x_0)-f(x_0)=0$ $(k \to \infty)$. Ein Widerspruch.
\end{proof}


\begin{definition}
	$f \colon D \rightarrow \R$ hei{\ss}t auf $D$ \textbf{Lipschitz-stetig} $:\iff$ 
	$$ \exists L \geq 0 ~ \forall x, y \in D: ~  | f(x) - f(y) | \leq L |x - y|.  $$
\end{definition}

Übung: Ist $f$ Lipschitz-stetig auf $D$, so ist $f$ gleichmäßig stetig auf $D$.


\begin{beispiele} \
        \begin{enumerate}
        \item $f:[0, 1] \to \R$, $f(x) = x^{2}$ ist Lipschitz-stetig (also gleichmäßig stetig): 
	\begin{align*}
		|f(x) - f(y) | & = |x^{2} - y^{2}| = |(x + y) (x - y)| = |x+y| |x-y| \\
			& \leq (|x| + |y|) |x - y| \leq 2 |x - y| \quad (x, y \in [0, 1]).
	\end{align*}  
	\item $g \colon [0, \infty) \rightarrow \R$, $g(x) = x^{2}$ ist nicht gleichmäßig stetig, insbesondere nicht Lipschitz-stetig: \\
	Betrachte $(x_n)=(n+\frac{1}{n})$, $(y_n)=(n)$. Es gilt $x_n-y_n=\frac{1}{n} \to 0$, aber
	$$
	g(x_n)-g(y_n)=2+ \frac{1}{n^2} \not\to 0 \quad (n \to \infty).
	$$
	\end{enumerate}
\end{beispiele}




\newpage

\chapter{Funktionenfolgen und -reihen}
In diesem $\S$en sei stets $\emptyset \neq D \subseteq \R$, $(f_{n})$ eine Folge von Funktionen $f_{n} \colon D \rightarrow \R$ und $s_{n} \coloneqq f_{1} + f_{n} + \cdots + f_{n}$ $(n \in \N)$.

\index{konvergent!punktweise} \index{Grenzfunktion} \index{Summenfunktion}
\begin{definition} ~\
 	\begin{enumerate}
		\item Die Funktionenfolge $(f_{n})$ hei{\ss}t \textbf{auf} $D$ \textbf{punktweise konvergent} $:\iff$ Für jedes $x \in D$ ist die Folge $(f_{n}(x))$ konvergent. \\
			In diesem Fall sei
			$$f(x) \coloneqq \lim_{n \to \infty} f_{n}(x) \quad (x \in D).$$ 
			Die Funktion $f \colon D \rightarrow \R$ hei{\ss}t die \textbf{Grenzfunktion} von $(f_{n})$.
		\item Die Funktionenreihe $\sum_{n=1}^{\infty} f_{n}$ hei{\ss}t \textbf{auf} $D$ \textbf{punktweise konvergent} $:\iff$ Für jedes $x \in D$ ist die Folge $(s_{n}(x))$ konvergent. \\
			In diesem Fall sei 
			$$
			f(x) \coloneqq \sum_{n=1}^{\infty} f_{n}(x) \quad (x \in D).
			$$
			Die Funktion $f \colon D \rightarrow \R$ hei{\ss}t die \textbf{Summenfunktion} von $(f_{n})$.
	\end{enumerate}
\end{definition}


\begin{beispiele} ~\
	\begin{enumerate}
		\item $D = [0,1]$, $f_{n}(x) = x^{n}$ $(n \in \N)$. Es gilt:
			$$ f(x) \coloneqq \lim_{n\rightarrow\infty} f_{n}(x) = \begin{cases} 0, & 0 \leq x < 1 \\  1, & x = 1 \end{cases} $$
			$(f_{n})$ konvergiert auf $[0, 1]$ punktweise gegen $f$.
		\item Es sei $\sum_{n=0}^{\infty} a_{n} (x-x_{0})^{n}$ eine Potenzreihe mit dem Konvergenzradius $r > 0$ und $D \coloneqq (x_{0} - r, x_{0} + r)$ 
		$(D \coloneqq \R$, falls $r = \infty$). Es sei $f_n:D \to \R$ definiert durch $f_{n}(x) = a_{n} (x - x_{0})^{n}$ $(n \in \N_0)$. Nach \ref{4.1:satz} gilt:
		$\sum_{n=0}^{\infty} f_{n}$ konvergiert auf $D$ punktweise gegen $$f(x) \coloneqq \sum_{n=0}^{\infty} a_{n} (x-x_{0})^{n}.$$
		\item $D = [0, \infty)$, $f_{n}(x) \coloneqq \frac{nx}{1+n^{2}x^{2}}$ $(n \in \N)$. Für jedes $x \in [0,\infty)$ gilt
		$$
		f_n(x) = \frac{\frac{x}{n}}{\frac{1}{n^2} + x^{2}} \rightarrow 0 \quad (n \rightarrow \infty).
		$$
		Also konvergiert $(f_{n})$ auf $D$ punktweise gegen $f:D \to \R$, $f(x)=0$. 
	\end{enumerate}
\end{beispiele}

\begin{bemerkung}
Punktweise Konvergenz von $(f_{n})$ auf $D$ gegen $f$ bedeutet: 
			$$\forall x\in D ~ \forall \varepsilon > 0 ~ \exists n_{0} = n_{0}(\varepsilon, x) \in \N ~  \forall n \geq n_{0}: ~  |f_{n}(x) - f(x)| < \varepsilon.$$
\end{bemerkung}

\begin{definition} ~\
	\begin{enumerate}
		\item $(f_{n})$ konvergiert \text{auf $D$ gleichmä{\ss}ig} (glm) gegen $f \colon D \rightarrow \R$ $:\iff$
		      $$\forall \varepsilon > 0 ~ \exists n_{0} = n_{0}(\varepsilon) \in \N ~ \forall n \geq n_{0} ~ \forall x\in D: ~  |f_{n}(x) - f(x)| < \varepsilon.$$
		\item $\sum_{n=1}^{\infty} f_{n}$  konvergiert \text{auf $D$ gleichmä{\ss}ig} (glm) gegen $f \colon D \rightarrow \R$ $:\iff$ 
		      $$\forall \varepsilon > 0 ~ \exists n_{0} = n_{0}(\varepsilon) \in \N ~ \forall n \geq n_{0} ~ \forall x\in D: ~  |s_{n}(x) - f(x)| < \varepsilon.$$
	\end{enumerate}
\end{definition}

\bigskip
 
Offensichtlich folgt aus gleichmä{\ss}iger Konvergenz stets punktweise Konvergenz. Die Umkehrung ist im allgemeinen falsch (siehe Beispiele unten).

\bigskip

$(f_{n})$ konvergiert auf $D$ gleichmä{\ss}ig gegen $f$ bedeutet anschaulich: Zu jedem $\varepsilon > 0$ existiert ein $n_{0} = n_{0}(\varepsilon) \in \N$ mit: 
$$ \text{Für } n \geq n_{0}  \text{ liegt der Graph von } f_{n} \text{ im "'}\varepsilon\text{-Schlauch"' um den Graphen von } f.$$

\begin{beispiele} ~\
	\begin{enumerate}
		\item Es sei $D = [0, 1], f_{n}(x) = x^{n}$ $(n \in \N)$. Bekannt: $(f_{n})$ konvergiert punktweise gegen 
			$$ f(x) = \begin{cases} 0, & \text{falls } x \in [0, 1) \\ 1, & \text{falls } x = 1 \end{cases} $$
			Es sei $0 < \varepsilon < \frac{1}{2}$. Wegen $f_{n}(\frac{1}{\sqrt[n]{2}}) = \frac{1}{2}$ und $\frac{1}{\sqrt[n]{2}} \in [0,1)$ gilt
				$$ \left|f_{n}(\frac{1}{\sqrt[n]{2}}) - f(\frac{1}{\sqrt[n]{2}}) \right| = \frac{1}{2} > \varepsilon \quad (n \in \N).$$
			Also konvergiert $(f_{n})$ auf $[0, 1]$ nicht gleichmä{\ss}ig gegen $f$.
		\item Wir betrachten $\sum_{n=0}^{\infty} x^{n}$ auf $D = (-1, 1)$. Es gilt: 
			$$ \forall x \in D: ~ s_{n}(x) = 1 + x + \dotsc + x^{n} = \frac{1 - x^{n+1}}{1 - x} \rightarrow \frac{1}{1 - x} \quad (n \to \infty).  $$
			Die Funktionenreihe $\sum_{n=0}^{\infty} x^{n}$ konvergiert also punktweise auf $D$ gegen die Summenfunktion $f(x):=\frac{1}{1 - x}$. \\ 
			Behauptung: $\sum_{n=0}^{\infty} x^{n}$ konvergiert auf $D$ nicht gleichmä{\ss}ig gegen $f$. \\
			Beweis: Annahme: $\sum_{n=0}^{\infty} x^{n}$ (also $(s_{n})$) konvergiert auf $D$ gleichmä{\ss}ig gegen $f$. Zu $\varepsilon = 1$ existiert dann ein 
			$n_{0} \in \N$ mit
			$$ | s_{n}(x) - f(x) | = \frac{|x|^{n+1}}{1 - x} < 1 \quad (n \geq n_{0}, ~ x \in D). $$
			Aber: $$\frac{|x|^{n+1}}{1 - x} \rightarrow \infty \quad (x \rightarrow 1-),$$ Widerspruch.
		\item Es sei $D = [0, \infty)$, $f_{n}(x) = \frac{nx}{1 + n^{2} x^{2}}$ $(n \in \N)$. Bekannt: 
		        $$
		        \forall x \in D: ~ f_{n}(x) \rightarrow 0 =: f(x).
		        $$
		        Es sei $0 < \varepsilon < \frac{1}{2}$. Es gilt $f_n(\frac{1}{n})=\frac{1}{2}$ $(n \in \N)$ und damit: 
			$$ \forall n \in \N: ~ |f_{n}(\frac{1}{n}) - f(\frac{1}{n})| = \frac{1}{2} > \varepsilon. $$
			Also konvergiert $(f_{n})$ auf $D$ nicht gleichmäßig gegen $f$.
	\end{enumerate}	
\end{beispiele}

\index{Konvergenzkriterium!Funktionen!Weierstra{\ss}}
\begin{satz} ~\ \label{8.1:satz}
	\begin{enumerate}
		\item Die Folge $(f_{n})$ konvergiere auf $D$ punktweise gegen $f \colon D \rightarrow \R$. Weiter sei $(\alpha_{n})$ eine Folge mit $\alpha_{n} \rightarrow 0$, $m \in \N$ und
			$$ \forall n \geq m ~\forall x \in D: ~  |f_{n}(x) - f(x) | \leq \alpha_{n}. $$
			Dann konvergiert $(f_{n})$ auf $D$ gleichmä{\ss}ig gegen $f$. \label{8.1.a:satz}
		\item \textbf{Kriterium von Weierstra{\ss}}: Es sei $m \in \N$, $(c_{n})$ eine Folge in $[0, \infty)$, $\sum_{n=1}^{\infty} c_{n}$ sei konvergent und
			$$\forall n \geq m ~\forall x \in D: ~  | f_{n}(x) | \leq c_{n}.  $$
			Dann konvergiert $\sum_{n=1}^{\infty} f_{n}$ auf $D$ gleichmä{\ss}ig. \label{8.1.b:satz}
	\end{enumerate}
\end{satz}

\begin{proof} ~\
	\begin{enumerate}
		\item Es sei $\varepsilon > 0$. Es gilt: 
		$$\exists n_{0} \geq m ~\forall n \geq n_{0}: ~ \alpha_{n} < \varepsilon,$$
		und damit
		$$\forall n \geq n_{0} ~\forall x \in D: ~ | f_{n}(x) - f(x) | < \varepsilon. $$
		\item Ohne Beweis.
	\end{enumerate}
\end{proof}


\begin{satz} \label{8.2:satz}
	Es sei $\sum_{n=0}^{\infty} a_{n} (x - x_{0})^{n}$ eine Potenzreihe mit Konvergenzradius $r > 0$, es sei $D \coloneqq (x_{0} - r, x_{0} + r)$ ($D \coloneqq \R$, 
	falls $r = \infty$). \\
	Ist $[a, b] \subseteq D$, so konvergiert die Potenzreihe auf $[a, b]$ gleichmä{\ss}ig.
\end{satz}

\begin{proof}
	Es sei o.B.d.A. $x_{0} = 0$. \\	
	Wähle $\delta > 0$ so, da{\ss} $-r < -\delta <a < b < \delta < r$. Für jedes $x \in [a, b]$ gilt dann $|x| \leq \delta$, also
		\begin{align*}
		(\ast) \quad \quad \forall n \in \N_0: ~ \left| a_{n} x^{n} \right| = |a_{n}| |x|^{n} \leq |a_{n}| \delta^{n} =: c_{n}. 
		\end{align*}
	Nach \ref{4.1:satz} konvergiert $\sum_{n=0}^{\infty} a_{n} \delta^{n}$ absolut, also ist $\sum_{n=0}^{\infty} c_{n}$ konvergent. Aus $(*)$ und \ref{8.1.b:satz}
	folgt die Behauptung.
\end{proof}


\begin{satz} \label{8.3:satz}
	$(f_{n})$ bzw. $\sum_{n=1}^{\infty} f_{n}$ konvergiere auf $D$ gleichmä{\ss}ig gegen $f \colon D \rightarrow \R$. Dann gilt:
	\begin{enumerate}
		\item Sind alle $f_{n}$ in $x_{0} \in D$ stetig, so ist $f$ in $x_{0}$ stetig. \label{8.3.a:satz}
		\item Sind alle $f_{n} \in C(D)$, so ist $f \in C(D)$. \label{8.3.b:satz}
	\end{enumerate}
\end{satz}

\begin{folgerungen} ~\
	\begin{enumerate}
		\item Konvergiert $(f_{n})$ auf $D$ punktweise gegen $f \colon D \rightarrow \R$ und gilt $f_{n} \in C(D)$ $(n\in \N)$ aber $f \notin C(D)$, 
		so ist die Konvergenz nicht gleichmä{\ss}ig.
		\item Unter den Voraussetzung von \ref{8.3.a:satz} gilt: Ist $x_{0}$ ein Häufungspunkt von $D$, so ist:
			\begin{align*}
				\lim_{x \rightarrow x_{0}} \left( \lim_{n \rightarrow \infty} f_{n}(x) \right) & = \lim_{x \rightarrow x_{0}} f(x) \overset{\ref{8.3.a:satz}}{=} f(x_{0}) = \lim_{n \rightarrow \infty} f_{n}(x_{0}) \\
					& = \lim_{n \rightarrow \infty} \left( \lim_{x \rightarrow x_{0}} f_{n}(x) \right).  
			\end{align*}
	\end{enumerate}	
\end{folgerungen}

\begin{proof}(von \ref{8.3:satz}) \
	\begin{enumerate}
		\item Es sei $(x_k)$ eine Folge in $D$ mit $x_k \to x_0$. Wir zeigen $f(x_k) \to f(x_0)$: Es sei $\varepsilon > 0$. Nach Voraussetzung gilt:
		        $$
		        \exists m \in \N ~\forall x \in D: ~  |f_{m}(x) - f(x)| < \frac{\varepsilon}{3}.
		        $$
		        Da $f_{m}$ stetig in $x_{0}$ ist gilt $f_m(x_k) \to f_m(x_0)$ $(k \to \infty)$. Damit folgt:
		        $$
		        \exists k_0 \in \N ~\forall k \ge k_0: ~ |f_{m}(x_k) - f_{m}(x_{0})| < \frac{\varepsilon}{3}.
		        $$
		        Für $k \ge k_0$ gilt damit:
			\begin{align*}
				|f(x_k) - f(x_{0})| & = |f(x_k) - f_{m}(x_k) + f_{m}(x_k) - f_{m}(x_{0}) + f_{m}(x_{0}) - f(x_{0})| \\ 
					& \leq |f(x_k) - f_{m}(x_k) | + | f_{m}(x_k) - f_{m}(x_{0})| + | f_{m}(x_{0}) - f(x_{0})| \\
					& < \frac{\varepsilon}{3} +  \frac{\varepsilon}{3} + \frac{\varepsilon}{3} = \varepsilon
			\end{align*}
			Damit folgt die Behauptung.
		\item folgt aus a).	
	\end{enumerate}
\end{proof}


\begin{proof}(von \ref{7.4:satz}) Es sei $\sum_{n=0}^{\infty}a_{n}(x - x_{0})^{n}$ sei eine Potenzreihe mit Konvergenzradius $r > 0$, $D \coloneqq$ 
($x_{0} - r, x_{0} + r) ~(D := \R$, falls $r = \infty$) und $f(x) \coloneqq \sum_{n=0}^{\infty} a_{n}(x-x_{0})^{n} ~(x \in D)$. \\ 
Es sei $x \in D$. Wähle $a, b \in \R$ so, da{\ss} $x \in (a, b) \subseteq [a, b] \subseteq D$. Nach \ref{8.2:satz} konvergiert die Potenzreihe auf 
$[a, b]$ gleichmä{\ss}ig. Nach \ref{8.3:satz} ist $f \in C([a, b])$. Also ist $f$ in $x$ stetig. Da $x \in D$ beliebig war ist $f \in C(D)$.
\end{proof}

\index{Identitätssatz für Potenzreihen}
\begin{satz}[Identitätssatz für Potenzreihen] \label{8.4:prop-IdentitätssatzFürPotenzreihe}
	Es sei $\sum_{n=0}^{\infty} a_{n} (x - x_{0})^{n}$ eine Potenzreihe mit Konvergenzradius $r > 0$, $D \coloneqq (x_{0} - r, x_{0} + r)$ ($D \coloneqq \R$, falls $r = \infty$) 
	und $f(x) \coloneqq \sum_{n=0}^{\infty} a_{n} (x - x_{0})^{n} ~(x \in D)$. \\
	Weiter sei $(x_{k})$ eine Folge in $D \setminus \{ x_{0} \}$ mit $x_{k} \rightarrow x_{0}$ und $f(x_{k}) = 0$ $(k \in \N)$. Dann gilt:
	$$\forall n \in \N_{0}: ~  a_{n} = 0.$$ 
	Insbesondere ist dann $r= \infty$ und $f(x)=0$ $(x \in \R)$.
\end{satz}

Ohne Beweis.

\newpage


\chapter{Differentialrechnung}

I.d. $\S$en sei $I \subseteq \R$ ein Intervall und $f \colon I \rightarrow \R$ eine Funktion. 


\index{differenzierbar} \index{Ableitung}
\begin{definition}
	$f$ hei{\ss}t \textbf{in $x_{0} \in I$ differenzierbar} (db) $:\iff$ Es existiert
		$$\lim_{x \rightarrow x_{0}} \frac{f(x) - f(x_{0})}{x - x_{0}} \in \R. $$
	Äquivalent ist: Es existiert 
	$$
	\lim_{h \rightarrow 0} \frac{f(x_{0} + h) - f(x_{0})}{h}\in \R.
	$$
	In diesem Fall hei{\ss}t obiger Grenzwert die \textbf{Ableitung von $f$ in $x_{0}$} und wird mit $f'(x_{0})$ bezeichnet. \\
	Ist $f$ in jedem $x \in I$ differenzierbar, so hei{\ss}t $f$ \textbf{auf $I$ differenzierbar} und die \textbf{Ableitung $f':I \to \R$ von $f$ auf $I$} ist gegeben durch 
	$x \mapsto f'(x)$.
\end{definition}


\begin{beispiele} ~\
	\begin{enumerate}
		\item Es sei $c \in \R$ und $f(x) \coloneqq c$ $(x \in \R)$. Dann ist $f$ auf $\R$ differenzierbar und $f'(x)=0$ $(x \in \R)$.
		\item Es sei $I = \R$, $f(x) = |x|$, $x_{0} = 0$. Es gilt: 
			$$ \frac{f(x) - f(x_{0})}{x - x_{0}} = \frac{|x|}{x} = \begin{cases} ~1, & x > 0 \\ -1, & x < 0 \end{cases} $$ 
			$f$ ist also in $x_{0} = 0$ nicht differenzierbar.
		\item Es sei $I = \R$, $n \in \N$, $f(x) = x^{n}$. Für $x_{0} \in \R$, $x \neq x_{0}$ gilt:
			\begin{align*}
				\frac{f(x) - f(x_{0})}{x - x_{0}} & = \frac{x^{n} - x_{0}^{n}}{x - x_{0}} \\
					& = \frac{(x - x_{0}) (x^{n-1} + x^{n-2} x_{0} + \dotsc + x x_{0}^{n-2} + x_{0}^{n-1})}{x - x_{0}} \\
					& = x^{n-1} + x^{n-2} x_{0} + \dotsc + x x_{0}^{n-2} + x_{0}^{n-1} \rightarrow n x_{0}^{n-1} ~(x \rightarrow x_{0}). 
			\end{align*} 
			Also ist $f$ auf $\R$ differenzierbar und $f'(x) = n x^{n-1}$ $(x \in \R)$, kurz:
				$$ (x^{n})' = n x^{n-1} \text{ auf } \R. $$
		\item Es sei $I = \R$, $f(x) = e^{x}$. Für $x_{0} \in \R$, $h \neq 0$ gilt: 
		 	$$ \frac{f(x_{0} + h) - f(x_{0})}{h} = \frac{e^{x_{0} + h} - e^{x_{0}}}{h} \xrightarrow[]{\ref{7.6:bsp}} e^{x_{0}} ~(h \rightarrow 0). $$ 
		 	Also ist $f$ auf $\R$ differenzierbar und $f'(x) = e^{x}$ $(x \in \R)$, kurz:
		 		$$ (e^{x})' = e^{x} \text{ auf } \R $$
	\end{enumerate}	
\end{beispiele}


\begin{satz} \label{9.1:satz}
	Ist $f$ in $x_{0} \in I$ differenzierbar, so ist $f$ in $x_{0}$ stetig.
\end{satz}

\begin{proof}
	Es sei $x \in I$, $x \neq x_{0}$. Es gilt:
		$$ f(x) - f(x_{0}) = \frac{f(x) - f(x_{0})}{x - x_{0}} (x - x_{0}) \rightarrow f'(x_0) \cdot 0 = 0 ~(x \rightarrow x_{0}) $$
	Also gilt $\lim_{x \rightarrow x_{0}} f(x) = f(x_{0})$.
\end{proof}

\begin{bemerkung}
Die Funktion $f: \R \to \R$, $f(x)=|x|$ ist in $x_0=0$ stetig aber in diesem Punkt nicht differenzierbar.
\end{bemerkung}


\begin{satz}[Differentiationsregeln]
	Die Funktionen $f,g \colon I \rightarrow \R$ seien in  $x_{0} \in I$ differenzierbar. Dann gilt:
	\begin{enumerate}
		\item Für $\alpha, \beta \in \R$ ist $\alpha f + \beta g$ differenzierbar in $x_{0}$ und
			$$ (\alpha f + \beta g)'(x_{0}) = \alpha f'(x_{0}) + \beta g'(x_{0}) $$
		\item $f g$ ist differenzierbar in $x_{0}$ und
			$$ (f g)'(x_{0}) = f'(x_{0})g(x_{0})+ f(x_{0})g'(x_{0}). $$
		\item Ist $g(x_{0}) \neq 0$, so existiert ein $\delta > 0$ mit $g(x) \neq 0 ~(x \in J \coloneqq I \cap U_{\delta}(x_{0}))$. Die Funktion $\frac{f}{g} \colon J \rightarrow \R$ ist differenzierbar in $x_{0}$ und 
			$$ \left( \frac{f}{g} \right)'(x_{0}) = \frac{f'(x_{0}) g(x_{0}) - f(x_{0})g'(x_{0})}{g(x_{0})^{2}}. $$
	\end{enumerate}		
\end{satz}

\begin{proof} ~\
	\begin{enumerate}
		\item Übung.
		\item Übung (man orientiere sich an c)).
		\item Nach \ref{9.1:satz} ist $g$ stetig in $x_{0}$. Wegen $g(x_{0}) \neq 0$ folgt mit \ref{6.3.b:satz}:  
			$$ \exists \delta > 0 ~ \forall x \in I \cap U_{\delta}(x_{0}) \eqqcolon J: ~ g(x) \neq 0. $$
			Sei $h \coloneqq \frac{f}{g}$ auf $J$. Für $x \neq x_{0}$, $x \in J$ gilt:
			\begin{align*}
				\frac{h(x) - h(x_{0})}{x - x_{0}} & = \frac{f(x) - f(x_{0})}{x- x_{0}} \frac{1}{g(x)} - f(x_{0}) 
				\frac{\frac{1}{g(x_{0})} - \frac{1}{g(x)}}{x - x_{0}} \\
					& = \underbrace{\frac{1}{g(x)g(x_{0})}}_{\rightarrow \frac{1}{g(x_{0})^{2}}} 
					\bigg( \underbrace{\frac{f(x) - f(x_{0})}{x - x_{0}}}_{\rightarrow f'(x_{0})} g(x_{0}) - f(x_{0}) 
					\underbrace{\frac{g(x) - g(x_{0})}{x - x_{0}}}_{\rightarrow g'(x_{0})} \bigg) \\
					& \to \frac{f'(x_{0}) g(x_{0}) - f(x_{0})g'(x_{0})}{g(x_{0})^{2}} \quad (x \to x_0).
			\end{align*} 
	\end{enumerate}
\end{proof}


\begin{satz} \label{9.3:satz}
	Es sei $f \in C(I)$ streng monoton, in $x_{0} \in I$ differenzierbar und es sei $f'(x_{0}) \neq 0$. Dann ist
	$f^{-1} \colon f(I) \rightarrow \R$ differenzierbar in $y_{0} \coloneqq f(x_{0})$ und
	$$ (f^{-1})'(y_{0}) = \frac{1}{f'(x_{0})} = \frac{1}{f'(f^{-1}(y_{0}))}. $$
\end{satz}

\begin{proof} Nach \ref{7.12:satz} ist $f(I)$ ein Intervall. Es sei $(y_{n})$ eine Folge in $f(I)$ mit $y_{n} \rightarrow y_{0}$ und $y_{n} \neq y_{0}$ $(n \in \N)$. 
        Setze $x_{n} \coloneqq f^{-1}(y_{n})$ $(n \in \N)$. Nach \ref{7.13:satz} ist $f^{-1} \in C(f(I))$, also gilt $x_{n}=f^{-1}(y_{n})\rightarrow f^{-1}(y_{0})=x_{0}$. Somit gilt:
	$$ \frac{f^{-1}(y_{n}) - f^{-1}(y_{0})}{y_{n} - y_{0}} = \frac{x_{n} - x_{0}}{f(x_{n}) - f(x_{0})} \rightarrow \frac{1}{f'(x_{0})} \quad (n \to \infty). $$
\end{proof}

\index{Kettenregel}
\begin{satz}[Kettenregel] \label{9.4:prop-Kettenregel}
	Es sei $J \subseteq \R$ sei ein weiteres Intervall, $g \colon J \rightarrow \R$ eine Funktion und $f(I) \subseteq J$. Weiter sei $f$ in $x_{0} \in I$ differenzierbar 
	und $g$ sei in $y_{0} \coloneqq f(x_{0})$ differenzierbar. Dann ist
		$$ g \circ f \colon I \rightarrow \R \text{ differenzierbar in } x_{0} $$
		und
		$$ (g \circ f)'(x_{0}) = g'(f(x_{0})) f'(x_{0}). $$
\end{satz}

\begin{proof}
	Für $y \in J$ sei
		$$ \tilde{g}(y) \coloneqq \begin{cases} \frac{g(y) - g(y_{0})}{y - y_{0}}, & y \neq y_{0} \\ g'(y_{0}), & y = y_{0} \end{cases} $$
	Nach Voraussetzung ist $g$ differenzierbar in $y_{0}$. Damit ist $\tilde{g}$ stetig in $y_{0}$, d.h. 
	        $$\tilde{g}(y) \rightarrow \tilde{g}(y_{0}) = g'(y_{0}) = g'(f(x_{0})) \quad ~(y \rightarrow y_{0}).$$
		$$ \Rightarrow \tilde{g}(f(x)) \rightarrow g'(f(x_{0})) \quad (x \rightarrow x_{0}) $$
	Es ist $g(y) - g(y_{0}) = \tilde{g}(y) (y - y_{0})$ $(y \in J)$. Damit folgt:
		$$ \frac{g(f(x)) - g(f(x_{0}))}{x - x_{0}} = \tilde{g}(f(x)) \frac{f(x) - f(x_{0})}{x - x_{0}} \rightarrow g'(f(x_{0})) f'(x_{0}) \quad (x \rightarrow x_{0}). $$
\end{proof}


\begin{beispiele} ~\
	\begin{enumerate}
		\item Es sei $a > 0$ und $h(x) = a^{x}$ $(x \in \R)$. \\
		        Mit $g(x) = e^{x}$ und $f(x) = x \log a$ gilt $h(x)=e^{x \log a} = g(f(x))$. 
		        Nach \ref{9.4:prop-Kettenregel} gilt: 
		        $$h'(x) = g'(f(x)) f'(x) = e^{x \log a} \cdot \log a = a^{x} \log a.$$
		        Kurz: $(a^{x})' = a^{x} \log a$ auf $\R$.
		\item Betrachte $f(x) = e^{x}$ $(x \in \R)$, $f^{-1}(y) = \log y ~(y \in (0, \infty))$. Nach \ref{9.3:satz} ist $f^{-1}$ auf $(0, \infty)$ differenzierbar und
			$$ (f^{-1})'(y) = \frac{1}{f'(f^{-1}(y))} = \frac{1}{e^{\log(y)}} = \frac{1}{y}. $$
			Kurz: $(\log x)' = \frac{1}{x}$ auf $(0, \infty)$.
		\item Es sei $\alpha \in \R$ und $f(x) = x^{\alpha} = e^{\alpha \log x} ~(x \in (0, \infty))$.
			$$ f'(x) = e^{\alpha \log x} (\alpha \log x)' = x^{\alpha} \alpha \frac{1}{x} = \alpha x^{\alpha - 1}. $$
			Kurz: $(x^{\alpha})' = \alpha x^{\alpha - 1}$ auf $(0, \infty)$.
		\item Aus Beispiel c) folgt: $(\sqrt{x})' = \frac{1}{2 \sqrt{x}}$ auf $(0, \infty)$.
	\end{enumerate}
\end{beispiele}


\begin{anwendung} \label{9.5:anwendung}
	Es sei $a \in \R$ und o.B.d.A. $a \neq 0$. Für $f(t) = \log(1 + t)$ $(t > -1)$ gilt: $f'(t) = \frac{1}{1 + t}$. Damit folgt:
	$$ \lim_{t \rightarrow 0}  \frac{\log(1+t)}{t} = \lim_{t \rightarrow 0} \frac{f(t) - f(0)}{t - 0} = f'(0) = 1 $$
	$$ \Rightarrow 1 = \lim_{x \rightarrow \infty} \frac{\log(1 + \frac{a}{x})}{\frac{a}{x}} =  \lim_{x \rightarrow \infty} \frac{1}{a} x \log(1 + \frac{a}{x}) 
	=  \lim_{x \rightarrow \infty} \frac{1}{a} \log (1 + \frac{a}{x})^{x} $$
	$$
	\Rightarrow  \lim_{x \rightarrow \infty} \log (1 + \frac{a}{x})^{x} = a  \Rightarrow \lim_{x \rightarrow \infty} (1 + \frac{a}{x})^{x} = e^{a}.
	$$
\end{anwendung}


\begin{definition}
	Es sei $M \subseteq \R$ und $g \colon M \rightarrow \R$ eine Funktion.
	\begin{enumerate}
		\item $x_{0} \in M$ hei{\ss}t ein \textbf{innerer Punkt von M} $:\iff \exists \delta > 0: ~ U_{\delta}(x_{0}) \subseteq M$
		\item $g$ hat in $x_{0} \in M$ ein \textbf{lokales Maximum [bzw. Minimum]} $:\iff$
		$$ 
		\exists \delta > 0 ~ \forall x \in U_{\delta}(x_{0}) \cap M: ~ g(x) \leq g(x_{0}) ~~ [\text{ bzw. } g(x) \geq g(x_{0})].
		$$
		Alternative Sprechweise: \textbf{Relatives Maximum [bzw. Minimum]}.
		\item $g$ hat in $x_{0} \in M$ ein \textbf{globales Maximum [bzw. Minimum]} $:\iff$
		$$ 
		\forall x \in M: ~ g(x) \leq g(x_{0}) ~~ [\text{ bzw. } g(x) \geq g(x_{0})].
		$$
		Alternative Sprechweise: \textbf{Absolutes Maximum [bzw. Minimum]}.
		\item \textbf{``Extremum''} bedeutet ``Maximum oder Minimum''.
	\end{enumerate}
	
	
\end{definition}


\begin{satz} \label{9.6:satz}
	Die Funktion $f \colon I \rightarrow \R$ habe in $x_{0} \in I$ ein lokales Extremum und sei in $x_{0}$ differenzierbar. Ist $x_{0}$ ein innerer Punkt von $I$, 
	so ist $f'(x_{0}) = 0$.
\end{satz}

\begin{proof}
	O.B.d.A. habe $f$ in $x_{0}$ ein lokales Maximum. Dann gilt: 
	$$ \exists \delta > 0: ~ U_{\delta}(x_{0}) \subseteq I \text{ und } f(x) \leq f(x_{0}) ~ (x \in U_{\delta}(x_{0})).$$
	Damit ist
	$$ D(x) \coloneqq \frac{f(x) - f(x_{0})}{x - x_{0}} \begin{cases} \leq 0, ~~ x \in (x_0,x_0+\delta) \\ \geq 0, ~~ x \in (x_0-\delta,x_0) \end{cases}. $$
	Also gilt $f'(x_{0}) = \lim_{x \rightarrow x_{0} +} D(x) \leq 0$ und $f'(x_{0}) = \lim_{x \rightarrow x_{0} -} D(x) \geq 0$. 
\end{proof}

\index{Mittelwertsatz}
\begin{satz}[Der Mittelwertsatz (MWS) der Differentialrechnung] \label{9.7:prop-Mittelwertsatz} ~\\
        Es sei $f \in C([a, b])$ und $f$ sei auf $(a, b)$ differenzierbar. Dann gilt: 
	$$ \exists \xi \in (a, b): ~ \frac{f(b) - f(a)}{b - a} = f'(\xi). $$		
\end{satz}

\begin{proof}
	Wir setzen
	$$g(x) \coloneqq f(x) - f(a) - \frac{f(b) - f(a)}{b - a} (x - a) \quad (x \in [a, b]).$$ Es gilt: $g \in C([a, b])$, $g$ ist differenzierbar auf $(a, b)$, $g(a) = g(b) = 0$ und 
		$$ g'(x) = f'(x) - \frac{f(b) - f(a)}{b - a} \quad (x \in (a, b)). $$
	Wir zeigen: $\exists \xi \in (a, b)$: $g'(\xi) = 0$. \\
	Fall 1: $g(x) = 0$ $(x \in [a,b])$. \checkmark \\
	Fall 2: $g(x_0) \not= 0$ f\"ur ein $x_0 \in [a,b]$. Nach \ref{7.11:satz} gilt:
	$$\exists x_{1}, x_{2} \in [a, b] ~ \forall x \in [a, b]: ~ g(x_{1}) \leq g(x) \leq g(x_{2}).$$
	Nun ist $x_{1} \in (a, b)$ oder $x_{2} \in (a, b)$ (sonst wäre $g = 0$ auf $[a,b]$). Mit \ref{9.6:satz} folgt: $g'(x_{1}) = 0$ oder $g'(x_{2}) = 0$.
\end{proof}


\begin{folgerung} \label{9.8:folg}
	Es sei $f \colon I \rightarrow \R$ differenzierbar auf $I$. Dann gilt: 
	$$f \text{ ist auf } I \text { konstant } \iff ~ \forall x \in I: ~ f'(x)=0. $$	
\end{folgerung}

\begin{proof}
	"'$\Rightarrow$"' \checkmark, "'$\Leftarrow$"' Es seien $x_{1}, x_{2} \in I$ mit $x_{1} < x_{2}$. Nach \ref{9.7:prop-Mittelwertsatz} gilt:
		$$ \exists \xi \in (x_{1}, x_{2}): ~ f(x_{2}) - f(x_{1}) = f'(\xi) (x_{2} - x_{1}) = 0, $$ 
	also $f(x_{1}) = f(x_{2})$.
\end{proof}


\begin{anwendung} \label{9.9:anwendung}
	Es sei $f \colon I \rightarrow \R$ differenzierbar. Dann gilt: 
		$$ f' = f \text{ auf } I \iff \exists c \in \R: f(x) = c e^{x} ~(x \in I) $$
\end{anwendung}

\begin{proof}
	"'$\Rightarrow$"' \checkmark, "'$\Leftarrow$"' Setze $g(x) \coloneqq \frac{f(x)}{e^{x}}$ $(x \in I)$. Dann gilt:
	$$ \forall x \in I: ~ g'(x) = \frac{f'(x) e^{x} - f(x) e^x}{e^{2x}} = 0. $$
	Mit \ref{9.8:folg} folgt: $\exists c \in \R ~\forall x \in I:  g(x) = c$, also $f(x)=c e^x$ $(x \in I)$.
\end{proof}


\begin{satz} \label{9.10:satz}
	$f, g \colon I \rightarrow \R$ seien auf $I$ differenzierbar. Dann gilt:
	\begin{enumerate}
		\item Ist $f' = g'$ auf $I$, so existiert ein $c \in \R$ mit $f = g + c$ auf $I$.
		\item Ist $f' \geq 0$ auf $I$, so ist $f$ monoton wachsend auf $I$. \\
				Ist $f' > 0$ auf $I$, so ist $f$ streng monoton wachsend auf $I$.
		\item Ist $f' \leq 0$ auf $I$, so ist $f$ monoton fallend auf $I$. \\
				Ist $f' < 0$ auf $I$, so ist $f$ streng monoton fallend auf $I$.
	\end{enumerate}
\end{satz}

\begin{proof}  ~\
	\begin{enumerate}
		\item Es gilt $(f - g)' = f'-g'=0$ auf $I$. Mit \ref{9.8:folg} folgt die Behauptung.
		\item Es sei z.B. $f' > 0$ auf $I$ und $x_{1}, x_{2} \in I$ mit $x_{1} < x_{2}$. Mit dem MWS folgt: 
		$$\exists \xi \in (x_{1}, x_{2}): ~ f(x_{2}) - f(x_{1}) = \underbrace{f'(\xi)}_{> 0} (x_{2} - x_{1}) > 0,$$
		also $f(x_{1}) < f(x_{2})$.
		\item Analog zur b).
	\end{enumerate}
\end{proof}


\begin{unnamedtheorem}[Die Regeln von de l'Hospital] \label{9.11:prop:lHopital} ~\\
	Es sei $I = (a, b)$, wobei $a = -\infty$ oder $b = \infty$ zugelassen ist. Es seien $f, g \colon I \rightarrow \R$ auf $I$ differenzierbar mit 
	$g'(x) \neq 0$ $(x \in I)$, und es sei $c=a$ oder $c=b$. 
	Es existiere
	$$ L \coloneqq \lim_{x \rightarrow c} \frac{f'(x)}{g'(x)}  \in \R \cup \{ - \infty, \infty \}. $$
	Gilt (I) $\lim_{x \rightarrow c} f(x) = \lim_{x \rightarrow c} g(x) = 0$ oder (II) $\lim_{x \rightarrow c} g(x) = \pm \infty$, \\
	so ist 
	$$ \lim_{x \rightarrow c} \frac{f(x)}{g(x)} = L. $$
	\end{unnamedtheorem}

Ohne Beweis.

\begin{beispiele} ~\
	\begin{enumerate}
		\item Für $a, b > 0$ gilt:
		$$\lim_{x \rightarrow 0} \frac{a^{x} - b^{x}}{x} = \lim_{x \rightarrow 0} \frac{a^{x} \log a - b^{x} \log b}{1} = \log a - \log b.$$
		\item $$\lim_{x \rightarrow \infty} \frac{\log x}{x} = \lim_{x \rightarrow \infty} \frac{\frac{1}{x}}{1} = 0.$$
		\item $$\lim_{x \rightarrow 0} x \log x = \lim_{x \rightarrow 0} \frac{\log x}{\frac{1}{x}} 
		= \lim_{x \rightarrow 0} \frac{\frac{1}{x}}{-\frac{1}{x^{2}}} = \lim_{x \rightarrow 0} (-x) = 0.$$
		Hieraus folgt:
		$$\lim_{x \rightarrow 0} x^{x} = \lim_{x \rightarrow 0} e^{x \log x} = e^{0} = 1.$$
		\item  $$0=\lim_{x \rightarrow 1} \frac{\log x }{x} \not= \lim_{x \rightarrow 1} \frac{\frac{1}{x}}{1} = 1.$$
		Die Voraussetzungen der Regeln von de l'Hospital sind hier nicht erfüllt.
	\end{enumerate}
\end{beispiele}


\begin{satz} \label{9.12:satz}
	Es sei $\sum_{n=0}^{\infty} a_{n} (x - x_{0})^{n}$ eine Potenzreihe mit Konvergenzradius $r > 0$, $I = (x_{0}- r, x_{0} + r)$ ($I = \R$, falls $r = \infty$) und 
	$f(x) \coloneqq \sum_{n = 0}^{\infty} a_{n} (x - x_{0})^{n} ~(x \in I)$. Dann gilt:
	\begin{enumerate}
		\item Die Potenzreihe $\sum_{n=1}^{\infty} n a_{n} (x - x_{0})^{n-1}$ hat den Konvergenzradius $r$.
		\item $f$ ist auf $I$ differenzierbar und
			$$ f'(x) = \sum_{n=1}^{\infty} n a_{n} (x - x_{0})^{n-1} \quad (x \in I). $$
	\end{enumerate}
\end{satz}

\begin{proof} ~\
	\begin{enumerate}
		\item Es gilt: $\sum_{n=1}^{\infty} n a_{n} (x - x_{0})^{n-1}$ konvergiert genau dann, wenn $\sum_{n=1}^{\infty} n a_{n} (x - x_{0})^{n}$
		konvergiert. Beide Potenzreihen haben also denselben Konvergenzradius. Wegen $\lim_{n \to \infty} \sqrt[n]{n} =1$ gilt
		$$
		\limsup_{n \to \infty} \sqrt[n]{n |a_n|}= \limsup_{n \to \infty} \sqrt[n]{|a_n|}.
		$$
		Mit \ref{4.1:satz} folgt die Behauptung.
		\item Ohne Beweis (kann mit \ref{10.17:satz} bewiesen werden).
	\end{enumerate}
\end{proof}

\begin{unnamedtheorem}[Sinus/Cosinus] \label{9.13:prop-SinusCosinus}
$\sin x =\sum_{n=0}^{\infty} (-1)^{n} \frac{x^{2n+1}}{(2n+1)!}$ $(x \in \R)$. \\
Nach \ref{9.12:satz} gilt: $\sin$ ist auf $\R$ differenzierbar und 
	$$ (\sin x)' = \sum_{n=0}^{\infty} (-1)^{n} \frac{(2n+1) x^{2n}}{(2n + 1)!} = \sum_{n=0}^{\infty} (-1)^{n} \frac{x^{2n}}{(2n)!} = \cos x. $$
	Analog: $\cos$ ist auf $\R$ differenzierbar und $(\cos x)' = - \sin x$.
\end{unnamedtheorem}
	
\begin{unnamedtheorem}[Definition von $\pi$] ~\ \label{9.14:prop-DefPi}
	\begin{enumerate}
		\item Für $x \in (0, 2)$ ist
			$$ \sin x = \underbrace{\left(x - \frac{x^{3}}{3!}\right)}_{> 0} + \underbrace{\left(\frac{x^{5}}{5!} - \frac{x^{7}}{7!}\right)}_{> 0} + 
			\underbrace{\left(\frac{x^{9}}{9!} - \frac{x^{11}}{11!}\right)}_{> 0} + \dotsc > x - \frac{x^{3}}{3!} > 0. $$
			Speziell: $\sin 1 > 1 - \frac{1}{6} = \frac{5}{6}$.
		\item $\exists \xi_{0} \in (0, 2)$: $\cos \xi_{0} = 0$ und $\cos x > 0$ $(x\in [0, \xi_{0}))$
		\begin{proof}
			Es gilt $\cos 0 = 1 > 0$ und 
			\begin{align*}
				\cos 2 & = \cos (1 + 1) \overset{\ref{4.4:prop-Sinus}}{=} \cos^{2} 1 - \sin^{2} 1 = \cos^{2} 1 + \sin^{2} 1 - 2 \sin^{2} 1 \\
					& = 1 - 2 \sin^{2} 1 \leq 1 - 2 \cdot \frac{25}{36} < 0.
			\end{align*} 
			Mit \ref{7.7:prop-Zwischenwertsatz} folgt:  $\exists \xi_{0} \in (0, 2):$ $\cos \xi_{0} = 0$. Weiter gilt:
			$$ \forall x \in (0,2): ~ (\cos x)' = - \sin x \overset{a)}< 0 ~ \Rightarrow ~ \forall x \in [0, \xi_{0}): ~ \cos x > 0.  $$
		\end{proof}
		\item Es sei $\xi_{0}$ wie in b). Wir definieren $$\pi \coloneqq 2 \xi_{0}.$$ 
		Es gilt $\xi_{0} \in (0, 2)$, also $\pi \in (0, 4)$ $(\pi \approx 3,14\dotsc)$. Es ist	$\frac{\pi}{2} = \xi_{0}$, also $\cos \frac{\pi}{2} = 0$.
		Damit gilt:
		$$ \sin^{2} \frac{\pi}{2}= 1 - \cos^{2} \frac{\pi}{2} = 1 \Rightarrow | \sin \frac{\pi}{2} | = 1 \xRightarrow[]{a)} \sin \frac{\pi}{2} = 1. $$
		\end{enumerate}	 
\end{unnamedtheorem}


\begin{figure*}[!ht] \centering
	\begin{tikzpicture}
     	\draw[->] (-2,0) -- (4.5,0) node[right] {$x$};
      	\draw[->] (0,-2) -- (0,2) node[above] {$y$};
      	\draw[scale=0.1,domain=-10:29,smooth,variable=\x] plot ({\x},{12.5*sin(deg(\x/6))}) node[below] {$\sin x$};
      	\draw[scale=0.1,domain=-10:29,smooth,variable=\x] plot ({\x},{12.5*cos(deg(\x/6))}) node[above] {$\cos x$};
		\draw ($(0,1.25) + (-4pt,0)$) -- ($(0,1.25) + (4pt,0)$) node [right] {$1$};
		\draw ($(0,-1.25) + (-4pt,0)$) -- ($(0,-1.25) + (4pt,0)$) node [right] {$-1$};
    \end{tikzpicture}
	\caption{Sinus und Cosinus.}	
\end{figure*}

\begin{unnamedtheorem}[Weitere Eigenschaften von Sinus und Cosinus] \label{9.15:prop-EigSinusCosinus} ~\
	\begin{enumerate}
		\item Aus \ref{4.4:prop-Sinus} folgt:
			\begin{align*}
				\sin(x + \frac{\pi}{2}) & = \sin x \cos \frac{\pi}{2} + \cos x \sin \frac{\pi}{2} = \cos x 
				\intertext{Analog:}
				\cos(x + \frac{\pi}{2}) & = - \sin x \\
				\sin(x + \pi) & = - \sin x , \quad \cos(x + \pi) = - \cos x \\
				\sin(x + 2\pi) &  = \sin x, \quad \cos(x + 2\pi) = \cos x
			\end{align*}
		\item $\cos$ hat in $[0, \pi]$ genau eine Nullstelle. Ohne Beweis.
		\item In der gro{\ss}en Übungen wird gezeigt:
			\begin{align*}
				\cos x = 0 & \iff x \in \{ (2k + 1) \frac{\pi}{2} : k \in \Z \} \\
				\sin x = 0 & \iff x \in \{ k \pi : k \in \Z \}
			\end{align*}
	\end{enumerate}
	
\end{unnamedtheorem}

\index{Tangens}
\begin{definition}
	Die Funktion 
	$$
	\tan \colon \R \setminus \{ (2k + 1) \frac{\pi}{2} \colon k \in \Z \} \rightarrow \R, ~ \tan x \coloneqq \frac{\sin x}{\cos x}
	$$
	hei{\ss}t \textbf{Tangens}. Es gilt:
	$$ (\tan x)' = \frac{\cos^{2} x + \sin^{2} x}{\cos^{2} x} = \frac{1}{\cos^{2} x} > 0. $$
	Also ist $\tan$ auf $(-\frac{\pi}{2}, \frac{\pi}{2})$ streng monoton wachsend.
\end{definition}

\index{Arkustangens}
\begin{definition}
	Es gilt (Übung): $\tan((-\frac{\pi}{2}, \frac{\pi}{2})) = \R$. Es existiert also die Umkehrfunktion
	$$ \arctan \coloneqq \tan^{-1} \colon \R \rightarrow (-\frac{\pi}{2},\frac{\pi}{2}). $$
	Sie hei{\ss}t \textbf{Arkustangens}. Mit \ref{9.3:satz} folgt:
	$$(\arctan x)' = \frac{1}{1 + x^{2}} \quad (x \in \R).$$
\end{definition}

\index{Abelscher Grenzwertsatz}
\begin{satz}[Abelscher Grenzwertsatz] \label{9.16:prop-AbelscherGrenzwertsatz}
Es sei $\sum_{n=0}^{\infty} a_{n} (x - x_{0})^{n}$ eine Potenzreihe mit Konvergenzradius $r \in (0,\infty)$. Dann gilt:
	\begin{enumerate}
		\item Konvergiert die Potenzreihe auch in $x_{0} + r$ und ist
			$$ f(x) \coloneqq \sum_{n=0}^{\infty} a_{n} (x - x_{0})^{n} \text{ für } x \in (x_{0} - r, x_{0} + r], $$
			so ist f stetig in $x_{0} + r$.
		\item  Konvergiert die Potenzreihe auch in $x_{0} - r$ und ist
			$$ f(x) \coloneqq \sum_{n=0}^{\infty} a_{n} (x - x_{0})^{n} \text{ für } x \in [x_{0} - r, x_{0} + r), $$
			so ist f stetig in $x_{0} - r$.
	\end{enumerate}
\end{satz}

Ohne Beweis.

\begin{anwendungen} ~\ \label{9.17:anwendungen}
	\begin{enumerate}
		\item Betrachte $f(x) = \log(1+x)$ $(x \in (-1, 1])$. Dann gilt: \label{9.17.a:anwendungen}
			$$ f'(x) = \frac{1}{1+x} = \frac{1}{1 - (-x)} = \sum_{n=0}^{\infty} (-x)^{n} = \sum_{n=0}^{\infty} (-1)^{n} x^{n} \quad (x \in (-1,1)=:I). $$
			Wir setzen $g(x) \coloneqq \sum_{n=1}^{\infty} (-1)^{n+1} \frac{x^{n}}{n}$ für $x \in (-1, 1]$. Nach \ref{9.12:satz} ist $g$ ist differenzierbar auf $I$ und 
			$$ g'(x) = \sum_{n=1}^{\infty} (-1)^{n+1} x^{n-1} = \sum_{n=1}^{\infty} (-1)^{n-1} x^{n-1} = f'(x) \quad (x \in I). $$
			Damit existiert ein $c \in \R$ mit $f(x) = g(x) + c$ $(x \in I)$. Mit $x = 0$ folgt $c = 0$. Also gilt:
			$$ f(x) = g(x) \quad (x \in I). $$
			Da $f$ und $g$ stetig auf $(0,1]$ sind (für $g$ vgl. \ref{9.16:prop-AbelscherGrenzwertsatz}) gilt
			$$f(x) = g(x) ~ \quad (x \in (-1, 1]), $$ also 
			$$ \log(1+x) = \sum_{n=1}^{\infty} (-1)^{n+1} \frac{x^{n}}{n} \quad (x \in (-1 , 1]). $$
			Insbesondere gilt mit $x = 1$: 
			$$\log(2) = \sum_{n=1}^{\infty} \frac{(-1)^{n+1}}{n}.$$ 
		\item  Ähnlich wie in a) zeigt man (\"Ubung): \label{9.17.b:anwendungen}
			$$ \arctan x = \sum_{n=0}^{\infty} (-1)^{n} \frac{x^{2n +1}}{2n + 1} \quad (x \in [-1, 1]). $$
			Insbesondere gilt mit $x = 1$: 
			$$\arctan 1 = \sum_{n=0}^{\infty} \frac{(-1)^{n}}{2n + 1} = 1 - \frac{1}{3} + \frac{1}{5} - \frac{1}{7} +- \dotsc.$$
			Es gilt:
		        $$ \cos \frac{\pi}{4} = \cos(-\frac{\pi}{4}) \overset{\ref{9.15:prop-EigSinusCosinus}}{=} \sin( \frac{\pi}{2} - \frac{\pi}{4}) 
		        = \sin \frac{\pi}{4} \Rightarrow \tan \frac{\pi}{4} = 1 \Rightarrow \arctan 1 = \frac{\pi}{4}. $$
		        Somit gilt:
		        $$\frac{\pi}{4} = \sum_{n=0}^{\infty} \frac{(-1)^{n}}{2n + 1}.$$
	\end{enumerate}
\end{anwendungen}

\index{differenzierbar!stetig} \index{differenzierbar!n-mal}  \index{Ableitung!n-te}
\begin{definition} ~\
	\begin{enumerate}
		\item Es sei $f \colon I \rightarrow \R$ auf $I$ differenzierbar. Ist $f'$ in $x_{0} \in I$ differenzierbar, so hei{\ss}t $f$ 
		\textbf{in $x_{0}$ zweimal differenzierbar} und
			 $$ f''(x_{0}) \coloneqq (f')'(x_{0})$$ 
		hei{\ss}t \textbf{die 2. Ableitung von $f$  in} $x_{0}$.
		\item Ist $f'$ auf $I$ differenzierbar, so hei{\ss}t $f$ \textbf{auf} $I$ \textbf{zweimal differenzierbar} und
			 $$ f'' \coloneqq (f')' $$
		 die 2. Ableitung von $f$ auf $I$. Entsprechend definiert man, falls vorhanden: 
		 $$ f'''(x_{0}), f^{(4)}(x_{0}), f^{(5)}(x_{0}), \dotsc \text{ und } f''', f^{(4)},  f^{(5)}, \dotsc $$ 
		\item Für $n \in \N$ hei{\ss}t $f$ \textbf{auf $I$ $n$-mal stetig differenzierbar} $:\iff f$ ist auf $I$ $n$-mal differenzierbar und $f^{(n)} \in C(I)$.
		      In diesem Fall gilt: $f,f',\dots, f^{(n)} \in C(I)$.
		      Wir setzen
		        $$C^{0}(I) \coloneqq C(I), \quad  f^{(0)} \coloneqq f,$$
		        $$ C^n(I):= \{f:I \to \R: ~ f \text{ ist auf } I ~ n\text{-mal stetig differenzierbar}\} \quad (n \in \N), $$
			$$C^{\infty}(I) \coloneqq \bigcap_{n \geq 0} C^{n}(I). $$
	\end{enumerate}
\end{definition}


\begin{beispiele} ~\
	\begin{enumerate}
		\item $(e^{x})''' = e^{x}$, $(\sin x)'' = (\cos x)' = - \sin x$. Es gilt: $E,\sin,\cos\in C^{\infty}(\R)$. \\
		      $\sin,\cos \in C^{\infty}(\R)$.
		\item Betrachte $f(x) = x |x|$ $(x \in \R)$. Es gilt:
			\begin{description}
				\item Für $x > 0$: $f(x) = x^{2}$, $f'(x) = 2x$.
				\item Für $x < 0$: $f(x) = -x^{2}$, $f'(x) = -2x$.
				\item Für $x = 0$: $\frac{f(t) - f(0)}{t - 0} = |t| \rightarrow 0$ $(t \rightarrow 0)$, also $f'(0)=0$.
			\end{description}
			Somit ist $f$ auf $\R$ differenzierbar, $f'(x) = 2 |x|$ $(x \in \R)$ und $f'$ ist stetig auf $\R$. In $x_{0} = 0$ ist $f$ nicht zweimal differenzierbar.
			Also gilt: $f \in C^1(\R)$ und $f \notin C^2(\R)$. \\ Bemerkung: Allgemein gilt: \\
				$$C^0(I)\subsetneqq C^1(I)\subsetneqq C^2(I)\subsetneqq C^3(I)\subsetneqq ...$$
	\end{enumerate}	
\end{beispiele}


\begin{beispiel} \label{9.18:bsp} ~\\
	Wir betrachten $f:[0,1] \to \R$ definiert durch 
	$$f(x) = \begin{cases} x^{\frac{3}{2}} \sin \frac{1}{x}, & x \in (0, 1] \\0, & x = 0 \end{cases}$$
	Auf $(0, 1]$ gilt: 
	$$ f'(x) = \frac{3}{2} \sqrt{x} \sin \frac{1}{x} + x^{\frac{3}{2}} (\cos \frac{1}{x}) (-\frac{1}{x^{2}})
	= \frac{3}{2} \sqrt{x} \sin \frac{1}{x} -  \frac{1}{\sqrt{x}} \cos \frac{1}{x}.$$
	Weiter gilt:
	$$ \frac{f(x) - f(0)}{x - 0} = \underbrace{\sqrt{x}}_{\rightarrow 0} \underbrace{\sin \frac{1}{x}}_{\text{beschr.}} \rightarrow 0 \quad (x \to 0).$$
	Also ist $f$ auf $[0, 1]$ differenzierbar (mit $f'(0) = 0$). Für $x_{n} \coloneqq \frac{1}{2 n \pi}$ $(n \in \N)$ gilt: $x_{n} \rightarrow 0$ und
		$$ f'(x_{n}) = -\sqrt{2 n \pi} \cos(2 n \pi) = -\sqrt{2 n \pi} \to -\infty \quad (n \to \infty). $$ 
	Damit ist $f'$ auf $[0, 1]$ nicht beschränkt, also insbesondere nicht stetig auf $[0,1]$. \\
	Also: $f$ ist auf $[0, 1]$ differenzierbar, aber $f \notin C^{1}([0, 1])$.
\end{beispiel}


\begin{satz} \label{9.19:satz}
	Es sei $\sum_{n=0}^{\infty} a_{n} (x - x_{0})^{n}$ eine Potenzreihe mit Konvergenzradius $r > 0$, $I \coloneqq (x_{0} - r, x_{0} + r)$ ($I = \R$, falls $r = \infty$) und 
	$$ f(x) \coloneqq \sum_{n=0}^{\infty} a_{n} (x - x_{0})^{n} \quad (x \in I). $$
	Dann gilt $f \in C^{\infty}(I)$ und 
	$$\forall k \in \N_{0} ~ \forall x \in I: ~ f^{(k)}(x) = \sum_{n=k}^{\infty} n(n-1) \cdots (n-k+1) a_{n} (x - x_{0})^{n-k}. $$
	Mit $x = x_{0}$ folgt insbesondere: $f^{(k)}(x_{0}) = k! a_{k}$, also
		$$\forall k \in \N_{0}: ~ a_{k} = \frac{f^{(k)}(x_{0})}{k!}. $$
\end{satz}

\begin{proof}
Folgt induktiv aus \ref{9.12:satz}. 
\end{proof}


\begin{satz}[Satz von Taylor] \label{9.20:satz-Taylor} ~\\
	Es sei $n \in \N_{0}$ und $f$ sei auf $I$ $(n+1)$-mal differenzierbar. Es seien $x, x_{0} \in I$ und $x \neq x_{0}$. Dann existiert ein 
	$\xi \in (\min\{x,x_0\},\max\{x,x_0\}) $ mit
		$$ f(x) = f(x_{0}) + \frac{f'(x_{0})}{1!} (x - x_{0}) + \dotsc + \frac{f^{(n)}(x_{0})}{n!} (x - x_{0})^{n} + \frac{f^{(n+1)}(\xi)}{(n+1)!} (x - x_{0})^{n+1},$$\\also\\ 
			$$f(x) = \sum_{k=0}^{n}\frac{f^{(k)}(x_0)}{k!}(x-x_0)^k+\frac{f^{(n+1)}(\xi)}{(n+1)!}(x-x_0)^{n+1}.$$
\end{satz}

Ohne Beweis.

\begin{bemerkung}
	Im Fall $n = 0$ folgt sie Aussage von \ref{9.20:satz-Taylor} direkt aus dem MWS.	
\end{bemerkung}


\begin{satz} \label{9.21:satz}
	Es sei $n \geq 2$, $f \in C^{n}(I)$, $x_{0} \in I$ sei ein innerer Punkt von $I$, und
	$$ f'(x_{0}) = f''(x_{0}) = \dotsc = f^{(n-1)}(x_{0}) = 0 \text{ und } f^{(n)}(x_{0}) \neq 0. $$
	Dann gilt:
	\begin{enumerate}
		\item Ist $n$ gerade und $f^{(n)}(x_{0}) < 0$, so hat $f$ in $x_{0}$ ein lokales Maximum.
		\item Ist $n$ gerade und $f^{(n)}(x_{0}) > 0$, so hat $f$ in $x_{0}$ ein lokales Minimum.
		\item Ist $n$ ungerade, so hat $f$ in $x_{0}$ kein lokales Extremum. 
	\end{enumerate}	
\end{satz}

\begin{proof}
	Aus $f^{(n)}(x_{0}) \neq 0$ und $f^{(n)} \in C(I)$ folgt: 
	$$ (\ast) \quad \exists \delta > 0: ~ U_{\delta}(x_{0}) \subseteq I \text{ und }  f^{(n)}(x) f^{(n)}(x_0) > 0 ~ (x \in U_{\delta}(x_{0})).$$
	Es sei $x \in U_{\delta}(x_{0}) \setminus \{ x_{0} \}.$ Nach \ref{9.20:satz-Taylor} existiert ein $\xi$ zwischen $x$ und $x_{0}$ mit:
	$$ f(x) = \underbrace{\sum_{k=0}^{n-1} \frac{f^{(k)}(x_{0})}{k!} (x - x_{0})^{k}}_{= f(x_{0})} + \underbrace{\frac{f^{(n)}(\xi)}{n!} (x - x_{0})^{n}}_{\eqqcolon R(x)} $$
	\begin{enumerate}
		\item Es gilt $f^{(n)}(x_{0}) < 0$. Mit $(\ast)$ folgt $f^{n}(\xi) < 0$. Da $n$ gerade ist gilt $(x - x_{0})^{n} > 0$. Also ist $R(x) < 0$. Somit gilt: 
		      $$\forall x \in U_{\delta}(x_{0}) \setminus \{ x_{0} \}: ~ f(x) < f(x_{0}).$$
		\item Analog zu a).
		\item Es sei o.B.d.A. $f^{(n)}(x_{0}) > 0$, also $f^{(n)}(\xi) > 0$. Da $n$ ungerade ist gilt
			$$ (x - x_{0})^{n} \begin{cases} > 0, & x > x_{0} \\ < 0, & x < x_{0} \end{cases} $$
			und damit
			$$ R(x) \begin{cases} > 0, & x \in (x_0,x_0+\delta) \\ < 0, & x \in (x_0-\delta,x_0) \end{cases} 
			\quad \Rightarrow  \quad f(x) \begin{cases} > f(x_{0}), & x \in (x_0,x_0+\delta)  \\ < f(x_{0}), & x  \in (x_0-\delta,x_0) \end{cases}.$$
			
	\end{enumerate}
\end{proof}


\newpage


\chapter{Das Riemann-Integral}

\index{Zerlegung}
\begin{vereinbarung}
In diesem $\S$en sei stets $a < b$, $f \colon [a, b] \rightarrow \R$ eine Funktion und $f$ beschränkt auf $[a, b]$. Wir setzen 
$m \coloneqq \inf f([a, b]), M \coloneqq \sup f([a, b])$.
\end{vereinbarung}

\begin{definition} ~\
\begin{enumerate}
 \item $Z = \{ x_{0}, x_{1}, \dotsc, x_{n} \}$ hei{\ss}t eine \textbf{Zerlegung} von $[a, b]$ $:\iff$ 
       $$a = x_{0} < x_{1} < \dotsc < x_{n} = b.$$ 
       $\mathcal{Z} \coloneqq \{ Z: Z$ ist eine Zerlegung von $[a, b] \}$. 

 \item Es sei $Z = \{ x_{0}, \dotsc, x_{n} \} \in \mathcal{Z}$. Wir definieren 
      $$I_{j} \coloneqq [x_{j-1} , x_{j}], ~ |I_{j}| \coloneqq x_{j} - x_{j-1}, ~ m_{j} \coloneqq \inf f(I_{j}), ~ M_{j} \coloneqq \sup f(I_{j})  \quad (j = 1, \dotsc, n),$$
      sowie
	\begin{align*}
		s_{f}(Z) & \coloneqq \sum_{j=1}^{n} m_{j} |I_{j}| \quad \text{(die \textbf{Untersumme} von $f$ bzgl. $Z$),} \\
		S_{f}(Z) & \coloneqq \sum_{j=1}^{n} M_{j} |I_{j}| \quad \text{(die \textbf{Obersumme} von $f$ bzgl. $Z$).}
	\end{align*}
\end{enumerate}
\end{definition}	

Für jedes $j \in\{1, \dotsc, n\}$ gilt $m \leq m_{j} \leq M_{j} \leq M$, also $m |I_{j}| \leq m_{j} |I_{j}| \leq M_{j} |I_{j}| \leq M |I_{j}|$ und somit
	$$
	(\ast) \quad m(b-a)= m\sum_{j=1}^{n} |I_{j}| \leq s_{f}(Z) \leq S_{f}(Z) \leq M \sum_{j=1}^{n} |I_{j}| = M (b - a). 
	$$
	



\begin{definition}
	Es seien $Z_{1}, Z_{2} \in \mathcal{Z}$. $Z_{2}$ hei{\ss}t eine Verfeinerung von $Z_{1}$ $:\iff$ $Z_{1} \subseteq Z_{2}$.
\end{definition}


\begin{satz} \label{10.1:satz}
Es seien $Z_{1}, Z_{2} \in \mathcal{Z}$. Dann gilt:
	\begin{enumerate}
		\item $s_{f}(Z_{1}) \leq S_{f}(Z_{2})$. \label{10.1.a:satz}
		\item Ist $Z_{1} \subseteq Z_{2}$, so gilt: 
		      $$s_{f}(Z_{1}) \leq s_{f}(Z_{2}), \quad S_{f}(Z_{1}) \geq S_{f}(Z_{2}). $$ \label{10.1.b:satz}
	\end{enumerate}	      
\end{satz}		      
Ohne Beweis. \\
Aus $(*)$ folgt: Es existieren
	$$ s_{f}  \coloneqq \sup \{ s_{f}(Z) \colon Z \in \mathcal{Z} \} \text{ und } S_{f}  \coloneqq \inf \{ S_{f}(Z) \colon Z \in \mathcal{Z} \}.$$
Aus $(*)$ und \ref{10.1.a:satz} folgt:
	$$ m (b - a) \leq s_{f} \leq  S_{f} \leq M (b - a). $$


\index{integrierbar} \index{integrierbar!Riemann} \index{Integral} \index{Integral!Riemann}
\begin{definition} ~\\
	Die Funktion $f$ hei{\ss}t (Riemann-)\textbf{integrierbar} (ib) über $[a, b]$ $:\iff$ $s_{f} = S_{f}$. In diesem Fall hei{\ss}t
		$$ \int_{a}^{b} f dx \coloneqq \int_{a}^{b} f(x) dx \coloneqq S_{f} (= s_{f}) $$
	das (Riemann-)\textbf{Integral}	von $f$ über $[a, b]$ und wir schreiben: \\
		$$f \in R([a, b]) oder f \in R([a, b],\R).$$
\end{definition}


\begin{beispiele} ~\
	\begin{enumerate}
		\item Es sei $c \in \R$ und $f(x) = c$ $(x \in [a, b])$. Dann gilt $c(b - a) \leq s_{f} \leq S_{f} \leq c(b- a)$, also $f \in R[a, b]$ und 
		$\int_{a}^{b}c dx = c(b - a)$.
		\item Es sei $Z = \{ x_{0}, \dotsc, x_{n} \}$ eine Zerlegung von $[0, 1]$ und $f:[0,1] \to \R$ definiert durch
			$$ f(x) \coloneqq \begin{cases} 1, & x \in [0, 1] \cap \Q \\ 0, & x \in [0, 1] \setminus \Q \end{cases}. $$  
			Hier gilt: $m_{j} = \inf f(I_{j}) = 0$, $ M_{j} = \sup f(I_{j}) = 1$ $(j=1,\dots,n)$, also $s_{f}(Z) = 0$, $S_{f}(Z) = 1$.
			Somit ist $s_{f} = 0 \neq 1 = S_{f}$ und damit $f \notin R([0, 1])$.
	\end{enumerate}	
\end{beispiele}

\begin{satz} \label{10.2:satz}
	Es seien $f, g \in R([a, b])$. Dann gilt:
	\begin{enumerate}
		\item Ist $f \leq g$ auf $[a, b]$, so ist $\int_{a}^{b} f dx \leq \int_{a}^{b} g dx$.
		\item Für $\alpha, \beta \in \R$ ist $\alpha f + \beta g \in R([a, b])$ und
			$$ \int_{a}^{b} (\alpha f + \beta g) dx = \alpha \int_{a}^{b} f dx + \beta \int_{a}^{b} g dx. $$
	\end{enumerate}
\end{satz}

\begin{proof}
	Nur a) ( b) Übung): Es sei $Z = \{ x_{0}, \dotsc x_{n} \} \in \mathcal{Z}$, $I_{j}$ und $m_{j}$ wie immer. 
	        Es sei $\widetilde{m}_{j} \coloneqq \inf g(I_{j})$ $(j = 1, \dotsc, n)$. Wegen $f \leq g$ auf $I_{j}$ gilt:
		$$ m_{j} \leq \widetilde{m}_{j} ~ (j = 1, \dotsc, n) ~ \text{und damit} ~ s_{f}(Z) \leq s_{g}(Z) \leq s_{g}. $$ 
		Da $Z\in \mathcal{Z}$ beliebig war folgt 
		$$ \int_{a}^{b} f dx = s_{f} \leq s_g= \int_{a}^{b} g dx. $$
\end{proof}

\index{Integrabilitätskriterium!Riemannsches}
\begin{satz}[Riemannsches Integrabilitätskriterium] \label{10.3:prop} ~\\
Es gilt: $$f \in R([a, b]) \iff \forall \varepsilon > 0 ~\exists Z = Z(\varepsilon) \in \mathcal{Z}: ~ S_{f}(Z) - s_{f}(Z) < \varepsilon. $$	
\end{satz}

Ohne Beweis.

\begin{satz} \label{10.4:satz}
	Ist $f:[a,b] \to \R$ monoton, so ist $f \in R([a, b])$.
\end{satz}

\begin{proof}
	O.B.d.A. sei $f$ monoton wachsend. Es sei $\varepsilon > 0$. Wähle $n \in \N$ so, da{\ss} 
	$$ \frac{b - a}{n} (f(b) - f(a)) < \varepsilon. $$
	Für $j = 0, \dotsc, n$ sei $x_{j} \coloneqq a + j \frac{b - a}{n}$. Damit ist $Z \coloneqq \{ x_{0}, \dotsc, x_{n} \} \in \mathcal{Z}$. Es seien $I_{j}, m_{j}$ und $M_{j}$ 
	wie immer. Es gilt: 
	$$
	|I_{j}| = \frac{b - a}{n}, ~ m_{j} = f(x_{j-1}), ~ M_{j} = f(x_{j}) \quad (j=1,\dotsc,n).
	$$
	Also:  
	\begin{align*}
		S_{f}(Z) - s_{f}(Z) & = \sum_{j=1}^{n} (M_{j} - m_{j})|I_{j}| \\
		& = \frac{b - a}{n} \sum_{j=1}^{n} ( f(x_{j}) - f(x_{j-1}) ) \\
		& = \frac{b - a}{n} ( f(b) - f(a) ) < \varepsilon.
	\end{align*} 
	Mit \ref{10.3:prop} folgt die Behauptung.
\end{proof}


\begin{satz} \label{10.5:satz}
	Es gilt: $C([a, b]) \subseteq R([a, b])$.
\end{satz}

\begin{proof}
	Es sei $f \in C([a, b])$ und $\varepsilon > 0$. Mit \ref{7.16:satz} folgt: 
	$$
	(\ast) \quad \exists \delta > 0~ \forall t, s \in [a, b]: ~ |t - s| < \delta ~ \Rightarrow  ~ |f(t) - f(s)| < \frac{\varepsilon}{b - a}.
	$$
	Es sei $Z = \{ x_{0}, \dotsc, x_{n} \} \in \mathcal{Z}$ so gewählt, da{\ss} $|I_{j}| < \delta$ $(j = 1, \dotsc, n)$, und $I_{j}, M_{j}, m_{j}$ seien wie immer.
	Betrachte $I_{j}$: Nach \ref{7.11:satz} gilt:
	$$ \exists \xi, \eta \in I_{j}: ~ f(\xi) = m_{j}, ~ f(\eta) = M_{j}. $$
	Wegen $|I_{j}| < \delta$ ist $|\xi - \eta| < \delta$ und mit $(\ast)$ folgt:
	$$ M_{j} - m_{j} = f(\eta) - f(\xi) = |f(\eta) - f(\xi)| < \frac{\varepsilon}{b - a}. $$
	Damit ist 
	$$S_{f}(Z) - s_{f}(Z) = \sum_{j=1}^{n} (M_{j} - m_{j}) |I_{j}| < \frac{\varepsilon}{b - a} \sum_{j=1}^{n} |I_{j}| = \varepsilon.$$
	Mit \ref{10.3:prop} folgt die Behauptung.
\end{proof}

\index{Stammfunktion}
\begin{definition}
	Es sei $I \subseteq \R$ ein Intervall und $G, g \colon I \rightarrow \R$ Funktionen. Die Funktion $G$ hei{\ss}t eine \textbf{Stammfunktion} von $g$ auf $I$ 
	$:\iff$ $G$ ist auf $I$ differenzierbar und $G' = g$ auf $I$.
\end{definition}


Beachte: Sind $G$ und $H$ Stammfunktionen von $g$ auf $I$, so ist $G' = g = H'$ auf $I$ und nach \ref{9.10:satz} gilt
$$\exists c \in \R ~ \forall x \in I: ~ G(x) = H(x) + c.$$

\index{Hauptsätze der Diff.- und Integralrechnung!1. Hauptsatz} 
\begin{satz}[Erster Hauptsatz der Differential- und Integralrechnung]\label{10.6:prop-1Hauptsatz} ~\\
	Ist $f \in R([a, b])$ und besitzt $f$ auf $[a, b]$ eine Stammfunktion $F$, so ist 
	$$ \int_{a}^{b} f(x) dx = F(b) - F(a).$$
\end{satz}

\begin{proof}
	Es sei $Z = \{ x_{0}, \dotsc, x_{n} \} \in \mathcal{Z}$, und $I_{j}, m_{j}, M_{j}$ seien wie immer. Für jedes $j\in\{1,\dotsc, n\}$ gilt:
	$$ F(x_{j}) - F(x_{j-1}) \overset{MWS}{=} F'(\xi_{j}) (x_{j} - x_{j-1}) = f(\xi_{j}) \underbrace{(x_{j} - x_{j-1})}_{ = |I_{j}|}, $$
	mit $\xi_{j} \in (x_{j-1}, x_{j})$. Wegen $m_{j} \leq f(\xi_{j}) \leq M_{j}$ gilt $m_{j} |I_{j}| \leq f(\xi_{j}) |I_{j}| \leq M_{j} |I_{j}|$. 
	Summation über $j$ liefert
	$$
	s_{f}(Z)  \leq \sum_{j=1}^{n} f(\xi_{j}) |I_{j}|  = \sum_{j=1}^{n} \left( F(x_{j}) - F(x_{j-1}) \right) = F(b) - F(a) \leq S_{f}(Z).
	$$
	Also gilt: 
	$$ \forall Z \in \mathcal{Z}: ~ s_{f}(Z) \leq F(b) - F(a) \leq S_{f}(Z).$$ 
	Wegen $f \in R([a, b])$ folgt: $$\int_{a}^{b} f dx = s_{f} \leq F(b) - F(a) \leq S_{f} = \int_{a}^{b} f dx.$$
\end{proof}

In Rechnungen ist folgende Schreibweise nützlich:
$$F(x) \Big|_{a}^{b} := \left[ F(x) \right]_{a}^{b}:= F(b)-F(a). $$

\begin{beispiele} ~\
	\begin{enumerate}
		\item Es sei $0 < a < b$, $f(x) = \frac{1}{x}$ $(x \in [a,b])$. Es gilt $f \in C([a, b]) \xRightarrow[]{\ref{10.5:satz}} f \in R([a, b])$, und
		$F(x) \coloneqq \log x$ ist eine Stammfunktion von $f$ auf $[a, b]$. Mit \ref{10.6:prop-1Hauptsatz} folgt:
			$$ \int_{a}^{b} \frac{1}{x} dx = \log x \Big|_{a}^{b} = \log b - \log a =\log \frac{b}{a}. $$
		\item  Es gilt: $$\int_{0}^{\frac{\pi}{2}} \cos x dx = \sin x \Big|_{0}^{\frac{\pi}{2}}= \sin \frac{\pi}{2} - \sin 0 = 1.$$
	\end{enumerate}
\end{beispiele}

\begin{bemerkung} ~\
\begin{enumerate}
	\item Es gibt integrierbare Funktionen, die keine Stammfunktion besitzen!
	\item Es gibt nicht integrierbare Funktionen, die Stammfunktionen besitzen!
\end{enumerate}
\end{bemerkung}

\begin{beispiele} ~\
	\begin{enumerate}
		\item Betrachte $$f(x) = \begin{cases} 1, & x \in (0, 1] \\ 0, & x = 0 \end{cases}.$$ 
			$f$ ist monoton $\xRightarrow[]{\ref{10.4:satz}} f \in R([0, 1])$. \\
			Annahme: $f$ besitzt auf $[0, 1]$ eine Stammfunktion $F$. Dann gilt:
			$$ F'(x) = f(x) ~ (x \in [0, 1]), \text{ also } F'(x) = 1  ~ (x \in (0, 1]). $$
			Mit \ref{9.10:satz} folgt: $\exists c \in \R: F(x) = x + c$ $(x \in (0, 1])$. Weiter gilt: \\
			$F$ ist differenzierbar in $0$ $\Rightarrow$ $F$ ist stetig in $0$ $\Rightarrow$ $F(0) = c$. Also ist $F(x) = x + c$ $(x \in [0, 1])$. 
			Es folgt
			$$ 0 = f(0) = F'(0) = \lim_{x \rightarrow 0} \frac{F(x) - F(0)}{x - 0} = \lim_{x \rightarrow 0} \frac{x + c - c}{x} = 1, $$
			ein Widerspruch.
		\item Betrachte 
		        $$F(x) \coloneqq \begin{cases} x^{\frac{3}{2}} \sin \frac{1}{x}, & x \in (0, 1] \\ 0, & x = 0 \end{cases}.$$
		        Nach \ref{9.18:bsp} ist $F$ ist auf $[0, 1]$ differenzierbar. Setze $f \coloneqq F'$. Dann ist $F$ eine Stammfunktion von $f$ auf $[0, 1]$.
		        Nach \ref{9.18:bsp} ist $f$ ist auf $[0, 1]$ nicht beschränkt, also $f \notin R([a, b])$. 
	\end{enumerate}	
\end{beispiele}



\begin{satz} \label{10.7:satz}
	Es sei $c \in (a, b)$. Dann gilt:
	$$  f \in R([a, b]) \iff f \in R([a, c]) \text{ und } f \in R([c, b]). $$	
	In diesem Fall gilt: $$ \int_{a}^{b} f dx = \int_{a}^{c} f dx + \int_{c}^{b} f dx.$$ 
\end{satz}

Ohne Beweis.

\textbf{Motivation}: Für $n \geq 2$ sei 
	$$f_{n} \colon [0, 1] \rightarrow \R,  \quad f_{n}(x) = \begin{cases} n^{2} x, & x \in [0, \frac{1}{n}), \\ n - (x - \frac{1}{n}) n^{2}, & x \in [\frac{1}{n}, \frac{2}{n}), \\ 
	0, & x \in [\frac{2}{n}, 1]. \end{cases} $$


\begin{figure*}[!ht] \centering
	\begin{tikzpicture}
		\begin{axis}
			\addplot[domain=0:0.2] {x*5*5};
			\addplot[domain=0.2:0.4] {10-x*5*5};
			\addplot[domain=0.4:1] {0*x};
		\end{axis}
	\end{tikzpicture}	
	\caption{$f_{n}$ für $n = 5$.}	
\end{figure*}

Es gilt: $$f_{n} \in C([0, 1]) \xRightarrow[]{\ref{10.5:satz}} f_{n} \in R([0, 1]) \xRightarrow[\ref{10.7:satz}]{\ref{10.6:prop-1Hauptsatz}} \int_{0}^{1} f_{n} dx = 1 ~(n \geq 2).$$
Übung: $(f_{n})$ konvergiert auf $[0, 1]$ punktweise gegen $f = 0$. Also gilt: 
	$$ \lim_{n \rightarrow 0} \int_{0}^{1} f_{n}(x) dx = 1 \neq 0 = \int_{0}^{1} f(x) dx = \int_{0}^{1} \left( \lim_{n \rightarrow \infty} f_{n}(x) \right) dx. $$

\begin{satz} \label{10.8:satz}
	Es sei $(f_{n})$ eine Folge in $R([a, b])$ und $(f_{n})$ konvergiere auf $[a, b]$ gleichmä{\ss}ig gegen $f \colon [a, b] \rightarrow \R$. Dann gilt: $f \in R([a, b])$ und
	$$ \lim_{n \rightarrow \infty} \int_{a}^{b} f_{n}(x) dx = \int_{a}^{b} f(x) dx. $$
\end{satz}

Ohne Beweis.

\begin{satz} \label{10.9:satz}
	Es sei $\sum_{n=0}^{\infty} a_{n} (x - x_{0})^{n}$ eine Potenzreihe mit Konvergenzradius $r > 0$, $I \coloneqq (x_{0} - r, x_{0} + r)$ ($I \coloneqq \R$, falls $r = \infty$) 
	und
	$$ g(x) \coloneqq \sum_{n=0}^{\infty} a_{n} (x - x_{0})^{n} \quad (x \in I) $$
	Dann hat die Potenzreihe $\sum_{n=0}^{\infty} \frac{a_{n}}{n + 1} (x - x_{0})^{n+1}$ den Konvergenzradius $r$ und für
	$$
	(\ast) \quad G(x) \coloneqq \sum_{n=0}^{\infty} \frac{a_{n}}{n + 1} (x - x_{0})^{n+1} \quad (x \in I). 
	$$
	gilt $G' = g$ auf $I$.
\end{satz}

\begin{proof}
	Es sei $\tilde{r}$ der Konvergenzradius der Potenzreihe in $(\ast)$. Nach \ref{9.12:satz} gilt $r = \tilde{r}$ und $G' = g$ auf $I$.	
\end{proof}


\begin{satz} \label{10.10:satz}
	Es seien $f, g \in R([a, b])$. Dann gilt:
	\begin{enumerate}
		\item Es sei $D \coloneqq f([a, b])$ und mit einem $L \geq 0$ gelte für $h \colon D \rightarrow \R$:
			$$ |h(s) - h(t)| \leq L |s - t| \quad (t, s \in D). $$
			Dann ist $h \circ f \in R([a, b])$.
		\item $|f| \in R([a, b])$ und $|\int_{a}^{b} f(x) dx| \leq \int_{a}^{b} |f(x)| dx$ ($\triangle$-Ungleichung für Integrale).
		\item $fg \in R([a, b])$.
		\item Ist $g(x) \neq 0$ $(x \in [a, b])$ und $\frac{1}{g}$ auf $[a, b]$ beschränkt, so ist $\frac{1}{g} \in R([a, b])$.
	\end{enumerate}
\end{satz}

\begin{proof} ~\
	\begin{enumerate}
		\item, c) und d) ohne Beweis.
		\item Es sei $D \coloneqq f([a, b])$ und $h(t) \coloneqq |t|$ $(t \in D)$. Dann ist $|f| = h \circ f$. Für $t, s \in D$ gilt: 
			$$ |h(t) - h(s)| = \left| |t| - |s| \right| \le |t - s|. $$
			Aus a) folgt $|f| \in R([a, b])$. Weiter ist $\pm f \leq |f|$ auf $[a, b]$. Mit \ref{10.2:satz} folgt 
			$$ \pm \int_{a}^{b} f dx \leq \int_{a}^{b} |f| dx, \text { also }  |\int_{a}^{b} f(x) dx| \leq \int_{a}^{b} |f(x)| dx. $$
			
	\end{enumerate}
\end{proof}


\begin{definition}
	Es sei $f \in R([a, b])$ und $\alpha, \beta \in [a, b]$. Wir setzen
	$$\int_{\alpha}^{\alpha} f(x) dx := 0.$$ 
	Ist $\alpha < \beta$, so ist nach \ref{10.7:satz} $f \in R([\alpha, \beta])$ und wir setzen
	$$ \int_{\beta}^{\alpha} f(x) dx \coloneqq - \int_{\alpha}^{\beta} f(x) dx. $$
\end{definition}

\index{Hauptsätze der Diff.- und Integralrechnung!2. Hauptsatz}
\begin{satz}[Zweiter Hauptsatz der Differential- und Integralrechnung]\label{10.11:2Hauptsatz} ~\\
	Es sei $f \in R([a, b])$ und
	$$ F(x) \coloneqq \int_{a}^{x} f(t) dt \quad \left( x \in [a, b] \right). $$
	Dann gilt:
	\begin{enumerate}
		\item $F(y) - F(x) = \int_{x}^{y} f(t) dt$ $(x, y \in [a, b])$.
		\item $F$ ist Lipschitz-stetig.
		\item Ist $f \in C([a, b])$, so ist $F \in C^{1}([a, b])$ und $F'(x) = f(x)$ $(x \in[a, b])$.
	\end{enumerate}
\end{satz}

\begin{proof} ~\
	\begin{enumerate}
		\item Es seien $x, y \in [a, b]$. \\
		Fall 1: Für $x \leq y$ gilt
			\begin{align*}
				F(y) - F(x) & ~ = \int_{a}^{y} f(t) dt - \int_{a}^{x} f(t) dt \\
				& \overset{\ref{10.7:satz}}{=} \int_{a}^{x} f(t) dt + \int_{x}^{y} f(t) - \int_{a}^{x} f(t) dt \\
				& ~ = \int_{x}^{y} f(t) dt
			\end{align*}
			Fall 2: Für $x > y$ gilt
			$$ F(y) - F(x) = - (F(x) - F(y)) \overset{Fall 1}{=} - \int_{y}^{x} f(t) dt = \int_{x}^{y} f(t) dt. $$
		\item Setze $L \coloneqq \sup \{ |f(t)| : t \in [a, b] \}$. Es seien $x, y \in [a, b]$ und o.B.d.A. sei $x \leq y$. Dann gilt:
			\begin{align*}
				|F(y) - F(x)| & \overset{a)}{=} | \int_{x}^{y} f(t) dt | \overset{\ref{10.10:satz}}{\leq} \int_{x}^{y} |f(t)| dt 
				\overset{\ref{10.2:satz}}{\leq} \int_{x}^{y} L dt \\
				& = L (y - x) = L |y - x|.
			\end{align*}
		\item Wir zeigen für $x_{0} \in [a, b)$:
			$$ \lim_{h \rightarrow 0+} \frac{F(x_{0} + h) - F(x_{0})}{h} = f(x_{0}), $$
			(analog zeigt man für $x_{0} \in (a, b]$: $\lim_{h \rightarrow 0 - } \frac{F(x_{0} + h) - F(x_{0})}{h} = f(x_{0})$). \\
			Sei also $x_{0} \in [a, b)$, $h > 0$ und $x_{0} + h \in [a, b]$. Es ist
			$$ \frac{1}{h} \int_{x_{0}}^{x_{0}+h} f(x_{0}) dt = f(x_{0}) $$
			und
			$$ \frac{F(x_{0} + h) - F(x_{0})}{h} \overset{a)}{=} \frac{1}{h} \int_{x_{0}}^{x_{0}+h} f(t) dt. $$
			Weiter gilt: 
			\begin{align*} 
				D(h) & \coloneqq \left| \frac{F(x_{0} + h) - F(x_{0})}{h} - f(x_{0}) \right| \\
					 & = \frac{1}{h} \left| \int_{x_{0}}^{x_{0}+h} ( f(t) - f(x_{0}) ) dt \right| \\
					 & \overset{\ref{10.10:satz}}{\leq} \frac{1}{h} \int_{x_{0}}^{x_{0}+h} | f(t) - f(x_{0})| dt. 
			\end{align*} 
			Die Funktion $t \mapsto |f(t) - f(x_{0})|$ ist stetig auf $[a,b]$. Nach \ref{7.11:satz} gilt daher:
			$$\exists \xi_{h} \in [x_{0}, x_{0} + h] ~\forall t \in [x_{0}, x_{0} + h]: ~  |f(t) - f(x_{0})| \leq | f(\xi_{h}) - f(x_{0})|.$$
			Also gilt: 
			$$ D(h) \leq \frac{1}{h} \int_{x_{0}}^{x_{0}+ h} |f(\xi_{h}) - f(x_{0})| dt = |f(\xi_{h}) - f(x_{0})|. $$ 
			Für $h \rightarrow 0+$ gilt $\xi_{h} \rightarrow x_{0}$. Da $f$ stetig ist folgt $f(\xi_{h}) \rightarrow f(x_{0})$ $(h \to 0+)$, 
			also $D(h) \rightarrow 0$ $(h \rightarrow 0+)$. 
	\end{enumerate}
\end{proof}


Aus \ref{10.10:satz} folgt (Übung):

\begin{folgerung} \label{10.12:folg}
	Es sei $I \subseteq \R$ ein Intervall, $g \in C(I)$ und $x_0 \in I$ (fest). Definiere $G \colon I \rightarrow \R$ durch
	$$ G(x) = \int_{x_0}^{x} f(t) dt. $$
	Dann gilt: $G \in C^{1}(I)$ und $G' = g$ auf $I$.
\end{folgerung}

\index{Integral!unbestimmtes}
\begin{definition}
	Es sei $I \subseteq \R$ ein Intervall. Besitzt $g \colon I \rightarrow \R$ auf $I$ eine Stammfunktion, so schreibt man für eine solche auch 
	$$\int g dx \text{  oder  } \int g(x) dx $$
	und nennt dies ein \textbf{unbestimmtes Integral} von $g$.
\end{definition}

\begin{beispiel} 
	$$\int \cos x dx = \sin x,  \quad  \int \cos x dx = \sin x  + 17.$$
\end{beispiel}

\index{Partielle Integration}
\begin{satz}[Partielle Integration] ~\\
	Es sei $I \subseteq \R$ ein Intervall und $f, g \in C^{1}(I)$. Dann gilt:
	\begin{enumerate}
		\item $\int f' g dx = f g - \int f g' dx$ auf $I$.
		\item Ist $I = [a, b]$, so ist $\int_{a}^{b} f' g dx = f g \Big|_{a}^{b} - \int_{a}^{b} f g' dx$.
	\end{enumerate}
\end{satz}

\begin{proof}
	Es gilt $(f g)' = f' g + f g' \Rightarrow f' g = (fg)' - fg'$ und damit a), sowie
		$$ \int_{a}^{b} f' g dx = \int_{a}^{b} (fg)' dx - \int_{a}^{b} fg' dx \overset{\ref{10.6:prop-1Hauptsatz}}{=} fg \Big|_{a}^{b} - \int_{a}^{b} fg' dx. $$ 
\end{proof}


\begin{beispiele} ~\
	\begin{enumerate}
		\item $$\int \sin^{2} x dx = \int \underbrace{\sin x}_{ f' } \underbrace{\sin x}_{ g } dx = - \cos x \sin x - \int - \cos^{2} x dx$$
			$$ = - \cos x \sin x + \int \cos^{2} x dx  = - \cos x \sin x + \int (1 - \sin^{2} x) dx$$
			$$= x - \cos x \sin x - \int \sin^{2} x dx $$
			$$\Rightarrow \int \sin^{2} x dx = \frac{1}{2} (x - \cos x \sin x).$$
		\item Ungeeignete Anwendung der partiellen Integration:
		$$\int \underbrace{x}_{f'} \underbrace{e^{x}}_{g} dx = \frac{1}{2} x^{2} e^{x} - \int \frac{1}{2} x^{2} e^{x} dx.$$ 
		Besser:
		$$\int \underbrace{x}_{g} \underbrace{e^{x}}_{f'} = x e^{x} - \int e^{x} dx = x e^{x}- e^{x}.$$
		\item $$\int \log x dx = \int \underbrace{1}_{f'} \underbrace{\log x}_{g} dx = x \log x - \int x \frac{1}{x} dx = x \log x - x.$$
	\end{enumerate}	
\end{beispiele}


\textbf{Bezeichnung}: Es seien $\alpha, \beta \in \R$ und $\alpha \neq \beta$. Wir setzen
	$$ \langle \alpha, \beta \rangle \coloneqq \begin{cases} [\alpha, \beta], & \text{ falls } \alpha < \beta \\ [\beta, \alpha], & \text{ falls } \alpha > \beta \end{cases} $$

\index{Substitutionsregeln}
\begin{satz}[Substitutionsregeln] \label{10.14:prop-Substitutionsregeln} ~\\
	Es seien $I$ und $J$ Intervalle in $\R$, es sei $f \in C(I)$, $g \in C^{1}(J)$ und $g(J) \subseteq I$. 
	\begin{enumerate}
		\item Es gilt $$\int f(g(t)) g'(t) dt = \int f(x) dx \Big|_{x = g(t)} ~ \text{ auf } J.$$
		\item Es sei $g'(t) \neq 0$ $(t \in J)$ ($\Rightarrow g' > 0$ auf $J$ oder $g' < 0$ auf $J$ $\Rightarrow$ $g$ ist streng monoton). Dann gilt:
			$$ \int f(x) dx = \int f(g(t)) g'(t) dt \Big|_{t=g^{-1}(x)} \text{ auf } I. $$
		\item Ist $I = \langle a, b \rangle$, $J = \langle \alpha, \beta \rangle$, $g(\alpha) = a$ und $g(\beta) = b$, so gilt
			$$ \int_{a}^{b} f(x) dx = \int_{\alpha}^{\beta} f(g(t)) g'(t) dt. $$
	\end{enumerate}
\end{satz}

\begin{proof}
	Nach \ref{10.2:satz} hat $f$ auf $I$ eine Stammfunktion $F$. Setze $G(t) \coloneqq F(g(t))$ $(t \in J)$. Es gilt (Kettenregel): $G \in C^{1}(J)$ und
	$$ G'(t) = F'(g(t)) g'(t) = f(g(t)) g'(t) \quad (t \in J)$$
	und damit
	\begin{enumerate}
		\item $$\int f(g(t)) g'(t) dt = \int G'(t) dt = G(t) = F(g(t)) = \int f(x) dx \Big|_{x=g(t)}.$$
		\item $$\int f(g(t)) g'(t) dt\Big|_{t = g^{-1}(x)} = G(g^{-1}(x)) = F(g(g^{-1}(x))) = F(x)=\int f(x) dx .$$
		\item $$\int_{\alpha}^{\beta} f(g(t)) g'(t) dt \overset{\ref{10.6:prop-1Hauptsatz}}{=} G(\beta) - G(\alpha) = F(g(\beta)) - F(g(\alpha))$$ 
		$$ = F(b) - F(a) \overset{\ref{10.6:prop-1Hauptsatz}}{=} \int_{a}^{b} f(x) dx. $$ 
	\end{enumerate}
\end{proof}


\textbf{Merkregel}:  Ist $y = y(x)$ eine differenzierbare Funktion, so schreibt man für $y'$ auch $\frac{dy}{dx}$.

Zu \ref{10.14:prop-Substitutionsregeln}: Substituiere $x = g(t)$, fasse also $x$ als Funktion von $t$ auf. Dann: $\frac{dx}{dt} = g'(t)$, also
	$$ \text{"'} dx = g'(t) dt \text{"'}. $$


\begin{beispiele} ~\
	\begin{enumerate}
		\item $$\int_{0}^{1} \frac{e^{2x} + 1}{e^{x}} dx ~ \begin{cases} x = \log t, e^{x} = t \\ \frac{dx}{dt} = \frac{1}{t}, dx = \frac{1}{t} dt \\ 
		x = 0 \Rightarrow t = 1, x = 1 \Rightarrow t = e \end{cases} $$
			\begin{align*}
				& = \int_{1}^{e} \frac{t^{2} + 1}{t} \cdot \frac{1}{t} dt = \int_{1}^{e} \frac{t^{2} + 1}{t^{2}} = \int_{1}^{e} (1 + \frac{1}{t^{2}}) dt \\
				& = \left[ t - \frac{1}{t} \right]_{1}^{e} = e - \frac{1}{e} - (1 - 1) = e - \frac{
				1}{e}.
			\end{align*}
		\item $$\int_{0}^{1} \sqrt{1 - x^{2}} dx ~ \begin{cases} x = \sin t, t \in [0, \frac{\pi}{2}] \\ \frac{dx}{dt} = \cos t, dx = \cos t dt \end{cases}$$
			\begin{align*}
				& = \int_{0}^{\frac{\pi}{2}} \sqrt{1 - \sin^{2}t} \cos t dt = \int_{0}^{\frac{\pi}{2}} \sqrt{\cos^{2} t} \cos t dt \\
				& = \int_{0}^{\frac{\pi}{2}} |\cos t| \cos t dt = \int_{0}^{\frac{\pi}{2}} \cos^{2} t dt = \int_{0}^{\frac{\pi}{2}} (1 - \sin^{2} t) dt \\
				& \overset{s.o.}{=} \left[ t - \frac{1}{2} (t - \cos t \sin t) \right]_{0}^{\frac{\pi}{2}} = \frac{\pi}{4}.
			\end{align*}
	\end{enumerate}
\end{beispiele}



\begin{satz} \label{10.15:satz}
	Es seien $f, g \colon [a, b] \rightarrow \R$ beschränkt. Dann gilt:
	\begin{enumerate}
		\item Ist $\{ x \in [a, b]: f$ ist in $x$ nicht stetig $\}$ endlich, so ist $f \in R([a, b])$.
		\item Ist $f \in R([a, b])$ und $\{ x \in [a, b] : f(x) \neq g(x) \}$ endlich, so ist $g \in R([a, b])$ und 
			$$ \int_{a}^{b} f(x) dx = \int_{a}^{b} g(x) dx. $$
	\end{enumerate}	
\end{satz}

Ohne Beweis.

\begin{satz} \label{10.16:satz-Mittelwertsätze}
	Es seien $f, g \in R([a, b])$, $g \geq 0$ auf $[a, b]$, $m \coloneqq \inf f([a, b])$ und $M \coloneqq \sup f([a, b])$. Dann gilt:
	\begin{enumerate}
		\item $\exists \mu \in [m, M]$: $\int_{a}^{b} fg dx = \mu \int_{a}^{b} g dx$.
		\item $\exists \mu \in [m, M]$: $\int_{a}^{b} f dx = \mu (b - a)$.
	\end{enumerate}
	Ist $f \in C([a, b])$, so existiert ein $\xi \in [a, b]$ mit $\mu = f(\xi)$ in a) bzw. b).
\end{satz}

\begin{proof} ~\
	\begin{enumerate}
		\item Aus $g \geq 0$ auf $[a, b]$ folgt $mg \leq fg \leq Mg$ auf $[a, b]$. Mit \ref{10.2:satz} folgt
			$$ m \underbrace{\int_{a}^{b} g dx}_{\eqqcolon A} \leq \underbrace{\int_{a}^{b} fg dx}_{\eqqcolon B} \leq M \int_{a}^{b} g dx, $$
			also $m A \leq B \leq M A$. Beachte: $A \ge 0$. \\
			Fall 1: $A = 0$. Dann ist $B = 0$ und jedes $\mu \in [m, M]$ leistet das Verlangte. \\
			Fall 2: $A > 0$. Es gilt: $m \leq \frac{B}{A} \leq M$. Nun leistet $\mu =\frac{B}{A}$ das Verlangte.
		\item Mit $g=1$ auf $[a,b]$ folgt die Behauptung aus a). 
	\end{enumerate}
	Der Zusatz folgt aus \ref{7.7:prop-Zwischenwertsatz} und \ref{7.11:satz}.
\end{proof}


\begin{satz} \label{10.17:satz}
	Es sei $(f_{n})$ eine Folge mit:
	\begin{enumerate}[label=\roman*\upshape)]
		\item $f_{n} \in C^{1}([a, b])$ $(n \in \N)$,
		\item $(f_{n}(a))$ ist konvergent,
		\item $(f_{n}')$ konvergiert auf $[a, b]$ gleichmä{\ss}ig gegen $g \colon [a, b] \rightarrow \R$.
	\end{enumerate}
	Dann konvergiert $(f_{n})$ auf $[a, b]$ gleichmä{\ss}ig und für $$f(x) \coloneqq \lim_{x \rightarrow \infty} f_{n}(x) ~(x \in [a, b])$$ gilt:
		$$ f \in C^{1}([a, b]) \text{ und } f'(x) = g(x) ~ (x \in  [a, b]). $$
\end{satz}

\begin{bemerkung}
Satz \ref{10.17:satz} enthält wieder eine Aussage über das Vertauschen von Grenzwerten:
$$\lim_{n \rightarrow \infty} f_{n}'(x) = g(x) = f'(x) = (\lim_{n \rightarrow \infty} f_{n}(x))' \quad (x \in [a,b]).$$ 
\end{bemerkung}

\begin{proof}
	Wir setzen $\alpha_{n} \coloneqq \int_{a}^{b} |f_{n}'(t)- g(t)| dt$ $(n \in \N)$. Nach iii) folgt: $(|f_{n}' - g|)$ konvergiert auf $[a, b]$ gleichmä{\ss}ig gegen 
	$0$. Damit folgt mit \ref{10.8:satz}: $\alpha_{n} \rightarrow 0$ $(n \to \infty)$. Wir setzen $c:= \lim_{n \to \infty} f_{n}(a)$. Für jedes $x \in [a, b]$ gilt:
	$$ 
	f_{n}(x) \overset{\ref{10.6:prop-1Hauptsatz}}{=} \underbrace{f_{n}(a)}_{\rightarrow c} + \int_{a}^{x} f_{n}'(t) dt 
	\xrightarrow[]{\ref{10.8:satz}} c + \int_{a}^{x} g(t) dt \eqqcolon f(x) \quad (n \to \infty). 
	$$
	Also: $(f_{n})$ konvergiert auf $[a, b]$ punktweise gegen $f$. Mit \ref{8.3.a:satz} folgt $g \in C([a, b])$, und nach \ref{10.11:2Hauptsatz} ist daher
	$f \in C^{1}([a, b])$ und $f' = g$ auf $[a,b]$. \\
	Weiter gilt:
	\begin{align*}
		|f_{n}(x) - f(x)| & = |f_{n}(x) - f_{n}(a) + f_{n}(a) - c - \int_{a}^{x} g(t) dt  | \\
		& \overset{\ref{10.6:prop-1Hauptsatz}}{=} |\int_{a}^{x} (f_{n}'(t) - g(t)) dt + f_{n}(a) - c | \\
		& \leq \int_{a}^{x} |f_{n}'(t) - g(t)| dt + |f_{n}(a)-c| \\
		& \leq \int_{a}^{b} |f'_{n}(t) - g(t)| dt + |f_{n}(a)-c| \\
		& = \underbrace{\alpha_{n} + |f_{n}(a)-c|}_{\rightarrow 0} \quad (x \in [a, b]).
	\end{align*}
	Mit \ref{8.1:satz} folgt: $(f_{n})$ konvergiert auf $[a, b]$ gleichmä{\ss}ig gegen $f$.
\end{proof}


\begin{bemerkung}
	Der Beweis von \ref{9.12:satz} b) kann mit \ref{8.2:satz} und \ref{10.17:satz} geführt werden.	
\end{bemerkung}


\newpage


\chapter{Uneigentliche Integrale}


\begin{vereinbarung}
	Ist $I \subseteq \R$ ein Intervall und $f \colon I \rightarrow \R$ eine Funktion, so soll stets gelten: $f \in R(J)$ für jedes kompakte Intervall $J \subseteq I$.	
\end{vereinbarung}

\index{uneigentliche Integral} \index{konvergent}
\begin{definition} ~\
	\begin{enumerate}
		\item Es sei $a \in \R$, $\beta \in \R \cup \{ \infty \}$, $a < \beta$ und $f \colon [a, \beta) \rightarrow \R$ eine Funktion. \\
		        Das \textbf{uneigentliche Integral} $\int_{a}^{\beta} f(x) dx$ hei{\ss}t \textbf{konvergent}  $: \iff$ 
			$$  \text{Es existiert } \lim_{t \rightarrow \beta-} \int_{a}^{t} f(x) dx  \in \R.$$
			In diesem Fall:
			$$ \int_{a}^{\beta} f(x) dx \coloneqq \lim_{t \rightarrow \beta- } \int_{a}^{t} f(x) dx. $$
		\item Es sei $b \in \R$, $\alpha \in \R \cup \{ -\infty \}$, $\alpha < b$ und $f \colon (\alpha, b] \rightarrow \R$ eine Funktion. \\
		        Das \textbf{uneigentliche Integral} $\int_{\alpha}^{b} f(x) dx$ hei{\ss}t \textbf{konvergent} $:\iff$ 
			$$ \text{Es existiert } \lim_{t \rightarrow \alpha + } \int_{t}^{b} f(x) dx \in \R. $$
			In diesem Fall:
			$$ \int_{\alpha}^{b} f(x) dx \coloneqq \lim_{t \rightarrow \alpha +} \int_{t}^{b} f(x) dx. $$
	\end{enumerate}
	Ein nicht konvergentes uneigentliches Integral hei{\ss}t divergent.
	
\end{definition}


\begin{beispiele} ~\
	\begin{enumerate}
		\item $\int_{1}^{\infty} \frac{1}{x^{\gamma}} dx, ~(\gamma >0)$ ($a = 1, \beta = \infty$). Für $t > 1$ gilt: \label{11.0:bsp-obigeBspa}
			$$ \int_{1}^{t} \frac{1}{x^{\gamma}} dx = \begin{cases} \log t, & \text{ falls } \gamma = 1 \\ 
			\frac{1}{1 - \gamma} (t^{1 - \gamma} - 1), & \text{ falls } \gamma \neq 1 \end{cases} $$
			Also gilt: $\int_{1}^{\infty} \frac{1}{x^{\gamma}} dx$ konvergiert $\iff$ $\gamma > 1$. In diesem Fall ist
			$$ \int_{1}^{\infty} \frac{1}{x^{\gamma}} dx = \frac{1}{\gamma - 1}. $$
		\item $\int_{0}^{\infty} \frac{1}{1 + x^{2}} dx$ ($a = 0, \beta = \infty$). Für $t > 0$ gilt:
			$$ \int_{0}^{t} \frac{1}{1 + x^{2}} dx = \arctan t \rightarrow \frac{\pi}{2} ~(t \rightarrow \infty). $$ \label{11.0:bsp-obigeBspb}
			Also ist $\int_{0}^{\infty} \frac{1}{1 + x^{2}} dx$ konvergent und $= \frac{\pi}{2}$.
		\item $\int_{0}^{1} \frac{1}{x^{\gamma}} dx, ~\gamma > 0$ ($\alpha = 0, b = 1$). Wie in Beispiel a) sieht man: \label{11.0:bsp-obigeBspc}
			$$ \int_{0}^{1} \frac{1}{x^{\gamma}} dx \text{ konvergiert} \iff \gamma < 1 $$.
			In diesem Fall ist \\
				$$ \int_{0}^{1} \frac{1}{x^{\gamma}} dx= \frac{1}{1-\gamma}. $$
		\item $\int_{-\infty}^{0} \frac{1}{1 + x^{2}} dx$ ($\alpha = -\infty, b = 0$). Wie in Beispiel b) sieht man: \label{11.0:bsp-obigeBspd}
			$$ \int_{-\infty}^{0} \frac{1}{1 + x^{2}} dx  \text{ konvergiert und } = \frac{\pi}{2}. $$
		\item $\int_{0}^{\infty} \sin x dx$  ($a = 0, \beta = \infty$). Es sei $t_{n} = n \pi$ $(n \in \N)$. Es gilt: $t_n \to \infty$ und \label{11.0:bsp-obigeBspe}
			$$ \int_{0}^{t_{n}} \sin x dx = -\cos \Big|_{0}^{t_{n}} = 1 - \cos t_{n} = 1 - \cos(n\pi) = 1 - (-1)^n \quad (n \in \N).$$
			Also ist $\int_{0}^{\infty} \sin x dx$ divergent.
	\end{enumerate}	
\end{beispiele}

\index{uneigentliche Integral!konvergiert}
\begin{definition}
	Es sei $\alpha < \beta$, $\alpha \in \R \cup \{ - \infty \}$, $\beta \in \R \cup \{ \infty \}$ und $f \colon (\alpha, \beta) \rightarrow \R$ eine Funktion. \\
	Das \textbf{uneigentliche Integral} $\int_{\alpha}^{\beta} f(x) dx$ hei{\ss}t \textbf{konvergent} $:\iff$ 
	$$\exists c \in (\alpha, \beta): ~ \int_{\alpha}^{c} f(x) dx \text{ und } \int_{c}^{\beta} f(x) dx \text{ sind konvergent }.$$ In diesem Fall:
	$$ \int_{\alpha}^{\beta} f(x) dx \coloneqq \int_{\alpha}^{c} f(x) dx + \int_{c}^{\beta} f(x) dx. $$
	Im anderen Fall hei{\ss}t das Integral divergent.
\end{definition}

\textbf{Übung}: Obige Definition ist unabhängig von $c \in (\alpha, \beta)$.

\begin{beispiele} ~\
	\begin{enumerate}
		\item $\int_{-\infty}^{\infty} x dx$ ist divergent, denn $\int_{0}^{\infty} x dx$ ist divergent. 
		\item Es sei $\gamma > 0$. Obige Beispiele a) und c) zeigen:
			$$ \int_{0}^{\infty} \frac{1}{x^{\gamma}} dx \text{ ist divergent.} $$
		\item Obige Beispiele b) und d) zeigen:
			$$ \int_{-\infty}^{\infty} \frac{1}{1 + x^{2}} dx \text{ ist konvergent und} = \pi. $$
	\end{enumerate}
\end{beispiele}

Die folgenden Definitionen und Sätze formulieren wir nur für uneigentliche Integrale der Form
	$$ \int_{a}^{\beta} f(x) dx. $$
Sie gelten sinngemä{\ss} auch für die beiden anderen Typen uneigentlicher Integrale.

\begin{bemerkung}
Für $t \in [a, \beta)$ sei $g(t) \coloneqq \int_{a}^{t} f(x) dx$. Dann gilt:
	$$ \int_{a}^{\beta} f(x) dx \text{ konvergiert} \iff \lim_{t \rightarrow \beta-} g(t) \text{ existiert und ist in }\R. $$
D.h. die Konvergenz eines uneigentlichen Integrals ist gleichbedeutend mit der Existenz eines Funktionenlimes.
\end{bemerkung}
	

\index{Cauchykriterium}
\begin{satz}[Cauchykriterium] \label{11.1:prop-Cauchykriterium}
	Es gilt: 
	$$\int_{a}^{\beta} f(x) dx \text{ konvergiert} \iff 
	\forall \varepsilon > 0 ~\exists c \in (a, \beta) ~ \forall u, v \in (c, \beta): ~ \left| \int_{u}^{v} f(x) dx \right| < \varepsilon.  $$
\end{satz}

\begin{proof}
Folgt aus \ref{6.2.c:satz}.
\end{proof}


\begin{beispiel*}
	Behauptung: $\int_{1}^{\infty} \frac{\sin x}{x} dx$ konvergiert.	
\end{beispiel*}

\begin{proof}
	Für $1 < u < v$ gilt:
	\begin{align*}
		        |\int_{u}^{v} \frac{\sin x}{x} dx | & = | \int_{u}^{v} \underbrace{\frac{1}{x}}_{g} \underbrace{\sin x}_{f'} dx | \\
			& = | \left[ -\frac{\cos x}{x} \right]_{v}^{u} - \int_{u}^{v} - \frac{1}{x^{2}} (-\cos x) dx | \\
			& = | \frac{\cos v}{v} - \frac{\cos u}{u} - \int_{u}^{v} \frac{\cos x}{x^{2}} dx | \\
			& \leq \frac{1}{v} + \frac{1}{u} + \int_{u}^{v} \frac{1}{x^{2}} dx = \frac{2}{u}
	\end{align*}
	Es sei $\varepsilon > 0$ und o.B.d.A $\varepsilon < 2$. Setze $c \coloneqq \frac{2}{\varepsilon}$. Für $\frac{2}{\varepsilon}=c < u < v$ gilt nun:
	$$ | \int_{u}^{v} \frac{\sin x}{x} dx | \leq \frac{2}{u} < \varepsilon. $$
	Mit \ref{11.1:prop-Cauchykriterium} folgt die Behauptung.
\end{proof}

\index{konvergent!absolut}
\begin{definition} ~\
	$$\int_{a}^{\beta} f(x) dx \text{ hei{\ss}t } \textbf{absolut konvergent} :\iff \int_{a}^{\beta} |f(x)| dx \text{ ist konvergent}.$$
\end{definition}

\begin{beispiel*}
$\int_{1}^{\infty} \frac{\sin x}{x} dx$ ist nicht absolut konvergent (Übung).
\end{beispiel*}


Den folgenden Satz beweist man mit \ref{11.1:prop-Cauchykriterium} ähnlich wie bei Reihen:

\index{Majorantenkriterium} \index{Minorantenkriterium}
\begin{satz} ~\ \label{11.2:satz}
	\begin{enumerate}
		\item Ist $\int_{a}^{\beta} f(x) dx$ absolut konvergent, so ist $\int_{a}^{\beta} f(x) dx$ konvergent und
			$$ | \int_{a}^{\beta} f(x) dx | \leq \int_{a}^{\beta} |f(x)| dx. $$
		\item \textbf{Majorantenkriterium}: Ist $|f| \leq h$ auf $[a, \beta)$ und $\int_{a}^{\beta} h(x) dx$ konvergent, so ist $\int_{a}^{\beta} f(x) dx$ konvergent.
		\item \textbf{Minorantenkriterium}: Ist $f \geq h \geq 0$ auf $[a, \beta)$ und $\int_{a}^{\beta} h(x) dx$ divergent, so ist $\int_{a}^{\beta} f(x) dx$ divergent.
	\end{enumerate}	
\end{satz}


\begin{beispiele} ~\
	\begin{enumerate}
		\item $\int_{1}^{\infty} \underbrace{\frac{x}{\sqrt{1 + x^{5}}}}_{\eqqcolon f(x)} dx$.  Für $x\ge 1$ gilt: 
		      $|f(x)| = f(x) \leq \frac{x}{\sqrt{x^{5}}} = \frac{1}{x^{\frac{3}{2}}} \eqqcolon g(x)$. 
		      $$ \int_{1}^{\infty} g(x) dx \text{ konvergiert } \Rightarrow \int_{1}^{\infty} f(x) dx \text{ konvergiert}. $$
		\item $\int_{1}^{\infty} \underbrace{\frac{x}{x^{2} + 7x}}_{\eqqcolon f(x)} dx$. Es sei $g(x) \coloneqq \frac{1}{x}$. Es gilt: 
		      $$\frac{f(x)}{g(x)} = \frac{x^{2}}{x^{2} + 7x} \rightarrow 1 ~(x \rightarrow \infty)$$
			$$ \Rightarrow \exists c \geq 1~ \forall  x \geq c: ~ \frac{f(x)}{g(x)} \geq \frac{1}{2} 
			~ \Rightarrow ~ \forall x \geq c: ~ f(x) \geq \frac{1}{2} g(x). $$
			Weiter gilt:
			$$
			\int_{c}^{\infty} \frac{1}{2}g(x) dx \text{ divergiert} \Rightarrow \int_{c}^{\infty} f(x) dx \text{ divergiert}
			\Rightarrow \int_{1}^{\infty} f(x) dx \text{ divergiert}.
			$$
	\end{enumerate}
\end{beispiele}

\newpage

\chapter{Die komplexe Exponentialfunktion}

Erinnerung (lineare Algebra): Die Menge $\C$ der komplexen Zahlen ist ein Körper. Alle aus den Körperaxiomen hergeleiteten Formeln 
gelten daher auch in $\C$. 

\begin{beispiele} ~\
\begin{enumerate}
 \item Die Binomische Formel gilt in $\C$.
 \item Die geometrische Summenformel gilt in $\C$:
 $$
 \sum_{k=0}^{n} z^{k} = \frac{1 - z^{k+1}}{1 - z} \quad (z \in \C, ~ z \neq 1).
 $$
\end{enumerate}
\end{beispiele}



\index{Betrag!einer komplexen Zahl}
Es sei $z = x + iy \in \C ~(x, y \in \R)$. 
\begin{description}  \addtolength{\itemindent}{0.4cm}
	\item $|z| \coloneqq \sqrt{x^{2} + y^{2}}$ hei{\ss}t \textbf{Betrag} von $z$.
	\item $\overline{z} \coloneqq x - iy$ hei{\ss}t \textbf{komplex Konjugierte} von $z$.
	\item $z \cdot \overline{z} = |z|^{2} ~(z \in \C)$.
	\item $|z \cdot w| = |z| \cdot |w| ~(z, w \in \C)$.
	\item $|z+ w| \le |z|+|w| ~(z, w \in \C)$.
\end{description}



\begin{definition}
Die auf $\C$ definierte Funktion $$ z=x+iy \mapsto  e^{z} \coloneqq e^{x} (\cos y + i \sin y) $$ hei{\ss}t \textbf{komplexe Exponentialfunktion}.
\end{definition}

Ist $z = x \in \R$, so ist $e^{z} = e^{x}$; ist $z = it$ $(t \in \R)$, so ist $e^{it} = \cos t + i \sin t$.


\begin{satz} \label{12.1:satz}
	Es gilt: 
	\begin{enumerate}
		\item $\forall z, w \in \C: ~ e^{z +w} = e^{z} e^{w}$; $\forall z \in \C ~ \forall n \in \Z: ~ e^{nz} = (e^{z})^n$. \label{12.1.a:satz}
		\item $\forall t \in \R: ~ |e^{it}| = 1,\quad e^{-it} = \overline{e^{it}}$. \label{12.1.b:satz}
		\item $e^{i \pi} + 1 = 0$. \label{12.1.c:satz}
		\item $\forall k \in \Z ~ \forall z \in \C: ~ e^{z + 2 k \pi i} = e^{z}$. \label{12.1.d:satz}
		\item $\forall t \in \R: ~\cos t = \frac{1}{2} \left( e^{it} + e^{-it} \right), \quad \sin t = \frac{1}{2i} \left( e^{it} - e^{-it} \right)$. \label{12.1.e:satz}
	\end{enumerate}
\end{satz}

\begin{proof} ~\
	\begin{enumerate}
		\item Übung (mit den Additionstheoremen von $E$, $\sin$, $\cos$).
		\item $$e^{it} = \cos t + i \sin t \Rightarrow |e^{it}| = (\cos^{2} t + \sin^{2} t)^{\frac{1}{2}} = 1,$$
		      $$ e^{-it} = \cos (-t) + i \sin (-t) = \cos t - i \sin t = \overline{\cos t + i \sin t} = \overline{e^{it}}. $$
		\item $e^{i\pi} = \cos \pi + i \sin \pi = -1$.
		\item $e^{z+2 k \pi i} = e^z e^{2 k \pi i}= e^z(\cos (2 k \pi) + i \sin (2k \pi)) = e^z.$ 
		\item $e^{it} + e^{-it} = 2 \cos t$, $e^{it} - e^{-it} = 2i \sin t$.
	\end{enumerate}
\end{proof}


\begin{definition}
	Für $z \in \C$ sei
	$$ \cos z \coloneqq \frac{1}{2} \left( e^{iz} + e^{-iz} \right), \quad \sin z \coloneqq \frac{1}{2 i} \left( e^{iz} - e^{-iz} \right). $$
\end{definition}


\begin{uebung}
	Für alle $z, w \in \C$ gilt:
	\begin{align*}
		\sin (z + w) & = \sin z \cos w + \sin w \cos z, \\
		\cos (z + w) & = \cos z \cos w - \sin z \sin w.
	\end{align*}	
\end{uebung}


\begin{satz} \label{12.2:satz}
	Es sei $z = x + i y \in \C ~(x, y \in \R)$. Dann gilt:
	$$ e^{z} = 1 \iff \exists k \in \Z: ~ z = 2 k \pi i. $$
\end{satz}


	
\begin{proof}
	"'$\Leftarrow$"': Folgt aus $\ref{12.1.d:satz}$. \\
	"'$\Rightarrow$"': Es sei $e^{z} = 1$, also 
	  $$1 = e^{x} ( \cos y + i \sin y) = e^{x} \cos y + i e^{x} \sin y$$
	  $$\Rightarrow e^{x} \cos y = 1, e^{x} \sin y = 0 \Rightarrow \sin y = 0 \Rightarrow \exists j \in \Z: y = j \pi.$$
	  Also ist $\cos y = (-1)^{j}$, somit $1 = e^{x} (-1)^{j}$ und daher $j = 2k$ für ein $k \in \Z$ und $x = 0$. 
	  Also gilt $z = 2k \pi i$.
\end{proof}

Aus \ref{12.2:satz} folgt:
	\begin{align*}
		e^{z} = e^{w} & \iff e^{z} e^{-w} = e^{w} e^{-w} \\
			& \iff e^{z-w} = e^{w - w} = e^{0} =1 \\
			& \overset{\ref{12.2:satz}}{\iff} \exists k \in \Z: z = w + 2 k \pi i
	\end{align*}


\index{Polarkoordinaten} \index{Argument}
\textbf{Polarkoordinaten}: Es sei $z = x + iy \in \C$ $(x, y \in \R)$ und $z \neq 0$. Wir setzen
	$$ r \coloneqq |z| = (x^{2} + y^{2})^{\frac{1}{2}} $$
	
	Die Gerade durch $0$ und $z$ schließt mit der positiven $x$-Achse einen Winkel $\varphi \in (-\pi, \pi]$ ein.
	Die Zahl $\varphi$ hei{\ss}t das \textbf{Argument von $z$}; $\arg z:= \varphi$. Es gilt
		$$ \cos \varphi = \frac{x}{r}, \quad \sin \varphi = \frac{y}{r}, $$
	also
		$$  z = x + iy = r \cos \varphi + i r \sin \varphi = r e^{i \varphi} =  |z| e^{i \arg z}. $$
	Ist weiter $w \in \C\setminus\{0\}$ und $\psi \coloneqq \arg w$, so gilt:
		$$ z w = |z| e^{i \varphi} |w| e^{i \psi} = |z| |w| e^{i(\varphi + \psi)} $$
	
	


\index{Fundamentalsatz der Algebra} 
\begin{satz}[Fundamentalsatz der Algebra] \label{12.3:prop-FundamentalsatzDerAlgebra} ~\\
	Es sei $p(z) = a_{0} + a_{1} z + \dotsc + a_{n} z^{n}$ ein Polynom mit $n \geq 1$, $a_{0}, \dotsc, a_{n} \in \C$ und $a_{n} \neq 0$. 
	Dann existieren eindeutig bestimmte Zahlen $z_{1}, \dotsc, z_{n} \in \C$ mit 
	$$p(z) = a_{n} (z - z_{1}) \cdot \dotsc \cdot (z - z_{n}) \quad (z \in \C).$$
	Insbesondere gilt: 
	$$ p(z)=0 \iff z \in \{z_{1}, \dotsc, z_{n}\}.$$
\end{satz}

Ohne Beweis.

\index{Wurzeln!komplexe}
\begin{definition}
	Es sei $a \in \C$ und $n \in \N$. Jedes $z \in \C$ mit $z^{n} = a$ hei{\ss}t eine \textbf{$n$-te Wurzel aus} $a$.
	$$ \sqrt[n]{a} \text{  bezeichnet eine $n$-te Wurzel aus } a.$$
\end{definition}


\begin{satz} \label{12.4:satz}
	Es sei $a \in \C \setminus \{ 0 \}$, $n \in \N$, $r \coloneqq |a|$ und $\varphi \coloneqq \arg a$ (also $a = |a| e^{i \varphi} = r e^{i \varphi}$). 
	Für $k = 0, 1, \dotsc, n - 1$ sei
		$$ z_{k} \coloneqq \sqrt[n]{r} e^{i \frac{\varphi + 2 k \pi}{n}} $$
	Dann gilt:
	\begin{enumerate}
		\item $z_{j} \neq z_{k}$ für $j \neq k$.
		\item $z$ ist eine $n$-te Wurzel aus $a \iff z \in \{ z_{0}, z_{1}, \dotsc, z_{n-1} \}$.
	\end{enumerate}
\end{satz}

\begin{proof} ~\
	\begin{enumerate}
		\item Es seien $j, k \in \{0, \dotsc, n - 1 \}$, $k \geq j$. Ist 
		        $$z_k=e^{i \frac{\varphi + 2k \pi}{n}} = e^{i \frac{\varphi + 2 j \pi}{n}}=z_j, $$ so existiert ein $l \in \Z$ mit:
			$$ 
			i \frac{\varphi + 2k \pi}{n} = i \frac{\varphi + 2 j \pi}{n} + 2 l \pi i \Rightarrow \frac{\varphi}{2 \pi} + k = \frac{\varphi}{2 \pi} + j + l n 
			\Rightarrow \frac{k - j}{n} = l.
			$$
			Somit ist
			$$ 0 \le  l= \frac{k - j}{n}  \leq \frac{k}{n} \leq \frac{n - 1}{n} = 1 - \frac{1}{n} < 1. $$
			Wegen $l \in \Z$ folgt damit $l = 0$, also $k = j$.
		\item   Es sei $p(z) \coloneqq z^{n} - a$. Dann gilt: $z$ ist eine $n$-te Wurzel aus $a$ $\iff$ $p(a) = 0$. Weiter gilt
			$$ z_{k}^{n} = r e^{i(\varphi +2 k \pi)} = r e^{i \varphi} e^{2 k \pi i} = r e^{i \varphi} = a \quad (k = 0, \dotsc, n - 1), $$
			also $p(z_{k}) = 0$ $(k = 0, \dotsc, n - 1)$. Aus a) und \ref{12.3:prop-FundamentalsatzDerAlgebra} folgt die Behauptung.
	\end{enumerate}
\end{proof}

\index{Einheitswurzeln!n-te}
\begin{bezeichnung} ~\\
	Ist $a = 1$, so hei{\ss}en die Zahlen $z_{0}, \dotsc, z_{n-1}$ aus \ref{12.4:satz} die \textbf{$n$-ten Einheitswurzeln}.	Diese sind also 
	$$z_{k} = e^{\frac{2 k \pi i}{n}} \quad (k = 0, \dotsc, n-1).$$
\end{bezeichnung}


\begin{bemerkung}
	Insbesondere gilt:
	$$
	z^{n} - 1 = \prod_{k=0}^{n-1} (z -e^{\frac{2 k \pi i}{n}}) \quad (z \in \C).
	$$	
\end{bemerkung}


\begin{beispiele} ~\
	\begin{enumerate}
	        \item Die 4. Einheitswurzel sind $1, -1, i, -i$.
	        \item Die 4. Wurzeln aus $16$ sind $2, -2, 2i, -2i$.
		\item Im Reellen ist $\sqrt{4} = 2$. Im Komplexen sind $2$ und $-2$ die Wurzeln aus $4$.
	\end{enumerate}	
\end{beispiele}


\begin{beispiel*}
        $\sqrt{-3 + 4i}=?$ Man kann Wurzeln auf verschiedene Weisen berechnen:
	\begin{description}
		\item 1. Möglichkeit: $w = u + iv$ $(u,v \in \R).$ Dann gilt: 
			$$w^{2} = u^{2} - v^{2} + 2iuv = -3 + 4i \iff u^{2} - v^{2} = -3, ~ 2uv = 4. $$
			Löse das Gleichungssystem.
		\item 2. Möglichkeit: $z = -3 + 4i$. Bestimme $|z|$ und $\varphi = \arg z$. Dann sind
			$$ \pm \sqrt{|z|} e^{i \frac{\arg z}{2}} \text{ die Wurzeln von } z. $$
		\item 3. Möglichkeit: Ist $z \in (-\infty, 0]$, so sind $w = \pm \sqrt{-z}$ die Wurzeln von $z$. \\
		        Behauptung: Ist $z \in \C \setminus (-\infty, 0]$, so sind
			$$ w = \pm \sqrt{|z|} \frac{z + |z|}{\left| z + |z| \right|} $$
			die Wurzeln von $z$.
			\begin{proof}
				Es gilt:
				\begin{align*}
					 \left( \pm \sqrt{|z|} \frac{z + |z|}{\left| z + |z| \right|} \right)^{2} & = |z| \frac{(z + |z|) (z + |z|)}{(z + |z|) (\overline{z} + |z|)} 
					 =  |z| \frac{(z + |z|)}{(\overline{z} + |z|)} \\
					 & = \frac{(|z|z + z \overline{z})}{(\overline{z} + |z|)} = z \frac{(|z| + \overline{z})}{(\overline{z} + |z|)} = z. 
				\end{align*}
			\end{proof}
			Also gilt: 
			$$\sqrt{-3 + 4i} = \pm \sqrt{5} \frac{-3 + 4i + 5}{|-3 + 4i + 5|} = \pm \sqrt{5} \frac{2 + 4i}{\sqrt{20}} = \pm (1 + 2i).$$
	\end{description}
\end{beispiel*}


\begin{satz} \label{12.5:satz}
	Es seien $p, q \in \C$. Für $z \in \C$ gilt:
	$$ z^{2} + pz + q = 0 \iff z = -\frac{p}{2} \pm \underbrace{\sqrt{\frac{p^{2}}{4} - q}}_{\text{doppeldeutig!}}. $$	
\end{satz}

\begin{proof}
	"'$\Leftarrow$"' nachrechnen. Rest mit \ref{12.3:prop-FundamentalsatzDerAlgebra}.
\end{proof}


\begin{beispiel}
	Löse $(*)$ $z^{2} + (1 - 2i)z - 2i = 0$. 
	\begin{align*}
		z & = \frac{2i - 1}{2} \pm \sqrt{ \frac{(2i - 1)^{2}}{4} + 2i} = i - \frac{1}{2} \pm \sqrt{\frac{-4 - 4i + 1}{4} + 2i} \\
		  & = i - \frac{1}{2} \pm \frac{1}{2}\sqrt{ -3 - 4i + 8i}  = i - \frac{1}{2} \pm \frac{1}{2} \sqrt{-3 + 4i}.
	\end{align*}
	Also sind 
	$$
		z_{1}  = i - \frac{1}{2} + \frac{1}{2} (1 + 2i) = 2i \text{ und }
		z_{2}  = i - \frac{1}{2} + \frac{1}{2} (-1 - 2i) = -1 
	$$
	die Lösungen von $(*)$. Es gilt $$z^{2} + (1 - 2i) z - 2i = ( z - z_{1})(z - z_{2}) = (z - 2i) (z + 1).$$
\end{beispiel}

\index{Logarithmus}
\begin{definition}
	Es sei $w \in \C \setminus \{ 0 \}$. Jedes $z \in \C$ mit $e^{z} = w$ hei{\ss}t ein \textbf{Logarithmus von $w$}.
\end{definition}


\begin{satz} \label{12.6:satz}
	Es sei $w \in \C \setminus \{ 0 \}$, $ r = |w|$ und $\varphi = \arg w$, also $w = r e^{i \varphi}$. Für $z \in \C$ gilt: 
	$$ z \text{ ist ein Logarithmus von } w \iff \exists k \in \Z: z = \underbrace{\log |w|}_{\log \text{ in } \R} + i \varphi + 2 k \pi i. $$	
\end{satz}

\begin{proof}
	"'$\Leftarrow$"': Es gilt $$e^{z} = e^{\log |w|} e^{i \varphi} e^{2 k \pi i} = |w| e^{i \varphi} = w.$$
	"'$\Rightarrow$"': Es sei $z = x + iy$ $(x, y \in \R)$ und $w = e^{z} = e^{x} e^{iy}$. Dann gilt $|w| = e^{x}$ $\Rightarrow$ $x = \log |w|$. Weiter ist
	$$ |w| e^{i \varphi} = w = e^{z} = e^{x} e^{iy} = |w| e^{iy} $$
	$$ \Rightarrow e^{i \varphi} = e^{iy} ~ \Rightarrow ~ \exists k \in \Z: ~ i y = i \varphi + 2k \pi i.$$
	Also gilt $z = \log |w| + i \varphi + 2k \pi i$.
\end{proof}


\begin{beispiele} ~\
	\begin{enumerate}
		\item $w = -1$; $|w| = 1$, $\arg w = \pi$. Alle Logarithmen von $-1$:
			$$ i \pi + 2 k \pi i \quad (k \in \Z). $$
		\item $w = 1$; $|w| = 1$, $\arg w = 0$. Alle Logarithmen von $1$:
			$$ 2 k \pi i \quad (k \in \Z). $$
		\item $w = 1 + i$; $|w| = \sqrt{2}$, $\arg w = \frac{\pi}{4}$. Alle Logarithmen von $1 + i$:
			$$ \log \sqrt{2} + i \frac{\pi}{4} + 2k \pi i \quad (k \in \Z). $$			
	\end{enumerate}
\end{beispiele}


\newpage

\chapter{Fourierreihen}

Für eine Funktion $f \colon \R \rightarrow \R$ betrachten wir die Eigenschaft 
	$$ (V) ~ \begin{cases}
				f \in R([-\pi, \pi]) \text{ und $f$ ist auf $\R$ $2\pi$-periodisch,} \\
				\text{d.h. } f(x + 2 \pi) = f(x) ~ (x \in \R).
			\end{cases} $$

\index{trigonometrische Reihe}
\begin{definition}
	Es seien $(a_{n})_{n=0}^{\infty}$ und $(b_{n})_{n=1}^{\infty}$ Folgen in $\R$. Eine Reihe der Form
	$$ \frac{a_{0}}{2} + \sum_{n=1}^{\infty} \left( a_{n} \cos(nx) + b_{n} \sin(nx) \right) $$
	hei{\ss}t eine \textbf{trigonometrische Reihe} (TR).
\end{definition}

\textbf{Fragen}: Wann ist $f$ mit der Eigenschaft $(V)$ durch eine trigonometrische Reihe darstellbar? Wie hängt dann $f$ mit $(a_{n})$ und $(b_{n})$ zusammen?

\index{Orthogonalitätsrelationen}
\begin{satz} \label{13.1:satz} Es gilt:
	\begin{enumerate}
		\item Die Funktion $f$ erfülle $(V)$. Dann gilt für jedes $a \in \R:$ 
		      $$f \in R([a, a + 2\pi]) \text{ und } \int_{a}^{a+ 2\pi} f(x) dx = \int_{-\pi}^{\pi} f(x) dx. $$
		\item \textbf{Orthogonalitätsrelationen}: Für alle $k, n \in \N$ gilt:
			\begin{align*}
				\int_{-\pi}^{\pi} \sin(nx) \cos(kx) dx & = 0
				\intertext{und}
				\int_{-\pi}^{\pi} \sin(nx) \sin(kx) dx & = \int_{-\pi}^{\pi} \cos(nx) \cos(kx) dx = \begin{cases} \pi, & k = n \\ 0, & k \neq n \end{cases}.				
			\end{align*}
	\end{enumerate}	
\end{satz}

\begin{proof}
a) Übung. \\
b) Die Funktion $x \mapsto \sin(nx) \cos(kx)$ ist ungerade. Damit folgt 
$$\int_{-\pi}^{\pi} \sin(nx) \cos(kx) dx  = 0.$$
Rest: Übung.
\end{proof}


\textbf{Motivation}: Es seien $(a_{n})_{n=0}^{\infty}$ und $(b_{n})_{n=1}^{\infty}$ Folgen und es gelte
	$$ f(x) = \frac{a_{0}}{2} + \sum_{n=1}^{\infty} \left( a_{n} \cos(nx) + b_{n} \sin(nx) \right) \quad (x \in \R), $$
	wobei diese trigonometrisch Reihe auf $\R$ gleichmä{\ss}ig konvergent sei. \\
	Für jedes $k \in \N$ gilt dann:
	$$ f(x) \sin(k x) = \frac{a_{0}}{2} \sin(kx) + \sum_{n=1}^{\infty} \left( a_{n} \cos(nx) \sin(kx) + b_{n} \sin(nx) \sin(kx) \right) \quad (x \in \R).$$
	Übung: Die letzte Reihe konvergiert auf $\R$ ebenfalls gleichmä{\ss}ig. Mit \ref{10.8:satz} folgt daher:
	\begin{align*}
		\int_{-\pi}^{\pi} f(x) \sin(kx) dx & = \frac{a_{0}}{2} \underbrace{\int_{-\pi}^{\pi} \sin(kx) dx}_{= 0} + 
		\sum_{n = 1}^{\infty} a_{n} \underbrace{\int_{-\pi}^{\pi} \cos(nx)\sin(kx) dx}_{\overset{\ref{13.1:satz}}{=} 0} \\
		& ~\qquad + \sum_{n = 1}^{\infty} b_{n} \underbrace{\int_{-\pi}^{\pi} \sin(nx) \sin(kx) dx}_{\overset{\ref{13.1:satz}}{=} 
		\begin{cases} \pi, & \text{ falls } k = n \\ 0, & \text{ falls } k \neq n \end{cases}} \\
		& = b_{k} \pi.
	\end{align*}
	Also gilt:
	$$ \forall k \in \N: ~ b_{k} = \frac{1}{\pi} \int_{-\pi}^{\pi} f(x) \sin(kx) dx. $$
	Analog zeigt man:
	$$ \forall k \in \N_{0}: ~ a_{k} = \frac{1}{\pi} \int_{-\pi}^{\pi} f(x) \cos(kx) dx. $$
	
	

	
\index{Fourierkoeffizienten} \index{Fourierreihe}
\begin{definition}
	Die Funktion $f$ erfülle $(V)$. Setze
	\begin{align*}
		a_{n} & \coloneqq \frac{1}{\pi} \int_{-\pi}^{\pi} f(x) \cos(nx) dx \quad (n \in \N_{0}),
		\intertext{und}
		b_{n} & \coloneqq \frac{1}{\pi} \int_{-\pi}^{\pi} f(x) \sin(nx) dx \quad (n \in \N).
	\end{align*} 
	Die Zahlen $a_{n}$, $b_{n}$ hei{\ss}en die \textbf{Fourierkoeffizienten} (FK) von $f$ und die mit $a_{n}$ und $b_{n}$ gebildete trigonometrische Reihe hei{\ss}t 
	die zu $f$ gehörenden \textbf{Fourierreihe}. Man schreibt:
	$$ f(x) \sim \frac{a_{0}}{2} + \sum_{n=1}^{\infty} \left( a_{n} \cos(nx) + b_{n} \sin(nx) \right). $$
\end{definition}

\textbf{Frage}: Für welche $x \in \R$ konvergiert die zu $f$ gehörige Fourierreihe, und wogegen?

\begin{satz} \label{13.2:satz}
	Für $f$ gelte $(V)$.
	\begin{enumerate}
		\item Ist $f$ gerade, also $f(x) = f(-x)$ $(x \in \R)$, so gilt für die Fourierkoeffizienten von $f$:
			$$ a_{n} = \frac{2}{\pi} \int_{0}^{\pi} f(x) \cos(nx) dx ~ (n \in \N_0) \text{ und } b_{n} = 0 ~ (n \in \N). $$
		\item Ist $f$ ungerade, also $f(x) = -f(-x)$ $(x \in \R)$, so gilt für die Fourierkoeffizienten von $f$:
			$$ a_{n} = 0 ~ (n \in \N_0) \text{ und } b_{n} = \frac{2}{\pi} \int_{0}^{\pi} f(x) \sin(nx) dx ~ (n \in \N).  $$
	\end{enumerate}	
\end{satz}

\begin{proof}
	Übung.
\end{proof}

\index{Grenzwert!rechtsseitiger} \index{Grenzwert!linksseitiger}
\begin{definition} ~\
Es sei $D \subseteq \R$, $x_{0}$ ein Häufungspunkt von $D$ und $g \colon D \rightarrow \R$ eine Funktion. Wir setzen
			$$ g(x_{0}\pm) \coloneqq \lim_{x \rightarrow x_{0} \pm } g(x), \text{ falls dieser Grenzwert vorhanden und $\in \R$ ist}. $$	
\end{definition}


\begin{definition}
	Es sei $f:\R \to \R$ $2\pi$-periodisch. Die Funktion $f$ hei{\ss}t \textbf{stückweise glatt} $:\iff$ es existiert eine Zerlegung $\{ t_{0}, t_{1}, \dotsc, t_{n} \}$ 
	von $[-\pi, \pi]$ (also $-\pi = t_{0} < t_{1} < \dotsc < t_{n-1} < t_{n} = \pi$) mit:
	\begin{enumerate}[label=\roman*\upshape)]
		\item $f \in C^{1} \left( (t_{j-1}, t_{j}) \right) ~(j = 1, \dotsc, n)$.
		\item Für jedes $x \in \R$ existieren die Grenzwerte:
			$$ f(x-), ~ f'(x-), ~ f(x+), ~ f'(x+) $$ 
			In diesem Fall setzen wir
			$$s_f(x):= \frac{f(x+)+f(x-)}{2} \quad (x \in \R). $$
	\end{enumerate}	
\end{definition}


\textbf{Beachte}: 
\begin{enumerate}
        \item In den Punkten $t_{j}$ muss $f$ nicht stetig sein.
	\item $f$ hat die Eigenschaft $(V)$, vgl. \ref{10.15:satz} a).   
\end{enumerate}



\begin{satz} \label{13.3:satz}
	Die Funktion $f$ sei $2\pi$-periodisch und stückweise glatt. Dann konvergiert die Fourierreihe von $f$ in jedem $x \in \R$ gegen $s_{f}(x)$. 
	Ist in diesem Fall $f$ in $x \in \R$ stetig, so konvergiert die Fourierreihe von $f$ also gegen $f(x)$.	
\end{satz}

Ohne Beweis.

\begin{beispiel} \label{13.4:bsp}
	$f \colon \R \rightarrow \R$ sei $2\pi$-periodisch und auf $(-\pi, \pi]$ definiert durch
	$$ f(x) = \begin{cases} x, & x \in (-\pi, \pi) \\ 0, & x = \pi \end{cases}. $$	
	Es gilt: $f$ ist stückweise glatt und $s_{f}(x) = f(x)$ $(x \in \R)$. Weiter ist $f$ ist ungerade. Nach \ref{13.2:satz} ist also $a_{n} = 0$ $(n \in \N_{0})$ und
	$$ b_{n} = \frac{2}{\pi} \int_{0}^{\pi} f(x) \sin(nx) dx \overset{\ref{10.15:satz}}{=} 
	\frac{2}{\pi} \int_{0}^{\pi} x \sin(nx) dx \overset{\text{Übung}}{=} (-1)^{n+1} \frac{2}{n} \quad (n \in \N). $$
	Mit \ref{13.3:satz} folgt nun:
	$$ f(x) = 2 \sum_{n=1}^{\infty} \frac{(-1)^{n+1}}{n} \sin(nx) \quad (x \in \R).$$
	$$ \Rightarrow \frac{x}{2} = \sum_{n=1}^{\infty} \frac{(-1)^{n+1}}{n} \sin(nx) \quad (x \in (-\pi, \pi)). $$
	Mit $x = \frac{\pi}{2}$ folgt: 
	$$\frac{\pi}{4} = 1 - \frac{1}{3} + \frac{1}{5} - \frac{1}{7} +- \dotsc,$$
	vgl. \ref{9.17.b:anwendungen}.
\end{beispiel}


\begin{beispiel} \label{13.5:bsp}
	$f \colon \R \rightarrow \R$ sei $2\pi$-periodisch und auf $[-\pi, \pi]$ definiert durch $f(x) = x^{2}$. \\
	Es gilt: $f$ ist stückweise glatt, $f$ ist gerade und $f(x) = s_{f}(x)$ $(x \in \R)$. Nach \ref{13.2:satz} ist also $b_{n} = 0$ $(n \in \N)$ und
	$$ a_{n} = \frac{2}{\pi} \int_{0}^{\pi} x^{2} \cos(nx) dx = \begin{cases} \frac{2 \pi^{2}}{3}, & n = 0 \\ 4 \frac{ (-1)^{n}}{n^{2}}, & n \ge 1 \end{cases}.  $$
	Mit \ref{13.3:satz} folgt:
	$$ f(x) = \frac{\pi^{2}}{3} - 4 \left( \frac{\cos x}{1^{2}} - \frac{\cos(2x)}{2^{2}} + \frac{\cos(3x)}{3^{2}} -+ \dotsc \right) \quad (x \in \R) $$
	$$\Rightarrow x^{2} = \frac{\pi^{2}}{3} - 4 \left( \frac{\cos x}{1^{2}} - \frac{\cos(2x)}{2^{2}} + \frac{\cos(3x)}{3^{2}} -+ \dotsc \right) ~ (x \in [-\pi, \pi]).$$
	Hieraus erhalten wir:
	\begin{align}
		& x = 0: \quad \frac{\pi^{2}}{12} = 1 - \frac{1}{2^{2}} + \frac{1}{3^{2}} - \frac{1}{4^{2}} +- \dotsc = \sum_{n=1}^{\infty} \frac{(-1)^{n+1}}{n^{2}} \tag{1} \\
		& x = \pi: \quad \frac{\pi^{2}}{6} = 1 + \frac{1}{2^{2}} + \frac{1}{3^{2}} + \frac{1}{4^{2}} + \dotsc = \sum_{n=1}^{\infty} \frac{1}{n^{2}} \tag{2}
	\end{align}
	Addition von $(1)$, $(2)$ liefert: 
	$$\frac{\pi^{2}}{8} = 1 + \frac{1}{3^{2}} + \frac{1}{5^{2}} + \dotsc = \sum_{n=0}^{\infty} \frac{1}{(2n +1)^{n}}.$$
\end{beispiel}


\begin{satz} \label{13.6:satz}
	Es sei $f \in C(\R)$ und $f$ sei $2\pi$-periodisch und stückweise glatt. Dann gilt:
	\begin{enumerate}
		\item Die Fourierreihe von $f$ konvergiert in jedem $x \in \R$ absolut.
		\item Die Fourierreihe von $f$ konvergiert auf $\R$ gleichmä{\ss}ig gegen $f$.
		\item Sind $a_{n}, b_{n}$ die Fourierkoeffizienten von $f$, so konvergieren die Reihen
			$$ \sum_{n=0}^{\infty} a_{n} \text{ und } \sum_{n=1}^{\infty} b_{n} \text{ absolut}.$$
			
	\end{enumerate}	
\end{satz}
Ohne Beweis.

\index{Besselsche Ungleichung}
\begin{satz} \label{13.7:satz}
	Die Funktion f erfülle (V), und $a_{n}$, $b_{n}$ seien ihre Fourrierkoeffizienten. Dann gilt:
	\begin{enumerate}
		\item $\sum_{n=1}^{\infty} (a_{n}^{2} + b_{n}^{2})$ ist konvergent. \label{13.7.a:satz}
		\item $\frac{a_{0}^{2}}{2} + \sum_{n=1}^{\infty} (a_{n}^{2} + b_{n}^{2}) \leq  \frac{1}{\pi} \int_{-\pi}^{\pi} g(x)^{2} dx$ (Besselsche Ungleichung). \label{13.7.b:satz}
		\item $a_{n} \rightarrow 0$, $b_{n} \rightarrow 0$ $(n \to \infty)$. \label{13.7.c:satz}
	\end{enumerate}	
\end{satz}

\begin{proof}
	Für $n \in \N$ und $x \in [-\pi, \pi]$:
		$$ s_{n}(x) \coloneqq \frac{a_{0}}{2} + \sum_{k=1}^{n} (a_{k} \cos(kx) + b_{k} \sin(kx)) $$
	Dann gilt: 
	\begin{align*}
		0 \leq \int_{-\pi}^{\pi} (g(x) - s_{n}(x))^{2} dx & = \int_{-\pi}^{\pi} (g(x)^{2} - 2g(x) s_{n}(x) + s_{n}(x)^{2} ) dx \\
		  & \overset{\ref{13.1:satz}}{\underset{nachr.}{=}} \int_{-\pi}^{\pi} g(x)^{2} dx - \pi \left( \frac{a_{0}^{2}}{2} + \sum_{k=1}^{n} ( a_{k}^{2} + b_{k}^{2}) \right),
	\end{align*}
	also
	$$ \alpha_{n} \coloneqq \frac{a_{0}^{2}}{2} + \underbrace{\sum_{k=1}^{b}(a_{k}^{2} + b_{k}^{2})}_{\eqqcolon \beta_{n}} 
	\leq \frac{1}{\pi} \int_{-\pi}^{\pi} g(x)^{2} dx \eqqcolon \alpha.$$
	Die Folge $(\alpha_{n})$ ist monoton wachsend und beschränkt, somit ist $(\alpha_{n})$ konvergent. Damit ist $(\beta_{n})$ konvergent und es folgt a). \\
	Aus $\alpha_{n} \leq \alpha$ $(n \in \N)$ folgt b). \\
	Aus $(1)$ und \ref{3.1:satz} folgt $a_{n}^{2} + b_{n}^{2} \rightarrow 0$. Damit gilt $a_{n}^{2} \rightarrow 0$, $b_{n}^2 \rightarrow 0$ und hieraus folgt c).
\end{proof}
\newpage

\chapter{\texorpdfstring{Der Raum $\R^{n}$}{Der Raum Rn}}
Es sei $n \in \N$. Erinnerung (lineare Algebra):
$$\R^{n} \coloneqq \{ (x_{1}, \dotsc, x_{n}) : x_{1}, \dotsc, x_{n} \in \R \}.$$
$\R^{n}$ ist mit der bekannten Addition und Skalarmultiplikation ein Vektorraum über $\R$ mit $\dim \R^{n} = n$.

\index{Einheitsvektoren}
Die Vektoren 
$$e_{1} \coloneqq (1, 0, \dotsc, 0), ~ e_{2} \coloneqq (0, 1, 0, \dotsc, 0), \dotsc, ~ e_{n} \coloneqq (0, \dotsc, 0, 1)$$
hei{\ss}en \textbf{Einheitsvektoren};
$\{ e_{1}, \dotsc, e_{n} \}$ ist eine Basis des $\R^{n}$. Ist $x = (x_{1}, \dotsc, x_{n}) \in \R^{n}$, so ist
$$ x = x_{1} e_{1} + \dotsc + x_{n} e_{n}. $$

\index{Innenprodukt} \index{Skalarprodukt} \index{Norm} \index{Länge}
\begin{definition}
	Es seien $x = (x_{1}, \dotsc, x_{n})$, $y = (y_{1}, \dotsc, y_{n}) \in \R^{n}$.
	\begin{enumerate}
		\item Die Zahl $xy \coloneqq x \cdot y \coloneqq x_{1} y_{1} + \dotsc + x_{n} y_{n}$ hei{\ss}t \textbf{Skalarprodukt} oder \textbf{Innenprodukt} von $x$ und $y$. 
		\item Die Zahl $\| x \| \coloneqq \sqrt{x \cdot x} = (x_{1}^{2} + \dotsc + x_{n}^{2})^{\frac{1}{2}}$ hei{\ss}t \textbf{Norm} oder \textbf{Länge} von $x$. 
		Beachte: $\|x\|^{2} = x \cdot x$. Im Fall $n = 1$ ist $\|x\| = |x|$.
		\item Die Zahl $\| x - y \|$ hei{\ss}t \textbf{Abstand} von $x$ und $y$. Beachte: $\| x - y \| = \| y - x \|$.
	\end{enumerate}
\end{definition}


\begin{beispiele} ~\
	\begin{enumerate}
		\item $(1, 2, -1) \cdot (1, 3, 4) = 1 + 6 - 4 = 3$.
		\item $\| (1, 2, -1) \| = (1 + 4 + 1)^{\frac{1}{2}} = \sqrt{6}$.
		\item $\| e_{j} \| = 1 ~(j = 1, \dotsc, n)$.
	\end{enumerate}
\end{beispiele}

\index{Cauchy-Schwarz Ungleichung} \index{Dreiecksungleichung}
\begin{satz} \label{14.1:satz}
	Es seien $x = (x_{1}, \dotsc, x_{n})$, $y, z \in \R^{n}$ und $\alpha \in \R$. Dann gilt:
	\begin{enumerate}
		\item $(x + y) \cdot z = x \cdot z + y \cdot y$, $x \cdot y = y \cdot x$.
		\item $(\alpha x) \cdot y = \alpha (x \cdot y) = x \cdot (\alpha y)$.
		\item $\| x \| \geq 0$; $\| x \| = 0 \iff x = 0 = (0, \dotsc, 0)$.
		\item $\| \alpha x \| = |\alpha| \| x \|$.
		\item $| x \cdot y | \leq \| x \| \| y \|$ (\textbf{Cauchy-Schwarzsche Ungleichung} (CSU)).
		\item $\| x + y\| \leq \| x \| + \| y \|$ (\textbf{Dreiecksungleichung}).
		\item $\left| \|x\| - \| y \| \right| \leq \| x - y \|$.
		\item $\forall j \in \{ 1, \dotsc, n \}: ~ |x_{j}| \leq \| x \| \leq \sum_{k=1}^{n} |x_{k}|.$
	\end{enumerate}
\end{satz}

\begin{proof}
	a) - d): Nachrechnen.
	\begin{enumerate} \setcounter{enumi}{4}
		\item O.B.d.A. sei $y \neq 0$, also $\| y \| > 0$. Es sei $A \coloneqq \| x \|^{2} = x \cdot x$, $B \coloneqq x \cdot y$, $C \coloneqq \| y \|^{2} = y \cdot y$ und
		$\alpha \coloneqq \frac{B}{C}$. Dann gilt:
			\begin{align*}
				0 & \leq \sum_{j=1}^{n} (x_{j} - \alpha y_{j})^{2} = \sum_{j=1}^{n} \left( x_{j}^{2} - 2\alpha x_{j} y_{j} + \alpha^{2} y_{j}^{2} \right) \\
				  & = A - 2 \alpha B + \alpha^{2} C = A - 2 \frac{B^{2}}{C} + \frac{B^{2}}{C} = A - \frac{B^{2}}{C} 	
			\end{align*}
			$$\Rightarrow B^{2} \leq AC \Rightarrow (x \cdot y)^{2} \leq \|x\|^{2} \|y\|^{2}.$$
		\item Es gilt: $$\|x + y \|^{2} = (x + y) \cdot (x + y) = x \cdot x + 2 x \cdot y + y \cdot y = \| x \|^{2} + 2x \cdot y + \| y \|^{2}$$
			$$  \leq \| x \|^{2} + 2 | x \cdot y | + \| y \|^{2} \overset{e)}{=} \| x \|^{2} + 2 \| x \| \| y \| + \| y \|^{2} = \left( \| x \|+ \| y \| \right)^{2}. $$
		\item Übung.
		\item Für jedes $j \in \{1,\dotsc,n\}$ gilt:
		        $$|x_{j}|^{2} = x_{j}^{2} \leq x_{1}^{2} + \dotsc + x_{n}^{2} = \|x\|^{2} \Rightarrow |x_{j}| \leq \| x \|.$$
		      Weiter gilt: 
			$$ x = x_{1} e_{1} + \dotsc + x_{n} e_{n} \Rightarrow \| x \| 
			\overset{d),f)}{\le} |x_{1}| \| e_{1} \| + \dotsc |x_{n}| \| e_{n} \| = |x_{1}| + \dotsc + |x_{n}|. $$
	\end{enumerate}
\end{proof}

\index{Norm!Matrizen}
\begin{definition}
	Es seien $l, m, n \in \N$ und $$A \coloneqq \begin{pmatrix} a_{11} & \dotsc	& a_{1n} \\ \vdots & & \vdots \\ a_{m1} & \dotsc & a_{mn} \end{pmatrix}$$
	eine reelle $m \times n$-Matrix.
	$$ \| A \| = \left( \sum_{j=1}^{m} \sum_{k=1}^{n} a_{jk}^{2} \right)^{\frac{1}{2}} \text{ hei{\ss}t} \textbf{ Norm von $A$}.	 $$

	Es sei $B$ eine reelle $n \times l$-Matrix (dann existiert $AB$). Es gilt (Übung):
	\begin{align*}
	(\ast) \quad \quad	\|AB\| \leq \|A\| \|B\|.	
	\end{align*}
	Es sei $x = (x_{1}, \dotsc, x_{n}) \in \R^{n}$.
	$$ A x \coloneqq A \cdot x^{T} = A \begin{pmatrix} x_{1} \\ \vdots \\ x_{n} \end{pmatrix} \quad (\textbf{Matrix-Vektorprodukt}) $$
\end{definition}

Aus $(*)$ folgt: $$\| A x \| \leq \| A \| \| x \|.$$

\index{Kugel} \index{Kugel!offene} \index{Kugel!abgeschlossene} \index{Umgebung}
\begin{definition}
	Es sei $x_{0} \in \R^{n}$ und $\varepsilon > 0$.
	\begin{enumerate}
		\item $U_{\varepsilon}(x_{0}) \coloneqq \{ x \in \R^{n}: \| x - x_{0} \| < \varepsilon \}$ hei{\ss}t \textbf{offene Kugel um $x_{0}$ mit Radius $\varepsilon$}, 
		oder auch \textbf{$\varepsilon$-Umgebung von $x_{0}$}.
		\item $\overline{U_{\varepsilon}(x_{0})} \coloneqq \{ x \in \R^{n}: \| x - x_{0} \| \leq \varepsilon \}$ hei{\ss}t \textbf{abgeschlossene Kugel um $x_{0}$ mit 
		Radius $\varepsilon$}. 
	\end{enumerate}
\end{definition}

\index{beschränkt} \index{offen} \index{abgeschlossen} \index{kompakt}
\begin{definition}
	Es sei $A \subseteq \R^{n}$.
	\begin{enumerate}
		\item $A$ hei{\ss}t \textbf{beschränkt} $:\iff \exists c \geq 0 ~\forall a \in A: ~ \| a \| \leq c$.
		\item $A$ hei{\ss}t \textbf{offen} $:\iff \forall a \in A ~\exists \varepsilon = \varepsilon(a) > 0: ~ U_{\varepsilon}(a) \subseteq A$. 
		\item $A$ hei{\ss}t \textbf{abgeschlossen} $:\iff$ $\R^{n} \setminus A$ ist offen.
		\item $A$ hei{\ss}t \textbf{kompakt} $:\iff A$ ist beschränkt und abgeschlossen.
	\end{enumerate}
\end{definition}


\begin{beispiele} ~\
	\begin{enumerate}
		\item Offene Kugeln sind offen, abgeschlossene Kugeln sind nicht offen.
		\item $\R^{n}$ ist offen, $\emptyset$ ist offen, $\R^{n}$ ist abgeschlossen, $\emptyset$ ist abgeschlossen.
		\item Abgeschlossene Kugeln sind kompakt.
		\item $A = \{ (x, y) \in \R^{2}: y = x^{2} \}$ ist nicht beschränkt, also auch nicht kompakt. $A$ ist nicht offen, aber $A$ ist abgeschlossen.
		\item $A = \{ (x, y) \in \R^{2} : y \geq 0, x > 0 \}$ ist nicht offen und auch nicht abgeschlossen.
	\end{enumerate}
\end{beispiele}


% Skript - Ende
\appendix 

% Inhaltsverzeichnis
\renewcommand{\indexname}{Stichwortverzeichnis}
\printindex


\end{document}