%%%%% Nicht übernommen
%% todo #Seite 9: nach "Groߟe Übungen:..." 
%% todo #Seite 12: unten Notiz am Ende übernehmen?
%% todo # Seite 23b: Fall 4, über \Rightarrow 2.1 (x^{k}) was steht da? "Bu"? Das Gleiche auch 5. Fall
%%%%% Fragen
%% todo im Skript waren einige Unterabschnitte, wie z.B."3.3 Leibnitzkriterium..." ich wusste nicht, wie ich es umsetzen sollte, also habe ich so etwas in Propositionen gepackt, hoffe das ist in Ordnung
%%%%% Ändern
%% todo beweis qed Zeichen
%% todo alle Hyperrefs überarbeiten
%% todo def überprüfen: Fett gedruckt? Index gesetzt?

\documentclass[a4paper,titlepage,ngerman,headsepline,DIV15,BCOR12mm,halfparskip*,14pt]{scrartcl}

\usepackage[utf8]{inputenc} 
\usepackage[T1]{fontenc} 
\usepackage[ngerman]{babel}

\title{Höhere Mathematik I}
\author{G. Herzog, C. Schmoeger}
\date{Wintersemester 2016/17}
\publishers{Karlsruher Institut für Technologie}

%%%%%%%%%%%%%%%% Setup	
\usepackage{amsmath}
\usepackage{amssymb}
\usepackage{amsthm}
\usepackage{color}
\usepackage[bookmarks,bookmarksnumbered,hypertexnames=false,pdfpagelayout=OneColumn,colorlinks,linkcolor=blue]{hyperref}
\usepackage{makeidx} 
\usepackage{mathtools} 
\usepackage{pgfplots}
\usepackage{tikz}
\usetikzlibrary{matrix}	
\usetikzlibrary{decorations.pathreplacing}

\renewcommand{\labelenumi}{\alph{enumi})}
\renewcommand{\labelenumii}{(\roman{enumii})}
\setlength{\parindent}{0pt}
\renewcommand{\baselinestretch}{1.15}
\setkomafont{subsection}{\fontsize{0.8em}{1em}\selectfont\bfseries}
\DeclareUnicodeCharacter{00A0}{ }
\renewcommand*{\qed}{\hfill\ensuremath{\square}}%

\newtheoremstyle{dotless}{}{}{}{}{\bfseries}{:}{ }{}

\newcommand{\Z}{\mathbb{Z}}
\newcommand{\R}{\mathbb{R}}
\newcommand{\N}{\mathbb{N}}
\newcommand{\C}{\mathbb{C}}
\newcommand{\Q}{\mathbb{Q}}

\theoremstyle{dotless}
\newtheorem{satz}{Satz}[section]	  	
\newtheorem{prop}[satz]{Proposition}
\newtheorem{folg}[satz]{Folgerung}
\newtheorem*{definition}{Definition}
\newtheorem*{beispiel}{Beispiel}
\newtheorem*{beispiele}{Beispiele}
\newtheorem*{bemerkung}{Bemerkung} 
\newtheorem*{bemerkungen}{Bemerkungen}
\newtheorem*{regeln}{Regeln}
\newtheorem*{vereinbarung}{Vereinbarung}
\newtheorem*{schreibweise}{Schreibweise}
\newtheorem*{folgerung}{Folgerung}
\newtheorem*{schreibweisen}{Schreibweisen}
\newtheorem*{hilfssatz}{Hilfssatz}
\newtheorem*{beweis}{Beweis}

%%%%%%%%%%%%%%%% Setup ende

\makeindex

\begin{document}
	
% Titleblatt
\maketitle

% Inhaltsverzeichnis
\tableofcontents
 \newpage
  
% Das Skript
\section{Reelle Zahlen}

Grundmenge der Analysis is die Menge $\R$, die Menge der \textbf{reellen Zahlen}. Diese führen wir \textbf{axiomatisch} ein, d.h. wir nehmen $\R$ als gegeben an und \textbf{fordern} in den folgenden 15 \textbf{Axiomen} Eigenschaften von $\R$ aus denen sich alle weiteren Rechenregeln herleiten lassen.
\newline

\index{Axiome!Körper-}
\textbf{Körperaxiome:} in $\R$ seien zwei Verknüpfungen $"'+"'$ und $"'\cdot"'$ gegeben, die jedem Paar $a, b \in \R$ genau ein $a + b \in \R$ und genau ein $a b \coloneqq a \cdot b \in \R$ zuordnen. Dabei soll gelten:


\begin{description} \label{k.axiom}
	\label{k.axiom-a1}
	\item[\hspace{0.4cm}$(A1)$] $\forall a, b, c \in \R \: a + \left( b + c \right) = \left( a + b \right) + c$  (Assoziativgesetz)
	\label{k.axiom-a5}
	\item[\hspace{0.4cm}$(A5)$] $\forall a, b, c \in \R \: a \cdot \left( b \cdot c \right) = \left( a \cdot b \right) \cdot c$
	\label{k.axiom-a2}
	\item[\hspace{0.4cm}$(A2)$] $\exists 0 \in \R$ mit $\forall a \in \R \: a + 0 = a$ (Null)
	\label{k.axiom-a6}
	\item[\hspace{0.4cm}$(A6)$] $\exists 1 \in \R$ mit $\forall a \in \R \: a \cdot 1 = a$ \textbf{und} $1 \neq 0$ (Eins)
	\label{k.axiom-a3}
	\item[\hspace{0.4cm}$(A3)$] $\forall a \in \R ~ \exists -a \in \R \: a + (-a) = 0$
	\label{k.axiom-a7}
	\item[\hspace{0.4cm}$(A7)$] $\forall a \in \R \setminus \{ 0 \} ~ \exists a^{-1} \in \R \: a \cdot a^{-1} = 1$
	\label{k.axiom-a4}
	\item[\hspace{0.4cm}$(A4)$] $\forall a, b \in \R \: a + b = b + a$ (Kommutativgesetz)
	\label{k.axiom-a8}
	\item[\hspace{0.4cm}$(A8)$] $\forall a, b \in \R \: a \cdot b = b \cdot a$ (Kommutativgesetz)
	\label{k.axiom-a9}
	\item[\hspace{0.4cm}$(A9)$] $\forall a, b, c \in \R \: a \cdot (b + c) = a \cdot b + a \cdot c$ (Distributivgesetz)
\end{description}


\begin{schreibweisen}
	für $a, b \in \R$: $a - b \coloneqq a + (-b)$ und für $b \neq 0$: $ \frac{a}{b} \coloneqq a \cdot b^{-1}$.
\end{schreibweisen}


\textbf{Alle} bekannten Regeln der Grundrechnungsarten lassen sich aus \hyperref[k.axiom]{$(A1) - (A9)$} herleiten. Diese Regeln seien von nun an bekannt.


\begin{beispiele}\
	\begin{enumerate}
		\item Beh.: $\exists_{1} 0 \in  \R$ mit $\forall a \in \R \: \ a + 0 = a$
		  \begin{beweis}
			Sei $\tilde{0} \in \R$ mit $\forall a \in \R \: a + \tilde{0} = a$. Mit $a = 0$ folgt: $0 + \tilde{0} = 0$. Mit $a = \tilde{0}$ in \hyperref[k.axiom-a2]{$(A2)$} folgt: $\tilde{0} + 0 = \tilde{0}$. Dann $0 = 0 + \tilde{0} =_{\hyperref[k.axiom-a4]{(A4)}} \tilde{0} + 0 = \tilde{0}$
		  \end{beweis}
		 \item Beh.: $\forall a \in \R \: a \cdot 0 = 0$
		   \begin{beweis}
		     Sei $a \in \R$ und $b \coloneqq a \cdot 0$. Dann: $b =_{\hyperref[k.axiom-a2]{(A2)}} a (0 + 0) =_{\hyperref[k.axiom-a9]{(A9)}} a \cdot 0 + a \cdot 0 = b + b$. \\
		     $0 =_{\hyperref[k.axiom-a2]{(A3)}} b + (-b) = (b + b) + (-b) =_{\hyperref[k.axiom-a1]{(A1)}} b + (b + (-b)) = b + 0 =_{\hyperref[k.axiom-a2]{(A2)}} b$
		   \end{beweis}
	\end{enumerate}
\end{beispiele}

\index{Axiome!Anordnungs-} 
\textbf{Anordnungsaxiome:} in $\R$ ist eine Relation $"'\leq"'$ gegeben. \\
Dabei sollen gelten:
\begin{description}
	\label{a.axiom-a10}
	\item[\hspace{0.4cm}$(A10)$] für $a, b \in \R$ gilt $a \leq b$ oder $b \leq a$
	\label{a.axiom-a11}
	\item[\hspace{0.4cm}$(A11)$] aus $a \leq b$ und $b \leq a$ folgt $a = b$
	\label{a.axiom-a12}
	\item[\hspace{0.4cm}$(A12)$] aus $a \leq b$ und $b \leq c$ folgt $a \leq c$
	\label{a.axiom-a13}
	\item[\hspace{0.4cm}$(A13)$] aus $a \leq b$ folgt $\forall c \in \R \: a + c \leq b + c$
	\label{a.axiom-a14}
	\item[\hspace{0.4cm}$(A14)$] aus $a \leq b$ und $0 \leq c$ folgt $ac \leq b c$
\end{description}


\begin{schreibweisen}
$b \geq a :\iff a \leq b$; $a < b :\iff a \leq b$ und $a \neq b$; $b > 0 :\iff a < b$
\end{schreibweisen}


Aus \hyperref[k.axiom]{$(A1) - (A14)$} lassen sich alle Regeln für Ungleichungen herleiten. Diese Regeln seien von nun an bekannt.


\begin{beispiele}[ohne Beweis]\
	\begin{enumerate}
		\item aus $a < b$ und $0 < c$ folgt $ac < bc$
		\item aus $a \leq b$ und $c \leq 0$ folgt $ac \geq bc$
		\item aus $a \leq b$ und $c \leq d$ folgt $a + c \geq b + d$
	\end{enumerate}
\end{beispiele}

\index{Intervalle}
\textbf{Intervalle:} Seien  $a, b \in \R$ und $a < b$
\begin{description}
	\item $[a, b] \coloneqq \{ x \in \R : a \leq x \leq b \}$ (abgeschlossenes Intervall) 
	\item $(a, b) \coloneqq \{ x \in \R : a < x < b \}$ (offenes Intervall)
	\item $(a, b] \coloneqq \{ x \in \R : a < x \leq b \}$ (halboffenes Intervall)
	\item $[a, b) \coloneqq \{ x \in \R : a \leq x < b \}$ (halboffenes Intervall)
	\item $[a, \infty) \coloneqq \{ x \in \R : x \geq a \}$, $(a , \infty) \coloneqq \{ x \in \R : x > a \}$
	\item $(-\infty, a] \coloneqq \{ x \in \R : x \leq a\}$, $(-\infty, a) \coloneqq \{ x \in \R : x < a\}$ 
	\item $(- \infty, \infty) \coloneqq \R$
\end{description}

\index{Betrag}
\subsection*{Der Betrag} 
Für $a \in \R$ hei{\ss}t $|a| \coloneqq \begin{cases} \hspace{0.35cm} a, & \text{falls } a \geq 0 \\ -a, & \text{falls } a < 0\end{cases}$ der Betrag von $a$.


\begin{beispiele}
	$|1| = 1$, $~|-7| = -(-7) = 7$. \\
		Anschaulich:  \tikz[baseline=-0.5ex]{  \draw(0,0)--(12,0);
    \foreach \x/\xtext in {0/$$,2/$$,4/$a$,6/{\small $0$},8/$$,10/{\small $b$},12/$$}
      \draw(\x,3pt)--(\x,-3pt) node[below] {\xtext};
    \draw[decorate,decoration={brace},yshift=2ex]  (6,0) -- node[above=0.4ex] {\small $\underset{\glqq \text{Abstand} \grqq \text{ von } a \text{ und } b}{|a - b| =}$}  (10,0);
    \draw[decorate,decoration={brace},yshift=-4ex] (6,0) -- node[below=0.3ex] {\small $\underset{\glqq \text{Abstand} \grqq \text{ von } 0}{|a| =}$} (4,0);} \\ \\
	Es ist $|-a| = |a|$ und $|a - b| = |b - a|$
\end{beispiele}


\begin{regeln}\
	\begin{enumerate}
		\item $|a| \geq 0$
		\item $|a| = 0 \iff a = 0$
		\item $|ab| = |a||b|$
		\item $\pm a \leq |a|$
		\item $|a + b| \leq |a| + |b|$ (Dreiecksungleichung)
		\item $\left| |a| - |b| \right| \leq |a - b|$
	\end{enumerate}	

	\begin{beweis}\
	  \begin{enumerate}
		\item[a)]- d) leichte Übung.
		\item[e)] Fall 1: $a +b \geq 0$. Dann: $|a + b| = a + b \leq_{d)} |a| + |b|$. \\
			Fall 2: $a + b < 0$. Dann: $|a + b| = - (a + b) = - a + (- b) \leq_{d)} |a| + |b|$.
		\item[f)] $c \coloneqq |a| - |b|$; $|a| = |a - b + b| \leq_{d)} |a - b | + |b|$
			$$
				\Rightarrow c = |a| - |b| \leq |a - b|. \text{ Analog: } -c = |b| - |a| \leq |b - a| = |a - b| 
			$$
			Also: $\pm c \leq |a - b|$.
	  \end{enumerate}
	\end{beweis}
\end{regeln}

\index{beschränkt!Menge} \index{Schranke} \index{Supremum} \index{Infimum}
\begin{definition}
	Sei $\emptyset \neq M \subseteq \R$. 
	\begin{enumerate}
		\item $M$ hei{\ss}t \textbf{nach oben beschränkt} $:\iff \exists \gamma \in \R ~ \forall x \in M \: x \leq \gamma$ \\
			In diesem Fall hei{\ss}t $\gamma$ eine \textbf{obere Schranke}
		\item Ist $\gamma$ eine obere Schranke von $M$ und gilt $\gamma \leq \delta$ für jede weitere obere Schranke $\delta$ von $M$, so hei{\ss}t $\gamma$ das \textbf{Supremum} von $M$ (kleinste obere Schranke von $M$)
		\item $M$ hei{\ss}t \textbf{nach unten beschränkt} $:\iff \exists \gamma \in \R ~ \forall x \in M \: \gamma \leq x$\\
			In diesem Fall hei{\ss}t $\gamma$ eine \textbf{untere Schranke} (US)
		\item Ist $\gamma$ eine untere Schranke von $M$ und gilt $\gamma \geq \delta$ für jede weitere untere Schranke $\delta$ von $M$, so hei{\ss}t $\gamma$ das \textbf{Infimum} von $M$ (grö{\ss}te untere Schranke von $M$)
	\end{enumerate}
\end{definition}

\textbf{Bez.}: in dem Fall: $\gamma = \sup M$ bzw. $\gamma = \inf M$.
\newline


Aus \hyperref[a.axiom-a11]{$(A11)$} folgt: ist $\sup M$ bzw. $\inf M$ vorhanden, so ist $\sup M$ bzw. $\inf M$ eindeutig bestimmt.
\newline


Ist $\sup M$ bzw. $\inf M$ vorhanden und gilt $\sup M \in M$ bzw. $\inf M \in M$, so hei{\ss}t $\sup M$ das Maximum bzw. $\inf M$ das Minimum von $M$ und wird mit $\max M$ bzw. $\min M$ bezeichnet.


\begin{beispiele} ~\
	\begin{enumerate}
		\item $M = (1, 2)$. $\sup M = 2 \notin M$, $\inf M = 1 \notin M$. $M$ hat kein Maximum und kein Minimum.
		\item $M = (1, 2]$. $\sup M = 2 \in M$, $\max M = 2$
		\item $M = (3, \infty)$. $M$ ist nicht nach oben beschränkt, $3 = \inf M \notin M$.
		\item $M = (-\infty, 0]$. $M$ ist nach unten unbeschränkt, $0 = \sup M = \max M$.
	\end{enumerate}
\end{beispiele}


 \index{Axiome!Vollständigkeits-}
\textbf{Vollständigkeitsaxiom:}
\begin{description} \label{v.axiom-a10}
	\item[\hspace{0.4cm}$(A15)$]Ist $\emptyset \neq M \subseteq \R$ und ist $M$ nach oben beschränkt, so ist $\sup M$ vorhanden.
\end{description}

\begin{satz} \label{satz:1.1}
	Ist $\emptyset \neq M \subseteq \R$ und ist $M$ nach unten beschränkt, so ist $\inf M$ vorhanden.
\end{satz} 

\begin{beweis}
	i. d. Übungen.
\end{beweis}

\index{beschränkt} 
\begin{definition}
	Sei $\emptyset \neq M \subseteq \R$. $M$ hei{\ss}t beschränkt $:\iff$ $M$ ist nach oben und nach unten beschränkt ($\iff \exists c \geq 0 ~\forall x \in M \: |x| \leq c \iff \exists c \geq 0 ~\forall x \in M \: - c \leq x \leq c$)
\end{definition}


\begin{satz} \label{satz:1.2}
	Es sei $\emptyset \neq B \subseteq A \subseteq \R$
	\begin{enumerate}
		\item Ist $A$ bechränkt $\Rightarrow$ $\inf A \leq \sup A$
		\item Ist $A$ nach oben bzw. unten beschränkt $\Rightarrow$ $B$ ist nach oben beschränkt und $\sup B \leq \sup A$ bzw. nach unten beschränkt und $\inf B \geq \inf A$
		\item $A$ sei nach oben bzw. unten beschränkt und $\gamma$ eine obere bzw. untere Schranke von $A$. Dann
			$$
				\gamma = \sup A \iff \forall \varepsilon > 0 ~\exists x = x(\varepsilon) \in A : x > \gamma - \varepsilon
			$$
			\center{ bzw. }
			$$
				\gamma = \inf A \iff \forall \varepsilon > 0 ~\exists x = x(\varepsilon) \in A : x < \gamma + \varepsilon
			$$			
	\end{enumerate}

	\begin{beweis} \ 
	  \begin{enumerate}
		\item $A \neq \emptyset \Rightarrow \exists x \in \R : x \in A$. Dann $\inf A \leq x$, $x \leq \sup A$ $(A12)$
			$$ \Rightarrow \inf A \leq \sup A $$
		\item Sei $x \in B$. Dann: $x \in A$, also $x \leq \sup A$. $B$ ist also nach oben beschränkt und $\sup A$ ist eine obere Schranke von $B$
			$$ \Rightarrow \sup B \leq \sup A $$
			Analog der Fall für $A$ nach unten beschränkt.
		\item $"'\Rightarrow"'$ Sei $\gamma = \sup A$ und $\varepsilon > 0$. Dann: $\gamma - \varepsilon < \varepsilon$. $\gamma - \varepsilon$ ist also keine obere Schranke von $A$. Also: $\exists x \in A : x > \gamma - \varepsilon$ \\
			$"'\Leftarrow"'$ Sei $\tilde{\gamma} \leq \gamma$. Annahme: $\gamma \neq \tilde{\gamma}$. Dann $\tilde{\gamma} < \gamma$, also $\varepsilon \coloneqq \gamma - \tilde{\gamma} > 0$.\\
			$\xRightarrow[]{Vor.} \exists x \in A: x > \gamma - \varepsilon = \gamma- (\gamma - \tilde{\gamma}) = \tilde{\gamma}$. Widerspruch zu $x \leq \tilde{\gamma}$.
			\end{enumerate}
	\end{beweis}
\end{satz}


\subsection*{Natürliche Zahlen} 

\index{Natürliche Zahlen} \index{Induktionsmenge}
\begin{definition}\
	\begin{enumerate}
		\item $A \subseteq \R$ hei{\ss}t eine Induktionsmenge (IM) \\
		$$ :\iff \begin{cases}1 . & 1 \in A; \\ 2. & \text{aus } x \in A \text{ folgt stets } x + 1 \in A \end{cases}$$ \\
		Beispiele: $\R, [1, \infty), \{ 1 \} \cup [2, \infty)$ sind Induktionsmengen
		\item $\N \coloneqq \{ x \in \R : x$ gehört zu \textbf{jeder} IM $\}$ = Durchschnitt aller IMn \\
			Also: $\N \subseteq A$ für jede Induktionsmenge $A$.
	\end{enumerate}	
\end{definition}

\begin{satz}\ \label{satz:1.3}
	\begin{enumerate}
		\item $\N$ ist eine Induktionsmenge
		\item $\N$ ist nicht nach oben beschränkt
		\item Ist $x \in \R$, so ex. ein $n \in \N: N > x$
	\end{enumerate}
\end{satz}


Von nun an sei $\N = \{ 1, 2, 3, \dotsc \}$ bekannt.


\begin{prop}[Prinzip der vollständigen Induktion]\ \index{vollständige Induktion} \label{prop-1.4}
	Ist $A \subseteq \N$ und $A$ eine Induktionsmenge, so ist $A = N$.

	\begin{beweis}
		$A \subseteq \N$ (nach Vor.) und $\N \subset A$ (nach Def.), also $A = \N$
	\end{beweis}
\end{prop}


\subsection*{Beweisverfahren durch vollständige Induktion}
$A(n)$ sei eine Aussage, die für jedes $n \in \N$ definiert ist. Für $A(n)$ gelte:
$$\begin{cases}
	(I) & A(1) \text{ ist wahr;} \\ (II) & \text{ist } n \in \N \text{ und } A(n) \text{ wahr, so ist auch } $A(n + 1)$ \text{ wahr;}
\end{cases}$$
Dann ist $A(n)$ wahr für \textbf{jedes} $n \in \N$!

\begin{beweis}
	Sei $A \coloneqq \{ n \in \N : A(n)$ ist wahr $\}$. Dann: \\
	$A \subseteq \N$ und, wg. $(I)$, $(II)$, $A$ ist eine Induktionsmenge $\xRightarrow[]{\hyperref[prop-1.4]{(1.4)}} A = \N$
\end{beweis}


\begin{beispiel}
	Beh.: ~ $\underbrace{1 + 2 + \dotsc + n = \frac{n (n + 1)}{2}}_{A(n)}, \quad \forall n \in \N$
	
	\begin{beweis}[induktiv]
		I.A.: $1 = \frac{1 (1 + 1)}{2} \checkmark$, $A(1)$ ist also wahr. \\ \\
		I.V.: Für ein $n \in \N$ gelte $1 + 2 + \dotsc + n = \frac{n (n + 1)}{2}$ \\
		I.S.: $n \curvearrowright n + 1$: 
		\begin{align*}
			1 + 2 + \dotsc + n + (n + 1) & =_{I.V.}  \frac{n (n + 1)}{2} + (n + 1) \\
									 	 & = (n + 1) \left( \frac{n}{2} + 1 \right) \\
									 	 & = \frac{(n + 1)(n + 2)}{2}
		\end{align*}
		$\Rightarrow A(n + 1)$ ist wahr.
	\end{beweis}
\end{beispiel}

\index{ganze Zahlen} \index{rationale Zahlen}
\begin{definition} ~\
	\begin{enumerate}
		\item $\N_{0} \coloneqq \N \cup \{ 0 \}$
		\item $\Z \coloneqq \N_{0} \cup \{ - n : n \in \N \}$ (ganze Zahlen)
		\item $\Q \coloneqq \{ \frac{p}{q} : p \in \Z, q \in \N \}$ (rationale Zahlen)
	\end{enumerate}
\end{definition}


\begin{satz} \label{satz:1.5}
	Sind $x, y \in \R$ und $x < y \Rightarrow \exists r \in \Q$:
	$$ x < r < y $$	

	\begin{beweis}
		i. d. Übungen.
	\end{beweis}
\end{satz}


\subsection*{Einige Definitionen und Formeln} \index{Fakultäten} \index{Binomialkoeffizient} \index{Binomischer Satz} \index{Bernoullische Ungleichung}
\begin{enumerate}
	\item Für $a \in \R$, $n \in \N: a^{n} \coloneqq \underbrace{a \cdot \dotsc \cdot a}_{n \text{ Faktoren}}$, $a^{0} \coloneqq 1$ und ist $a \neq 0: a^{-n} \coloneqq \frac{1}{a^{n}}$ \\
		Es gelten die bekannten Rechenregeln.
	\item Für $n \in \N: n! \coloneqq 1 \cdot 2 \cdot \dotsc \cdot n$, $0! \coloneqq 1$ (\textbf{Fakultäten})
	\item \textbf{Binomialkoeffizienten}: für $n \in \N_{0}, k \in \N_{0}$ und $k \leq n$:
		$$
			\binom{n}{k} \coloneqq \frac{n!}{k!(n - k)!}
		$$
		z.B. $\binom{n}{0} = 1 = \binom{n}{n}$. Es gilt (nachrechnen!): \\
		$$
			\binom{n}{k} + \binom{n}{k - 1} = \binom{n + 1}{k} \quad \text{für } 1 \leq k \leq n
		$$
	\item Für $a, b \in \R$ und $n \in \N$ gilt: 
		\begin{align*}
			a^{n + 1} - b^{n + 1} & = (a - b) \left(a^{n} + a^{n-1}b + a^{n-2}b^{2} + \dotsc + a b^{n-1} + b^{n} \right) \\
				& = (a - b) \sum_{k = 0}^{n} a^{n -k}b^{k}
		\end{align*}
	\item \textbf{Binomischer Satz}: $a, b \in \R ~\forall n \in \N:$ $(a + b)^{n} = \sum_{k = 0}^{n} \binom{n}{k} a^{n-k}b^{k}$
		\begin{beweis}
			i. d. Übungen.
		\end{beweis}
	\item \textbf{Bernoullische Ungleichung}: Sei $x \in \R$ und $x \geq -1$. Dann:
		$$ (1 + x)^{n} \geq 1 + n x$$
		\begin{beweis}[induktiv]
			I.A.: $n = 1$: $1 + x \geq 1 + x$ \\
			I.V.: Für ein $n \in \N$ gelte $(1 + x)^{n} \geq 1 + nx$ \\
			I.S.: $n \curvearrowright n + 1$: $\xRightarrow[]{I.V.} (1 + x)^{n} \geq 1 + n x$ und da $1 + x \geq 0$:
			\begin{align*}
				(1 + x)^{n + 1} & \geq (1 + nx)(1 + x) \\
								& = 1 + nx + x + \underbrace{nx^{n}}_{\geq 0} \\
								& \geq 1 + nx + x \\
								& = 1 + (n + 1)x
			\end{align*}
		\end{beweis}
\end{enumerate}


\begin{hilfssatz}[HS]
	Für $x, y \geq 0$ und $n \in \N$ gilt: $x \leq y \iff x^{n} \leq y^{n}$	

	\begin{beweis}
		i. d. Übungen.
	\end{beweis}
\end{hilfssatz}


\begin{satz} \label{satz:1.6} \index{Wurzel}
	Sei $a \geq 0$ und $n \in \N$. Dann gibt es genau ein $x \geq 0$ mit: $x^{n} = a$. \\
	Dieses $x$ hei{\ss}t \textbf{n-te Wurzel aus a}; Bez.: $x = \sqrt[n]{a}$. ($\sqrt[2]{a} \eqqcolon \sqrt{a}$)
	
	\begin{beweis}
		Existenz: später in \S 7. \\
		Eindeutigkeit: seien $x, y \geq 0$ und $x^{n} = a = y^{n}$. $\xRightarrow[]{HS} x = y$
	\end{beweis}
\end{satz}


\begin{bemerkungen}\
	\begin{enumerate}
		\item $\sqrt{2} \notin \Q$ (s. Schule)
		\item Für $a \geq 0$ ist $\sqrt[n]{a} \geq 0$. Bsp.: $\sqrt{4} = 2$, $\sqrt{4} \neq - 2$. Die Gleichung $x^{2} = 4$ hat zwei Lösungen: $x = \pm \sqrt{4} = \pm 2$.
		\item $\sqrt{x^{2}} |x|$ $\forall x \in \R$
	\end{enumerate}
\end{bemerkungen}


\subsection*{Rationale Exponenten}
\begin{enumerate}
	\item Sei zunächste $a > 0$ und $r \in \Q, r > 0$. Dann ex. $m, n \in \N : r = \frac{m}{n}$. Wir wollen definieren:
		$$
			a^{r} \coloneqq \left( \sqrt[n]{a} \right)^{m} \quad (*)
		$$
		Problem: gilt auch noch $r = \frac{p}{q}$ mit $p, q \in \N$, gilt dann $\left( \sqrt[n]{a} \right)^{m} = \left( \sqrt[q]{a} \right)^{p}$? \\
		Antwort: ja (d.h. obige Def. $(*)$ ist sinnvoll).
		\begin{beweis}
			$x \coloneqq \left( \sqrt[n]{a} \right)^{m}$, $y \coloneqq \left( \sqrt[q]{a} \right)^{p}$, dann: $x, y \geq 0$ und $mq = np$, also
			\begin{align*}
				x^{q} & = \left( \sqrt[n]{a} \right)^{mq} = \left( \sqrt[n]{a} \right)^{np} = \left(  \left( \sqrt[n]{a} \right)^{m}\right)^{p} = a^{p} \\
					  & = \left( \left( \sqrt[q]{a} \right)^{q}\right)^{p} = \left( \left( \sqrt[q]{a} \right)^{p}\right)^{q} = y^{q}
			\end{align*}
			$\xRightarrow[]{HS} x = y$.  
		\end{beweis}
	\item Sei $a > 0, r \in \Q$ und $r < 0$. $a^{r} \coloneqq \frac{1}{a^{-r}}$. Es gelten die bekannten Rechenregeln:
		$$
			\left( ~ a^{r} a^{s} = a^{r + s}, \left( a^{r} \right)^{s} = a^{rs}, \dotsc \right)
		$$
\end{enumerate}

\newpage

\section{Folgen und Konvergenz}

\index{Folge} \index{Folge!reelle}
\begin{definition}
	Es sei $X$ eine Menge, $X \neq \emptyset$. Eine Funktion $a \colon \N \to X$ hei{\ss}t eine \textbf{Folge in X}. Ist $X = \R$, so hei{\ss}t $a$ eine \textbf{reelle Folge}.
\end{definition}


\begin{schreibweisen}
$a_{n}$ statt $a(n)$ (n-tes Folgenglied) \\
$(a_{n})$ oder $(a_{n})_{n = 1}^{\infty}$ oder $(a_{1}, a_{2}, \dotsc)$ statt $a$
\end{schreibweisen}


\begin{beispiele} ~\
	\begin{enumerate}
		\item $a_{n} \coloneqq \frac{1}{n}$ $~(n \in \N)$, also $(a_{n}) = (1, \frac{1}{2}, \frac{1}{3}, \dotsc)$
		\item $a_{2n} \coloneqq 0$, $a_{2n-1} \coloneqq 1$ $~(n \in \N)$, also $(a_{n}) = (1, 0, 1, 0, \dotsc)$
	\end{enumerate}
\end{beispiele}


\begin{bemerkung}
	Ist $p \in \Z$ und $a \colon \{ p, p + 1, \dotsc \} \to X$ eine Funktion, so spricht man ebenfalls von einer Folge in $X$. Bez.: $(a_{n})_{n = p}^{\infty}$. Meist $p = 0$ oder $p = 1$.
\end{bemerkung}

\index{abzählbar} \index{uberabzahlbar@überabzählbar}
\begin{definition}
	Sei $X$ eine Menge, $X \neq \emptyset$.
	\begin{enumerate}
		\item $X$ hei{\ss}t \textbf{abzählbar} $:\iff \exists$ Folge $(a_{n})$ in $X$: $X = \{ a_{1}, a_{2}, a_{3}, \dotsc \}$
		\item $X$ hei{\ss}t \textbf{überabzählbar} $:\iff X$ ist nicht abzählbar
	\end{enumerate}
\end{definition}


\begin{beispiele} ~\
	\begin{enumerate}
		\item Ist $X$ endlich, so ist $X$ abzählbar.
		\item $\N$ ist abzählbar, denn $\N = \{ a_{1}, a_{2}, a_{3}, \dotsc \}$ mit $a_{n} \coloneqq n$ $(n \in \N)$
		\item $\Z$ ist abzählbar, denn $\Z = \{ a_{1}, a_{2}, a_{3}, \dotsc \}$ mit $a_{1} \coloneqq 0, a_{2} \coloneqq 1, a_{3} \coloneqq -1, a_{4} \coloneqq 2, a_{5} \coloneqq -2, \dotsc$ also
			$$ a_{2n} \coloneqq n, \quad a_{2n + 1} \coloneqq -n \quad (n \in \N) $$
		\item $\Q$ ist abzählbar!
			\begin{center}
 			  \begin{tikzpicture}
				\matrix(m)[matrix of math nodes,column sep=1cm,row sep=1cm]{
    				1 & 2 & 3 & 4 & 5 & 6 & \cdots \\
    				\frac{1}{2} & \frac{2}{2} & \frac{3}{2} & \frac{4}{2} & \frac{5}{2} & \cdots & \cdots \\
    				\frac{1}{3} & \frac{2}{3} & \frac{3}{3} & \frac{4}{3} & \frac{5}{3} & \cdots \\
    				\frac{1}{4} & \frac{2}{4} & \frac{3}{4} & \frac{4}{4} & \cdots \\
    				\frac{1}{5} & \frac{2}{5} & \cdots & \cdots \\
    				\cdots & \cdots &  \\
				};
				\draw[->]
					(m-1-1)edge(m-1-2) (m-1-2)edge(m-2-1) (m-2-1)edge(m-3-1) (m-3-1)edge(m-2-2) (m-2-2)edge(m-1-3) (m-1-3)edge(m-1-4) (m-1-4)edge(m-2-3) (m-2-3)edge(m-3-2) (m-3-2)edge(m-4-1) (m-4-1)edge(m-5-1) (m-5-1)edge(m-4-2) (m-4-2)edge(m-3-3) (m-3-3)edge(m-2-4) (m-2-4)edge(m-1-5) (m-1-5)edge(m-1-6); 
  			  \end{tikzpicture}
			\end{center}
			Durchnummerieren in Pfeilrichtung liefert
				$$ \{ x \in \Q : x > 0 \} = \{ a_{1}, a_{2}, a_{3}, \dotsc \} $$
			$b_{1} \coloneqq 0, b_{2n} \coloneqq a_{n}, b_{2n + 1} \coloneqq - a_{n}$ $(n \in \N)$. Dann:
			$$ \Q = \{ b_{1}, b_{2}, b_{3}, \dotsc \} $$
		\item $\R$ ist überabzählbar (Beweis in \S 5).
	\end{enumerate}	
\end{beispiele}


\begin{vereinbarung}
	Solange nichts anderes gesagt wird, seien alle vorkommenden Folgen stets Folgen in $\R$. \\                                                                                                  
	Die folgenden Sätze und Definitionen formulieren wir nur für Folgen der Form $(a_{n})_{n=1}^{\infty}$. Sie gelten sinngemä{\ss} für Folgen der Form $(a_{n})_{n=p}^{\infty}$ $(p \in \Z)$.
\end{vereinbarung}

\index{beschränkt!Folge}
\begin{definition}
	Sei $(a_{n})$ eine Folge und $M \coloneqq \{ a_{1}, a_{2}, \dotsc \}$.
	\begin{enumerate}
		\item$(a_{n})$ hei{\ss}t \textbf{nach oben beschränkt} $:\iff M$ ist nach oben beschränkt. I.d. Fall: $\sup_{n \in \N} a_{n} \coloneqq \sup_{n = 1}^{\infty} a_{n} \coloneqq \sup M$.
		\item$(a_{n})$ hei{\ss}t \textbf{nach unten beschränkt} $:\iff M$ ist nach unten beschränkt. I.d. Fall: $\inf_{n \in \N} a_{n} \coloneqq \inf_{n = 1}^{\infty} a_{n} \coloneqq \inf M$.
		\item$(a_{n})$ hei{\ss}t \textbf{beschränkt} $~:\iff M$ ist beschränkt 
			$$ \iff \exists c \geq 0: |a_{n}| \leq c ~\forall n \in \N $$
	\end{enumerate}
\end{definition}

\index{für fast alle}
\begin{definition}
	Sei $A(n)$ eine für jedes $n \in \N$ definierte Aussage. \\
	$A(n)$ gilt \textbf{für fast alle} (ffa) $n \in \N$ $:\iff \exists n_{0} \in \N: A(n)$ ist wahr $\forall n \geq n_{0}$
\end{definition}


\begin{definition} \index{Umgebung}
	Sei $a \in \R$ und $\varepsilon > 0$
		$$ U_{\varepsilon}(a) \coloneqq (a - \varepsilon, a + \varepsilon) = \{ x \in \R : | x - a| < \varepsilon \} $$
	hei{\ss}t $\varepsilon$\textbf{-Umgebung von a}.
\end{definition}

\index{konvergent} \index{Grenzwert} \index{Limes} \index{divergent}
\begin{definition}
	Eine Folge $(a_{n})$ hei{\ss}t \textbf{konvergent}
	$$ :\iff \exists a \in \R : \begin{cases} \text{zu jedem } \varepsilon > 0 \text{ ex. } n_{0} = n_{0}(\varepsilon) \in \N: \\
		|a_{n} - a| < \varepsilon ~\forall n \geq n_{0}
	\end{cases} $$
	I. d. Fall hei{\ss}t $a$ \textbf{Grenzwert} (GW) oder \textbf{Limes} von $(a_{n})$ und man schreibt
	$$ 
		a_{n} \rightarrow a ~(n \rightarrow \infty) \text{ oder } a_{n} \rightarrow a \text{ oder } \lim_{n \rightarrow \infty} a_{n} = a
	$$
	Ist $(a_{n})$ nicht konvergent, so hei{\ss}t $(a_{n})$ \textbf{divergent}. Beachte:
	\begin{align*}
		a_{n} \rightarrow a ~(n \rightarrow \infty) & \iff \forall \varepsilon > 0 ~\exists n_{0} \in \N: a_{n} \in U_{\varepsilon}(a) ~\forall n \geq n_{0} \\
				& \iff \forall \varepsilon > 0 \text{ gilt: } a_{n} \in U_{\varepsilon}(a) \text{ ffa } n \in \N \\
				& \iff \forall \varepsilon > 0 \text{ gilt: } a_{n} \notin U_{\varepsilon}(a) \text{ für höchstens endlich viele } n \in \N
	\end{align*}
\end{definition}


\begin{satz} \label{satz:2.1}
	$(a_{n})$ sei konvergent und $a = \lim a_{n}$
	\begin{enumerate}
		\item Gilt auch noch $a_{n} \rightarrow b$, so ist $a = b$
		\item $(a_{n})$ ist beschränkt
	\end{enumerate}
	
	\begin{beweis}\
	  \begin{enumerate}
		\item Annahme $a \neq b$. Dann ist $\varepsilon \coloneqq \frac{|a - b|}{2} > 0$.
			$$
			\exists n_{0} \in \N: |a_{n_{0}} - a| < \varepsilon \quad \forall n \geq n_{0} \text{ und } \exists n_{1} \in \N: |a_n - b| < \varepsilon \quad \forall n \geq n_{1}
			$$
			$N \coloneqq \max \{ n_{0}, n_{1} \}$. Dann:
			$$
				2 \varepsilon = |a - b| = | a - a_{N} + a_{N} - b| \leq |a_{N} - a| + |a_{N} - b| < 2 \varepsilon
			$$
			Widerspruch! Also $ a = b$
		\item  Zu $\varepsilon = 1 ~\exists n_{0} \in \N: |a_{n} - a| < 1 ~\forall n \geq n_{0}$. Dann:
			$$
				|a_{n}| = |a_{n} - a + a| \leq |a_{n} - a| + |a| \leq 1 + |a| \quad \forall n \geq n_{0}
			$$
			$c \coloneqq \max \{ 1 + |a|, |a_{1}|, \dotsc, |a_{n_{0} - 1}| \}$. Dann: $|a_{n}| \leq \varepsilon ~\forall n \geq 1$.
	  \end{enumerate}
	\end{beweis}	
\end{satz}


\begin{beispiele}\
	\begin{enumerate}
		\item Sei $c \in \R$ und $a_{n} \coloneqq c ~\forall n \in \N$. Dann:
			$$
				| a_{n} - c | = 0 \quad \forall n \in \N
			$$
			Also: $a_{n} \rightarrow c$.
		\item $a_{n} \coloneqq \frac{1}{n} ~(n \in \N)$. Beh: $a_{n} \rightarrow 0 ~(n \rightarrow \infty)$.
			\begin{beweis}
				Sei $\varepsilon > 0: |a_{n} - 0 | = |a_{n}| = \frac{1}{n} < \varepsilon \iff n > \frac{1}{\varepsilon}$
				$$
						\xRightarrow[]{\hyperref[satz:1.3]{\text{1.3 c)}}} \exists n_{0} \in \N: n_{0} > \frac{1}{\varepsilon}
				$$
				Für $n \geq n_{0}$ ist $n > \frac{1}{\varepsilon}$, also $\frac{1}{n} < \varepsilon$. Somit $|a_{n} - 0| < \varepsilon ~\forall n \geq n_{0}$
			\end{beweis}
		\item $a_{n} \coloneqq (-1)^{n}$. Es ist $|a_{n}| = 1 ~\forall n \in \N$, $(a_{n})$ ist also beschränkt. Behauptung: $(a_{n})$ ist divergent.
			\begin{beweis}
				$\forall n \in \N: |a_{n} - a_{n+1}| = |(-1)^{n} - (-1)^{n+1}| = |(-1)^{n}| \left( 1 - (-1) \right) = 2$. \\
				Annahme: $(a_{n})$ konvergiert. Definiere $a \coloneqq \lim a_{n}$, dann 
				$$
					 \exists n_{0} \in \N: ~ |a_{n} - a| < \frac{1}{2} \quad \forall n \geq n_{0}
				$$
				Für $n \geq n_{0}$ gilt dann aber:
				$$
					2 = |a_{n} - a_{n+1}| = |a_{n} - a + a - a_{n + 1}| \leq |a_{n} - a| + |a_{n+1} - a| < \frac{1}{2} + \frac{1}{2} = 1
				$$
				Widerspruch!
			\end{beweis}
		\item $a_{n} \coloneqq n ~(n \in \N)$. $(a_{n})$ ist nicht beschränkt $\xRightarrow[]{\hyperref[satz:2.1]{2.1 b)}} (a_{n})$ ist divergent.
		\item $a_{n} \coloneqq  \frac{1}{\sqrt{n}} (n \in \N)$. Beh.: $a_{n} \rightarrow 0$
			\begin{beweis}
				Sei $\varepsilon > 0$.
				$$
					|a_{n} - 0| = \frac{1}{\sqrt{n}} < \varepsilon \iff \sqrt{n} > \frac{1}{n} \iff n > \frac{1}{\varepsilon^{2}}
				$$
				$\xRightarrow[]{\hyperref[satz:1.3]{1.3 c)}} \exists n_{0} \in \N: n_{0} > \frac{1}{\varepsilon^{2}}$. Ist $n \geq n_{0} \Rightarrow n > \frac{1}{\varepsilon^{2}} \Rightarrow \frac{1}{\sqrt{n}} < \varepsilon \Rightarrow |a_{n} - 0 | < \varepsilon$ 
			\end{beweis}
		\item $a_{n} \coloneqq \sqrt{n + 1} - \sqrt{n}$. 
			\begin{beweis}
				$$
					a_{n} = \frac{(\sqrt{n + 1} - \sqrt{n})(\sqrt{n + 1} + \sqrt{n})}{\sqrt{n + 1} + \sqrt{n}} = \frac{1}{\sqrt{n + 1} + \sqrt{n}} \leq \frac{1}{\sqrt{n}}
				$$
				$\Rightarrow |a_{n} - 0| \leq \frac{1}{\sqrt{n}} ~\forall n \in \N$. Sei $\varepsilon > 0$, nach Beispiel e) folgt:
				$$
					\exists n_{0} \in \N: ~ \frac{1}{\sqrt{n}} < \varepsilon \quad \forall n \geq n_{0} \Rightarrow |a_{n} - 0| < \varepsilon \quad \forall n \geq n_{0}
				$$
				Also $a_{n} \rightarrow 0$.
			\end{beweis}
	\end{enumerate}
\end{beispiele}


\begin{definition}
	$(a_{n})$ und $(b_{n})$ seien Folgen und $\alpha \in \R$
	$$
		(a_{n}) \pm (b_{n}) \coloneqq (a_{n} \pm b_{n}); ~
		\alpha (a_{n}) \coloneqq (\alpha a_{n}); ~
		(a_{n}) (b_{n}) \coloneqq (a_{n} b_{n}) 		
	$$	
	Gilt $b_{n} \neq 0 ~\forall n \geq m$, so ist die Folge $\left( \frac{a_{n}}{b_{n}} \right)_{n = m}^{\infty}$ definiert.
\end{definition}


\begin{satz}\ \label{satz:2.2}
	$(a_{n}), (b_{n}), (c_{n})$ und $(\alpha_{n})$ seien Folge und $a, b, \alpha \in \R$

	\begin{enumerate}
		\item $a_{n} \rightarrow a \iff |a_{n} - a| \rightarrow 0$
		\item Gilt $|a_{n} - a| \leq \alpha_{n}$ ffa $n \in \N$ und $\alpha_{n} \rightarrow 0$, so gilt $a_{n} \rightarrow a$
		\item Es gelte $a_{n} \rightarrow a$ und $b_{n} \rightarrow b$. Dann:
			\begin{enumerate}
				\item $|a_{n}| \rightarrow |a|$ 
				\item $a_{n} + b_{n} \rightarrow a + b$
				\item $\alpha a_{n} \rightarrow \alpha a$
				\item $a_{n} b_{n} \rightarrow a b$
				\item ist $a \neq 0$, so ex. ein $m \in \N$:
					$$
						a_{n} \neq 0 ~\forall n \geq m \text{ und für die Folge } \left( \frac{1}{a_{n}} \right)_{n = m}^{\infty} \text{ gilt: } \frac{1}{a_{n}} \rightarrow \frac{1}{a}
					$$
			\end{enumerate}
		\item Es gelte $a_{n} \rightarrow a$, $b_{n} \rightarrow b$ und $a_{n} \leq b_{n}$ ffa $n \in \N \Rightarrow a \leq b$
		\item Es gelte $a_{n} \rightarrow a$, $b_{n} \rightarrow a$ und $a_{n} \leq c_{n} \leq b_{n}$ ffa $n \in \N$. Dann $c_{n} \rightarrow a$.
	\end{enumerate}
\end{satz}


\begin{beispiele}\
	\begin{enumerate}
		\item Sei $p \in \N$ und $a_{n} \coloneqq \frac{1}{n^{p}}$. Es ist $n \leq n^{p} ~\forall n \in \N$. \\
			Dann: $0 \leq a_{n} \leq \frac{1}{n} ~\forall n \in \N \xRightarrow[]{\hyperref[satz:2.2]{2.2 e)}} a_{n} \rightarrow 0$, also $\frac{1}{n^{p}} \rightarrow 0$.
		\item $a_{n} \coloneqq \frac{5n^{2} + 3n + 1}{4n^{2} - n + 2} = \frac{5 + \frac{3}{n} + \frac{1}{n^{2}}}{4 - \frac{1}{n} + \frac{2}{n^{2}}} \xrightarrow[]{\label{satz:2.2}} \frac{5}{4}$
	\end{enumerate}
	
	\begin{beweis}[von 2.2]\
		\begin{enumerate}
			\item folgt aus der Definition der Konvergenz
			\item $\exists m \in \N: |a_{n} - a | \leq \alpha_{m} ~\forall n \geq m$. Sei $\varepsilon > 0$
				$$
		 		\exists n_{1} \in \N: \alpha_{n} < \varepsilon ~\forall n \geq n_{1}.
		 		$$
		 		$n_{0} \coloneqq \max \{ m , n_{1} \}$. Für $n \geq n_{0}$: $|a_{n} - a| \leq \alpha_{n} < \varepsilon$
			\item \begin{enumerate}
				\item $| |a_{n}| - |a|| \leq_{\S 1} |a_{n} - a| ~\forall n \in \N \xRightarrow[a)]{b)} |a_{n}| \rightarrow |a|$
				\item Sei $\varepsilon > 0$. $\exists n_{1}, n_{2} \in \N; |a_{n} - a| < \frac{\varepsilon}{2} ~\forall n \geq n_{1}$, $|b_{n} - b| < \frac{\varepsilon}{2} ~\forall n \geq n_{2}$ \\
					$n_{0} \coloneqq \max \{ n_{1}, n_{2} \}$. Für $n \geq n_{0}$:
					$$
						|a_{n} + b_{n} - (a + b)| = |a_{n} - a + b_{n} - b| \leq |a_{n} - a| + |b_{n} - b| < \frac{\varepsilon}{2} + \frac{\varepsilon}{2} = \varepsilon
					$$
				\item Übung
				\item $c_{k} \coloneqq |a_{n} b_{n} - ab|$. z. z.: $c_{n} \rightarrow 0$
					\begin{align*}
						c_{n} & = |a_{n}b_{n} - a_{n}b + a_{n}b - ab| = |a{n}(b_{n} - b)+ (a_{n} - a)b| \\
							  & \leq |a_{n}||b_{n} - b| + |b||a_{n}-a|
					\end{align*}
					$\xRightarrow[]{\hyperref[satz:2.1]{2.1 b)}} \exists c \geq 0 : |a_{n}| \leq c ~\forall n \in \N$ und $c \geq |b|$. Dann:
					$$
						c_{n} \leq c(|b_{n}-b| + |a_{n}-a|) \eqqcolon \alpha_{n} \xRightarrow[c) (ii), c) (iii)]{a)} \alpha_{n} \rightarrow 0
					$$
					Also: $|c_{n} - 0| = c_{n} \leq \alpha_{n} ~\forall n \in \N$ und $\alpha_{n} \rightarrow 0 \xRightarrow[]{b)} c_{n} \rightarrow 0$.
				\item $\varepsilon \coloneqq \frac{|a|}{2}$; aus (i): $|a_{n}| \rightarrow |a| \Rightarrow \exists n \in N$:
					$$
						 |a_{n}| \in U_{\varepsilon}(|a|) = (|a| - \varepsilon, |a| + \varepsilon) = (\frac{|a|}{2}, \frac{3}{2} |a|) \quad \forall n \geq m
					$$
					$\Rightarrow |a_{n}| > \frac{|a|}{2} > 0 ~\forall n \geq m \Rightarrow a_{n} \neq 0 ~\forall n \geq m$. \\
					Für $n \geq m$:
					$$
						\left| \frac{1}{a_{n}} - \frac{1}{a} \right| = \frac{|a_{n} - a|}{|a_{n}||a|} \leq \frac{2|a_{n} - a|}{|a|^{2}} \eqqcolon \alpha_{n}
					$$
					$\alpha_{n} \rightarrow 0 \xRightarrow[]{b)} \frac{1}{a_{n}} \rightarrow \frac{1}{a}$.
			  \end{enumerate}
			\item Annahme $b < a$, $\varepsilon \coloneqq \frac{a-b}{2} > 0$ ~ \tikz[baseline=-0.5ex]{  \draw(0,0)--(8,0);
    \foreach \x/\xtext in {0/$$,1.75/$$,3/$b$,4.5/$$,6.25/{\small $a$},8/$$}
      \draw(\x,3pt)--(\x,-3pt) node[below] {\xtext};
      \foreach \x/\xtext in {4/$x$,5.5/$y$}
      \draw(\x,2pt)--(\x,-2pt) node[below] {\xtext};
    \draw[decorate,decoration={brace},yshift=2ex]  (1.75,0) -- node[above=0.4ex] {\small $U_{\varepsilon}(b)$}  (4.25,0);
    \draw[decorate,decoration={brace},yshift=2ex] (5.25,0) -- node[above=0.4ex] {\small $U_{\varepsilon}(a)$} (7.25,0);} \\
				Dann: $x < y ~\forall x \in U_{\varepsilon}(b) ~\forall y \in U_{\varepsilon}(a)$. \\
				\begin{align*}
				\exists n_{0} \in \N: b_{n} \in U_{\varepsilon}(b) ~\forall n \geq n_{0} \\
				\exists m \in \N: a_{n} \leq b_{n} ~\forall n \geq m
				\end{align*} 
				$m_{0} \coloneqq \max \{ n_{0}, m \}$. Für $n \geq m_{0}$: $a_{n} \leq b_{n} < b + \varepsilon$, also $a_{n} \notin U_{\varepsilon}(a)$. Widerspruch!   
			\item $\exists m \in \N: a_{n} \leq c_{n} \leq b_{n} ~\forall n \geq m$. Sei $\varepsilon > 0$. $\exists n_{1}, n_{2} \in \N$: 
				\begin{align*}
					a - \varepsilon < a_{n} < a + \varepsilon ~\forall n \geq n_{1} \\
					a - \varepsilon < b_{n} < a + \varepsilon ~\forall n \geq n_{2}
				\end{align*}
				$n_{0} \coloneqq \max \{ n_{1}, n_{2}, m \}$. Für $n \geq n_{0}$:
				$$
					a - \varepsilon < a_{n} \leq c_{n} \leq b_{n} < a + \varepsilon
				$$
				Also: $|a_{n} - a| < \varepsilon \forall n \geq n_{0}$.
		\end{enumerate}	
	\end{beweis}	
\end{beispiele}


\begin{definition}\ \index{monoton}  \index{monoton! wachsend}   \index{monoton! streng wachsend} \index{monoton! streng fallend} \index{monoton! fallend}
	\begin{enumerate}
		\item $(a_{n})$ hei{\ss}t \textbf{monoton wachsend} $: \iff a_{n+1} \geq a_{n} ~\forall n \in \N$.
		\item $(a_{n})$ hei{\ss}t \textbf{streng monoton wachsend} $: \iff a_{n+1} > a_{n} ~\forall n \in \N$.
		\item Entsprechend definiert man \textbf{monoton fallend} und \textbf{streng monoton fallend}.
		\item $(a_{n})$ hei{\ss}t \textbf{monoton} $:\iff (a_{n})$n ist monoton wachsend oder monoton fallend.
	\end{enumerate}
\end{definition}

\label{prop:2.3} \index{Monotoniekriterium}
\begin{prop}[Monotoniekriterium]\
	\begin{enumerate}
		\item $(a_{n})$ sei monoton wachsend und nach oben beschränkt. Dann ist $(a_{n})$ konvergent und 
			$$
				\lim_{n \rightarrow \infty} a_{n} = \sup_{n = 1}^{\infty} a_{n}
			$$
		\item $(a_{n})$ sei monoton fallend und nach unten beschränkt. Dann ist $(a_{n})$ konvergent und 
			$$
				\lim_{n \rightarrow \infty} a_{n} = \inf_{n = 1}^{\infty} a_{n}
			$$
	\end{enumerate}
\end{prop}

\begin{beweis}
		$a \coloneqq \sup_{n = 1}^{\infty} a_{n}$. Sei $\varepsilon > 0$. Dann ist $a - \varepsilon$ keine obere Schranke von $\{ a_{1}, a_{2}, \cdots \}$, also existiert ein $n_{0} \in \N: a_{n_{0}} > a - \varepsilon$. Für $n \geq n_{0}$:
			$$
				a - \varepsilon < a_{n_{0}} \leq a_{n} \leq a \leq a + \varepsilon
			$$
			also $|a_{n} - a| \leq \varepsilon ~\forall n \geq n_{0}$.
\end{beweis}

\begin{figure*}[h]
	\centering
 	\begin{tikzpicture} % todo image
		\begin{axis}[ticks=none,axis x line=bottom,axis y line=left, domain=0:10, x=1cm, xmin=0, xmax=10,
        ymin=0, ymax=1.1]
        	\coordinate (A) at (rel axis cs:1,0);
 			\begin{scope}[black]
      			\draw[black] ({rel axis cs:1,0}|-{axis cs:0,1}) -- ({rel axis cs:0,0}|-{axis cs:0,1});
      			\draw[black, dotted, thick] ({rel axis cs:1,0}|-{axis cs:0,0.8}) -- ({rel axis cs:0,0}|-{axis cs:0,0.8});
     		\end{scope}
     		\addplot+[only marks,mark size=1pt] {(5*x/(5*x+10))+0.1};
 		\end{axis}
	\end{tikzpicture}	
\end{figure*}


\begin{beispiel} $a_{1} \coloneqq \sqrt[3]{6}$, $a_{n + 1} \coloneqq \sqrt[3]{6 + a_{n}}$ $(n \geq 2)$.
	\begin{description}
		\item $a_{1} = \sqrt[3]{6} < \sqrt[3]{8} = 2$;
		\item $a_{2} = \sqrt[3]{6 + a_{1}} < \sqrt[3]{6 + 2} = 2$;
		\item $a_{2} = \sqrt[3]{6 + a_{1}} < \sqrt[3]{6} = a_{1}$;
	\end{description}
	Behauptung: $0 < a_{n} < 2$ und $a_{n + 1} > a_{n}$ $\forall n \in \N$

	\begin{beweis}[induktiv]~\\
		I.A.: s.o. \\
		I.V.: Sei $n \in \N$ und $0 < a_{n} < 2$ und $a_{n+1} > a_{n}$. $n \curvearrowright n + 1$: $a_{n + 1} = \sqrt[3]{6 + a_{n}} >_{I.V.} 0$
		$$
			a_{n +1} = \sqrt[3]{6 + a_{n}} <_{I.V.} \sqrt[3]{6 + 2} = 2; \quad a_{n + 2} = \sqrt[3]{6 + a_{n+1}} >_{I.V.} \sqrt[3]{6 + a_{n}} = a_{n + 1}
		$$
		Also: $(a_{n})$ ist nach oben beschränkt und monoton wachsend. \\
		$\xRightarrow[]{2.3} (a_{n})$ ist konvergent. $a \coloneqq \lim a_{n}$, $a_{n} \geq 0 ~\forall n \xRightarrow[]{2.2} a \geq 0$. Es ist
		$$
			a_{n+1}^{3} = 6 + a_{n} \quad \forall n \in \N
		$$
		$$
			\xRightarrow[]{2.2} a^{3} = 6 + a \Rightarrow 0 = a^{3} - a + 6 = (a-2)(\underbrace{a^{2}-2a+3}_{\geq 3})
		$$
		$\Rightarrow a = 2$.
	\end{beweis}
\end{beispiel}


\textbf{Wichtige Beispiele:} \\

Vorbemerkung: Seien $x, y \geq 0$ und $p \in \N$: es ist (s. \S 1)
$$
	x^{p} - y^{p} = (x - y) \sum_{k = 0}^{p-1} x^{p-1-k}y^{k}
$$
$\Rightarrow |x^{p} - y^{p}| = |x-y| \sum_{k=0}^{p-1} x^{p-1-k}y^{k} \geq y^{p-1} |x - y|$
\newline

\begin{beispiel} \label{bsp:2.4}
	Sei $a_{n} \geq 0 ~\forall n \in \N$, $a_{n} \rightarrow a (\geq 0)$ und $p \in \N$. Dann $\sqrt[p]{a_{n}} \rightarrow \sqrt[p]{a}$
	
	\begin{beweis}~\\
		Fall 1: $a = 0$. Sei $\varepsilon > 0, \exists n_{0} \in \N: |a_{n}| < \varepsilon^{p} ~\forall n \geq n_{0}$
		$$
			\Rightarrow | \sqrt[p]{a_{n}} = \sqrt[p]{|a_{n}|} < \varepsilon ~\forall n \geq n_{0}
		$$
		Also $\sqrt[p]{a_{n}} \rightarrow 0$. \newline
		
		Fall 2: $a \neq 0$.
		\begin{align*}
			|a_{n} - a| & = | (\underbrace{\sqrt[p]{a_{n}}}_{\eqqcolon x})^{p} - |\underbrace{\sqrt[p]{a}}_{\eqqcolon y}|^{p} | =  |x^{p} - y^{p}| \\
					& \geq_{s.o.} \underbrace{y^{p-1}}_{\coloneqq c} |x - y| = c | \sqrt[p]{a_{n}} - \sqrt[p]{a} |, \quad c > 0
		\end{align*}
		$\Rightarrow |\sqrt[p]{a_{n}} - \sqrt[p]{a}| \leq \frac{1}{c} |a_{n} - a| \eqqcolon \alpha_{n}$. $\alpha_{n} \rightarrow 0 \Rightarrow \sqrt[p]{a_{n}} \rightarrow \sqrt[p]{a}$
	\end{beweis} 
\end{beispiel}


\begin{beispiel} \label{bsp:2.5}
	Für $x \in \R$ gilt $(x^{n})$ ist konvergent $\iff x \in (-1,1]$, i. d. Fall:
	$$
		\lim_{n \rightarrow \infty} x^{n} = \begin{cases} 1, & \text{falls } x = 1 \\ 0, & \text{falls } x \in (-1 , 1) \end{cases}
	$$
	
	\begin{beweis}~\\
		Fall 1: $x = 0$. Dann $x^{k} \rightarrow 0$. Fall 2: $x = 1$. Dann $x^{k} \rightarrow 1$. \\
		Fall 3: $x = -1$. Dann $(x^{k}) = ((-1)^{k})$, ist divergent. \\
		Fall 4: $|x| > 1$. $\exists \delta > 0: |x| = 1 + \delta \Rightarrow |x^{k}| = |x|^{k} = (1 + \delta)^{k} \geq 1 + n \delta \geq n \delta$ \\
		 $\Rightarrow$ ist nicht beschränkt $\xRightarrow[]{2.1} (x^{k})$ ist divergent.
		Fall 5: $0 < |x| < 1 \Rightarrow \frac{1}{|x|} > 1 \Rightarrow \exists \eta > 0: \frac{1}{|x|} = 1 + \eta$.
		$$
			\Rightarrow |\frac{1}{x^{n}}| = \left( \frac{1}{|x|} \right)^{n} = (1 + \eta)^{n} \geq 1 + n \eta \geq n \eta
		$$
		$\Rightarrow |x^{n}| \leq \frac{1}{\eta} \cdot \frac{1}{n} \Rightarrow x^{n} \rightarrow 0$.
	\end{beweis}	
\end{beispiel}


\begin{beispiel} \label{bsp:2.6}
	Sei $x \in \R$ und $s_{n} \coloneqq 1 + x + x^{n} + \dotsc x^{n} = \sum_{k = 0}^{n} x^{k}$ \\
	Fall 1: $x = 1$. Dann: $x_{n} = n + 1$, $(s_{n})$ ist also divergent. \\
	Fall 2: $x \neq 1 \Rightarrow s_{n} = \frac{1 - x^{n+1}}{1 - x}$. Aus \hyperref[bsp:2.5]{(2.5)}:
	$$
		(s_{n}) \text{ konvergent} \quad \iff \quad |x| < 1
	$$
	i.d. Fall: $\lim s_{n} = \frac{1}{1 - x}$
\end{beispiel}


\begin{beispiel} \label{bsp:2.7}
	Behauptung: $\sqrt[n]{n} \rightarrow 1$.
	
	\begin{beweis}
		Es ist $\sqrt[n]{n} \geq 1 ~\forall n \in \N$, also $a_{n} \coloneqq \sqrt[n]{n} - 1 \geq 0 ~\forall n \in \N$. Z. z.: $a_{n} \rightarrow 0$. \\
		Für $n \geq 2$:
		$$
			n = \left( \sqrt[n]{n} \right)^{n} = \left( a_{n} + 1 \right)^{n} =_{\S 1} \sum_{k=0}^{n} \binom{n}{k} a_{n}^{k} \geq \binom{n}{2} a_{n}^{2} = \frac{n(n-1)}{2} a_{n}^{2}
		$$
		$\Rightarrow \frac{n-1}{2}a_{n}^{2} \leq 1$. Also $\xRightarrow[a_{n} \geq 0]{} 0 \leq a_{n} \leq \frac{\sqrt{2}}{\sqrt{n-1}} (n \geq 2)$. $\Rightarrow a_{n} \rightarrow 0$.
	\end{beweis}
\end{beispiel}


\begin{beispiel} \label{bsp:2.8}
	Sei $c > 0$. Beh.: $\sqrt[n]{c} \rightarrow 1$.
	
	\begin{beweis}
		Fall 1: $c \geq 1$. $\exists m \in \N: 1 \leq c \leq m$
		$$
		 \Rightarrow 1 \leq c \leq n ~\forall n \geq m \Rightarrow 1 \leq \sqrt[n]{c} \leq \sqrt[n]{n} ~\forall n\geq m \Rightarrow \text{Beh.}
		$$
		Fall 2: $0 < c <1 \Rightarrow \frac{1}{c} > 1 \Rightarrow \sqrt[n]{c} = \frac{1}{\sqrt[n]{\frac{1}{c}}} \xrightarrow[Fall 1]{} 1 (n \rightarrow \infty)$ $\Rightarrow$ Beh.
	\end{beweis}
\end{beispiel}


\begin{beispiel} \label{bsp:2.9}
	$a_{n} \coloneqq \left( 1 + \frac{1}{n} \right)^{n}; b_{n} \coloneqq \sum_{k = 0}^{n} \frac{1}{k!} = 1 + 1 + \frac{1}{2!} + \dotsc + \frac{1}{n!}$ \\
	Beh.: $(a_{n})$ und $(b_{n})$ sind konvergent und $\lim a_{n} = \lim b_{n}$
	
	\begin{beweis}
		I. d. gr. Übungen wird gezeigt: $2 \leq a_{n} < a_{n+1} < 3 ~\forall n \in \N$
		$$
			\xRightarrow []{\hyperref[prop:2.3]{2.3}} (a_{n}) \text{ konvergiert, } a \coloneqq \lim a_{n}
		$$
		Es ist $b_{n} > 0$ und $b_{n+1} = b_{n} + \frac{1}{(n+1)!} > b_{n}$. $(b_{n})$ ist also monoton wachsend. Für $n > 3$:
		\begin{align*} 
			b_{n} & = 1 + 1 + \frac{1}{2} + \underbrace{\frac{1}{2 \cdot 2}}_{< \left(\frac{1}{2}\right)^{2}} + \underbrace{\frac{1}{2 \cdot 3 \cdot 4}}_{< \left(\frac{1}{2}\right)^{3}} + \dotsc + \underbrace{\frac{1}{2 \cdot \dotsc \cdot n}}_{< \left(\frac{1}{2}\right)^{n-1}} \\
				  & < 1 + \left( 1 + \frac{1}{2} + (\frac{1}{2})^{2} + \dotsc + (\frac{1}{2})^{n-1} \right) = 1 + \frac{1 - \left( \frac{1}{2} \right)^{n}}{1 - \frac{1}{2}} \\
				  & < 1 + \frac{1}{1 - \frac{1}{2}} = 3 \quad \forall n \in \N
		\end{align*} 
		$\xRightarrow[]{\hyperref[prop:2.3]{2.3}} (b_{n})$ konvergiert. $b \coloneqq \lim b_{n}$. Für $n \geq 2$:
		\begin{align*}
			a_{n} & = \left( 1 + \frac{1}{n} \right)^{n} =_{\S 1} \sum_{k=0}{n} \binom{n}{k} \frac{1}{n^{k}} \\
				  & = 1 + 1 + \sum_{k = 2}^{n} \frac{1}{k!} \frac{n!}{(n-k)!} \frac{1}{n^{k}} = 1 + 1 + \sum_{k=2}^{n} \frac{1}{k!} \frac{n(n-1) \cdot \dotsc \cdot (n-(k-1))}{n \cdot n \cdot \dotsc \cdot n} \\
				  & = 1 + 1 + \sum_{k=2}^{n} \frac{1}{k!} \underbrace{(1 - \frac{1}{n})}_{< 1} \underbrace{(1 - \frac{2}{n})}_{< 1} \cdot \dotsc \cdot \underbrace{(1 - \frac{k-1}{n})}_{< 1} \\
				  & \leq 1 + 1 + \sum_{k=2}^{n} \frac{1}{k!} = b_{n}
		\end{align*}
		Also $a_{n} \leq b_{n} ~\forall n \geq 2$. Z. z.: $\Rightarrow a \leq b$ \\
		Sei $j \in \N, j \geq 2$ (zunächst fest). Für $n \in \N, n \geq j$:
		\begin{align*}
			a_{n} & =_{s.o.} 1 + 1 + \sum_{k=2}^{n} \frac{1}{k!} (1-\frac{1}{n})(1-\frac{2}{n}) \cdot \dotsc \cdot (1-\frac{k-1}{n}) \\
				  & \geq 1 + 1 + \sum_{k = 2}^{j} \frac{1}{k!} \underbrace{(1-\frac{1}{n})}_{\rightarrow 1} \underbrace{(1-\frac{2}{n})}_{\rightarrow 1} \cdot \dotsc \cdot \underbrace{(1-\frac{k-1}{n})}_{\rightarrow 1} \\
				  & \rightarrow 1 + 1 + 1 \sum_{k=2}^{j} \frac{1}{k!} = b_{j} \quad (n \rightarrow \infty)
		\end{align*}
		Also $a \geq b_{j} ~\forall j \geq 2 \xRightarrow[]{j \rightarrow \infty} a \geq b$.
	\end{beweis}
\end{beispiel}


\begin{definition} \index{Eulersche Zahl}
	$$
		e \coloneqq \lim_{n \rightarrow \infty} \left( 1 + \frac{1}{n} \right)^{n} ~( = \lim_{n \rightarrow \infty} \sum_{k = 0}^{n} \frac{1}{k!} )
	$$
	hei{\ss}t \textbf{Eulersche Zahl}. Übung: $2 < e < 3$. \\
	$e \approx 2,718\dotsc$
\end{definition}


\begin{definition} \index{Teilfolge}
	Sei $(a_{n})$ eine Folge und $(n_{1}, n_{2}, n_{3}, \dotsc)$ eine Folge in $\N$ mit \\
	$n_{1} < n_{2} < n_{3} < \dotsc$. Für $k \in \N$ setze
	$$
		b_{k} \coloneqq a_{n_{k}}
	$$
	also $b_{1} = a_{n_{1}}, b_{2} = a_{n_{2}}, \dotsc$. Dann hei{\ss}t $(b_{k}) = (a_{n_{k}})$ eine \textbf{Teilfolge} (TF) von $(a_{n})$.
\end{definition}


\begin{beispiele}\
	\begin{enumerate}
		\item $(a_{2}, a_{4}, a_{6}, \dotsc)$ ist eine Teilfolge von $(a_{n})$; hier: $n_{k} = 2k$
		\item $(a_{1}, a_{4}, a_{9}, \dotsc)$ ist eine Teilfolge von $(a_{n})$; hier: $n_{k} = k^2$
		\item $(a_{2}, a_{6}, a_{4}, a_{10}, a_{8}, a_{14}, \dotsc)$ ist keine Teilfolge von $(a_{n})$.
	\end{enumerate}
\end{beispiele}


\begin{definition}
	$(a_{n})$ sei eine Folge und $\alpha \in \R$. $\alpha$ hei{\ss}t ein \textbf{Häufungswert} (HW) von $(a_{n})$
	$$
		:\iff \exists (TF) (a_{n_{k}}) \text{ von } (a_{n}): a_{n_{k}} \rightarrow \alpha (k \rightarrow \infty) 
	$$	
	$H(a_{n}) \coloneqq \{ \alpha \in \R: \alpha$ ist ein Häufungswert von $(a_{n}) \}$.
\end{definition}


\begin{satz} \label{satz:2.10}
	$\alpha \in \R$ ist ein Häufungswert von $(a_{n})$
	$$
		\iff \forall \epsilon > 0: a_{n_{k}} \in U_{\epsilon}(\alpha) \quad (*)
	$$
	für unendlich viele $n \in \N$.
\end{satz}

\begin{beweis} ~\\
	$"'\Rightarrow"'$ Sei $(a_{n_{k}})$ eine Teilfolge mit $a_{n_{k}} \rightarrow \infty$. Sei $\epsilon > 0 \exists k_{0} \in \N: a_{n_{k}} \in U_{\epsilon}(\alpha)$ für $k \geq k_{0} \Rightarrow (*)$ \\
	$"'\Leftarrow"'$ $\exists n_{1} \in \N: a_{n_{1}} \in U_{1}(\alpha)$. $\exists n_{2} \in \N: a_{n_{2}} \in U_{\frac{1}{2}}(\alpha)$ und $n_{2} > n_{1}$. $\exists n_{3} \in \N: a_{n_{3}} \in U_{\frac{1}{3}}(\alpha)$ und $n_{3} > n_{2}$. Etc. ... Man erhält eine Teilfolge $(a_{n_{k}})$ von $(a_{n})$ mit
	$$
		a_{n_{k}} \in U_{\frac{1}{k}}(\alpha) ~\forall k \in \N, \text{ also } |a_{n_{k}} - \alpha| < \frac{1}{k} ~\forall k
	$$
	Somit: $a_{n_{k}} \rightarrow \alpha$. 
\end{beweis}


\begin{beispiele}\
	\begin{enumerate}
		\item $a_{n} = (-1)^{n}$, $a_{2k} = 1 \rightarrow 1, a_{2k+1} \rightarrow -1$, also $1, -1 \in H(a_{n})$. Sei $\alpha \in \R, \alpha \neq 1, \alpha \neq -1$ \\
			% todo Zeichnung
			Wähle $\epsilon>0$ so, dass $1, -1 \notin U_{\epsilon}(\alpha)$. Dann $a_{n} \in U_{\epsilon}(\alpha)$ für kein $n \in \N$ $\xRightarrow[]{\hyperref[satz:2.10]{2.10}} \alpha \notin H(a_{n})$. Fazit: $H(a_{n}) = \{ 1, -1 \}$.
		\item $a_{n} = n$. Ist $\alpha \in \R$ und $\epsilon > 0$, so gilt: $a_{n} \in U_{\epsilon}(\alpha)$ für höchstens endlich viele $n$, also $\alpha \notin H(a_{n})$. Fazit: $H(a_{n}) = \emptyset$.
		\item $\Q$ ist abzählbar. Sei $(a_{n})$ eine Folge mit $Q = \{ a_{1}, a_{2}, a_{3}, \dotsc \}$. Sei $\alpha \in \R$ und $\epsilon > 0$ $\xRightarrow[]{\hyperref[satz:1.5]{1.5}} U_{\epsilon}(\alpha) = (\alpha - \epsilon, \alpha + \epsilon)$ enthält unendlich viele verschiedene rationale Zahlen $\xRightarrow[]{\hyperref[satz:2.10]{2.10}} \alpha \in H(a_{n})$. Fazit: $H(a_{n}) = \R$.
	\end{enumerate}	
\end{beispiele}

\begin{folgerung}
Ist $x \in \R$, so existieren Folgen $(r_{m})$ in $\Q : r_{n} \rightarrow \alpha$.	
\end{folgerung}


\begin{satz} \label{satz:2.11} 
	$(a_{n})$ sei konvergent, $a \coloneqq \lim a_{n}$ und $(a_{n_{k}})$ eine Teilfolge von $(a_{n})$. Dann:
	$$ a_{n_{k}} \rightarrow a (k \rightarrow \infty) $$
	Insbesondere: $H(a_{n}) = \{ \lim a_{n} \}$
\end{satz}

\begin{beweis}
	Sei $\epsilon > 0$. Dann: $a{n} \in U_{\epsilon}(a)$ ffa $n \in \N$, also auch $a_{n_{k}} \in U_{\epsilon}(a)$ ffa $k \in \N$. Somit: $a_{n_{k}} \rightarrow \alpha$.
\end{beweis}


\begin{definition} Sei $(a_{n})$ eine Folge. \index{niedrig}
	\begin{enumerate}
		\item $m \in \N$. $m$ hei{\ss}t \textbf{niedrig} (für $(a_{n})$)
			$$ :\iff a_{n} \geq a_{m} \quad \forall n \geq m $$
		\item $m \in \N$ hei{\ss}t nicht niedrig
			$$ :\iff \exists n \geq m: a_{n} < a_{m} \Rightarrow n > m: a_{n} < a_{m} $$
	\end{enumerate}
\end{definition}


\begin{hilfssatz}
	$(a_{n})$ sei eine Folge. Dann enthält $(a_{n})$ eine monotone Teilfolge.	
\end{hilfssatz}

\begin{beweis} ~\\
	Fall 1: es existieren höchstens endlich viele niedrige Indizes. Also existiert $n_{1} \in \N$: jedes $n \geq n_{1}$ ist nicht niedrig.
	\begin{description}
		\item $n_{1}$ nicht niedrig $\Rightarrow \exists n_{2} > n_{1} : a_{n_{2}} < a_{n_{1}}$
		\item $n_{2}$ nicht niedrig $\Rightarrow \exists n_{3} > n_{2} : a_{n_{3}} < a_{n_{2}}$
		\item Etc$\dotsc$
	\end{description}
	Wir erhalten so eine streng monoton fallende Teilfolge $(a_{n_{k}})$. \\ \\
	Fall 2: es existieren unendlich viele niedrige Indizes $n_{1}, n_{2}, \dotsc$, etwa $n_{1} < n_{2} < \dotsc$
	\begin{description}
		\item $n_{1}$ ist niedrig und $n_{2} > n_{1} \rightarrow a_{n_{2}} \geq a_{n_{1}}$
		\item $n_{2}$ nicht niedrig $\Rightarrow \exists n_{3} > n_{2} : a_{n_{3}} \geq a_{n_{2}}$
		\item Etc$\dotsc$
	\end{description}
	Wir erhalten so eine monoton wachsende Teilfolge $(a_{n_{k}})$.
\end{beweis}


\begin{satz}[Bolzano-Weierstra{\ss}] \label{satz:2.12-BolzanoWeierstrass} \index{Satz!Bolzano-Weierstra{\ss}} ~\\
	$(a_{n})$ sei beschränkt, dann: $H(a_{n}) \neq \emptyset$. $(a_{n})$ enthält also eine konvergente Teilfolge
\end{satz}

\begin{beweis}
	$\exists c \geq 0: |a_{n}| \leq c$ $\forall n \in \N$. $\xRightarrow[]{Hilfssatz} (a_{n})$ enthält eine monotone Teilfolge $(a_{n_{k}})$. Dann: $|a_{n_{k}}| \leq c \forall k \in \N$ \\
	$(a_{n_{k}})$ ist also beschränkt $\xRightarrow[]{\hyperref[prop:2.3]{2.3}} (a_{n_{k}})$ ist konvergent. Also $\lim_{k \rightarrow \infty} a_{n_{k}} \in H(a_{n})$.
\end{beweis}

\begin{satz} \label{satz:2.13}
	$(a_{n})$ sei beschränkt ($\xRightarrow[]{\hyperref[satz:2.12-BolzanoWeierstrass]{2.12}} H(a_{n}) \neq \emptyset$)
	\begin{enumerate}
		\item $H(a_{n})$ ist beschränkt
		\item $\sup H(a_{n}), \inf H(a_{n}) \in H(a_{n})$; es existieren also
			$$ \max H(a_{n}), \min H(a_{n}) $$
	\end{enumerate}
\end{satz}


\begin{definition} \index{Limes superior} \index{oberer Limes} \index{Limes inferior} \index{unterer Limes}
	Ist $(a_{n})$ beschränkt, so nennen wir 
	\begin{enumerate}
		\item $\limsup_{n \rightarrow \infty} a_{n} \coloneqq \limsup a_{n} \coloneqq \overline{\lim} a_{n} \coloneqq \max H(a_{n})$ hei{\ss}t \textbf{Limes superior} oder \textbf{oberer Limes} von $(a_{n})$.
		\item $\liminf_{n \rightarrow \infty} a_{n} \coloneqq \liminf a_{n} \coloneqq \underline{\lim} a_{n} \coloneqq \min H(a_{n})$ hei{\ss}t \textbf{Limes inferior} oder \textbf{unterer Limes} von $(a_{n})$.
	\end{enumerate}
\end{definition}


\begin{beweis}\
	\begin{enumerate}
		\item $\exists c \geq 0: |a_{n}| \leq c$ $\forall n \in \N$. Sei $\alpha \in H(a_{n})$. Es existiert eine Teilfolge $(a_{n_{k}})$ mit $a_{n_{k}} \rightarrow \alpha ~(k \rightarrow \infty)$. Es ist
			$$ |a_{n_{k}}| \leq c \quad \forall k, \text{ also } -c \leq a_{n_{k}} \leq c \quad \forall k $$ 
			$\Rightarrow - c \leq \alpha \leq c$. Also $|\alpha| \leq c$ $\forall \alpha \in H(a_{n})$.
		\item ohne Beweis.
	\end{enumerate}
\end{beweis}


\begin{satz} \label{satz:2.14}
	$(a_{n})$ sei beschränkt.
	\begin{enumerate}
		\item $\liminf a_{n} \leq \alpha \leq \limsup a_{n} ~\forall \alpha \in H(a_{n})$
		\item Ist $(a_{n})$ konvergent $\Rightarrow \limsup a_{n} = \liminf a_{n} = \lim a_{n}$
		\item $\limsup(\alpha a_{n}) = \alpha \limsup a_{n} ~\forall \alpha \geq 0$
		\item $\limsup(-a_{n}) = - \liminf a_{n}$
	\end{enumerate}
\end{satz}

\begin{beweis}
	a) klar, b) folgt aus \hyperref[satz:2.11]{2.11}	, c) und d) Übung.
\end{beweis}


\textbf{Motivation:} $(a_{n})$ sei konvergent und $\lim a_{n} \eqqcolon a$. Sei $\epsilon > 0$,
	$$ \exists n_{0} \in \N: |a_{n} - a| < \frac{\epsilon}{2} \quad \forall n \geq n_{0} $$
Für $n, m \geq n_{0}$:
	$$ |a_{n} - a_{m}| = |a_{n} - a + a - a_{m} | \leq |a_{n} - a| + |a_{m} - a| < \frac{\epsilon}{2} + \frac{\epsilon}{2} = \epsilon $$
D.h.: $(a_{n})$ hat die folgende Eigenschaft:
	\[ \forall \epsilon > 0 \exists n_{0} = n_{0}(\epsilon) \in \N: |a_{n} - a_{m}| < \epsilon \quad \forall n,m \geq n_{0} \tag*{(c)} \]
$(\iff\forall \epsilon > 0 ~\exists n_{0} = n_{0}(\epsilon) \in \N: |a_{n} - a_{n+k}| < \epsilon \quad \forall n \geq n_{0} ~\forall k \in \N)$


\begin{definition} \index{Cauchyfolge}
	Eine Folge $(a_{n})$ hei{\ss}t eine \textbf{Cauchyfolge} (CF)
	$$ :\iff (a_{n}) \text{ hat die Eigenschaft } (c) $$	
\end{definition}


Konvergente Folgen sind also Cauchy-Folgen!

\index{Cauchykriterium}
\begin{prop}[Cauchykriterium] \label{prop:2.15}
	$$ (a_{n}) \text{ ist konvergent} \iff (a_{n}) \text{ ist eine Cauchyfolge} $$
\end{prop}

\begin{beweis}
	$"'\Rightarrow"'$ s.o. $"'\Leftarrow"'$ ohne Beweis
\end{beweis}


\begin{beispiel}
	$a_{1} \coloneqq 1, a_{n+1} \coloneqq \frac{1}{1 + a_{n}}$ $(n \in \N)$. Mit Induktion folgt:
	\begin{enumerate}
		\item[1)] $0 < a_{n} \leq 1 ~(n \in \N)$ Damit:
		\item[2)] $a_{n} \geq \frac{1}{2} ~(n \in \N)$
	\end{enumerate}
	Für $n \geq 2, k \in \N$ gilt daher:
	\begin{align*}
		|a_{n+k} - a_{n} | & = \left| \frac{1}{1+a_{n+k-1}} - \frac{1}{1 - a_{n - 1}} \right| = \frac{|a_{n-1} - a_{n +k-1}|}{(1+a_{n+k-1})(1+a_{n-1})} \\
			& \leq \frac{1}{(1+\frac{1}{2})^{2}} |a_{n+k-1} a_{n-1}| = \frac{4}{9} |a_{n+k-1} - a_{n-1}| \\
			& \leq \left(\frac{4}{9} \right)^{2} |a_{n-k-2} - a_{n-2}| \leq \dotsc \leq \left( \frac{4}{9} \right)^{n-1} |a_{k+1} - a_{1}| \\
			& \leq \left( \frac{4}{9} \right)^{n-1} \left( |a_{k+1}| + |a_{1}|\right) \leq 2 \left( \frac{4}{9} \right)^{n-1} 
	\end{align*}
	$\exists n_{0} \in \N \setminus \{ 1 \}$: $2\left(\frac{4}{9}\right)^{n-1} < \epsilon ~(n \geq n_{0})$. Damit: $|a_{n+k} - a_{n}| < \epsilon ~(n \geq n_{0}, k \in \N)$. Also ist $(a_{n})$ Cauchyfolge. $a \coloneqq \lim_{n \rightarrow \infty} a_{n}$. Klar: $a \geq \frac{1}{2}$ und $a = \frac{1}{1 + a}$. Also $a^{2} + a - 1 = 0 \iff a = - \frac{1}{2} \pm \frac{\sqrt{5}}{2}$. Wegen $a \geq \frac{1}{2}$ folgt $a = \frac{\sqrt{5} - 1}{2}$.
\end{beispiel}



\newpage



\section{Unendliche Reihen}

\index{Reihe!unendliche} \index{Teilsumme} \index{konvergent} \index{divergent} \index{Reihenwert}
\begin{definition} $(a_{n})$ sei eine Folge;
	\begin{enumerate}
		\item $ s_{n} \coloneqq a_{1} + a_{2} + \dotsc a_{n} \quad (n \in \N)$
		(also $a_{1} = a_{1}, a_{2} = a_{1} + a_{2}, \dotsc$). $(s_{n})$ hei{\ss}t \textbf{(unendliche) Reihe} und wird mit $\sum_{n = 1}^{\infty} a_{n}$ bezeichnet. Weitere Bezeichnungen: $a_{1} + a_{2} + a_{3} + \dotsc$
		\item $s_{n}$ hei{\ss}t \textbf{n-te Teilsumme} von $\sum_{n=1}^{\infty} a_{n}$.	
		\item $\sum_{n=1}^{\infty} a_{n}$ hei{\ss}t konvergent bzw. divergent $: \iff (s_{n})$ ist konvergent 	bzw. divergent.
		\item Ist $\sum_{n = 1}^{\infty} a_{n}$ konvergent, so hei{\ss}t $\lim s_{n}$ der Reiehenwert und wird ebenfalls mit $\sum_{n=1}^{\infty} a_{n}$ bezeichnet (schlecht, aber so üblich)
	\end{enumerate} 	
\end{definition}


\begin{bemerkung}
	Ist $p \in \Z$ und $(a_{n})_{n=p}^{\infty}$ eine Folge, so definiert man entsprechend
		$$ s_{n} = a_{p} + a_{p+1} + \dotsc + a_{n} \quad (n \geq p) $$
	und $\sum_{n=p}^{\infty} a_{n}$ (meist: $p = 1$ oder $p = 0$)
\end{bemerkung}


Die folgenden Sätze und Definitionen formulieren wir nun für Reihen der Form $\sum_{n=1}^{\infty} a_{n}$. Diese Sätze und Definitionen gelten entsprechend für Reihen der Form $\sum_{n=p}^{\infty} a_{n}$ $(p \in \Z)$

\index{Reihe!geometrische} \index{Reihe!harmonische}
\begin{beispiele} ~\
	\begin{enumerate}
		\item Sei $x \in \R$. $\sum_{n=0}^{\infty} x^{n} = 1 + x + x^{2} + \dotsc$ hei{\ss}t \textbf{geometrische Reihe}. \\
			$s_{m} = 1 + x + \dotsc x^{m} \xRightarrow[]{\hyperref[bsp:2.6]{2.6}} (s_{n})$ konvergiert $\iff |x| < 1$ und $\lim s_{n} = \frac{1}{1 - x}$ für $|x| < 1$. Also: $\sum_{n=0}^{\infty} x^{n}$ konvergent $\iff |x| < 1$ und $\sum_{n=0}^{\infty} x^{n} = \frac{1}{1 - x}$ für $|x] < 1$.
		\item $\sum_{n=1}^{\infty} \frac{1}{n(n+1)}; a_{n} ) \frac{1}{n(n+1)} = \frac{1}{n} - \frac{1}{n+1}$
			\begin{align*}
				\Rightarrow s_{n} & = a_{1} + \dotsc + a_{n} \\
						& = (1 - \frac{1}{2}) + (\frac{1}{2} - \frac{1}{3}) + \dotsc + (\frac{1}{n-1} - \frac{1}{n}) + (\frac{1}{n} - \frac{1}{n+1}) \\
						& = 1 - \frac{1}{n+1} \rightarrow 1
			\end{align*}
			Also $\sum_{n=1}^{\infty} \frac{1}{n(n+1))}$ konvergent und $\sum_{n=1}^{\infty} \frac{1}{n(n+1)} = 1$.
		\item $\sum_{n=0}^{\infty} \frac{1}{n!} = 1 + 1 + \frac{1}{2!} + \frac{1}{3!} + \dotsc$; $s_{n} = 1 + 1 + \frac{1}{2!} + \dotsc + \frac{1}{n!} \xRightarrow[]{\hyperref[bsp:2.9]{2.9}} s_{n} \rightarrow e$. \\
			Also: $\sum_{n = 0}^{\infty} \frac{1}{n!}$ konvergiert und $\sum_{n=1}^{\infty} \frac{1}{n!} = e$.
		\item Die \textbf{harmonischen Reihe} $\sum_{n=1}^{\infty} \frac{1}{n}$. Dann ist $s_{n} = 1 + \frac{1}{2} + \dotsc \frac{1}{n}$, \\
			$s_{2n} = 1 + \frac{1}{2} + \dotsc + \frac{1}{n} + \frac{1}{n+1} + \dotsc + \frac{1}{2n} = s_{n} + \underbrace{\frac{1}{n+1}}_{\geq \frac{1}{2n}} + \underbrace{\frac{1}{n+2}}_{\geq \frac{1}{2n}} + \dotsc + + \underbrace{\frac{1}{2n}}_{\geq \frac{1}{2n}} \geq s_{n} + \frac{1}{2}$ \\
			Annahme $(s_{n})$ ist konvergent. $s \coloneqq \lim s_{n} \xRightarrow[satz:2.11]{2.11} s_{2n} \rightarrow s \Rightarrow s \geq s + \frac{1}{2} \rightarrow 0 \geq \frac{1}{2}$. Widerspruch! Also: $\sum_{n=1}^{\infty} \frac{1}{n}$ ist divergent!
	\end{enumerate}	
\end{beispiele}

\label{satz:3.1} \index{Monotoniekriterium} \index{Cauchykriterium}
\begin{satz}
	$(a_{n})$ sei eine Foge und $s_{n} = a_{1} + \dotsc + a_{n}$.
	\begin{enumerate}
		\item \textbf{Monotoniekriterium:} Sind alle $a_{n} \geq 0$ und ist $(s_{n})$ beschränkt, so ist $\sum_{n = 1}^{\infty} a_{n}$ konvergent.
		\item \textbf{Cauchykriterium:} $\sum a_{n}$ konvergiert $\iff \forall \epsilon > 0 \exists n_{0} = n_{0}(\epsilon) \in \N$:
			$$ \left| \sum_{k = n+1}^{m} a_{k} \right| < \epsilon ~\forall m > n \geq n_{0} $$
		\item Ist $\sum_{n=1}^{\infty} a_{n}$ konvergent $\Rightarrow a_{n} \rightarrow 0$.
		\item $\sum_{n=1}^{\infty} a_{n}$ sei konvergent. Dann ist für jedes $\nu \in \N$ die Reihe $\sum_{n=\nu+1}^{\infty} a_{n}$ konvergent und für $r_{\nu} \coloneqq \sum_{n = \nu+1}^{\infty} a_{n}$ gilt: $r_{\nu} \rightarrow 0$.
	\end{enumerate}
\end{satz}

\begin{beweis} ~\
	\begin{enumerate}
		\item $s_{n+1} = a_{1} + \dotsc + a_{n} + a_{n+1} = s_{n} + a_{n+1} \geq s_{n}$. $(s_{n})$ ist also wachsend und beschränkt $\xRightarrow[]{\hyperref[prop.2.3]{2.3}} (s_{n})$ konvergent.
		\item Für $m > n: |s_{m} - s_{n}| = | a_{1} + \dotsc + a_{n} + a_{n+1} + \dotsc + a_{m} - (a_{1} + \dotsc a_{n})| = |a_{n+1} + \dotsc + a_{m}| = |\sum_{k=n+1}^{m} a_{k}|$. Behauptung folgt aus \hyperref[prop:2.15]{2.15}.
		\item $s_{n+1} - s_{n} = a_{n+1}$. Ist $(s_{n})$ konvergent, so folgt $a_{n+1} \rightarrow 0$
		\item ohne Beweis!
	\end{enumerate}	
\end{beweis}


\begin{bemerkung}
	Ist $(a_{n})$ eine Folge und gilt $a_{n} \not\rightarrow 0$, so ist $\sum a_{n}$ divergent!
\end{bemerkung}

\label{satz:3.2}
\begin{satz}
	Die Reihen $\sum a_{n}$ und $\sum b_{n}$ seien konvergent und es seien $\alpha, \beta \in \R$. Dann konvergiert
		$$ \sum ( \alpha a_{n} + \beta b_{n}) $$
	und $\sum ( \alpha a_{n} + \beta b_{n}) = \alpha \sum a_{n} + \beta \sum b_{n}$
\end{satz}

\begin{beweis}
	\hyperref[satz:2.2]{2.2}	
\end{beweis}

\index{Leibnitzkriterium}
\begin{prop}[Leibnitzkriterium] \label{prop:3.3-leibnitzkriterium}
	Sei $(b_{n})$ eine Folge mit:
		$$ b_{n} \geq 0 ~\forall n \in \N, (b_{n}) \text{ ist monoton fallend und } b_{n} \rightarrow 0  $$
		Dann ist $\sum_{n=1}^{\infty} (-1)^{n+1}b_{n}$ konvergent-.
\end{prop}

\index{Reihe!alternierende harmonische Reihe}
\begin{beispiel}
	Aus \hyperref[prop:3.3-leibnitzkriterium]{3.3} folgt: \\
	Die \textbf{alternierende harmonische Reihe} $\sum_{n=1}^{\infty} \frac{(-1)^{n+1}}{n}$ ist konvergent.
\end{beispiel}

\begin{beweis}[von 3.3]
	$a_{n} \coloneqq (-1)^{n+1} b_{n}$, $s_{n} \coloneqq a_{1} + \dotsc + a_{n}$. $s_{2n+2} = s_{2n} + a_{2n+1} + a_{2n+2} = s_{2n} + \underbrace{b_{2n+1}-b_{2n+2}}_{\geq 0} \geq s_{2n}$. $(s_{2n})$ ist also monoton fallend. Es gilt:
	\[ \forall n \in \N: s_{2n} = s_{2n-1} - a_{2n} = s_{2n-1} - b_{2n} \leq s_{2n-1} \tag*{$(*)$} \]
	Also:
	$$ s_{2} \leq s_{4} \leq \dotsc \leq s_{2n} \overset{(*)}{\leq} s_{2n-1} \leq \dotsc \leq s_{3} \leq s_{1} $$
	$(s_{2n})$ und $(s_{2n+1})$ sind also beschränkt $\xRightarrow[]{\hyperref[prop:2.3]{2.3}} (s_{2n})$ und $(s_{2n+1})$ sind konvergent. $s \coloneqq \lim s_{2n} \xRightarrow[]{(*)} s = \lim s_{2n+1}$. \\
	Sei $\epsilon > 0$:
	$$
		\begin{rcases*}
	 		s_{2n} \in U_{\epsilon}(s) \text{ ffa } n \in \N \\
	 		s_{2n-1} \in U_{\epsilon}(s) \text{ ffa } n \in \N  	
		\end{rcases*} \Rightarrow s_{n} \in U_{\epsilon}(s) \text{ ffa } n \in \N
	$$
	Also: $s_{n} \rightarrow s$.
\end{beweis}

\index{konvergent!absolut}
\begin{definition}
	$\sum a_{n}$ hei{\ss}t \textbf{absolut konvergent} $: \iff \sum |a_{n}|$ ist konvergent.
\end{definition}


\begin{beispiel}
	$\sum_{n=1}^{\infty} \frac{(-1)^{n+1}}{n}$ ist konvergent, aber nicht absolut konvergent.
\end{beispiel}

\label{satz:3.4}
\begin{satz}
	$\sum a_{n}$ sei absolut konvergent. Dann:
	\begin{enumerate}
		\item $\sum a_{n}$ ist konvergent
		\item $|\sum_{n=1}^{\infty} a_{n}| \leq \sum_{n=1}^{\infty} |a_{n}|$ ($\triangle$-Ungleichung für Reihen)
	\end{enumerate}
\end{satz}

\begin{beweis} ~\
	\begin{enumerate}
		\item Seien $m,n \in \N, m > n$
			\[ \underbrace{| \sum_{k = n+1}^{m} a_{k} |}_{\eqqcolon \sigma_{m, n}} \leq \underbrace{\sum_{k = n+1}^{m}|a_{k}|}_{\eqqcolon \tau_{m, n}} \tag*{$(*)$} \]
			Sei $\epsilon > 0$ \hyperref[satz:3.1]{3.1 b)} n. Vor $\Rightarrow \exists n_{0} \in \N: \tau_{m, n} < \epsilon$ für $m > n > n_{0} \xRightarrow[]{(*)} \sigma_{m, n} < \epsilon$ für $m > n \geq n_{0} \xRightarrow[]{\hyperref[satz:3.1]{\text{3.1 b)}}} \sum a_{n}$ konvergiert
		\item Sei $s_{k} \coloneqq a_{1} + \dotsc + a_{k}$, $s \coloneqq \lim s_{n}$, $\sigma_{k} \coloneqq |a_{1}| + \dotsc |a_{k}|$ und $\sigma = \lim \sigma_{n}$. Dann: $|s_{n}| \rightarrow |s|$ und 
			$$ |s| \leq \sigma \quad \forall n $$
			$\Rightarrow |s| \leq \sigma \Rightarrow$ $\triangle$-Ungleichung
	\end{enumerate}
\end{beweis}

\label{satz:3.5} \index{Majorantenkriterium} \index{Minorantenkriterium}
\begin{satz} ~\
	\begin{enumerate}
		\item \textbf{Majorantenkriterium}: Gilt $|a_{n}| \leq b_{n}$ ffa $n \in \N$ und ist $\sum b_{n}$ konvergent, so ist $\sum a_{n}$ absolut konvergent.
		\item \textbf{Minorantenkriterium}: Gilt $a_{n} \geq b_{n} \geq 0$ ffa $n \in \N$ und ist $\sum b_{n}$ divergent, so ist $\sum a_{n}$ divergent.
	\end{enumerate}
\end{satz}

\begin{beweis} ~\
	\begin{enumerate}
		\item $\exists j \in \N$: $|a_{n}| \leq b_{n} ~\forall n \geq j$. Sei $m > n \geq j$, dann
			\[ \underbrace{\sum_{k=n+1}^{m}|a_{n}|}_{\eqqcolon \sigma_{m, n}} \leq \underbrace{\sum_{k=n+1}^{m} b_{k}}_{\eqqcolon \tau_{m,n}}  \]
			Sei $\epsilon > 0$ Vor. n. $\hyperref[satz:3.1]{3.1 b)} \Rightarrow \exists n_{0} \in \N: n_{0} \geq j$ und $\tau_{m,n} < \epsilon$ für $m > n \geq n_{0}$. Dann: $\sigma_{m,n} < \epsilon$ für $m > n \geq n_{0} \xRightarrow[]{\hyperref[satz:3.1]{3.1 b)}} \sum |a_{n}|$ konvergiert.
		\item Annahme: $\sum a_{n}$ konvergent $\xRightarrow[]{a)} \sum b_{n}$ konvergent, Widerspruch.
	\end{enumerate}	
\end{beweis}


\begin{beispiele} ~\
	\begin{enumerate}
		\item $\sum_{n=1}^{\infty} \frac{1}{(n+1)^{2}}$, $\forall n \in \N$:
			$$ a_{n} = \frac{1}{(n+1)^{2}} = |a_{n}| = \frac{1}{n^{2} + 2n +1} \leq \frac{1}{n^{2} + 2n} \leq \frac{1}{n(n+1)} \eqqcolon b_{n} $$
			Bekannt: $\sum b_{n}$ konvergiert $\xRightarrow[]{\hyperref[satz:3.5]{3.5 a)}} \sum_{n=1}^{\infty} \frac{1}{(n+1)^{2}}$ konvergiert
		\item Aus Beispiel a): $\sum_{n=1}^{\infty} \frac{1}{n^{2}}$ ist konvergent.
		\item Sei $\alpha > 0$ und $\alpha \in \Q$: Betrachte: $\sum_{n=1}^{\infty} \frac{1}{n^{\alpha}}$. \\
			Fall 1: $\alpha \in (0, 1]$.
				$$ \forall n \in \N: ~ \frac{1}{n^{\alpha}} \geq \frac{1}{n} \geq 0 \xRightarrow[]{\hyperref[satz:3.5]{3.5 b)}} \sum \frac{1}{n^{\alpha}} $$
			Fall 2: $\alpha \geq 2$: \\
				$$ \forall n \in \N: ~ 0 \leq \frac{1}{n^{\alpha}} \frac{1}{n^{2}} \xRightarrow[]{\hyperref[satz:3.5]{3.5 a)}} \sum_{n=1}^{\infty} \frac{1}{n^{\alpha}} \text{ konvergent} $$
			Fall 3: $\alpha \in (1, 2)$: vgl. Übungsblatt, $\sum \frac{1}{n^{\alpha}}$ konvergent. \\ \\
			\textbf{Fazit}: Ist $\alpha > 0$ und $\alpha \in \Q$, so gilt $\sum_{n=1}^{\infty} \frac{1}{n^{\alpha}}$ konvergiert $\Leftrightarrow \alpha > 1$
		\item $\sum_{n=1}^{\infty} (-1)^{n} \frac{n^{2} + 2}{n^{3} + 1}$; $|a_{n}| = \frac{n+2}{n^{3} + 1} \leq \frac{n+2}{n^{3}} \leq \frac{2n}{n^{3}} = \frac{2n}{n^{2}} \eqqcolon b_{n}$. Für $n \geq 2$ $\sum b_{n}$ konvergiert $\xRightarrow[]{\hyperref[satz:3.5]{3.5 a)}} \sum a_{n}$ konvergiert absolut
		\item $\sum_{n=1}^{\infty} \frac{\sqrt{n}}{n+1}$; $a_{n} = |a_{n}| = \frac{\sqrt{n}}{n+1} \geq \frac{\sqrt{n}}{2n} = \frac{1}{2} \cdot \frac{1}{\sqrt{n}} = \underbrace{\frac{1}{2} \cdot \frac{1}{n^{\frac{1}{2}}}}_{\geq 0} \eqqcolon b_{n}$. \\
			$\sum b_{n}$ divergiert $\xRightarrow[]{\hyperref[satz:3.5]{3.5 b)}} \sum a_{n}$ divergiert
	\end{enumerate}		
\end{beispiele}


\begin{bemerkung}
	Ist später später (in \S 7) die allgemeine Potenz $a^{x}$ ($a > 0, x \in \R)$ eingeführt, so zeigt man analog:
	$$ \text{Für } \alpha > 0 \text{ gilt:} \sum_{n=1}^{\infty} \frac{1}{n^{\alpha}} \text{ konvergiert} \iff \alpha > 1 $$
\end{bemerkung}


\begin{hilfssatz}
	$(c_{n})$ sei beschränkt
	\begin{enumerate}
		\item Ist $\alpha \coloneqq \limsup c_{n}$ und $x > \alpha$, so gilt: $c_{n} < x$ ffa $n$
				% todo image
		\item Ist $\alpha \coloneqq \liminf c_{n}$ und $x < \alpha$, so gilt: $c_{n} > x$ ffa $n$
		\item Ist $c_{n} \geq 0 ~\forall n \in \N$ und $\limsup c_{n} = 0$, so gilt $c_{n} \rightarrow 0$
	\end{enumerate}
\end{hilfssatz}

\begin{beweis} ~\
	\begin{enumerate}
		\item[b)] Sei $\epsilon > 0$. $x \coloneq \epsilon \xRightarrow[]{a)} -\epsilon < 0 \leq c_{n} < \epsilon$ ffa $n \in \N$, also $c_{n} \in U_{\epsilon}(0)$ ffa $n$.
		\item[a)] Annahme: $c_{n} \geq x$ für unendlich viele $n$, etwa für $n_{1}, n_{2}, n_{3}, \dotsc$ mit $n_{1} < n_{2} < n_{3} < \dotsc$, Die Teilfolge $(c_{n_{k}})$ ist beschränkt $\xRightarrow[]{\hyperref[satz:2.11]{2.11}} (c_{n_{k}})$ enthält eine konvergente Teilfolge $(c_{n_{k_{j}}})$. $\beta \coloneqq \lim_{j\rightarrow \infty} c_{n_{k_{j}}}$. Es ist $c_{n_{k_{j}}} \geq x ~\forall j \Rightarrow \beta \geq x > \alpha$; $(c_{n_{k_{j}}})$ ist eine Teilfolge von $(c_{n}) \Rightarrow \beta \in H(a_{n}) \Rightarrow \beta \leq \alpha$, Widerspruch.
	\end{enumerate}	
\end{beweis}

\label{Wurzelkriterium} \index{Wurzelkriterium}
\begin{prop}[Wurzelkriterium (WK)]
	Sei $(a_{n})$ eine Folge, $c_{n} \coloneqq \sqrt[n]{|a_{n}|}$.
	\begin{enumerate}
		\item Ist $(c_{n})$ unbeschränkt, so ist $\sum_{n=1}^{\infty} a_{n}$ divergent.
		\item Sei $(c_{n}$ beschränkt und $\alpha \coloneqq \limsup_{n \rightarrow \infty} c_{n}$
			\begin{enumerate}
				\item Ist $\alpha < 1$, so ist $\sum a_{n}$ absolut konvergent.
				\item Ist $\alpha > 1$, so ist $\sum a_{n}$ divergent
			\end{enumerate}
			Im Falle $\alpha = 1$ ist keine allgemeine Aussage möglich.
	\end{enumerate}
\end{prop}

\begin{beweis} ~\
	\begin{enumerate}
		\item $(c_{n})$ unbeschränkt $\Rightarrow c_{n} \geq 1$ für unendlich viele $n \Rightarrow |a_{n}| \geq 1$ für unendlich viele $n \Rightarrow a_{n} \rightarrow 0 \xRightarrow[]{\hyperref[satz:3.1]{3.1 c)}}$ Beh.
		\item  % todo beide Zeichnungen
			\begin{enumerate}
				\item Sei $\alpha < 1$, sei $x \in (\alpha, 1) \xRightarrow[]{Hilfssatz} c_{n} \leq x$ ffa n $\Rightarrow |a_{n}| \leq x^{n}$ ffa $n$. $\sum x^{n}$ konvergiert $\xRightarrow[]{\hyperref[satz:3.5]{3.5 a)}} \sum a_{n}$ konvergiert absolut 
				\item Sei $\alpha > 1$, wähle $\epsilon > 0$ so, dass $\alpha - \epsilon > 1$. Es gilt $c_{n} U_{\epsilon}(\alpha)$ für unendlich viele $n$. Dann: $c_{n} > \alpha - \epsilon > 1$ für unendlich viele $n$. Wie bei $a)$: $\sum a_{n}$ divergiert 
			\end{enumerate}
	\end{enumerate}
\end{beweis}


\begin{beispiele} ~\
	\begin{enumerate}
		\item $a_{n} \coloneqq \frac{1}{n}$; $c_{n} = \sqrt[n]{|a_{n}|} = \frac{1}{\sqrt[n]{n}} \rightarrow 1$, also $\alpha = 1$ und $\sum a_{n}$ divergiert.
		\item $a_{n} \coloneqq \frac{1}{n^{2}}$; $c_{n} = \sqrt[n]{|a_{n}|} = \frac{1}{(\sqrt[n]{n})^{2}} \rightarrow 1$, also $\alpha = 1$ und $\sum a_{n}$ konvergiert.
		\item Sei $x \in \R$ und $a_{n} \coloneqq \begin{cases} \frac{1}{2^{n}}, & \text{falls } n = 2k \\ n x^{n}, & \text{falls } n = 2k - 1 \end{cases}$ \\
			Frage: Wann ist $\sum a_{n}$ (abs.) konvergent? Es ist
			$$c_{n} = \sqrt[n]{|a_{n}|} = \begin{cases}
				\frac{1}{2}, & \text{falls } n = 2k \\ \sqrt[n]{n}|x|, & \text{falls } n = 2k - 1
			\end{cases}$$
			$(c_{n})$ ist also beschränkt, $H(c_{n}) = \left\{ \frac{1}{2}, |x| \right\}$. \\ \\
			Fall 1: $|x| < 1$. Dann: $\alpha = \limsup c_{n} < 1$, also ist $\sum a_{n}$ absolut konvergent. \\
			Fall 2: $|x| > 1$. Dann: $\alpha = \limsup c_{n} < 1$, also ist $\sum a_{n}$ divergent. \\
			Fall 3: $|x| = 1$. Dann: $\alpha = \limsup c_{n} = 1$. Es ist $|a_{n}| = n$ falls $n = 2k - 1$. Also: $a_{n} \not\rightarrow 0$. $\sum a_{n}$ ist also divergent.			
	\end{enumerate}	
\end{beispiele}

\label{prop:3.7-Quotientenkriterium} \index{Quotientenkriterium}
\begin{prop}[Quotientenkriterium (QK)]
	Es sei $a_{n} \neq 0 ~\forall n \in \N$ und $c_{n} \coloneqq \left| \frac{a_{n+1}}{a_{n}} \right| ~(n \in \N)$.
	\begin{enumerate}
		\item Ist $c_{n} \geq 1$ ffa $n \in \N$, so ist $\sum a_{n}$ divergent
		\item Sei $(c_{n})$ beschränkt, $\alpha \coloneqq \limsup c_{n}$ und $\beta \coloneqq \liminf c_{n}$
			\begin{enumerate}
				\item Ist $\alpha < 1$, so ist $\sum a_{n}$ absolut konvergent
				\item Ist $\beta > 1$, so ist $\sum a_{n}$ divergent.
			\end{enumerate}
	\end{enumerate}
\end{prop}

\label{folg:3.8}
\begin{folg}
	$(a_{n})$ und $(c_{n})$ seien wie in \hyperref[Quotientenkriterium]{3.7}, $(c_{n})$ sei konvergent und $\alpha \coloneqq \lim c_{n}$.
	$$ \sum a_{n} \text{ ist} \begin{cases} \text{absolut konv.}, & \text{falls } \alpha < 1 \\ \text{divergent}, & \text{falls } \alpha > 1 \end{cases} $$
	Im Falle $\alpha = 1$ ist keine allg. Aussage möglich.
\end{folg}


\begin{beispiele}
	\begin{enumerate}
		\item $a_{n} = \frac{1}{n}$, $\left| \frac{a_{n+1}}{a_{n}} \right| = \frac{n}{n+1} \rightarrow 1$, $\sum a_{n}$ divergiert
		\item $a_{n} = \frac{1}{n^{2}}$, $ \left| \frac{a_{n+1}}{a_{n}} \right| = \frac{n^{2}}{(n+1)^{2}} \rightarrow 1$, $\sum a_{n}$ konvergiert 
	\end{enumerate}	
\end{beispiele}

\index{Exponentialreihe} \index{Exponentialfunktion}
\begin{prop}[Die Exponentialreihe]
	Für $x \in \R$ betrachte die Reihe 
	$$ \sum_{n=0}^{\infty} \frac{x^{n}}{n!} = 1 + x + \frac{x^{2}}{2!} + \frac{x^{3}}{3!} + \dotsc $$
	Frage: für welche $x \in \R$ konvergiert die Reihe (absolut). \\
	Klar, die Reihe konvergiert für $x = 0$. Sei $x \ne q$ und $a_{n} \coloneqq \frac{x^{n}}{n!}$.
		$$ \left| \frac{a_{n+1}}{a_{n}} \right| = \left| \frac{x^{n+1}}{(n+1)!} \cdot \frac{n!}{x^{n}} \right| = \frac{|x|}{n+1} \rightarrow 0 \quad (n \rightarrow \infty) $$
	Aus \hyperref[folg:3.8]{3.8} folgt:
		$$ \sum_{n=0}^{\infty} \frac{x^{n}}{n!} \text{ konv. absolut für jedes } x \in \R $$
	Damit ist auf $\R$ eine Funktion $E \colon \R \rightarrow \R$ definiert:
		$$ E(x) \coloneqq \sum_{n=0}^{\infty} \frac{x^{n}}{n!} \quad \text{Exponentialfunktion} $$
	Es ist $E(0) = 1$, $E(1) \overset{\S 2}{=} e$. \\
	Später zeigen wir: $E(r) = e^{r}$ für $r \in \Q$. Desweiteren definieren wir später $e^{x} \coloneqq E(X)$ für $x \in \R \setminus \Q$. Dann: $e^{x} = E(x) \quad (x \in \R)$
\end{prop}

\index{Umordnung}
\begin{definition}
	Sei $(a_{n})$ eine Folge und $\varphi \colon \N \rightarrow \N$ eine Bijektion. Setze $b_{n} \coloneqq a_{\varphi(n)}$ $(n \in \N)$. also 
		$$ b_{1} = a_{\varphi(1)}, b_{2} = a_{\varphi(2)}, \dotsc $$
	Dann hei{\ss}t $(b_{n})$ eine \textbf{Umordnung} von $(a_{n})$.
\end{definition}

\begin{beispiel}
$(a_{2}, a_{4}, a_{1} a_{3}, a_{6}, a_{8}, a_{5}, a_{7}, \dotsc)$ ist eine Umordnung von $(a_{n})$.	
\end{beispiel}

\label{satz:3.10}
\begin{satz}
	$(b_{n})$ sei eine Umordnung von $(a_{n})$.
	\begin{enumerate}
		\item Ist $(a_{n})$ konvergent, so ist $(b_{n})$ konvergent und $\lim b_{n} = \lim a_{n}$.
		\item Ist $\sum a_{n}$ absolut konvergent, so ist $\sum b_{n}$ absolut konvergent und $\sum a_{n} = \sum b_{n}$
	\end{enumerate}	
\end{satz}

\begin{beweis}
	\begin{enumerate}
		\item $a \coloneqq \lim a_{n}$; Sei $\epsilon > 0$. $\exists n_{0} \in \N: |a_{n} - a| < \epsilon ~\forall n \geq n_{0}$. Dann: $|a_{\varphi(n)} - a| < \epsilon$ ffa $n \in \N$.
		\item ohne Beweis.
	\end{enumerate}	
\end{beweis}


\begin{bemerkung}[ohne Beweis]
	$\sum a_{n}$ sei konvergent, aber nicht absolut konvergent.
	\begin{enumerate}
		\item Ist $s \in \R$, so existiert eine Umordnung $\sum b_{n}$ von $\sum a_{n}$ mit: $\sum b_{n}$ ist konvergent und $\sum b_{n} = s$.
		\item Es existiert eine Umordnung $\sum c_{n}$ von $\sum a_{n}$ mit: $\sum c_{n}$ ist divergent.
	\end{enumerate}
\end{bemerkung}

\index{Cauchyprodukt}
\begin{definition}
	Gegeben seien die Reihen $\sum_{n=0}^{\infty} a_{n}$ und $\sum_{n=0}^{\infty} b_{n}$. \\
	Setze für $n \in \N$:
	\begin{align*}
		c_{n} & \coloneqq \sum_{k=0}^{\infty} a_{k} b_{n-k}, \text{ also: } \\
		c_{n} & = a_{0} b_{n} + a_{1} b_{n-1} + \dotsc + a_{n} b_{0}
	\end{align*} 
	Die Reihe $\sum_{n=0}^{\infty} c_{n}$ hei{\ss}t das \textbf{Cauchyprodukt (CP)} von $\sum a_{n}$ und $\sum b_{n}$.
\end{definition}

\label{satz:3.11}
\begin{satz}[ohne Beweis]
$\sum_{n=0}^{\infty} a_{n}$ und $\sum_{n=0}^{\infty} b_{n}$ seien absolut konvergent. Für ihr Cauchyprodukt $\sum_{n=0}^{\infty} c_{n}$ gilt dann:
	$$ \sum_{n=0}^{\infty} c_{n} \text{ ist absolut konvergent und } \sum_{n=0}^{\infty} c_{n} = (\sum_{n=0}^{\infty} a_{n}) (\sum_{n=0}^{\infty} b_{n}) $$
\end{satz}


\begin{beispiel}
	Sei $x \in \R$ und $|x | < 1$. \\
	Bekannt: $\sum_{n=0}^{\infty} x^{n}$ konvergiert absolut und	 $\sum_{k=0}^{\infty} x^{n} = \frac{1}{1-x}$. Also
	$$ \frac{1}{(1-x)^{2}} = (\sum_{n=0}^{\infty} x^{n})(\sum_{n=0}^{\infty} x^{n}) \overset{\hyperref[satz:3.11]{3.11}}{=} \sum_{n=0}^{\infty} c_{n} $$
	mit $c_{n} = \sum_{k=0}^{n} x^{k} x^{n-k} = (n+1)x^{n}$. Also:
	$$ \frac{1}{(1-x)^{2}} = \sum_{n=0}^{\infty} (n+1) x^{n} \quad (|x| < 1) $$
	z.B.: $(x = \frac{1}{2}): 4 = \sum_{n=0}^{\infty} \frac{(n+1)}{2^{n}}$. Weiter:
	$$ \frac{x}{(1-x)^{2}} = \sum_{n=0}^{\infty} (n+1) x^{n+1} = \sum_{n=1}^{\infty} n x^{n} $$
	z.B.: $(x = \frac{1}{2}): 2 = \sum_{n=1}^{\infty} \frac{n}{2^{n}}$, also $1 = \sum_{n=1}^{\infty} \frac{n}{2^{n+1}}$
\end{beispiel}

\label{prop:3.12-Exponentialfunktion}
\begin{prop}
	$E(X) = \sum_{n=0}^{\infty} \frac{x^{n}}{n!}$ $(x \in \R)$
	\begin{enumerate}
		\item $E(0) = 1, E(1) = e$
		\item $E(x + y) = E(x) E(y)$ $\forall x, y \in \R$
		\item $E(x_{1} + \dotsc + x_{m}) = E(x_{1}) \cdot \dotsc \cdot E(x_{m})$ $\forall x_{1}, \dotsc, x_{m} \in \R$
		\item $E(x) > 1 ~\forall x > 0$; $E(X) > 0 ~\forall x \in \R$; $E(-x) = E(x)^{-1} ~\forall x \in \R$
		\item $E(rx) = E(x)^{r} ~\forall x \in \R, \forall r \in \Q$
		\item $E(r) = e^{r} ~\forall r \in \Q$
		\item $E$ ist auf $\R$ streng monoton wachsend, d.h. aus $x < y$ folgt stets $E(x) < E(y)$
	\end{enumerate}	
\end{prop}

\begin{beweis}
	todo	
\end{beweis}


\newpage


% Inhaltsverzeichnis
\appendix \cleardoublepage \phantomsection \renewcommand{\indexname}{Stichwortverzeichnis} \addcontentsline{toc}{section}{\indexname} \printindex


\end{document}