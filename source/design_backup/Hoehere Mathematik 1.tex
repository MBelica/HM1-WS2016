%%%%% Meine Sachen:
%% todo Korrekturlesen, d.h. Tippfehler, quad bzw. Abstände und Punkte und Kommata
%% $$ und \[ \]
\documentclass[14pt,titlepage,ngerman,a4paper,headsepline,DIV15,halfparskip*]{scrartcl}

\usepackage[utf8]{inputenc} 
\usepackage[T1]{fontenc} 
\usepackage[ngerman]{babel}

\title{Höhere Mathematik I}
\author{G. Herzog, C. Schmoeger}
\date{Wintersemester 2016/17}
\publishers{Karlsruher Institut für Technologie}

%%%%%%%%%%%%%%%% Setup - Anfang
%% Packages
\usepackage{amsmath}
\usepackage{amssymb}
\usepackage{amsthm}
\usepackage{chngcntr}
\usepackage{enumitem}
\usepackage[colorlinks]{hyperref}
\usepackage{makeidx} 
\usepackage{mathtools} 
\usepackage{pgfplots}
\usepackage{tikz}
\usetikzlibrary{matrix}	
%% Hacks
\counterwithin{figure}{section}

\DeclareUnicodeCharacter{00A0}{ }
\newcommand{\indexsection}{section}
\renewcommand{\baselinestretch}{1.05}
\renewcommand{\labelenumi}{\alph{enumi})}
\renewcommand{\labelenumii}{(\roman{enumii})}
\setkomafont{subsection}{\fontsize{0.8em}{0em}\normalfont\bfseries}
\setlist[enumerate,1]{ref={\thesatz ~ \alph*)}}
\setlength{\parindent}{0pt} 
%% Math. Definitions
\newcommand{\C}{\mathbb{C}}
\newcommand{\N}{\mathbb{N}}
\newcommand{\Q}{\mathbb{Q}}
\newcommand{\R}{\mathbb{R}}
\newcommand{\Z}{\mathbb{Z}}
%% Theorems
\newtheoremstyle{named}{}{}{\normalfont}{}{\bfseries}{:}{0.25em}{#2 \thmnote{#3}}
\theoremstyle{named}
\newtheorem{namedtheorem}{Theorem} \counterwithin{namedtheorem}{section}
\newtheoremstyle{dotless}{}{}{}{}{\bfseries}{:}{ }{}
\renewcommand*{\qed}{\hfill\ensuremath{\square}}
\theoremstyle{dotless}
\newtheorem{satz}[namedtheorem]{Satz}	
\newtheorem{beispiel}[namedtheorem]{Beispiel}
\newtheorem{folg}[namedtheorem]{Folgerung}
\newtheorem{hilfssatz}[namedtheorem]{Hilfssatz}
\newtheorem{prop}[namedtheorem]{Proposition}
\newtheorem{anwendung}[namedtheorem]{Anwendung}
\newtheorem{anwendungen}[namedtheorem]{Anwendungen}
\newtheorem*{beispiel*}{Beispiel}
\newtheorem*{beispiele}{Beispiele}
\newtheorem*{bemerkung}{Bemerkung} 
\newtheorem*{bemerkungen}{Bemerkungen}
\newtheorem*{bezeichnung}{Bezeichnung}
\newtheorem*{definition}{Definition}
\newtheorem*{eigenschaften}{Eigenschaften}
\newtheorem*{folgerung}{Folgerung}
\newtheorem*{folgerungen}{Folgerungen}
\newtheorem*{hilfssatz*}{Hilfssatz}
\newtheorem*{regeln}{Regeln}
\newtheorem*{schreibweise}{Schreibweise}
\newtheorem*{schreibweisen}{Schreibweisen}
\newtheorem*{uebung}{Übung}
\newtheorem*{vereinbarung}{Vereinbarung}

%%%%%%%%%%%%%%%% Setup - Ende

\makeindex

\begin{document}

\maketitle
	
% Inhaltsverzeichnis
\tableofcontents
 \newpage 
  
% Skript - Anfang
\section{Reelle Zahlen}

Grundmenge der Analysis is die Menge $\R$, die Menge der \textbf{reellen Zahlen}. Diese führen wir \textbf{axiomatisch} ein, d.h. wir nehmen $\R$ als gegeben an und \textbf{fordern} in den folgenden 15 \textbf{Axiomen} Eigenschaften von $\R$ aus denen sich alle weiteren Rechenregeln herleiten lassen.
\newline

\index{Axiome!Körper-}
\textbf{Körperaxiome:} in $\R$ seien zwei Verknüpfungen $"'+"'$ und $"'\cdot"'$ gegeben, die jedem Paar $a, b \in \R$ genau ein $a + b \in \R$ und genau ein $a b \coloneqq a \cdot b \in \R$ zuordnen. Dabei soll gelten:


\begin{description} \addtolength{\itemindent}{0.4cm} \label{k.axiom}
	\item[$(A1)$] $\forall a, b, c \in \R \: a + \left( b + c \right) = \left( a + b \right) + c$  (Assoziativgesetz) \label{k.axiom-a1}
	\item[$(A5)$] $\forall a, b, c \in \R \: a \cdot \left( b \cdot c \right) = \left( a \cdot b \right) \cdot c$ \label{k.axiom-a5}
	\item[$(A2)$] $\exists 0 \in \R$ mit $\forall a \in \R : a + 0 = a$ (Null) \label{k.axiom-a2}
	\item[$(A6)$] $\exists 1 \in \R$ mit $\forall a \in \R : a \cdot 1 = a$ und $1 \neq 0$ (Eins) \label{k.axiom-a6}
	\item[$(A3)$] $\forall a \in \R ~ \exists -a \in \R : a + (-a) = 0$ \label{k.axiom-a3}
	\item[$(A7)$] $\forall a \in \R \setminus \{ 0 \} ~ \exists a^{-1} \in \R : a \cdot a^{-1} = 1$ \label{k.axiom-a7}
	\item[$(A4)$] $\forall a, b \in \R : a + b = b + a$ (Kommutativgesetz) \label{k.axiom-a4}
	\item[$(A8)$] $\forall a, b \in \R : a \cdot b = b \cdot a$ \label{k.axiom-a8}(Kommutativgesetz)
	\item[$(A9)$] $\forall a, b, c \in \R : a \cdot (b + c) = a \cdot b + a \cdot c$ (Distributivgesetz) \label{k.axiom-a9}
\end{description}


\begin{schreibweisen}
	für $a, b \in \R$: $a - b \coloneqq a + (-b)$ und für $b \neq 0$: $ \frac{a}{b} \coloneqq a \cdot b^{-1}$.
\end{schreibweisen}


\textbf{Alle} bekannten Regeln der Grundrechnungsarten lassen sich aus \hyperref[k.axiom]{$(A1) - (A9)$} herleiten. Diese Regeln seien von nun an bekannt.


\begin{beispiele} ~\
	\begin{enumerate}
		\item Beh.: $\exists_{1} 0 \in  \R$ mit $\forall a \in \R \: \ a + 0 = a$
		  \begin{proof}
			Sei $\tilde{0} \in \R$ mit $\forall a \in \R \: a + \tilde{0} = a$. Mit $a = 0$ folgt: $0 + \tilde{0} = 0$. Mit $a = \tilde{0}$ in \hyperref[k.axiom-a2]{$(A2)$} folgt: $\tilde{0} + 0 = \tilde{0}$. Dann $0 = 0 + \tilde{0} \overset{\hyperref[k.axiom-a4]{(A4)}}{=} \tilde{0} + 0 = \tilde{0}$
		  \end{proof}
		 \item Beh.: $\forall a \in \R \: a \cdot 0 = 0$
		   \begin{proof}
		     Sei $a \in \R$ und $b \coloneqq a \cdot 0$. Dann: $b \overset{\hyperref[k.axiom-a2]{(A2)}}{=} a (0 + 0) \overset{\hyperref[k.axiom-a9]{(A9)}}{=} a \cdot 0 + a \cdot 0 = b + b$. \\
		     $0 \overset{\hyperref[k.axiom-a2]{(A3)}}{=} b + (-b) = (b + b) + (-b) \overset{\hyperref[k.axiom-a1]{(A1)}}{=} b + (b + (-b)) = b + 0 \overset{\hyperref[k.axiom-a2]{(A2)}}{=} b$
		   \end{proof}
	\end{enumerate}
\end{beispiele}

\index{Axiome!Anordnungs-} 
\textbf{Anordnungsaxiome:} in $\R$ ist eine Relation $"'\leq"'$ gegeben. \\
Dabei sollen gelten:
\begin{description} \addtolength{\itemindent}{0.4cm}
	\item[$(A10)$] für $a, b \in \R$ gilt $a \leq b$ oder $b \leq a$ \label{a.axiom-a10}
	\item[$(A11)$] aus $a \leq b$ und $b \leq a$ folgt $a = b$ \label{a.axiom-a11}
	\item[$(A12)$] aus $a \leq b$ und $b \leq c$ folgt $a \leq c$ \label{a.axiom-a12}
	\item[$(A13)$] aus $a \leq b$ folgt $\forall c \in \R \: a + c \leq b + c$ \label{a.axiom-a13}
	\item[$(A14)$] aus $a \leq b$ und $0 \leq c$ folgt $ac \leq b c$ \label{a.axiom-a14}
\end{description}


\begin{schreibweisen}
$b \geq a \iff a \leq b$; $a < b \iff a \leq b$ und $a \neq b$; $b > 0 \iff a < b$
\end{schreibweisen}

Aus \hyperref[k.axiom]{$(A1) - (A14)$} lassen sich alle Regeln für Ungleichungen herleiten. Diese Regeln seien von nun an bekannt.


\begin{beispiele}[ohne Beweis] ~\
	\begin{enumerate}
		\item aus $a < b$ und $0 < c$ folgt $ac < bc$
		\item aus $a \leq b$ und $c \leq 0$ folgt $ac \geq bc$
		\item aus $a \leq b$ und $c \leq d$ folgt $a + c \geq b + d$
	\end{enumerate}
\end{beispiele}

\index{Intervalle}
\textbf{Intervalle:} Seien  $a, b \in \R$ und $a < b$
\begin{description} \addtolength{\itemindent}{0.4cm}
	\item $[a, b] \coloneqq \{ x \in \R : a \leq x \leq b \}$ (abgeschlossenes Intervall) 
	\item $(a, b) \coloneqq \{ x \in \R : a < x < b \}$ (offenes Intervall)
	\item $(a, b] \coloneqq \{ x \in \R : a < x \leq b \}$ (halboffenes Intervall)
	\item $[a, b) \coloneqq \{ x \in \R : a \leq x < b \}$ (halboffenes Intervall)
	\item $[a, \infty) \coloneqq \{ x \in \R : x \geq a \}$, $(a , \infty) \coloneqq \{ x \in \R : x > a \}$
	\item $(-\infty, a] \coloneqq \{ x \in \R : x \leq a\}$, $(-\infty, a) \coloneqq \{ x \in \R : x < a\}$ 
	\item $(- \infty, \infty) \coloneqq \R$
\end{description}

\index{Betrag}
\subsection*{Der Betrag} 
Für $a \in \R$ hei{\ss}t $|a| \coloneqq \begin{cases} \hspace{0.35cm} a, & \text{falls } a \geq 0 \\ -a, & \text{falls } a < 0\end{cases}$ der Betrag von $a$.


\begin{beispiele}
	$|1| = 1$, $~|-7| = -(-7) = 7$. \\
		Anschaulich:  \tikz[baseline=-0.5ex]{  \draw(0,0)--(12,0);
    \foreach \x/\xtext in {0/$$,2/$$,4/$a$,6/{\small $0$},8/$$,10/{\small $b$},12/$$}
      \draw(\x,3pt)--(\x,-3pt) node[below] {\xtext};
    \draw[decorate,decoration={brace},yshift=2ex]  (6,0) -- node[above=0.4ex] {\small $\underset{\glqq \text{Abstand} \grqq \text{ von } a \text{ und } b}{|a - b| =}$}  (10,0);
    \draw[decorate,decoration={brace},yshift=-4ex] (6,0) -- node[below=0.3ex] {\small $\underset{\glqq \text{Abstand} \grqq \text{ von } 0}{|a| =}$} (4,0);} \\ \\
	Es ist $|-a| = |a|$ und $|a - b| = |b - a|$
\end{beispiele}


\begin{regeln} ~\
	\begin{enumerate}
		\item $|a| \geq 0$
		\item $|a| = 0 \iff a = 0$
		\item $|ab| = |a||b|$
		\item $\pm a \leq |a|$
		\item $|a + b| \leq |a| + |b|$ (Dreiecksungleichung)
		\item $\left| |a| - |b| \right| \leq |a - b|$
	\end{enumerate}	

	\begin{proof} ~\
	  \begin{enumerate}
		\item[a)]- d) leichte Übung.
		\item[e)] Fall 1: $a +b \geq 0$. Dann: $|a + b| = a + b \overset{d)}{\leq} |a| + |b|$. \\
			Fall 2: $a + b < 0$. Dann: $|a + b| = - (a + b) = - a + (- b) \overset{d)}{\leq} |a| + |b|$.
		\item[f)] $c \coloneqq |a| - |b|$; $|a| = |a - b + b| \overset{d)}{\leq} |a - b | + |b|$
			$$
				\Rightarrow c = |a| - |b| \leq |a - b|. \text{ Analog: } -c = |b| - |a| \leq |b - a| = |a - b| 
			$$
			Also: $\pm c \leq |a - b|$.
	  \end{enumerate}
	\end{proof}
\end{regeln}

\index{beschränkt!Menge} \index{Schranke} \index{Supremum} \index{Infimum}
\begin{definition}
	Sei $\emptyset \neq M \subseteq \R$. 
	\begin{enumerate}
		\item $M$ hei{\ss}t \textbf{nach oben beschränkt} $\iff \exists \gamma \in \R ~ \forall x \in M \: x \leq \gamma$ \\
			In diesem Fall hei{\ss}t $\gamma$ eine \textbf{obere Schranke} (OS)
		\item Ist $\gamma$ eine obere Schranke von $M$ und gilt $\gamma \leq \delta$ für jede weitere obere Schranke $\delta$ von $M$, so hei{\ss}t $\gamma$ das \textbf{Supremum} von $M$ (kleinste obere Schranke von $M$)
		\item $M$ hei{\ss}t \textbf{nach unten beschränkt} $\iff \exists \gamma \in \R ~ \forall x \in M \: \gamma \leq x$\\
			In diesem Fall hei{\ss}t $\gamma$ eine \textbf{untere Schranke} (US)
		\item Ist $\gamma$ eine untere Schranke von $M$ und gilt $\gamma \geq \delta$ für jede weitere untere Schranke $\delta$ von $M$, so hei{\ss}t $\gamma$ das \textbf{Infimum} von $M$ (grö{\ss}te untere Schranke von $M$)
	\end{enumerate}
\end{definition}

\textbf{Bez.}: in dem Fall: $\gamma = \sup M$ bzw. $\gamma = \inf M$.

Aus \hyperref[a.axiom-a11]{$(A11)$} folgt: ist $\sup M$ bzw. $\inf M$ vorhanden, so ist $\sup M$ bzw. $\inf M$ eindeutig bestimmt.

Ist $\sup M$ bzw. $\inf M$ vorhanden und gilt $\sup M \in M$ bzw. $\inf M \in M$, so hei{\ss}t $\sup M$ das Maximum bzw. $\inf M$ das Minimum von $M$ und wird mit $\max M$ bzw. $\min M$ bezeichnet.


\begin{beispiele} ~\
	\begin{enumerate}
		\item $M = (1, 2)$. $\sup M = 2 \notin M$, $\inf M = 1 \notin M$. $M$ hat kein Maximum und kein Minimum.
		\item $M = (1, 2]$. $\sup M = 2 \in M$, $\max M = 2$
		\item $M = (3, \infty)$. $M$ ist nicht nach oben beschränkt, $3 = \inf M \notin M$.
		\item $M = (-\infty, 0]$. $M$ ist nach unten unbeschränkt, $0 = \sup M = \max M$.
	\end{enumerate}
\end{beispiele}


\index{Axiome!Vollständigkeits-}
\textbf{Vollständigkeitsaxiom:}
\begin{description} \addtolength{\itemindent}{0.4cm}
	\item[$(A15)$]Ist $\emptyset \neq M \subseteq \R$ und ist $M$ nach oben beschränkt, so ist $\sup M$ vorhanden. \label{v.axiom-a15}
\end{description}

\begin{satz} \label{1.1:satz}
	Ist $\emptyset \neq M \subseteq \R$ und ist $M$ nach unten beschränkt, so ist $\inf M$ vorhanden.
\end{satz} 

\begin{proof}
	i. d. Übungen.
\end{proof}

\index{beschränkt} 
\begin{definition}
	Sei $\emptyset \neq M \subseteq \R$. $M$ hei{\ss}t beschränkt $\iff$ $M$ ist nach oben und nach unten beschränkt ($\iff \exists c \geq 0 ~\forall x \in M \: |x| \leq c \iff \exists c \geq 0 ~\forall x \in M \: - c \leq x \leq c$)
\end{definition}


\begin{satz} \label{1.2:satz}
	Es sei $\emptyset \neq B \subseteq A \subseteq \R$
	\begin{enumerate}
		\item Ist $A$ beschränkt $\Rightarrow$ $\inf A \leq \sup A$
		\item Ist $A$ nach oben bzw. unten beschränkt $\Rightarrow$ $B$ ist nach oben beschränkt und $\sup B \leq \sup A$ bzw. nach unten beschränkt und $\inf B \geq \inf A$
		\item $A$ sei nach oben beschränkt und $\gamma$ eine obere Schranke von $A$. Dann:
			$$
				\gamma = \sup A \iff \forall \varepsilon > 0 ~\exists x = x(\varepsilon) \in A : x > \gamma - \varepsilon
			$$
		\item $A$ sei nach unten beschränkt und $\gamma$ eine untere Schranke von $A$. Dann:
			$$
				\gamma = \inf A \iff \forall \varepsilon > 0 ~\exists x = x(\varepsilon) \in A : x < \gamma + \varepsilon
			$$	
	\end{enumerate}

	\begin{proof} ~\ 
		\begin{enumerate}
			\item $A \neq \emptyset \Rightarrow \exists x \in \R : x \in A$. Dann $\inf A \leq x$, $x \leq \sup A$ $(A12)$
			$$ \Rightarrow \inf A \leq \sup A $$
			\item Sei $x \in B$. Dann: $x \in A$, also $x \leq \sup A$. $B$ ist also nach oben beschränkt und $\sup A$ ist eine obere Schranke von $B$
			$$ \Rightarrow \sup B \leq \sup A $$
			Analog der Fall für $A$ nach unten beschränkt.
			\item $"'\Rightarrow"'$ Sei $\gamma = \sup A$ und $\varepsilon > 0$. Dann: $\gamma - \varepsilon < \varepsilon$. $\gamma - \varepsilon$ ist also keine obere Schranke von $A$. Also: $\exists x \in A : x > \gamma - \varepsilon$ \\
				$"'\Leftarrow"'$ Sei $\tilde{\gamma} \leq \gamma$. Annahme: $\gamma \neq \tilde{\gamma}$. Dann $\tilde{\gamma} < \gamma$, also $\varepsilon \coloneqq \gamma - \tilde{\gamma} > 0$.\\
				$\xRightarrow[]{Vor.} \exists x \in A: x > \gamma - \varepsilon = \gamma- (\gamma - \tilde{\gamma}) = \tilde{\gamma}$. Widerspruch zu $x \leq \tilde{\gamma}$.
			\item Analog zu c)
		\end{enumerate}
	\end{proof}
\end{satz}

\subsection*{Natürliche Zahlen} 

\index{Natürliche Zahlen} \index{Induktionsmenge}
\begin{definition} ~\
	\begin{enumerate}
		\item $A \subseteq \R$ hei{\ss}t eine Induktionsmenge (IM) \\
		$$ \iff \begin{cases}1 . & 1 \in A; \\ 2. & \text{aus } x \in A \text{ folgt stets } x + 1 \in A \end{cases}$$ \\
		Beispiele: $\R$, $[1, \infty)$, $\{ 1 \} \cup [2, \infty)$ sind Induktionsmengen
		\item $\N \coloneqq \{ x \in \R : x$ gehört zu jeder IM $\}$ = Durchschnitt aller IMn. \\
			Also: $\N \subseteq A$ für jede Induktionsmenge $A$.
	\end{enumerate}	
\end{definition}

\begin{satz} ~\ \label{1.3:satz}
	\begin{enumerate}
		\item $\N$ ist eine Induktionsmenge
		\item $\N$ ist nicht nach oben beschränkt
		\item Ist $x \in \R$, so ex. ein $n \in \N: N > x$ \label{1.3.c:satz}
	\end{enumerate}
\end{satz}

Von nun an sei $\N = \{ 1, 2, 3, \dotsc \}$ bekannt.

\index{vollständige Induktion}
\begin{namedtheorem}[Prinzip der vollständigen Induktion] \label{1.4:prop} ~\\
	Ist $A \subseteq \N$ und $A$ eine Induktionsmenge, so ist $A = N$.
\end{namedtheorem}

\begin{proof}
	$A \subseteq \N$ (nach Vor.) und $\N \subset A$ (nach Def.), also $A = \N$
\end{proof}



\subsection*{Beweisverfahren durch vollständige Induktion}
$A(n)$ sei eine Aussage, die für jedes $n \in \N$ definiert ist. Für $A(n)$ gelte:
$$\begin{cases}
	(I) & A(1) \text{ ist wahr;} \\ (II) & \text{ist } n \in \N \text{ und } A(n) \text{ wahr, so ist auch } $A(n + 1)$ \text{ wahr;}
\end{cases}$$
Dann ist $A(n)$ wahr für \textbf{jedes} $n \in \N$!

\begin{proof}
	Sei $A \coloneqq \{ n \in \N : A(n)$ ist wahr $\}$. Dann: \\
	$A \subseteq \N$ und wegen $(I)$, $(II)$ ist $A$ eine Induktionsmenge $\xRightarrow[]{\ref{1.4:prop}} A = \N$
\end{proof}


\begin{beispiel*}
	Beh.: ~ $\underbrace{1 + 2 + \dotsc + n = \frac{n (n + 1)}{2}}_{A(n)}, ~\forall n \in \N$
	
	\begin{proof}(induktiv) \\
		I.A.: $1 = \frac{1 (1 + 1)}{2} \checkmark$, $A(1)$ ist also wahr. \\
		I.V.: Für ein $n \in \N$ gelte $1 + 2 + \dotsc + n = \frac{n (n + 1)}{2}$ \\
		I.S.: $n \curvearrowright n + 1$: 
		\begin{align*}
			1 + 2 + \dotsc + n + (n + 1) & \overset{I.V.}{=}  \frac{n (n + 1)}{2} + (n + 1) \\
									 	 & = (n + 1) \left( \frac{n}{2} + 1 \right) \\
									 	 & = \frac{(n + 1)(n + 2)}{2}
		\end{align*}
		$\Rightarrow A(n + 1)$ ist wahr.
	\end{proof}
\end{beispiel*}

\index{ganze Zahlen} \index{rationale Zahlen}
\begin{definition} ~\
	\begin{enumerate}
		\item $\N_{0} \coloneqq \N \cup \{ 0 \}$
		\item $\Z \coloneqq \N_{0} \cup \{ - n : n \in \N \}$ (ganze Zahlen)
		\item $\Q \coloneqq \{ \frac{p}{q} : p \in \Z, q \in \N \}$ (rationale Zahlen)
	\end{enumerate}
\end{definition}


\begin{satz} \label{1.5:satz}
	Sind $x, y \in \R$ und $x < y \Rightarrow \exists r \in \Q$: $x < r < y$	

	\begin{proof}
		i. d. Übungen.
	\end{proof}
\end{satz}

\index{Fakultät} \index{Binomialkoeffizient} \index{Binomischer Satz} \index{Bernoullische Ungleichung}
\subsection*{Einige Definitionen und Formeln} 
\begin{enumerate}
	\item Für $a \in \R$, $n \in \N: a^{n} \coloneqq \underbrace{a \cdot \dotsc \cdot a}_{n \text{ Faktoren}}$, $a^{0} \coloneqq 1$ und ist $a \neq 0: a^{-n} \coloneqq \frac{1}{a^{n}}$ \\
		Es gelten die bekannten Rechenregeln.
	\item Für $n \in \N: n! \coloneqq 1 \cdot 2 \cdot \dotsc \cdot n$, $0! \coloneqq 1$ (\textbf{Fakultäten})
	\item \textbf{Binomialkoeffizienten} (BK): für $n \in \N_{0}, k \in \N_{0}$ und $k \leq n$:
		$$
			\binom{n}{k} \coloneqq \frac{n!}{k!(n - k)!}
		$$
		z.B. $\binom{n}{0} = 1 = \binom{n}{n}$. Es gilt (nachrechnen!): \\
		$$
			\binom{n}{k} + \binom{n}{k - 1} = \binom{n + 1}{k} \quad \text{für } 1 \leq k \leq n
		$$
	\item Für $a, b \in \R$ und $n \in \N$ gilt: 
		\begin{align*}
			a^{n + 1} - b^{n + 1} & = (a - b) \left(a^{n} + a^{n-1}b + a^{n-2}b^{2} + \dotsc + a b^{n-1} + b^{n} \right) \\
				& = (a - b) \sum_{k = 0}^{n} a^{n -k}b^{k}
		\end{align*}
	\item \textbf{Binomischer Satz}: $a, b \in \R ~\forall n \in \N:$ $(a + b)^{n} = \sum_{k = 0}^{n} \binom{n}{k} a^{n-k}b^{k}$
		\begin{proof}
			i. d. Übungen.
		\end{proof}
	\item \textbf{Bernoullische Ungleichung}: Sei $x \in \R$ und $x \geq -1$. Dann:
		$$ (1 + x)^{n} \geq 1 + n x$$
		\begin{proof}(induktiv) \\
			I.A.: $n = 1$: $1 + x \geq 1 + x$ \\
			I.V.: Für ein $n \in \N$ gelte $(1 + x)^{n} \geq 1 + nx$ \\
			I.S.: $n \curvearrowright n + 1$: $\xRightarrow[]{I.V.} (1 + x)^{n} \geq 1 + n x$ und da $1 + x \geq 0$:
			\begin{align*}
				(1 + x)^{n + 1} & \geq (1 + nx)(1 + x) \\
								& = 1 + nx + x + \underbrace{nx^{n}}_{\geq 0} \\
								& \geq 1 + nx + x \\
								& = 1 + (n + 1)x
			\end{align*}
		\end{proof}
\end{enumerate}


\begin{hilfssatz*}[HS]
	Für $x, y \geq 0$ und $n \in \N$ gilt: $x \leq y \iff x^{n} \leq y^{n}$	

	\begin{proof}
		i. d. Übungen.
	\end{proof}
\end{hilfssatz*}

\index{Wurzel}
\begin{satz} \label{1.6:satz}
	Sei $a \geq 0$ und $n \in \N$. Dann gibt es genau ein $x \geq 0$ mit: $x^{n} = a$. \\
	Dieses $x$ hei{\ss}t \textbf{n-te Wurzel aus a}; Bez.: $x = \sqrt[n]{a}$. ($\sqrt[2]{a} \eqqcolon \sqrt{a}$)
	
	\begin{proof}
		Existenz: später in \S 7. \\
		Eindeutigkeit: seien $x, y \geq 0$ und $x^{n} = a = y^{n}$. $\xRightarrow[]{HS} x = y$
	\end{proof}
\end{satz}


\begin{bemerkungen} \
	\begin{enumerate}
		\item $\sqrt{2} \notin \Q$ (s. Schule)
		\item Für $a \geq 0$ ist $\sqrt[n]{a} \geq 0$. Bsp.: $\sqrt{4} = 2$, $\sqrt{4} \neq - 2$. Die Gleichung $x^{2} = 4$ hat zwei Lösungen: $x = \pm \sqrt{4} = \pm 2$.
		\item $\sqrt{x^{2}} |x|$ $\forall x \in \R$
	\end{enumerate}
\end{bemerkungen}


\subsection*{Rationale Exponenten}
\begin{enumerate}
	\item Sei zunächst $a > 0$ und $r \in \Q, r > 0$. Dann ex. $m, n \in \N : r = \frac{m}{n}$. Wir wollen definieren:
		$$
			a^{r} \coloneqq \left( \sqrt[n]{a} \right)^{m} \quad (*)
		$$
		Problem: gilt auch noch $r = \frac{p}{q}$ mit $p, q \in \N$, gilt dann $\left( \sqrt[n]{a} \right)^{m} = \left( \sqrt[q]{a} \right)^{p}$? \\
		Antwort: ja (d.h. obige Def. $(*)$ ist sinnvoll).
		\begin{proof}
			$x \coloneqq \left( \sqrt[n]{a} \right)^{m}$, $y \coloneqq \left( \sqrt[q]{a} \right)^{p}$, dann: $x, y \geq 0$ und $mq = np$, also
			\begin{align*}
				x^{q} & = \left( \sqrt[n]{a} \right)^{mq} = \left( \sqrt[n]{a} \right)^{np} = \left(  \left( \sqrt[n]{a} \right)^{m}\right)^{p} = a^{p} \\
					  & = \left( \left( \sqrt[q]{a} \right)^{q}\right)^{p} = \left( \left( \sqrt[q]{a} \right)^{p}\right)^{q} = y^{q}
			\end{align*}
			$\xRightarrow[]{HS} x = y$.  
		\end{proof}
	\item Sei $a > 0, r \in \Q$ und $r < 0$. $a^{r} \coloneqq \frac{1}{a^{-r}}$. Es gelten die bekannten Rechenregeln:
		$$
			\left( ~ a^{r} a^{s} = a^{r + s}, \left( a^{r} \right)^{s} = a^{rs}, \dotsc \right)
		$$
\end{enumerate}


\newpage


\section{Folgen und Konvergenz}

\index{Folge}
\begin{definition}
	Es sei $X$ eine Menge, $X \neq \emptyset$. Eine Funktion $a \colon \N \to X$ hei{\ss}t eine \textbf{Folge in X}. Ist $X = \R$, so hei{\ss}t $a$ eine \textbf{reelle Folge}.
\end{definition}


\begin{schreibweisen}
$a_{n}$ statt $a(n)$ (n-tes Folgenglied) \\
$(a_{n})$ oder $(a_{n})_{n = 1}^{\infty}$ oder $(a_{1}, a_{2}, \dotsc)$ statt $a$
\end{schreibweisen}


\begin{beispiele} ~\
	\begin{enumerate}
		\item $a_{n} \coloneqq \frac{1}{n}$ $~(n \in \N)$, also $(a_{n}) = (1, \frac{1}{2}, \frac{1}{3}, \dotsc)$
		\item $a_{2n} \coloneqq 0$, $a_{2n-1} \coloneqq 1$ $~(n \in \N)$, also $(a_{n}) = (1, 0, 1, 0, \dotsc)$
	\end{enumerate}
\end{beispiele}


\begin{bemerkung}
	Ist $p \in \Z$ und $a \colon \{ p, p + 1, \dotsc \} \to X$ eine Funktion, so spricht man ebenfalls von einer Folge in $X$. Bez.: $(a_{n})_{n = p}^{\infty}$. Meist $p = 0$ oder $p = 1$.
\end{bemerkung}

\index{abzählbar} \index{uberabzahlbar@überabzählbar}
\begin{definition}
	Sei $X$ eine Menge, $X \neq \emptyset$.
	\begin{enumerate}
		\item $X$ hei{\ss}t \textbf{abzählbar} $\iff \exists$ Folge $(a_{n})$ in $X$: $X = \{ a_{1}, a_{2}, a_{3}, \dotsc \}$
		\item $X$ hei{\ss}t \textbf{überabzählbar} $\iff X$ ist nicht abzählbar
	\end{enumerate}
\end{definition}


\begin{beispiele} ~\
	\begin{enumerate}
		\item Ist $X$ endlich, so ist $X$ abzählbar.
		\item $\N$ ist abzählbar, denn $\N = \{ a_{1}, a_{2}, a_{3}, \dotsc \}$ mit $a_{n} \coloneqq n$ $(n \in \N)$
		\item $\Z$ ist abzählbar, denn $\Z = \{ a_{1}, a_{2}, a_{3}, \dotsc \}$ mit $a_{1} \coloneqq 0, a_{2} \coloneqq 1, a_{3} \coloneqq -1, a_{4} \coloneqq 2, a_{5} \coloneqq -2, \dotsc$ also
			$$ a_{2n} \coloneqq n, \quad a_{2n + 1} \coloneqq -n \quad (n \in \N) $$
		\item $\Q$ ist abzählbar!
			\begin{figure*}[!ht] \centering
				\begin{tikzpicture}
					\matrix(m)[matrix of math nodes,column sep=1cm,row sep=1cm]{1 & 2 & 3 & 4 & 5 & 6 & \cdots \\
    					\frac{1}{2} & \frac{2}{2} & \frac{3}{2} & \frac{4}{2} & \frac{5}{2} & \cdots & \cdots \\
    					\frac{1}{3} & \frac{2}{3} & \frac{3}{3} & \frac{4}{3} & \frac{5}{3} & \cdots \\
    					\frac{1}{4} & \frac{2}{4} & \frac{3}{4} & \frac{4}{4} & \cdots \\
    					\frac{1}{5} & \frac{2}{5} & \cdots & \cdots \\
    					\cdots & \cdots &  \\
					};
					\draw[->]
					(m-1-1)edge(m-1-2) (m-1-2)edge(m-2-1) (m-2-1)edge(m-3-1) (m-3-1)edge(m-2-2) (m-2-2)edge(m-1-3) (m-1-3)edge(m-1-4) (m-1-4)edge(m-2-3) (m-2-3)edge(m-3-2) (m-3-2)edge(m-4-1) (m-4-1)edge(m-5-1) (m-5-1)edge(m-4-2) (m-4-2)edge(m-3-3) (m-3-3)edge(m-2-4) (m-2-4)edge(m-1-5) (m-1-5)edge(m-1-6); 
				\end{tikzpicture}
    			\caption{Zum Beweis der Abzählbarkeit von $\Q$.}
			\end{figure*} \\
			Durchnummerieren in Pfeilrichtung liefert:
				$$ \{ x \in \Q : x > 0 \} = \{ a_{1}, a_{2}, a_{3}, \dotsc \} $$
			$b_{1} \coloneqq 0, b_{2n} \coloneqq a_{n}, b_{2n + 1} \coloneqq - a_{n}$ $(n \in \N)$. Dann:
			$$ \Q = \{ b_{1}, b_{2}, b_{3}, \dotsc \} $$
		\item $\R$ ist überabzählbar (Beweis in \S 5).
	\end{enumerate}	
\end{beispiele}


\begin{vereinbarung}
	Solange nichts anderes gesagt wird, seien alle vorkommenden Folgen stets Folgen in $\R$. 
	                                                                                               
	Die folgenden Sätze und Definitionen formulieren wir nur für Folgen der Form $(a_{n})_{n=1}^{\infty}$. Sie gelten sinngemä{\ss} für Folgen der Form $(a_{n})_{n=p}^{\infty}$ $(p \in \Z)$.
\end{vereinbarung}

\index{beschränkt!Folge}
\begin{definition}
	Sei $(a_{n})$ eine Folge und $M \coloneqq \{ a_{1}, a_{2}, \dotsc \}$.
	\begin{enumerate}
		\item$(a_{n})$ hei{\ss}t \textbf{nach oben beschränkt} $\iff M$ ist nach oben beschränkt. I.d. Fall: $\sup_{n \in \N} a_{n} \coloneqq \sup_{n = 1}^{\infty} a_{n} \coloneqq \sup M$.
		\item$(a_{n})$ hei{\ss}t \textbf{nach unten beschränkt} $\iff M$ ist nach unten beschränkt. I.d. Fall: $\inf_{n \in \N} a_{n} \coloneqq \inf_{n = 1}^{\infty} a_{n} \coloneqq \inf M$.
		\item$(a_{n})$ hei{\ss}t \textbf{beschränkt} $~\iff M$ ist beschränkt 
			$$ \iff \exists c \geq 0: |a_{n}| \leq c ~\forall n \in \N $$
	\end{enumerate}
\end{definition}

\index{für fast alle}
\begin{definition}
	Sei $A(n)$ eine für jedes $n \in \N$ definierte Aussage. \\
	$A(n)$ gilt \textbf{für fast alle} (ffa) $n \in \N$ $\iff \exists n_{0} \in \N: A(n)$ ist wahr $\forall n \geq n_{0}$
\end{definition}

\index{Umgebung}
\begin{definition}
	Sei $a \in \R$ und $\varepsilon > 0$
		$$ U_{\varepsilon}(a) \coloneqq (a - \varepsilon, a + \varepsilon) = \{ x \in \R : | x - a| < \varepsilon \} $$
	hei{\ss}t $\varepsilon$\textbf{-Umgebung von a}.
\end{definition}

\index{konvergent} \index{Grenzwert} \index{Limes} \index{divergent}
\begin{definition}
	Eine Folge $(a_{n})$ hei{\ss}t \textbf{konvergent}
	$$ \iff \exists a \in \R : \begin{cases} \text{zu jedem } \varepsilon > 0 \text{ ex. } n_{0} = n_{0}(\varepsilon) \in \N: \\
		|a_{n} - a| < \varepsilon ~\forall n \geq n_{0}
	\end{cases} $$
	I. d. Fall hei{\ss}t $a$ \textbf{Grenzwert} (GW) oder \textbf{Limes} von $(a_{n})$ und man schreibt
	$$ 
		a_{n} \rightarrow a ~(n \rightarrow \infty) \text{ oder } a_{n} \rightarrow a \text{ oder } \lim_{n \rightarrow \infty} a_{n} = a
	$$
	Ist $(a_{n})$ nicht konvergent, so hei{\ss}t $(a_{n})$ \textbf{divergent}. Beachte:
	\begin{align*}
		a_{n} \rightarrow a ~(n \rightarrow \infty) & \iff \forall \varepsilon > 0 ~\exists n_{0} \in \N: a_{n} \in U_{\varepsilon}(a) ~\forall n \geq n_{0} \\
				& \iff \forall \varepsilon > 0 \text{ gilt: } a_{n} \in U_{\varepsilon}(a) \text{ ffa } n \in \N \\
				& \iff \forall \varepsilon > 0 \text{ gilt: } a_{n} \notin U_{\varepsilon}(a) \text{ für höchstens endlich viele } n \in \N
	\end{align*}
\end{definition}


\begin{satz} \label{2.1:satz}
	$(a_{n})$ sei konvergent und $a = \lim a_{n}$
	\begin{enumerate}
		\item Gilt auch noch $a_{n} \rightarrow b$, so ist $a = b$
		\item $(a_{n})$ ist beschränkt \label{2.1.b:satz}
	\end{enumerate}
	
	\begin{proof}\
	  \begin{enumerate}
		\item Annahme $a \neq b$. Dann ist $\varepsilon \coloneqq \frac{|a - b|}{2} > 0$.
			$$
			\exists n_{0} \in \N: |a_{n_{0}} - a| < \varepsilon \quad \forall n \geq n_{0} \text{ und } \exists n_{1} \in \N: |a_n - b| < \varepsilon \quad \forall n \geq n_{1}
			$$
			$N \coloneqq \max \{ n_{0}, n_{1} \}$. Dann:
			$$
				2 \varepsilon = |a - b| = | a - a_{N} + a_{N} - b| \leq |a_{N} - a| + |a_{N} - b| < 2 \varepsilon
			$$
			Widerspruch! Also $ a = b$
		\item  Zu $\varepsilon = 1 ~\exists n_{0} \in \N: |a_{n} - a| < 1 ~\forall n \geq n_{0}$. Dann:
			$$
				|a_{n}| = |a_{n} - a + a| \leq |a_{n} - a| + |a| \leq 1 + |a| \quad \forall n \geq n_{0}
			$$
			$c \coloneqq \max \{ 1 + |a|, |a_{1}|, \dotsc, |a_{n_{0} - 1}| \}$. Dann: $|a_{n}| \leq \varepsilon ~\forall n \geq 1$.
	  \end{enumerate}
	\end{proof}	
\end{satz}


\begin{beispiele}\
	\begin{enumerate}
		\item Sei $c \in \R$ und $a_{n} \coloneqq c ~\forall n \in \N$. Dann:
			$$
				| a_{n} - c | = 0 \quad \forall n \in \N
			$$
			Also: $a_{n} \rightarrow c$.
		\item $a_{n} \coloneqq \frac{1}{n} ~(n \in \N)$. Beh: $a_{n} \rightarrow 0 ~(n \rightarrow \infty)$.
			\begin{proof}
				Sei $\varepsilon > 0: |a_{n} - 0 | = |a_{n}| = \frac{1}{n} < \varepsilon \iff n > \frac{1}{\varepsilon}$
				$$
						\xRightarrow[]{\ref{1.3.c:satz}} \exists n_{0} \in \N: n_{0} > \frac{1}{\varepsilon}
				$$
				Für $n \geq n_{0}$ ist $n > \frac{1}{\varepsilon}$, also $\frac{1}{n} < \varepsilon$. Somit $|a_{n} - 0| < \varepsilon ~\forall n \geq n_{0}$
			\end{proof}
		\item $a_{n} \coloneqq (-1)^{n}$. Es ist $\forall n \in \N$: $|a_{n}| = 1$, $(a_{n})$ ist also beschränkt. Behauptung: $(a_{n})$ ist divergent.
			\begin{proof}
				Für alle $n \in \N$: 
				$$ |a_{n} - a_{n+1}| = |(-1)^{n} - (-1)^{n+1}| = |(-1)^{n}| \left( 1 - (-1) \right) = 2 $$
				Annahme: $(a_{n})$ konvergiert. Definiere $a \coloneqq \lim a_{n}$, dann 
				$$
					 \exists n_{0} \in \N: ~ |a_{n} - a| < \frac{1}{2} \quad \forall n \geq n_{0}
				$$
				Für $n \geq n_{0}$ gilt dann aber:
				$$
					2 = |a_{n} - a_{n+1}| = |a_{n} - a + a - a_{n + 1}| \leq |a_{n} - a| + |a_{n+1} - a| < \frac{1}{2} + \frac{1}{2} = 1
				$$
				Widerspruch!
			\end{proof}
		\item $a_{n} \coloneqq n ~(n \in \N)$. $(a_{n})$ ist nicht beschränkt $\xRightarrow[]{\ref{2.1.b:satz}} (a_{n})$ ist divergent.
		\item $a_{n} \coloneqq  \frac{1}{\sqrt{n}} (n \in \N)$. Beh.: $a_{n} \rightarrow 0$
			\begin{proof}
				Sei $\varepsilon > 0$.
				$$
					|a_{n} - 0| = \frac{1}{\sqrt{n}} < \varepsilon \iff \sqrt{n} > \frac{1}{n} \iff n > \frac{1}{\varepsilon^{2}}
				$$
				$\xRightarrow[]{\ref{1.3.c:satz}} \exists n_{0} \in \N: n_{0} > \frac{1}{\varepsilon^{2}}$. Ist $n \geq n_{0} \Rightarrow n > \frac{1}{\varepsilon^{2}} \Rightarrow \frac{1}{\sqrt{n}} < \varepsilon \Rightarrow |a_{n} - 0 | < \varepsilon$ 
			\end{proof}
		\item $a_{n} \coloneqq \sqrt{n + 1} - \sqrt{n}$. 
			\begin{proof}
				$$
					a_{n} = \frac{(\sqrt{n + 1} - \sqrt{n})(\sqrt{n + 1} + \sqrt{n})}{\sqrt{n + 1} + \sqrt{n}} = \frac{1}{\sqrt{n + 1} + \sqrt{n}} \leq \frac{1}{\sqrt{n}}
				$$
				$\Rightarrow |a_{n} - 0| \leq \frac{1}{\sqrt{n}} ~\forall n \in \N$. Sei $\varepsilon > 0$, nach Beispiel e) folgt:
				$$
					\exists n_{0} \in \N: ~ \frac{1}{\sqrt{n}} < \varepsilon \quad \forall n \geq n_{0} \Rightarrow |a_{n} - 0| < \varepsilon \quad \forall n \geq n_{0}
				$$
				Also $a_{n} \rightarrow 0$.
			\end{proof}
	\end{enumerate}
\end{beispiele}


\begin{definition}
	$(a_{n})$ und $(b_{n})$ seien Folgen und $\alpha \in \R$
	$$
		(a_{n}) \pm (b_{n}) \coloneqq (a_{n} \pm b_{n}); ~
		\alpha (a_{n}) \coloneqq (\alpha a_{n}); ~
		(a_{n}) (b_{n}) \coloneqq (a_{n} b_{n}) 		
	$$	
	Gilt $\forall n \geq m$ $b_{n} \neq 0$, so ist die Folge $\left( \frac{a_{n}}{b_{n}} \right)_{n = m}^{\infty}$ definiert.
\end{definition}


\begin{satz} \label{2.2:satz}
	$(a_{n}), (b_{n}), (c_{n})$ und $(\alpha_{n})$ seien Folge und $a, b, \alpha \in \R$

	\begin{enumerate}
		\item $a_{n} \rightarrow a \iff |a_{n} - a| \rightarrow 0$
		\item Gilt $|a_{n} - a| \leq \alpha_{n}$ ffa $n \in \N$ und $\alpha_{n} \rightarrow 0$, so gilt $a_{n} \rightarrow a$
		\item Es gelte $a_{n} \rightarrow a$ und $b_{n} \rightarrow b$. Dann:
			\begin{enumerate}
				\item $|a_{n}| \rightarrow |a|$ 
				\item $a_{n} + b_{n} \rightarrow a + b$
				\item $\alpha a_{n} \rightarrow \alpha a$
				\item $a_{n} b_{n} \rightarrow a b$
				\item ist $a \neq 0$, so ex. ein $m \in \N$:
					$$
						a_{n} \neq 0 ~\forall n \geq m \text{ und für die Folge } \left( \frac{1}{a_{n}} \right)_{n = m}^{\infty} \text{ gilt: } \frac{1}{a_{n}} \rightarrow \frac{1}{a}
					$$
			\end{enumerate}
		\item Es gelte $a_{n} \rightarrow a$, $b_{n} \rightarrow b$ und $a_{n} \leq b_{n}$ ffa $n \in \N \Rightarrow a \leq b$
		\item Es gelte $a_{n} \rightarrow a$, $b_{n} \rightarrow a$ und $a_{n} \leq c_{n} \leq b_{n}$ ffa $n \in \N$. Dann $c_{n} \rightarrow a$. \label{2.2.e:satz}
	\end{enumerate}
\end{satz}

\begin{beispiele} \
	\begin{enumerate}
		\item Sei $p \in \N$ und $a_{n} \coloneqq \frac{1}{n^{p}}$. Es ist $n \leq n^{p} ~\forall n \in \N$. \\
			Dann: $0 \leq a_{n} \leq \frac{1}{n} ~\forall n \in \N \xRightarrow[]{\label{2.2.e:satz}} a_{n} \rightarrow 0$, also $\frac{1}{n^{p}} \rightarrow 0$.
		\item $a_{n} \coloneqq \frac{5n^{2} + 3n + 1}{4n^{2} - n + 2} = \frac{5 + \frac{3}{n} + \frac{1}{n^{2}}}{4 - \frac{1}{n} + \frac{2}{n^{2}}} \xrightarrow[]{\ref{2.2:satz}} \frac{5}{4}$
	\end{enumerate}
	
	\begin{proof}(von 2.2) ~\
		\begin{enumerate}
			\item folgt aus der Definition der Konvergenz
			\item $\exists m \in \N: |a_{n} - a | \leq \alpha_{m} ~\forall n \geq m$. Sei $\varepsilon > 0$
				$$
		 		\exists n_{1} \in \N: \alpha_{n} < \varepsilon ~\forall n \geq n_{1}.
		 		$$
		 		$n_{0} \coloneqq \max \{ m , n_{1} \}$. Für $n \geq n_{0}$: $|a_{n} - a| \leq \alpha_{n} < \varepsilon$
			\item \begin{enumerate}
				\item $| |a_{n}| - |a|| \overset{\S 1}{\leq} |a_{n} - a| ~\forall n \in \N \xRightarrow[]{a), b)} |a_{n}| \rightarrow |a|$
				\item Sei $\varepsilon > 0$. $\exists n_{1}, n_{2} \in \N; |a_{n} - a| < \frac{\varepsilon}{2} ~\forall n \geq n_{1}$, $|b_{n} - b| < \frac{\varepsilon}{2} ~\forall n \geq n_{2}$ \\
					$n_{0} \coloneqq \max \{ n_{1}, n_{2} \}$. Für $n \geq n_{0}$:
					$$
						|a_{n} + b_{n} - (a + b)| = |a_{n} - a + b_{n} - b| \leq |a_{n} - a| + |b_{n} - b| < \frac{\varepsilon}{2} + \frac{\varepsilon}{2} = \varepsilon
					$$
				\item Übung
				\item $c_{k} \coloneqq |a_{n} b_{n} - ab|$. z. z.: $c_{n} \rightarrow 0$
					\begin{align*}
						c_{n} & = |a_{n}b_{n} - a_{n}b + a_{n}b - ab| = |a_{n}(b_{n} - b)+ (a_{n} - a)b| \\
							  & \leq |a_{n}||b_{n} - b| + |b||a_{n}-a|
					\end{align*}
					$\xRightarrow[]{\ref{2.1.b:satz}} \exists c \geq 0 : |a_{n}| \leq c ~\forall n \in \N$ und $c \geq |b|$. Dann:
					$$
						c_{n} \leq c(|b_{n}-b| + |a_{n}-a|) \eqqcolon \alpha_{n} \xRightarrow[c) (ii), c) (iii)]{a)} \alpha_{n} \rightarrow 0
					$$
					Also: $|c_{n} - 0| = c_{n} \leq \alpha_{n} ~\forall n \in \N$ und $\alpha_{n} \rightarrow 0 \xRightarrow[]{b)} c_{n} \rightarrow 0$.
				\item $\varepsilon \coloneqq \frac{|a|}{2}$; aus (i): $|a_{n}| \rightarrow |a| \Rightarrow \exists n \in N$:
					$$
						 |a_{n}| \in U_{\varepsilon}(|a|) = (|a| - \varepsilon, |a| + \varepsilon) = (\frac{|a|}{2}, \frac{3}{2} |a|) \quad \forall n \geq m
					$$
					$\Rightarrow |a_{n}| > \frac{|a|}{2} > 0 ~\forall n \geq m \Rightarrow a_{n} \neq 0 ~\forall n \geq m$. \\
					Für $n \geq m$:
					$$
						\left| \frac{1}{a_{n}} - \frac{1}{a} \right| = \frac{|a_{n} - a|}{|a_{n}||a|} \leq \frac{2|a_{n} - a|}{|a|^{2}} \eqqcolon \alpha_{n}
					$$
					$\alpha_{n} \rightarrow 0 \xRightarrow[]{b)} \frac{1}{a_{n}} \rightarrow \frac{1}{a}$.
			  \end{enumerate}
			\item Annahme $b < a$, $\varepsilon \coloneqq \frac{a-b}{2} > 0$ ~ \tikz[baseline=-0.5ex]{  \draw(0,0)--(8,0);
    				\foreach \x/\xtext in {0/$$,1.75/$$,3/$b$,4.5/$$,6.25/{\small $a$},8/$$}
     					\draw(\x,3pt)--(\x,-3pt) node[below] {\xtext};
      				\foreach \x/\xtext in {4/$x$,5.5/$y$}
      						\draw(\x,2pt)--(\x,-2pt) node[below] {\xtext};
    				\draw[decorate,decoration={brace},yshift=2ex]  (1.75,0) -- node[above=0.4ex] {\small $U_{\varepsilon}(b)$}  (4.25,0);
   					\draw[decorate,decoration={brace},yshift=2ex] (5.25,0) -- node[above=0.4ex] {\small $U_{\varepsilon}(a)$} (7.25,0);
   				} \\
				Dann: $x < y ~\forall x \in U_{\varepsilon}(b) ~\forall y \in U_{\varepsilon}(a)$. \\
				$$ \exists n_{0} \in \N: b_{n} \in U_{\varepsilon}(b) ~\forall n \geq n_{0} $$
				$$ \exists m \in \N: a_{n} \leq b_{n} ~\forall n \geq m $$
				$m_{0} \coloneqq \max \{ n_{0}, m \}$. Für $n \geq m_{0}$: $a_{n} \leq b_{n} < b + \varepsilon$, also $a_{n} \notin U_{\varepsilon}(a)$. Widerspruch!   
			\item $\exists m \in \N: a_{n} \leq c_{n} \leq b_{n} ~\forall n \geq m$. Sei $\varepsilon > 0$. $\exists n_{1}, n_{2} \in \N$: 
				\begin{align*}
					a - \varepsilon < a_{n} < a + \varepsilon ~\forall n \geq n_{1} \\
					a - \varepsilon < b_{n} < a + \varepsilon ~\forall n \geq n_{2}
				\end{align*}
				$n_{0} \coloneqq \max \{ n_{1}, n_{2}, m \}$. Für $n \geq n_{0}$:
				$$
					a - \varepsilon < a_{n} \leq c_{n} \leq b_{n} < a + \varepsilon
				$$
				Also: $|a_{n} - a| < \varepsilon \forall n \geq n_{0}$.
		\end{enumerate}	
	\end{proof}	
\end{beispiele}

\index{monoton}  \index{monoton! wachsend}   \index{monoton! streng wachsend} \index{monoton! streng fallend} \index{monoton! fallend}
\begin{definition}\ 
	\begin{enumerate}
		\item $(a_{n})$ hei{\ss}t \textbf{monoton wachsend} $\iff a_{n+1} \geq a_{n} ~\forall n \in \N$.
		\item $(a_{n})$ hei{\ss}t \textbf{streng monoton wachsend} $\iff a_{n+1} > a_{n} ~\forall n \in \N$.
		\item Entsprechend definiert man \textbf{monoton fallend} und \textbf{streng monoton fallend}.
		\item $(a_{n})$ hei{\ss}t \textbf{monoton} $\iff (a_{n})$ ist monoton wachsend oder monoton fallend.
	\end{enumerate}
\end{definition}

\index{Monotoniekriterium}
\begin{namedtheorem}[Monotoniekriterium] ~\ \label{2.3:prop}
	\begin{enumerate}
		\item $(a_{n})$ sei monoton wachsend und nach oben beschränkt. Dann ist $(a_{n})$ konvergent und 
			$$
				\lim_{n \rightarrow \infty} a_{n} = \sup_{n = 1}^{\infty} a_{n}
			$$
		\item $(a_{n})$ sei monoton fallend und nach unten beschränkt. Dann ist $(a_{n})$ konvergent und 
			$$
				\lim_{n \rightarrow \infty} a_{n} = \inf_{n = 1}^{\infty} a_{n}
			$$
	\end{enumerate}
\end{namedtheorem}

\begin{proof}
		$a \coloneqq \sup_{n = 1}^{\infty} a_{n}$. Sei $\varepsilon > 0$. Dann ist $a - \varepsilon$ keine obere Schranke von $\{ a_{1}, a_{2}, \cdots \}$, also existiert ein $n_{0} \in \N: a_{n_{0}} > a - \varepsilon$. Für $n \geq n_{0}$:
			$$
				a - \varepsilon < a_{n_{0}} \leq a_{n} \leq a \leq a + \varepsilon
			$$
			also $|a_{n} - a| \leq \varepsilon ~\forall n \geq n_{0}$.
\end{proof}

\begin{figure*}[!ht] \centering
	\begin{tikzpicture}
      \draw[->] (-0.5,0) -- (6,0) node[right] {$x$};
      \draw[->] (0,-0.5) --  (0,4) node[above] {$y$};
      \draw[-] (-0.5,3.25) --  (6,3.25) node[above] {$a$};
      \draw[dotted] (-0.5,2.75) --  (6,2.75) node[below] {$a - \epsilon$};
      \draw[scale=1,domain=0:6,loosely dotted,variable=\x,thick] plot ({\x},{4*(5*\x/(5*\x+10))+0.1});
    \end{tikzpicture}
    \caption{Zum Beweis des Monotonie-Kriteriums.}
\end{figure*}


\begin{beispiel*} $a_{1} \coloneqq \sqrt[3]{6}$, $a_{n + 1} \coloneqq \sqrt[3]{6 + a_{n}}$ $(n \geq 2)$.
	\begin{description}
		\item $a_{1} = \sqrt[3]{6} < \sqrt[3]{8} = 2$;
		\item $a_{2} = \sqrt[3]{6 + a_{1}} < \sqrt[3]{6 + 2} = 2$;
		\item $a_{2} = \sqrt[3]{6 + a_{1}} < \sqrt[3]{6} = a_{1}$;
	\end{description}
	Behauptung: $0 < a_{n} < 2$ und $a_{n + 1} > a_{n}$ $\forall n \in \N$

	\begin{proof}(induktiv) \\
		I.A.: s.o. \\
		I.V.: Sei $n \in \N$ und $0 < a_{n} < 2$ und $a_{n+1} > a_{n}$. \\
		$n \curvearrowright n + 1$: $a_{n + 1} = \sqrt[3]{6 + a_{n}} >_{I.V.} 0$
		$$
			a_{n +1} = \sqrt[3]{6 + a_{n}} <_{I.V.} \sqrt[3]{6 + 2} = 2; \quad a_{n + 2} = \sqrt[3]{6 + a_{n+1}} >_{I.V.} \sqrt[3]{6 + a_{n}} = a_{n + 1}
		$$
		Also: $(a_{n})$ ist nach oben beschränkt und monoton wachsend. \\
		$\xRightarrow[]{\ref{2.3:prop}} (a_{n})$ ist konvergent. $a \coloneqq \lim a_{n}$, $a_{n} \geq 0 ~\forall n \xRightarrow[]{\ref{2.2:satz}} a \geq 0$. Es ist
		$$
			a_{n+1}^{3} = 6 + a_{n} \quad \forall n \in \N
		$$
		$\xRightarrow[]{\ref{2.2:satz}} a^{3} = 6 + a \Rightarrow 0 = a^{3} - a + 6 = (a-2)(\underbrace{a^{2}-2a+3}_{\geq 3}) \Rightarrow a = 2$.
	\end{proof}
\end{beispiel*}


\textbf{Wichtige Beispiele:} \\

Vorbemerkung: Seien $x, y \geq 0$ und $p \in \N$: es ist (s. \S 1)
$$
	x^{p} - y^{p} = (x - y) \sum_{k = 0}^{p-1} x^{p-1-k}y^{k}
$$
$\Rightarrow |x^{p} - y^{p}| = |x-y| \sum_{k=0}^{p-1} x^{p-1-k}y^{k} \geq y^{p-1} |x - y|$
\newline

\begin{beispiel} \label{2.4:bsp}
	Sei $a_{n} \geq 0 ~\forall n \in \N$: $a_{n} \rightarrow a \geq 0$ und $p \in \N$. Dann $\sqrt[p]{a_{n}} \rightarrow \sqrt[p]{a}$
	
	\begin{proof} ~\\
		Fall 1: $a = 0$. Sei $\varepsilon > 0, \exists n_{0} \in \N: |a_{n}| < \varepsilon^{p} ~\forall n \geq n_{0}$
		$$
			\Rightarrow | \sqrt[p]{a_{n}} = \sqrt[p]{|a_{n}|} < \varepsilon ~\forall n \geq n_{0}
		$$
		Also $\sqrt[p]{a_{n}} \rightarrow 0$. \newline
		
		Fall 2: $a \neq 0$.
		\begin{align*}
			|a_{n} - a| & = | (\underbrace{\sqrt[p]{a_{n}}}_{\eqqcolon x})^{p} - |\underbrace{\sqrt[p]{a}}_{\eqqcolon y}|^{p} | =  |x^{p} - y^{p}| \\
					& \geq_{s.o.} \underbrace{y^{p-1}}_{\coloneqq c} |x - y| = c | \sqrt[p]{a_{n}} - \sqrt[p]{a} |, \quad c > 0
		\end{align*}
		$\Rightarrow |\sqrt[p]{a_{n}} - \sqrt[p]{a}| \leq \frac{1}{c} |a_{n} - a| \eqqcolon \alpha_{n}$. $\alpha_{n} \rightarrow 0 \Rightarrow \sqrt[p]{a_{n}} \rightarrow \sqrt[p]{a}$
	\end{proof} 
\end{beispiel}


\begin{beispiel} \label{2.5:bsp}
	Für $x \in \R$ gilt $(x^{n})$ ist konvergent $\iff x \in (-1,1]$, i. d. Fall:
	$$
		\lim_{n \rightarrow \infty} x^{n} = \begin{cases} 1, & \text{falls } x = 1 \\ 0, & \text{falls } x \in (-1 , 1) \end{cases}
	$$
	
	\begin{proof} ~\\
		Fall 1: $x = 0$. Dann $x^{k} \rightarrow 0$. Fall 2: $x = 1$. Dann $x^{k} \rightarrow 1$. \\
		Fall 3: $x = -1$. Dann $(x^{k}) = ((-1)^{k})$, ist divergent. \\
		Fall 4: $|x| > 1$. $\exists \delta > 0: |x| = 1 + \delta \Rightarrow |x^{k}| = |x|^{k} = (1 + \delta)^{k} \geq 1 + n \delta \geq n \delta$ \\
		 $\Rightarrow$ ist nicht beschränkt $\xRightarrow[]{2.1} (x^{k})$ ist divergent.
		Fall 5: $0 < |x| < 1 \Rightarrow \frac{1}{|x|} > 1 \Rightarrow \exists \eta > 0: \frac{1}{|x|} = 1 + \eta$.
		$$
			\Rightarrow |\frac{1}{x^{n}}| = \left( \frac{1}{|x|} \right)^{n} = (1 + \eta)^{n} \geq 1 + n \eta \geq n \eta
		$$
		$\Rightarrow |x^{n}| \leq \frac{1}{\eta} \cdot \frac{1}{n} \Rightarrow x^{n} \rightarrow 0$.
	\end{proof}	
\end{beispiel}


\begin{beispiel} \label{2.6:bsp}
	Sei $x \in \R$ und $s_{n} \coloneqq 1 + x + x^{n} + \dotsc x^{n} = \sum_{k = 0}^{n} x^{k}$ \\
	Fall 1: $x = 1$. Dann: $x_{n} = n + 1$, $(s_{n})$ ist also divergent. \\
	Fall 2: $x \neq 1 \Rightarrow s_{n} = \frac{1 - x^{n+1}}{1 - x}$. Aus \ref{2.5:bsp}:
	$$
		(s_{n}) \text{ konvergent} \quad \iff \quad |x| < 1
	$$
	i.d. Fall: $\lim s_{n} = \frac{1}{1 - x}$
\end{beispiel}


\begin{beispiel} \label{2.7:bsp}
	Behauptung: $\sqrt[n]{n} \rightarrow 1$.
	
	\begin{proof}
		Es ist $\sqrt[n]{n} \geq 1 ~\forall n \in \N$, also $a_{n} \coloneqq \sqrt[n]{n} - 1 \geq 0 ~\forall n \in \N$. Z. z.: $a_{n} \rightarrow 0$. \\
		Für $n \geq 2$:
		$$
			n = \left( \sqrt[n]{n} \right)^{n} = \left( a_{n} + 1 \right)^{n} \overset{\S 1}{=} \sum_{k=0}^{n} \binom{n}{k} a_{n}^{k} \geq \binom{n}{2} a_{n}^{2} = \frac{n(n-1)}{2} a_{n}^{2}
		$$
		$\Rightarrow \frac{n-1}{2}a_{n}^{2} \leq 1$. Also $\xRightarrow[]{a_{n} \geq 0} 0 \leq a_{n} \leq \frac{\sqrt{2}}{\sqrt{n-1}} (n \geq 2)$. $\Rightarrow a_{n} \rightarrow 0$.
	\end{proof}
\end{beispiel}


\begin{beispiel} \label{2.8:bsp}
	Sei $c > 0$. Beh.: $\sqrt[n]{c} \rightarrow 1$.
	
	\begin{proof}
		Fall 1: $c \geq 1$. $\exists m \in \N: 1 \leq c \leq m$
		$$
		 \Rightarrow 1 \leq c \leq n ~\forall n \geq m \Rightarrow 1 \leq \sqrt[n]{c} \leq \sqrt[n]{n} ~\forall n\geq m \Rightarrow \text{Beh.}
		$$
		Fall 2: $0 < c <1 \Rightarrow \frac{1}{c} > 1 \Rightarrow \sqrt[n]{c} = \frac{1}{\sqrt[n]{\frac{1}{c}}} \xrightarrow[Fall 1]{} 1 (n \rightarrow \infty)$ $\Rightarrow$ Beh.
	\end{proof}
\end{beispiel}


\begin{beispiel} \label{2.9:bsp}
	$a_{n} \coloneqq \left( 1 + \frac{1}{n} \right)^{n}; b_{n} \coloneqq \sum_{k = 0}^{n} \frac{1}{k!} = 1 + 1 + \frac{1}{2!} + \dotsc + \frac{1}{n!}$ \\
	Beh.: $(a_{n})$ und $(b_{n})$ sind konvergent und $\lim a_{n} = \lim b_{n}$
	
	\begin{proof}
		I. d. gr. Übungen wird gezeigt: $2 \leq a_{n} < a_{n+1} < 3 ~\forall n \in \N$
		$$
			\xRightarrow[]{\ref{2.3:prop}} (a_{n}) \text{ konvergiert, } a \coloneqq \lim a_{n}
		$$
		Es ist $b_{n} > 0$ und $b_{n+1} = b_{n} + \frac{1}{(n+1)!} > b_{n}$. $(b_{n})$ ist also monoton wachsend. Für $n > 3$:
		\begin{align*} 
			b_{n} & = 1 + 1 + \frac{1}{2} + \underbrace{\frac{1}{2 \cdot 2}}_{< \left(\frac{1}{2}\right)^{2}} + \underbrace{\frac{1}{2 \cdot 3 \cdot 4}}_{< \left(\frac{1}{2}\right)^{3}} + \dotsc + \underbrace{\frac{1}{2 \cdot \dotsc \cdot n}}_{< \left(\frac{1}{2}\right)^{n-1}} \\
				  & < 1 + \left( 1 + \frac{1}{2} + (\frac{1}{2})^{2} + \dotsc + (\frac{1}{2})^{n-1} \right) = 1 + \frac{1 - \left( \frac{1}{2} \right)^{n}}{1 - \frac{1}{2}} \\
				  & < 1 + \frac{1}{1 - \frac{1}{2}} = 3 \quad \forall n \in \N
		\end{align*} 
		$\xRightarrow[]{\ref{2.3:prop}} (b_{n})$ konvergiert. $b \coloneqq \lim b_{n}$. Für $n \geq 2$:
		\begin{align*}
			a_{n} & = \left( 1 + \frac{1}{n} \right)^{n} \overset{\S 1}{=} \sum_{k=0}{n} \binom{n}{k} \frac{1}{n^{k}} \\
				  & = 1 + 1 + \sum_{k = 2}^{n} \frac{1}{k!} \frac{n!}{(n-k)!} \frac{1}{n^{k}} = 1 + 1 + \sum_{k=2}^{n} \frac{1}{k!} \frac{n(n-1) \cdot \dotsc \cdot (n-(k-1))}{n \cdot n \cdot \dotsc \cdot n} \\
				  & = 1 + 1 + \sum_{k=2}^{n} \frac{1}{k!} \underbrace{(1 - \frac{1}{n})}_{< 1} \underbrace{(1 - \frac{2}{n})}_{< 1} \cdot \dotsc \cdot \underbrace{(1 - \frac{k-1}{n})}_{< 1} \\
				  & \leq 1 + 1 + \sum_{k=2}^{n} \frac{1}{k!} = b_{n}
		\end{align*}
		Also $a_{n} \leq b_{n} ~\forall n \geq 2$. Z. z.: $\Rightarrow a \leq b$ \\
		Sei $j \in \N, j \geq 2$ (zunächst fest). Für $n \in \N, n \geq j$:
		\begin{align*}
			a_{n} & \overset{s.o.}{=} 1 + 1 + \sum_{k=2}^{n} \frac{1}{k!} (1-\frac{1}{n})(1-\frac{2}{n}) \cdot \dotsc \cdot (1-\frac{k-1}{n}) \\
				  & \geq 1 + 1 + \sum_{k = 2}^{j} \frac{1}{k!} \underbrace{(1-\frac{1}{n})}_{\rightarrow 1} \underbrace{(1-\frac{2}{n})}_{\rightarrow 1} \cdot \dotsc \cdot \underbrace{(1-\frac{k-1}{n})}_{\rightarrow 1} \\
				  & \rightarrow 1 + 1 + 1 \sum_{k=2}^{j} \frac{1}{k!} = b_{j} \quad (n \rightarrow \infty)
		\end{align*}
		Also $a \geq b_{j} ~\forall j \geq 2 \xRightarrow[]{j \rightarrow \infty} a \geq b$.
	\end{proof}
\end{beispiel}

\index{Eulersche Zahl}
\begin{definition} 
	$$
		e \coloneqq \lim_{n \rightarrow \infty} \left( 1 + \frac{1}{n} \right)^{n} ~( = \lim_{n \rightarrow \infty} \sum_{k = 0}^{n} \frac{1}{k!} )
	$$
	hei{\ss}t \textbf{Eulersche Zahl}. Übung: $2 < e < 3$. \\
	$e \approx 2,718\dotsc$
\end{definition}

\index{Teilfolge}
\begin{definition} 
	Sei $(a_{n})$ eine Folge und $(n_{1}, n_{2}, n_{3}, \dotsc)$ eine Folge in $\N$ mit \\
	$n_{1} < n_{2} < n_{3} < \dotsc$. Für $k \in \N$ setze
	$$
		b_{k} \coloneqq a_{n_{k}}
	$$
	also $b_{1} = a_{n_{1}}, b_{2} = a_{n_{2}}, \dotsc$. Dann hei{\ss}t $(b_{k}) = (a_{n_{k}})$ eine \textbf{Teilfolge} (TF) von $(a_{n})$.
\end{definition}


\begin{beispiele}\
	\begin{enumerate}
		\item $(a_{2}, a_{4}, a_{6}, \dotsc)$ ist eine Teilfolge von $(a_{n})$; hier: $n_{k} = 2k$
		\item $(a_{1}, a_{4}, a_{9}, \dotsc)$ ist eine Teilfolge von $(a_{n})$; hier: $n_{k} = k^2$
		\item $(a_{2}, a_{6}, a_{4}, a_{10}, a_{8}, a_{14}, \dotsc)$ ist keine Teilfolge von $(a_{n})$.
	\end{enumerate}
\end{beispiele}


\begin{definition}
	$(a_{n})$ sei eine Folge und $\alpha \in \R$. $\alpha$ hei{\ss}t ein \textbf{Häufungswert} (HW) von $(a_{n})$
	$$
		\iff \exists \text{ Teilfolge } (a_{n_{k}}) \text{ von } (a_{n}) : ~a_{n_{k}} \rightarrow \alpha ~(k \rightarrow \infty) 
	$$	
	$H(a_{n}) \coloneqq \{ \alpha \in \R: \alpha$ ist ein Häufungswert von $(a_{n}) \}$.
\end{definition}


\begin{satz} \label{2.10:satz}
	$\alpha \in \R$ ist ein Häufungswert von $(a_{n})$
	$$
		\iff \forall \epsilon > 0: a_{n_{k}} \in U_{\epsilon}(\alpha) \quad (*)
	$$
	für unendlich viele $n \in \N$.
\end{satz}

\begin{proof} ~\\
	$"'\Rightarrow"'$ Sei $(a_{n_{k}})$ eine Teilfolge mit $a_{n_{k}} \rightarrow \infty$. Sei $\epsilon > 0 \exists k_{0} \in \N: a_{n_{k}} \in U_{\epsilon}(\alpha)$ für $k \geq k_{0} \Rightarrow (*)$ \\
	$"'\Leftarrow"'$ $\exists n_{1} \in \N: a_{n_{1}} \in U_{1}(\alpha)$. $\exists n_{2} \in \N: a_{n_{2}} \in U_{\frac{1}{2}}(\alpha)$ und $n_{2} > n_{1}$. $\exists n_{3} \in \N: a_{n_{3}} \in U_{\frac{1}{3}}(\alpha)$ und $n_{3} > n_{2}$. Etc. ... Man erhält eine Teilfolge $(a_{n_{k}})$ von $(a_{n})$ mit
	$$
		a_{n_{k}} \in U_{\frac{1}{k}}(\alpha) ~\forall k \in \N, \text{ also } |a_{n_{k}} - \alpha| < \frac{1}{k} ~\forall k
	$$
	Somit: $a_{n_{k}} \rightarrow \alpha$. 
\end{proof}


\begin{beispiele}\
	\begin{enumerate}
		\item $a_{n} = (-1)^{n}$, $a_{2k} = 1 \rightarrow 1, a_{2k+1} \rightarrow -1$, also $1, -1 \in H(a_{n})$. Sei $\alpha \in \R, \alpha \neq 1, \alpha \neq -1$ \\
			% todo image
			Wähle $\epsilon>0$ so, dass $1, -1 \notin U_{\epsilon}(\alpha)$. Dann $a_{n} \in U_{\epsilon}(\alpha)$ für kein $n \in \N$ $\xRightarrow[]{\ref{2.10:satz}} \alpha \notin H(a_{n})$. Fazit: $H(a_{n}) = \{ 1, -1 \}$.
		\item $a_{n} = n$. Ist $\alpha \in \R$ und $\epsilon > 0$, so gilt: $a_{n} \in U_{\epsilon}(\alpha)$ für höchstens endlich viele $n$, also $\alpha \notin H(a_{n})$. Fazit: $H(a_{n}) = \emptyset$.
		\item $\Q$ ist abzählbar. Sei $(a_{n})$ eine Folge mit $Q = \{ a_{1}, a_{2}, a_{3}, \dotsc \}$. Sei $\alpha \in \R$ und $\epsilon > 0$ $\xRightarrow[]{\ref{1.5:satz}} U_{\epsilon}(\alpha) = (\alpha - \epsilon, \alpha + \epsilon)$ enthält unendlich viele verschiedene rationale Zahlen $\xRightarrow[]{\ref{2.10:satz}} \alpha \in H(a_{n})$. Fazit: $H(a_{n}) = \R$.
	\end{enumerate}	
\end{beispiele}

\begin{folgerung}
Ist $x \in \R$, so existieren Folgen $(r_{m})$ in $\Q : r_{n} \rightarrow \alpha$.	
\end{folgerung}


\begin{satz} \label{2.11:satz} 
	$(a_{n})$ sei konvergent, $a \coloneqq \lim a_{n}$ und $(a_{n_{k}})$ eine Teilfolge von $(a_{n})$. Dann:
	$$ a_{n_{k}} \rightarrow a (k \rightarrow \infty) $$
	Insbesondere: $H(a_{n}) = \{ \lim a_{n} \}$
\end{satz}

\begin{proof}
	Sei $\epsilon > 0$. Dann: $a_{n} \in U_{\epsilon}(a)$ ffa $n \in \N$, also auch $a_{n_{k}} \in U_{\epsilon}(a)$ ffa $k \in \N$. Somit: $a_{n_{k}} \rightarrow \alpha$.
\end{proof}

\index{niedrig}
\begin{definition} Sei $(a_{n})$ eine Folge. 
	\begin{enumerate}
		\item $m \in \N$. $m$ hei{\ss}t \textbf{niedrig} (für $(a_{n})$) $\iff a_{n} \geq a_{m} ~\forall n \geq m $
		\item $m \in \N$ hei{\ss}t nicht niedrig $\iff \exists n \geq m: a_{n} < a_{m} \Rightarrow n > m: a_{n} < a_{m}$
	\end{enumerate}
\end{definition}


\begin{hilfssatz*}
	$(a_{n})$ sei eine Folge. Dann enthält $(a_{n})$ eine monotone Teilfolge.	
\end{hilfssatz*}

\begin{proof} ~\\
	Fall 1: es existieren höchstens endlich viele niedrige Indizes. Also existiert $n_{1} \in \N$: jedes $n \geq n_{1}$ ist nicht niedrig.
	\begin{description}
		\item $n_{1}$ nicht niedrig $\Rightarrow \exists n_{2} > n_{1} : a_{n_{2}} < a_{n_{1}}$
		\item $n_{2}$ nicht niedrig $\Rightarrow \exists n_{3} > n_{2} : a_{n_{3}} < a_{n_{2}}$
		\item Etc$\dotsc$
	\end{description}
	Wir erhalten so eine streng monoton fallende Teilfolge $(a_{n_{k}})$. \\ \\
	Fall 2: es existieren unendlich viele niedrige Indizes $n_{1}, n_{2}, \dotsc$, etwa $n_{1} < n_{2} < \dotsc$
	\begin{description}
		\item $n_{1}$ ist niedrig und $n_{2} > n_{1} \rightarrow a_{n_{2}} \geq a_{n_{1}}$
		\item $n_{2}$ nicht niedrig $\Rightarrow \exists n_{3} > n_{2} : a_{n_{3}} \geq a_{n_{2}}$
		\item Etc$\dotsc$
	\end{description}
	Wir erhalten so eine monoton wachsende Teilfolge $(a_{n_{k}})$.
\end{proof}

\index{Satz!Bolzano-Weierstra{\ss}}
\begin{satz}[Bolzano-Weierstra{\ss}] \label{2.12:satz-BolzanoWeierstrass}  ~\\
	$(a_{n})$ sei beschränkt, dann: $H(a_{n}) \neq \emptyset$. $(a_{n})$ enthält also eine konvergente Teilfolge
\end{satz}

\begin{proof}
	$\exists c \geq 0: |a_{n}| \leq c$ $\forall n \in \N$. $\xRightarrow[]{hilfssatz*} (a_{n})$ enthält eine monotone Teilfolge $(a_{n_{k}})$. Dann: $|a_{n_{k}}| \leq c \forall k \in \N$ \\
	$(a_{n_{k}})$ ist also beschränkt $\xRightarrow[]{\ref{2.3:prop}} (a_{n_{k}})$ ist konvergent. Also $\lim_{k \rightarrow \infty} a_{n_{k}} \in H(a_{n})$.
\end{proof}

\begin{satz} \label{2.13:satz}
	$(a_{n})$ sei beschränkt $\left( å\xRightarrow[]{\ref{2.12:satz-BolzanoWeierstrass}} H(a_{n}) \neq \emptyset \right)$
	\begin{enumerate}
		\item $H(a_{n})$ ist beschränkt
		\item $\sup H(a_{n}), \inf H(a_{n}) \in H(a_{n})$; es existieren also
			$$ \max H(a_{n}), \min H(a_{n}) $$
	\end{enumerate}
\end{satz}

\index{Limes superior} \index{oberer Limes} \index{Limes inferior} \index{unterer Limes}
\begin{definition} 
	Ist $(a_{n})$ beschränkt, so nennen wir 
	\begin{enumerate}
		\item $\limsup_{n \rightarrow \infty} a_{n} \coloneqq \limsup a_{n} \coloneqq \overline{\lim} a_{n} \coloneqq \max H(a_{n})$ hei{\ss}t \textbf{Limes superior} oder \textbf{oberer Limes} von $(a_{n})$.
		\item $\liminf_{n \rightarrow \infty} a_{n} \coloneqq \liminf a_{n} \coloneqq \underline{\lim} a_{n} \coloneqq \min H(a_{n})$ hei{\ss}t \textbf{Limes inferior} oder \textbf{unterer Limes} von $(a_{n})$.
	\end{enumerate}
\end{definition}


\begin{proof}\
	\begin{enumerate}
		\item $\exists c \geq 0: |a_{n}| \leq c$ $\forall n \in \N$. Sei $\alpha \in H(a_{n})$. Es existiert eine Teilfolge $(a_{n_{k}})$ mit $a_{n_{k}} \rightarrow \alpha ~(k \rightarrow \infty)$. Es ist
			$$ |a_{n_{k}}| \leq c \quad \forall k, \text{ also } -c \leq a_{n_{k}} \leq c \quad \forall k $$ 
			$\Rightarrow - c \leq \alpha \leq c$. Also $|\alpha| \leq c$ $\forall \alpha \in H(a_{n})$.
		\item ohne Beweis.
	\end{enumerate}
\end{proof}


\begin{satz} \label{2.14:satz}
	$(a_{n})$ sei beschränkt.
	\begin{enumerate}
		\item $\liminf a_{n} \leq \alpha \leq \limsup a_{n} ~\forall \alpha \in H(a_{n})$
		\item Ist $(a_{n})$ konvergent $\Rightarrow \limsup a_{n} = \liminf a_{n} = \lim a_{n}$
		\item $\limsup(\alpha a_{n}) = \alpha \limsup a_{n} ~\forall \alpha \geq 0$
		\item $\limsup(-a_{n}) = - \liminf a_{n}$
	\end{enumerate}
\end{satz}

\begin{proof}
	a) klar, b) folgt aus \ref{2.11:satz}, c) und d) Übung.
\end{proof}


\textbf{Motivation:} $(a_{n})$ sei konvergent und $\lim a_{n} \eqqcolon a$. Sei $\epsilon > 0$,
	$$ \exists n_{0} \in \N: |a_{n} - a| < \frac{\epsilon}{2} \quad \forall n \geq n_{0} $$
Für $n, m \geq n_{0}$:
	$$ |a_{n} - a_{m}| = |a_{n} - a + a - a_{m} | \leq |a_{n} - a| + |a_{m} - a| < \frac{\epsilon}{2} + \frac{\epsilon}{2} = \epsilon $$
D.h.: $(a_{n})$ hat die folgende Eigenschaft:
	\[ \forall \epsilon > 0 \exists n_{0} = n_{0}(\epsilon) \in \N: |a_{n} - a_{m}| < \epsilon \quad \forall n,m \geq n_{0} \tag*{(c)} \]
$$ (\iff \forall \epsilon > 0 ~\exists n_{0} = n_{0}(\epsilon) \in \N: |a_{n} - a_{n+k}| < \epsilon \quad \forall n \geq n_{0} ~\forall k \in \N) $$

\index{Cauchyfolge}
\begin{definition} 
	Eine Folge $(a_{n})$ hei{\ss}t eine \textbf{Cauchyfolge} (CF)
	$$ \iff (a_{n}) \text{ hat die Eigenschaft } (c) $$	
\end{definition}


Konvergente Folgen sind also Cauchy-Folgen!

\index{Cauchykriterium}
\begin{namedtheorem}[Cauchykriterium] $(a_{n}) \text{ ist konvergent} \iff (a_{n}) \text{ ist eine Cauchyfolge}$. \label{2.15:prop} 
\end{namedtheorem}

\begin{proof}
	$"'\Rightarrow"'$ s.o. $"'\Leftarrow"'$ ohne Beweis
\end{proof}


\begin{beispiel*}
	$a_{1} \coloneqq 1, a_{n+1} \coloneqq \frac{1}{1 + a_{n}}$ $(n \in \N)$. Mit Induktion folgt:
	\begin{enumerate}
		\item[1)] $0 < a_{n} \leq 1 ~(n \in \N)$ Damit:
		\item[2)] $a_{n} \geq \frac{1}{2} ~(n \in \N)$
	\end{enumerate}
	Für $n \geq 2, k \in \N$ gilt daher:
	\begin{align*}
		|a_{n+k} - a_{n} | & = \left| \frac{1}{1+a_{n+k-1}} - \frac{1}{1 - a_{n - 1}} \right| = \frac{|a_{n-1} - a_{n +k-1}|}{(1+a_{n+k-1})(1+a_{n-1})} \\
			& \leq \frac{1}{(1+\frac{1}{2})^{2}} |a_{n+k-1} a_{n-1}| = \frac{4}{9} |a_{n+k-1} - a_{n-1}| \\
			& \leq \left(\frac{4}{9} \right)^{2} |a_{n-k-2} - a_{n-2}| \leq \dotsc \leq \left( \frac{4}{9} \right)^{n-1} |a_{k+1} - a_{1}| \\
			& \leq \left( \frac{4}{9} \right)^{n-1} \left( |a_{k+1}| + |a_{1}|\right) \leq 2 \left( \frac{4}{9} \right)^{n-1} 
	\end{align*}
	$\exists n_{0} \in \N \setminus \{ 1 \}$: $2\left(\frac{4}{9}\right)^{n-1} < \epsilon ~(n \geq n_{0})$. Damit: $|a_{n+k} - a_{n}| < \epsilon ~(n \geq n_{0}, k \in \N)$. \\
	Also ist $(a_{n})$ eine Cauchyfolge. $a \coloneqq \lim_{n \rightarrow \infty} a_{n}$. Klar: 
	$$ a \geq \frac{1}{2} \text{ und } a = \frac{1}{1 + a} $$
	Also $a^{2} + a - 1 = 0 \iff a = - \frac{1}{2} \pm \frac{\sqrt{5}}{2}$. Wegen $a \geq \frac{1}{2}$ folgt $a = \frac{\sqrt{5} - 1}{2}$.
\end{beispiel*}


\newpage


\section{Unendliche Reihen}


\index{Reihe!unendliche} \index{Teilsumme} \index{konvergent} \index{divergent} \index{Reihenwert}
\begin{definition} $(a_{n})$ sei eine Folge;
	\begin{enumerate}
		\item $ s_{n} \coloneqq a_{1} + a_{2} + \dotsc a_{n} \quad (n \in \N)$
		(also $a_{1} = a_{1}, a_{2} = a_{1} + a_{2}, \dotsc$). $(s_{n})$ hei{\ss}t \textbf{(unendliche) Reihe} und wird mit $\sum_{n = 1}^{\infty} a_{n}$ bezeichnet. Weitere Bezeichnungen: $a_{1} + a_{2} + a_{3} + \dotsc$
		\item $s_{n}$ hei{\ss}t \textbf{n-te Teilsumme} von $\sum_{n=1}^{\infty} a_{n}$.	
		\item $\sum_{n=1}^{\infty} a_{n}$ hei{\ss}t konvergent bzw. divergent $\iff (s_{n})$ ist konvergent 	bzw. divergent.
		\item Ist $\sum_{n = 1}^{\infty} a_{n}$ konvergent, so hei{\ss}t $\lim s_{n}$ der Reihenwert und wird ebenfalls mit $\sum_{n=1}^{\infty} a_{n}$ bezeichnet (schlecht, aber so üblich)
	\end{enumerate} 	
\end{definition}


\begin{bemerkung}
	Ist $p \in \Z$ und $(a_{n})_{n=p}^{\infty}$ eine Folge, so definiert man entsprechend
		$$ s_{n} = a_{p} + a_{p+1} + \dotsc + a_{n} \quad (n \geq p) $$
	und $\sum_{n=p}^{\infty} a_{n}$ (meist: $p = 1$ oder $p = 0$)
\end{bemerkung}


Die folgenden Sätze und Definitionen formulieren wir nun für Reihen der Form $\sum_{n=1}^{\infty} a_{n}$. Diese Sätze und Definitionen gelten entsprechend für Reihen der Form $\sum_{n=p}^{\infty} a_{n}$ $(p \in \Z)$

\index{Reihe!geometrische} \index{Reihe!harmonische}
\begin{beispiele} ~\
	\begin{enumerate}
		\item Sei $x \in \R$. $\sum_{n=0}^{\infty} x^{n} = 1 + x + x^{2} + \dotsc$ hei{\ss}t \textbf{geometrische Reihe}. \\
			$s_{m} = 1 + x + \dotsc x^{m} \xRightarrow[]{\ref{2.6:bsp}} (s_{n})$ konvergiert $\iff |x| < 1$ und $\lim s_{n} = \frac{1}{1 - x}$ für $|x| < 1$. Also: $\sum_{n=0}^{\infty} x^{n}$ konvergent $\iff |x| < 1$ und $\sum_{n=0}^{\infty} x^{n} = \frac{1}{1 - x}$ für $|x] < 1$.
		\item $\sum_{n=1}^{\infty} \frac{1}{n(n+1)}; a_{n} = \frac{1}{n(n+1)} = \frac{1}{n} - \frac{1}{n+1}$
			\begin{align*}
				\Rightarrow s_{n} & = a_{1} + \dotsc + a_{n} \\
						& = (1 - \frac{1}{2}) + (\frac{1}{2} - \frac{1}{3}) + \dotsc + (\frac{1}{n-1} - \frac{1}{n}) + (\frac{1}{n} - \frac{1}{n+1}) \\
						& = 1 - \frac{1}{n+1} \rightarrow 1
			\end{align*}
			Also $\sum_{n=1}^{\infty} \frac{1}{n(n+1))}$ konvergent und $\sum_{n=1}^{\infty} \frac{1}{n(n+1)} = 1$.
		\item $\sum_{n=0}^{\infty} \frac{1}{n!} = 1 + 1 + \frac{1}{2!} + \frac{1}{3!} + \dotsc$; $s_{n} = 1 + 1 + \frac{1}{2!} + \dotsc + \frac{1}{n!} \xRightarrow[]{\ref{2.9:bsp}} s_{n} \rightarrow e$. \\
			Also: $\sum_{n = 0}^{\infty} \frac{1}{n!}$ konvergiert und $\sum_{n=1}^{\infty} \frac{1}{n!} = e$.
		\item Die \textbf{harmonischen Reihe} $\sum_{n=1}^{\infty} \frac{1}{n}$. Dann ist $s_{n} = 1 + \frac{1}{2} + \dotsc \frac{1}{n}$, \\
			$s_{2n} = 1 + \frac{1}{2} + \dotsc + \frac{1}{n} + \frac{1}{n+1} + \dotsc + \frac{1}{2n} = s_{n} + \underbrace{\frac{1}{n+1}}_{\geq \frac{1}{2n}} + \underbrace{\frac{1}{n+2}}_{\geq \frac{1}{2n}} + \dotsc + + \underbrace{\frac{1}{2n}}_{\geq \frac{1}{2n}} \geq s_{n} + \frac{1}{2}$ \\
			Annahme: $(s_{n})$ ist konvergent. $s \coloneqq \lim s_{n} \xRightarrow[2.11:satz]{2.11} s_{2n} \rightarrow s \Rightarrow s \geq s + \frac{1}{2} \rightarrow 0 \geq \frac{1}{2}$. Widerspruch! Also: $\sum_{n=1}^{\infty} \frac{1}{n}$ ist divergent!
	\end{enumerate}	
\end{beispiele}

\index{Monotoniekriterium} \index{Cauchykriterium}
\begin{satz} \label{3.1:satz} 
	$(a_{n})$ sei eine Foge und $s_{n} = a_{1} + \dotsc + a_{n}$.
	\begin{enumerate}
		\item \textbf{Monotoniekriterium:} Sind alle $a_{n} \geq 0$ und ist $(s_{n})$ beschränkt, so ist $\sum_{n = 1}^{\infty} a_{n}$ konvergent.
		\item \textbf{Cauchykriterium:} $\sum a_{n}$ konvergiert $\iff \forall \epsilon > 0 \exists n_{0} = n_{0}(\epsilon) \in \N$:
			$$ \left| \sum_{k = n+1}^{m} a_{k} \right| < \epsilon \quad \forall m > n \geq n_{0} $$ \label{3.1.b:satz} 
		\item Ist $\sum_{n=1}^{\infty} a_{n}$ konvergent $\Rightarrow a_{n} \rightarrow 0$. \label{3.1.c:satz} 
		\item $\sum_{n=1}^{\infty} a_{n}$ sei konvergent. Dann ist für jedes $\nu \in \N$ die Reihe $\sum_{n=\nu+1}^{\infty} a_{n}$ konvergent und für $r_{\nu} \coloneqq \sum_{n = \nu+1}^{\infty} a_{n}$ gilt: $r_{\nu} \rightarrow 0$.
	\end{enumerate}
\end{satz}

\begin{proof} ~\
	\begin{enumerate}
		\item $s_{n+1} = a_{1} + \dotsc + a_{n} + a_{n+1} = s_{n} + a_{n+1} \geq s_{n}$. $(s_{n})$ ist also wachsend und beschränkt $\xRightarrow[]{\ref{2.3:prop}} (s_{n})$ konvergent.
		\item Für $m > n: |s_{m} - s_{n}| = | a_{1} + \dotsc + a_{n} + a_{n+1} + \dotsc + a_{m} - (a_{1} + \dotsc a_{n})| = |a_{n+1} + \dotsc + a_{m}| = |\sum_{k=n+1}^{m} a_{k}|$. Behauptung folgt aus \ref{2.15:prop}.
		\item $s_{n+1} - s_{n} = a_{n+1}$. Ist $(s_{n})$ konvergent, so folgt $a_{n+1} \rightarrow 0$
		\item ohne Beweis!
	\end{enumerate}	
\end{proof}


\begin{bemerkung}
	Ist $(a_{n})$ eine Folge und gilt $a_{n} \not\rightarrow 0$, so ist $\sum a_{n}$ divergent!
\end{bemerkung}


\begin{satz} \label{3.2:satz}
	Die Reihen $\sum a_{n}$ und $\sum b_{n}$ seien konvergent und es seien $\alpha, \beta \in \R$. Dann konvergiert
		$$ \sum ( \alpha a_{n} + \beta b_{n}) $$
	und $\sum ( \alpha a_{n} + \beta b_{n}) = \alpha \sum a_{n} + \beta \sum b_{n}$
\end{satz}

\begin{proof}
	\ref{2.2:satz}
\end{proof}

\index{Konvergenzkriterium!Reihen!Leibnitz}
\begin{namedtheorem}[Leibnitzkriterium] \label{3.3:prop-LeibnitzKriterium}
	Sei $(b_{n})$ eine Folge mit:
		$$ b_{n} \geq 0 ~\forall n \in \N, (b_{n}) \text{ ist monoton fallend und } b_{n} \rightarrow 0  $$
		Dann ist $\sum_{n=1}^{\infty} (-1)^{n+1}b_{n}$ konvergent.
\end{namedtheorem}

\index{Reihe!alternierende harmonische Reihe}
\begin{beispiel*}
	Aus \ref{3.3:prop-LeibnitzKriterium} folgt: \\
	Die \textbf{alternierende harmonische Reihe} $\sum_{n=1}^{\infty} \frac{(-1)^{n+1}}{n}$ ist konvergent.
\end{beispiel*}

\begin{proof}(von 3.3) $a_{n} \coloneqq (-1)^{n+1} b_{n}$, \\
	$s_{n} \coloneqq a_{1} + \dotsc + a_{n}$. $s_{2n+2} = s_{2n} + a_{2n+1} + a_{2n+2} = s_{2n} + \underbrace{b_{2n+1}-b_{2n+2}}_{\geq 0} \geq s_{2n}$. $(s_{2n})$ ist also monoton fallend. Es gilt:
	\[ \forall n \in \N: s_{2n} = s_{2n-1} - a_{2n} = s_{2n-1} - b_{2n} \leq s_{2n-1} \tag*{$(*)$} \]
	Also:
	$$ s_{2} \leq s_{4} \leq \dotsc \leq s_{2n} \overset{(*)}{\leq} s_{2n-1} \leq \dotsc \leq s_{3} \leq s_{1} $$
	$(s_{2n})$ und $(s_{2n+1})$ sind also beschränkt $\xRightarrow[]{\ref{2.3:prop}} (s_{2n})$ und $(s_{2n+1})$ sind konvergent. $s \coloneqq \lim s_{2n} \xRightarrow[]{(*)} s = \lim s_{2n+1}$. \\
	Sei $\epsilon > 0$:
	$$
		\begin{rcases*}
	 		s_{2n} \in U_{\epsilon}(s) \text{ ffa } n \in \N \\
	 		s_{2n-1} \in U_{\epsilon}(s) \text{ ffa } n \in \N  	
		\end{rcases*} \Rightarrow s_{n} \in U_{\epsilon}(s) \text{ ffa } n \in \N
	$$
	Also: $s_{n} \rightarrow s$.
\end{proof}

\index{konvergent!absolut}
\begin{definition}
	$\sum a_{n}$ hei{\ss}t \textbf{absolut konvergent} $\iff \sum |a_{n}|$ ist konvergent.
\end{definition}


\begin{beispiel*}
	$\sum_{n=1}^{\infty} \frac{(-1)^{n+1}}{n}$ ist konvergent, aber nicht absolut konvergent.
\end{beispiel*}


\begin{satz} \label{3.4:satz}
	$\sum a_{n}$ sei absolut konvergent. Dann:
	\begin{enumerate}
		\item $\sum a_{n}$ ist konvergent
		\item $|\sum_{n=1}^{\infty} a_{n}| \leq \sum_{n=1}^{\infty} |a_{n}|$ ($\triangle$-Ungleichung für Reihen)
	\end{enumerate}
\end{satz}

\begin{proof} ~\
	\begin{enumerate}
		\item Seien $m,n \in \N, m > n$
			\[ \underbrace{| \sum_{k = n+1}^{m} a_{k} |}_{\eqqcolon \sigma_{m, n}} \leq \underbrace{\sum_{k = n+1}^{m}|a_{k}|}_{\eqqcolon \tau_{m, n}} \tag*{$(*)$} \]
			Sei $\epsilon > 0$, Voraussetzung nach $\ref{3.1.b:satz} \Rightarrow \exists n_{0} \in \N: \tau_{m, n} < \epsilon$ für $m > n > n_{0} \xRightarrow[]{(*)} \sigma_{m, n} < \epsilon$ für $m > n \geq n_{0} \xRightarrow[]{\ref{3.1.b:satz}} \sum a_{n}$ konvergiert
		\item Sei $s_{k} \coloneqq a_{1} + \dotsc + a_{k}$, $s \coloneqq \lim s_{n}$, $\sigma_{k} \coloneqq |a_{1}| + \dotsc |a_{k}|$ und $\sigma = \lim \sigma_{n}$. Dann: $|s_{n}| \rightarrow |s|$ und 
			$$ |s| \leq \sigma \quad \forall n $$
			$\Rightarrow |s| \leq \sigma \Rightarrow$ $\triangle$-Ungleichung
	\end{enumerate}
\end{proof}

\index{Konvergenzkriterium!Reihen!Majoranten} \index{Konvergenzkriterium!Reihen!Minoranten}
\begin{satz} ~\ \label{3.5:satz}
	\begin{enumerate}
		\item \textbf{Majorantenkriterium}: Gilt $|a_{n}| \leq b_{n}$ ffa $n \in \N$ und ist $\sum b_{n}$ konvergent, so ist $\sum a_{n}$ absolut konvergent. \label{3.5.a:satz}
		\item \textbf{Minorantenkriterium}: Gilt $a_{n} \geq b_{n} \geq 0$ ffa $n \in \N$ und ist $\sum b_{n}$ divergent, so ist $\sum a_{n}$ divergent. \label{3.5.b:satz}
	\end{enumerate}
\end{satz}

\begin{proof} ~\
	\begin{enumerate}
		\item $\exists j \in \N$: $|a_{n}| \leq b_{n} ~\forall n \geq j$. Sei $m > n \geq j$, dann
			\[ \underbrace{\sum_{k=n+1}^{m}|a_{n}|}_{\eqqcolon \sigma_{m, n}} \leq \underbrace{\sum_{k=n+1}^{m} b_{k}}_{\eqqcolon \tau_{m,n}}  \]
			Sei $\epsilon > 0$ Voraussetzung nach $\ref{3.1.b:satz} \Rightarrow \exists n_{0} \in \N: n_{0} \geq j$ und $\tau_{m,n} < \epsilon$ für $m > n \geq n_{0}$. Dann: $\sigma_{m,n} < \epsilon$ für $m > n \geq n_{0} \xRightarrow[]{\ref{3.1.b:satz}} \sum |a_{n}|$ konvergiert.
		\item Annahme: $\sum a_{n}$ konvergent $\xRightarrow[]{a)} \sum b_{n}$ konvergent, Widerspruch.
	\end{enumerate}	
\end{proof}


\begin{beispiele} ~\
	\begin{enumerate}
		\item $\sum_{n=1}^{\infty} \frac{1}{(n+1)^{2}}$, $\forall n \in \N$:
			$$ a_{n} = \frac{1}{(n+1)^{2}} = |a_{n}| = \frac{1}{n^{2} + 2n +1} \leq \frac{1}{n^{2} + 2n} \leq \frac{1}{n(n+1)} \eqqcolon b_{n} $$
			Bekannt: $\sum b_{n}$ konvergiert $\xRightarrow[]{\ref{3.5.a:satz}} \sum_{n=1}^{\infty} \frac{1}{(n+1)^{2}}$ konvergiert
		\item Aus Beispiel a): $\sum_{n=1}^{\infty} \frac{1}{n^{2}}$ ist konvergent.
		\item Sei $\alpha > 0$ und $\alpha \in \Q$: Betrachte: $\sum_{n=1}^{\infty} \frac{1}{n^{\alpha}}$. \\
			Fall 1: $\alpha \in (0, 1]$.
				$$ \forall n \in \N: ~ \frac{1}{n^{\alpha}} \geq \frac{1}{n} \geq 0 \xRightarrow[]{\ref{3.5.b:satz}} \sum \frac{1}{n^{\alpha}} \text{ divergiert.} $$
			Fall 2: $\alpha \geq 2$: \\
				$$ \forall n \in \N: ~ 0 \leq \frac{1}{n^{\alpha}} \leq \frac{1}{n^{2}} \xRightarrow[]{\ref{3.5.a:satz}} \sum_{n=1}^{\infty} \frac{1}{n^{\alpha}} \text{ konvergiert.} $$
			Fall 3: $\alpha \in (1, 2)$: vgl. Übungsblatt, $\sum \frac{1}{n^{\alpha}}$ konvergiert. \\ \\
			\textbf{Fazit}: Ist $\alpha > 0$ und $\alpha \in \Q$, so gilt $\sum_{n=1}^{\infty} \frac{1}{n^{\alpha}}$ konvergiert $\Leftrightarrow \alpha > 1$
		\item $\sum_{n=1}^{\infty} (-1)^{n} \frac{n^{2} + 2}{n^{3} + 1}$; $|a_{n}| = \frac{n+2}{n^{3} + 1} \leq \frac{n+2}{n^{3}} \leq \frac{2n}{n^{3}} = \frac{2n}{n^{2}} \eqqcolon b_{n}$. Für $n \geq 2$ $\sum b_{n}$ konvergiert $\xRightarrow[]{\ref{3.5.a:satz}} \sum a_{n}$ konvergiert absolut
		\item $\sum_{n=1}^{\infty} \frac{\sqrt{n}}{n+1}$; $a_{n} = |a_{n}| = \frac{\sqrt{n}}{n+1} \geq \frac{\sqrt{n}}{2n} = \frac{1}{2} \cdot \frac{1}{\sqrt{n}} = \underbrace{\frac{1}{2} \cdot \frac{1}{n^{\frac{1}{2}}}}_{\geq 0} \eqqcolon b_{n}$. \\
			$\sum b_{n}$ divergiert $\xRightarrow[]{\ref{3.5.b:satz}} \sum a_{n}$ divergiert
	\end{enumerate}		
\end{beispiele}


\begin{bemerkung}
	Ist später später (in \S 7) die allgemeine Potenz $a^{x}$ ($a > 0, x \in \R)$ eingeführt, so zeigt man analog:
	$$ \text{Für } \alpha > 0 \text{ gilt:} \sum_{n=1}^{\infty} \frac{1}{n^{\alpha}} \text{ konvergiert} \iff \alpha > 1 $$
\end{bemerkung}


\begin{hilfssatz*}
	$(c_{n})$ sei beschränkt
	\begin{enumerate}
		\item Ist $\alpha \coloneqq \limsup c_{n}$ und $x > \alpha$, so gilt: $c_{n} < x$ ffa $n$
				% todo image
		\item Ist $\alpha \coloneqq \liminf c_{n}$ und $x < \alpha$, so gilt: $c_{n} > x$ ffa $n$
		\item Ist $c_{n} \geq 0 ~\forall n \in \N$ und $\limsup c_{n} = 0$, so gilt $c_{n} \rightarrow 0$
	\end{enumerate}
\end{hilfssatz*}

\begin{proof} ~\
	\begin{enumerate}
		\item[b)] Sei $\epsilon > 0$. $x \coloneqq \epsilon \xRightarrow[]{a)} -\epsilon < 0 \leq c_{n} < \epsilon$ ffa $n \in \N$, also $c_{n} \in U_{\epsilon}(0)$ ffa $n$.
		\item[a)] Annahme: $c_{n} \geq x$ für unendlich viele $n$, etwa für $n_{1}, n_{2}, n_{3}, \dotsc$ mit $n_{1} < n_{2} < n_{3} < \dotsc$, Die Teilfolge $(c_{n_{k}})$ ist beschränkt $\xRightarrow[]{\ref{2.11:satz}} (c_{n_{k}})$ enthält eine konvergente Teilfolge $(c_{n_{k_{j}}})$. $\beta \coloneqq \lim_{j\rightarrow \infty} c_{n_{k_{j}}}$. Es ist $c_{n_{k_{j}}} \geq x ~\forall j \Rightarrow \beta \geq x > \alpha$; $(c_{n_{k_{j}}})$ ist eine Teilfolge von $(c_{n}) \Rightarrow \beta \in H(a_{n}) \Rightarrow \beta \leq \alpha$, Widerspruch.
	\end{enumerate}	
\end{proof}

\index{Konvergenzkriterium!Reihen!Wurzel}
\begin{namedtheorem}[Wurzelkriterium (WK)] \label{3.6:prop-Wurzelkriterium}
	Sei $(a_{n})$ eine Folge, $c_{n} \coloneqq \sqrt[n]{|a_{n}|}$.
	\begin{enumerate}
		\item Ist $(c_{n})$ unbeschränkt, so ist $\sum_{n=1}^{\infty} a_{n}$ divergent.
		\item Sei $(c_{n}$ beschränkt und $\alpha \coloneqq \limsup_{n \rightarrow \infty} c_{n}$
			\begin{enumerate}
				\item Ist $\alpha < 1$, so ist $\sum a_{n}$ absolut konvergent.
				\item Ist $\alpha > 1$, so ist $\sum a_{n}$ divergent
			\end{enumerate}
			Im Falle $\alpha = 1$ ist keine allgemeine Aussage möglich.
	\end{enumerate}
\end{namedtheorem}

\begin{proof} ~\
	\begin{enumerate}
		\item $(c_{n})$ unbeschränkt $\Rightarrow c_{n} \geq 1$ für unendlich viele $n \Rightarrow |a_{n}| \geq 1$ für unendlich viele $n \Rightarrow a_{n} \rightarrow 0 \xRightarrow[]{\ref{3.1.c:satz}}$ Beh.
		\item  % todo image (2x)
			\begin{enumerate}
				\item Sei $\alpha < 1$, sei $x \in (\alpha, 1) \xRightarrow[]{hilfssatz*} c_{n} \leq x$ ffa n $\Rightarrow |a_{n}| \leq x^{n}$ ffa $n$. $\sum x^{n}$ konvergiert $\xRightarrow[]{\ref{3.5.a:satz}} \sum a_{n}$ konvergiert absolut 
				\item Sei $\alpha > 1$, wähle $\epsilon > 0$ so, dass $\alpha - \epsilon > 1$. Es gilt $c_{n} U_{\epsilon}(\alpha)$ für unendlich viele $n$. Dann: $c_{n} > \alpha - \epsilon > 1$ für unendlich viele $n$. Wie bei $a)$: $\sum a_{n}$ divergiert 
			\end{enumerate}
	\end{enumerate}
\end{proof}


\begin{beispiele} ~\
	\begin{enumerate}
		\item $a_{n} \coloneqq \frac{1}{n}$; $c_{n} = \sqrt[n]{|a_{n}|} = \frac{1}{\sqrt[n]{n}} \rightarrow 1$, also $\alpha = 1$ und $\sum a_{n}$ divergiert.
		\item $a_{n} \coloneqq \frac{1}{n^{2}}$; $c_{n} = \sqrt[n]{|a_{n}|} = \frac{1}{(\sqrt[n]{n})^{2}} \rightarrow 1$, also $\alpha = 1$ und $\sum a_{n}$ konvergiert.
		\item Sei $x \in \R$ und $a_{n} \coloneqq \begin{cases} \frac{1}{2^{n}}, & \text{falls } n = 2k \\ n x^{n}, & \text{falls } n = 2k - 1 \end{cases}$ \\
			Frage: Wann ist $\sum a_{n}$ (abs.) konvergent? Es ist
			$$c_{n} = \sqrt[n]{|a_{n}|} = \begin{cases}
				\frac{1}{2}, & \text{falls } n = 2k \\ \sqrt[n]{n}|x|, & \text{falls } n = 2k - 1
			\end{cases}$$
			$(c_{n})$ ist also beschränkt, $H(c_{n}) = \left\{ \frac{1}{2}, |x| \right\}$. \\ \\
			Fall 1: $|x| < 1$. Dann: $\alpha = \limsup c_{n} < 1$, also ist $\sum a_{n}$ absolut konvergent. \\
			Fall 2: $|x| > 1$. Dann: $\alpha = \limsup c_{n} < 1$, also ist $\sum a_{n}$ divergent. \\
			Fall 3: $|x| = 1$. Dann: $\alpha = \limsup c_{n} = 1$. Es ist $|a_{n}| = n$ falls $n = 2k - 1$. Also: $a_{n} \not\rightarrow 0$. $\sum a_{n}$ ist also divergent.			
	\end{enumerate}	
\end{beispiele}

\index{Konvergenzkriterium!Reihen!Quotienten}
\begin{namedtheorem}[Quotientenkriterium (QK)] \label{3.7:prop-Quotientenkriterium}
	Es sei $a_{n} \neq 0 ~\forall n \in \N$ und $c_{n} \coloneqq \left| \frac{a_{n+1}}{a_{n}} \right| ~(n \in \N)$.
	\begin{enumerate}
		\item Ist $c_{n} \geq 1$ ffa $n \in \N$, so ist $\sum a_{n}$ divergent
		\item Sei $(c_{n})$ beschränkt, $\alpha \coloneqq \limsup c_{n}$ und $\beta \coloneqq \liminf c_{n}$
			\begin{enumerate}
				\item Ist $\alpha < 1$, so ist $\sum a_{n}$ absolut konvergent
				\item Ist $\beta > 1$, so ist $\sum a_{n}$ divergent.
			\end{enumerate}
	\end{enumerate}
\end{namedtheorem}

\begin{folg} \label{3.8:folg}
	$(a_{n})$ und $(c_{n})$ seien wie in \ref{3.7:prop-Quotientenkriterium}, $(c_{n})$ sei konvergent und $\alpha \coloneqq \lim c_{n}$.
	$$ \sum a_{n} \text{ ist} \begin{cases} \text{absolut konv.}, & \text{falls } \alpha < 1 \\ \text{divergent}, & \text{falls } \alpha > 1 \end{cases} $$
	Im Falle $\alpha = 1$ ist keine allg. Aussage möglich.
\end{folg}


\begin{beispiele} ~\
	\begin{enumerate}
		\item $a_{n} = \frac{1}{n}$, $\left| \frac{a_{n+1}}{a_{n}} \right| = \frac{n}{n+1} \rightarrow 1$, $\sum a_{n}$ divergiert
		\item $a_{n} = \frac{1}{n^{2}}$, $ \left| \frac{a_{n+1}}{a_{n}} \right| = \frac{n^{2}}{(n+1)^{2}} \rightarrow 1$, $\sum a_{n}$ konvergiert 
	\end{enumerate}	
\end{beispiele}

\index{Exponentialreihe} \index{Exponentialfunktion}
\begin{namedtheorem}[Die Exponentialreihe] \label{3.9:prop-Exponentialreihe}
	Für $x \in \R$ betrachte die Reihe 
	$$ \sum_{n=0}^{\infty} \frac{x^{n}}{n!} = 1 + x + \frac{x^{2}}{2!} + \frac{x^{3}}{3!} + \dotsc $$
	Frage: für welche $x \in \R$ konvergiert die Reihe (absolut). \\
	Klar, die Reihe konvergiert für $x = 0$. Sei $x \ne q$ und $a_{n} \coloneqq \frac{x^{n}}{n!}$.
		$$ \left| \frac{a_{n+1}}{a_{n}} \right| = \left| \frac{x^{n+1}}{(n+1)!} \cdot \frac{n!}{x^{n}} \right| = \frac{|x|}{n+1} \rightarrow 0 \quad (n \rightarrow \infty) $$
	Aus \ref{3.8:folg} folgt:
		$$ \sum_{n=0}^{\infty} \frac{x^{n}}{n!} \text{ konv. absolut für jedes } x \in \R $$
	Damit ist auf $\R$ eine Funktion $E \colon \R \rightarrow \R$ definiert:
		$$ E(x) \coloneqq \sum_{n=0}^{\infty} \frac{x^{n}}{n!} \quad \text{Exponentialfunktion} $$
	Es ist $E(0) = 1$, $E(1) \overset{\S 2}{=} e$. \\
	Später zeigen wir: $E(r) = e^{r}$ für $r \in \Q$. Des Weiteren definieren wir später $e^{x} \coloneqq E(X)$ für $x \in \R \setminus \Q$. Dann: $e^{x} = E(x) \quad (x \in \R)$
\end{namedtheorem}

\index{Umordnung}
\begin{definition}
	Sei $(a_{n})$ eine Folge und $\varphi \colon \N \rightarrow \N$ eine Bijektion. Setze $b_{n} \coloneqq a_{\varphi(n)}$ $(n \in \N)$. also 
		$$ b_{1} = a_{\varphi(1)}, b_{2} = a_{\varphi(2)}, \dotsc $$
	Dann hei{\ss}t $(b_{n})$ eine \textbf{Umordnung} von $(a_{n})$.
\end{definition}

\begin{beispiel*}
$(a_{2}, a_{4}, a_{1} a_{3}, a_{6}, a_{8}, a_{5}, a_{7}, \dotsc)$ ist eine Umordnung von $(a_{n})$.	
\end{beispiel*}


\begin{satz} \label{3.10:satz}
	$(b_{n})$ sei eine Umordnung von $(a_{n})$.
	\begin{enumerate}
		\item Ist $(a_{n})$ konvergent, so ist $(b_{n})$ konvergent und $\lim b_{n} = \lim a_{n}$.
		\item Ist $\sum a_{n}$ absolut konvergent, so ist $\sum b_{n}$ absolut konvergent und $\sum a_{n} = \sum b_{n}$
	\end{enumerate}	
\end{satz}

\begin{proof} ~\
	\begin{enumerate}
		\item $a \coloneqq \lim a_{n}$; Sei $\epsilon > 0$. $\exists n_{0} \in \N: |a_{n} - a| < \epsilon ~\forall n \geq n_{0}$. Dann: $|a_{\varphi(n)} - a| < \epsilon$ ffa $n \in \N$.
		\item ohne Beweis.
	\end{enumerate}	
\end{proof}


\begin{bemerkung}[ohne Beweis]
	$\sum a_{n}$ sei konvergent, aber nicht absolut konvergent.
	\begin{enumerate}
		\item Ist $s \in \R$, so existiert eine Umordnung $\sum b_{n}$ von $\sum a_{n}$ mit: $\sum b_{n}$ ist konvergent und $\sum b_{n} = s$.
		\item Es existiert eine Umordnung $\sum c_{n}$ von $\sum a_{n}$ mit: $\sum c_{n}$ ist divergent.
	\end{enumerate}
\end{bemerkung}

\index{Cauchyprodukt}
\begin{definition}
	Gegeben seien die Reihen $\sum_{n=0}^{\infty} a_{n}$ und $\sum_{n=0}^{\infty} b_{n}$. \\
	Setze für $n \in \N$:
	\begin{align*}
		c_{n} & \coloneqq \sum_{k=0}^{\infty} a_{k} b_{n-k}, \text{ also: } \\
		c_{n} & = a_{0} b_{n} + a_{1} b_{n-1} + \dotsc + a_{n} b_{0}
	\end{align*} 
	Die Reihe $\sum_{n=0}^{\infty} c_{n}$ hei{\ss}t das \textbf{Cauchyprodukt} (CP) von $\sum a_{n}$ und $\sum b_{n}$.
\end{definition}


\begin{satz}[ohne Beweis] \label{3.11:satz}
$\sum_{n=0}^{\infty} a_{n}$ und $\sum_{n=0}^{\infty} b_{n}$ seien absolut konvergent. Für ihr Cauchyprodukt $\sum_{n=0}^{\infty} c_{n}$ gilt dann:
	$$ \sum_{n=0}^{\infty} c_{n} \text{ ist absolut konvergent und } \sum_{n=0}^{\infty} c_{n} = (\sum_{n=0}^{\infty} a_{n}) (\sum_{n=0}^{\infty} b_{n}) $$
\end{satz}


\begin{beispiel*}
	Sei $x \in \R$ und $|x | < 1$. \\
	Bekannt: $\sum_{n=0}^{\infty} x^{n}$ konvergiert absolut und	 $\sum_{k=0}^{\infty} x^{n} = \frac{1}{1-x}$. Also
	$$ \frac{1}{(1-x)^{2}} = (\sum_{n=0}^{\infty} x^{n})(\sum_{n=0}^{\infty} x^{n}) \overset{\ref{3.11:satz}}{=} \sum_{n=0}^{\infty} c_{n} $$
	mit $c_{n} = \sum_{k=0}^{n} x^{k} x^{n-k} = (n+1)x^{n}$. Also:
	$$ \frac{1}{(1-x)^{2}} = \sum_{n=0}^{\infty} (n+1) x^{n} \quad (|x| < 1) $$
	z.B.: $(x = \frac{1}{2}): 4 = \sum_{n=0}^{\infty} \frac{(n+1)}{2^{n}}$. Weiter:
	$$ \frac{x}{(1-x)^{2}} = \sum_{n=0}^{\infty} (n+1) x^{n+1} = \sum_{n=1}^{\infty} n x^{n} $$
	z.B.: $(x = \frac{1}{2}): 2 = \sum_{n=1}^{\infty} \frac{n}{2^{n}}$, also $1 = \sum_{n=1}^{\infty} \frac{n}{2^{n+1}}$
\end{beispiel*}


\begin{namedtheorem}[Exponentialfunktion] \label{3.12:prop-Exponentialfunktion}
	$E(X) = \sum_{n=0}^{\infty} \frac{x^{n}}{n!}$ $(x \in \R)$
	\begin{enumerate}
		\item $E(0) = 1, E(1) = e$
		\item $E(x + y) = E(x) E(y)$ $\forall x, y \in \R$
		\item $E(x_{1} + \dotsc + x_{m}) = E(x_{1}) \cdot \dotsc \cdot E(x_{m})$ $\forall x_{1}, \dotsc, x_{m} \in \R$
		\item $E(x) > 1 ~\forall x > 0$; $E(X) > 0 ~\forall x \in \R$; $E(-x) = E(x)^{-1} ~\forall x \in \R$
		\item $E(rx) = E(x)^{r} ~\forall x \in \R, \forall r \in \Q$
		\item $E(r) = e^{r} ~\forall r \in \Q$
		\item $E$ ist auf $\R$ streng monoton wachsend, d.h. aus $x < y$ folgt stets $E(x) < E(y)$
	\end{enumerate}	
\end{namedtheorem}

\begin{proof} ~\
	\begin{enumerate}
		\item klar.
		\item $E(x)E(y) = (\sum_{n=0}^{\infty} \frac{x^{n}}{n!})(\sum_{n=0}^{\infty} \frac{y^{n}}{n!}) \overset{\ref{3.11:satz}}{=} \sum_{n=0}^{\infty} c_{n}$, wobei
			$$ c_{n} = \sum_{k=0}^{\infty} \frac{x^{k}}{k!} \cdot \frac{y^{n-k}}{(n-k)!} = \frac{1}{n!} \sum_{k=0}^{n} \underbrace{\frac{n!}{k! (n-k)!}}_{\binom{n}{k}} x^{k} y^{n-k} \overset{\S1}{=} \frac{1}{n!} (x+y)^{n} $$
			Also: $E(x)E(y) = \sum_{n=0}^{\infty} \frac{(x+y)^{n}}{n!} = E(x+y)$
		\item folgt aus b)
		\item Für $x > 0$: $E(x) = 1 + \underbrace{x + \frac{x^{2}}{2!} + \frac{x^{3}}{3!} + \dotsc}_{> 0} > 1$ \\
			$1 = E\left(x + (-x)\right) \overset{b)}{=} E(x) E(-x) ~\forall x \in \R$; Insb.: $E(x) > 0 ~(x < 0)$ und $E(-x) = E(x)^{-1}$.
		\item Sei $x \in \R$. Für $n \in \N$:
			$$ E(nx) = E(x + \dotsc + x) \overset{c)}{=} E(x)^{n} $$
			$E(x) = (E n(\frac{x}{n}) = E(\frac{x}{n})^{n}$, also $E(\frac{1}{n} x) = E(x)^{\frac{1}{n}}$. \\
			Für $m, n \in \N$:
			$E(\frac{m}{n} x) = E(m \frac{x}{n}) = E(\frac{x}{n})^{m} = (E(x)^{\frac{1}{n}})^{m} = E(x)^{\frac{m}{b}}$
			Also $E(rx) = E(x)^{r} ~\forall r \in \Q$ mit $r > 0$. Sei $r \in Q$ und $r < 0$. Dann: $-r > 0$, also
			$$ \underbrace{E(-rx)}_{\overset{d)}{=} \frac{1}{E(rx)}} =e(x)^{-r} = \frac{1}{E(x)^{r}} $$
			Also $E(rx) = E(x)^{r}$
		\item folgt aus d) mit $x = 1$.
		\item Sei $x < y \Rightarrow y - x > 0 \xRightarrow[]{d)} E(y - x) > 1$
			$$ \Rightarrow 1 < E(y - x) \overset{b)}{=} E(y)E(-x) \overset{d)}{=} \frac{E(y)}{E(x)} \xRightarrow[]{d)} E(x) < E(y) $$
	\end{enumerate}
\end{proof}


\newpage


\section{Potenzreihen}

\index{Potenzreihe}
\begin{definition}
	$(a_{n})_{n=0}^{\infty}$ sei eine Folge in $\R$ und $x_{0} \in \R$. Eine Reihe der Form
		$$ \sum_{n=0}^{\infty} a_{n} (x - x_{0})^{n} = a_{0} + a_{1} (x - x_{0}) + a_{2} (x - x_{0})^{2} + \dotsc $$
		hei{\ss}t \textbf{Potenzreihe} (PR). 
\end{definition}

Frage: für welche $x \in \R$ konvergiert die Potenzreihe (absolut)? Klar: die Potenzreihe konvergiert absolut für $x = x_{0}$.

\begin{beispiele} ~\
	\begin{enumerate}
		\item $\sum_{n=0}^{\infty} \frac{x^{n}}{n!}$. Hier: $a_{n} = \frac{1}{n!}, x_{0} = 0$. Bekannt: die Potenzreihe konvergiert absolut in jeden $x \in \R$.
		\item $\sum_{n=0}^{\infty} (x - x_{0})^{n}$. Hier: $a_{n} = 1$. Setze $q \coloneqq x - x_{0}$. Bekannt: $\sum_{n=0}^{\infty} q^{n}$ konvergiert absolut $\iff |q| < 1$. D.g. die Potenzreihe konvergiert absolut $\iff |x - x_{0} | < 1$
			% todo image
		\item $\sum_{n=0}^{\infty} n^{n} (x - x_{0})^{n}$. Hier: $a_{n} = n^{n}$. Sei $x \neq x_{0}$ und $b_{n} \coloneqq n^{n} (x - x_{0})^{n}$; $\sqrt[n]{|b_{n}|} = n |x - x_{0}| \xRightarrow[]{x \neq x_{0}} \left( \sqrt[n]{|b_{n}|} \right)$ ist unbeschränkt $\xRightarrow[]{\ref{3.6:prop-Wurzelkriterium}} \sum n^{n} (x - x_{0})^{n}$ ist divergent. \\
			Also: $\sum_{n=0}^{\infty} n^{n} (x - x_{0})^{n}$ konvergiert nur für $x = x_{0}$.
	\end{enumerate}	
\end{beispiele}

\index{Konvergenzradius}
\begin{definition}
	$\sum_{n=0}^{\infty} a_{n} (x - x_{0})^{n}$ sei eine Potenzreihe. Setze
		$$ \rho \coloneqq \begin{cases}
			\infty, & \text{falls } \left( \sqrt[n]{|b_{n}|} \right) \text{ unbeschränkt}; \\
			\limsup \sqrt[n]{|b_{n}|}, & \text{falls } \left( \sqrt[n]{|b_{n}|} \right) \text{ beschränkt}
		\end{cases} $$
	und
		$$ r \coloneqq \begin{cases}
			0, & \text{falls } \rho = \infty \\
			\infty, & \text{falls } \rho = 0 \\
			\frac{1}{\rho}, & \text{falls } \rho \in (0, \infty)
		\end{cases} $$
	(kurz: \glqq $r = \frac{1}{\rho}$\grqq). $r$ hei{\ss}t der \textbf{Konvergenzradius} (KR) der Potenzreihe.
\end{definition}


\begin{satz} \label{4.1:satz}
	$\sum_{n=0}^{\infty} a_{n} (x - x_{0})^{n}$ sei eine Potenzreihe und $\rho$ und $r$ seien wie oben.
	\begin{enumerate}
		\item Ist $r = 0$, so konvergiert die Potenzreihe nur für $x = x_{0}$.
		\item Ist $r = \infty$, so konvergiert die Potenzreihe absolut für jedes $x \in \R$.
		\item Ist $r \in (0, \infty)$, so konvergiert die Potenzreihe absolut für $x \in \R$ mit $|x - x_{0}| < r$, sie divergiert für $x \in \R$ mit $|x - x_{0}| > r$. Für $x = x_{0} \pm r$ ist keine allg. Aussage möglich.
			% todo image
	\end{enumerate}
\end{satz}

\begin{proof}
	Für $x \in \R$ sei $b_{n} \coloneqq a_{n} (x - x_{0})^{n} ~(n \in \N_{0})$. \\
	Damit: $\sqrt[n]{|b_{n}(x)|} = \sqrt[n]{|a_{n}|} |x - x_{0}|$
	\begin{enumerate}
		\item Sei $x \neq x_{0}$. $r = 0 \iff \rho = 0 \iff \left( \sqrt[n]{|b_{n}(x)|} \right)$ unbeschränkt $\xRightarrow[]{\ref{3.6:prop-Wurzelkriterium}} \sum b_{n}(x)$ divergiert.
		\item $r = \infty \iff \rho = 0 \iff \limsup \sqrt[n]{|b_{n}(x)|} = 0 ~\forall x \in \R \xRightarrow[]{\ref{3.6:prop-Wurzelkriterium}}$ Beh.
		\item $\limsup \sqrt[b]{|b_{n}(x)|} = \limsup \sqrt[n]{|a_{n}|} |x - x_{0}| = \rho |x - x_{0}| = \frac{1}{r} |x - x_{0}| < 1$ 
			$$ \iff |x - x| < r $$
			Analog für $|x - x| > r$. Behauptung folgt aus \ref{3.6:prop-Wurzelkriterium}.
	\end{enumerate}	
\end{proof}


\begin{folgerung}
	$\lim_{n \rightarrow \infty} \frac{1}{\sqrt[n]{n!}} = 0$	
\end{folgerung}

\begin{proof}
	$\sum_{n=0}^{\infty} \frac{x^{n}}{n!}$ hat den Konvergenzradius $r = \infty$; $a_{n} = \frac{1}{n!} \xRightarrow[]{\ref{4.1:satz}} \rho = 0$, also $\limsup \sqrt[n]{|a_{n}|} = 0$ Hilfssatz vor \ref{3.6:prop-Wurzelkriterium} $\Rightarrow \lim \sqrt[n]{|a_{n}|} = 0$.
\end{proof}


\begin{beispiele} ~\
	\begin{enumerate}
		\item $\sum_{n=0}^{\infty} x^{n}$; $a_{n} = 1 ~(n \in \N_{0})$, $x_{0}$ = 1; $\rho = 1, r = 1$. Die Potenzreihe konvergiert für $|x| <$ absolut; sie divergiert für $|x| > 1$. $|x| = 1$: die Potenzreihe divergiert.
		\item $\sum_{n=1}^{\infty} \frac{x^{n}}{n}$, $a_{0} = 0, a_{n} = \frac{1}{n} ~(n \geq 1)$, $x_{0} = 0$; $\sqrt[n]{|a_{n}|} = \frac{1}{\sqrt[n]{n}} \rightarrow 1 \Rightarrow \rho = 1 \Rightarrow r = 1$. Die Potenzreihe konvergiert absolut für $|x| < 1$ und sie divergiert für $|x| > 1$. $x = 1$: $\sum_{n=1}^{\infty} \frac{1}{n}$ divergiert; $x = -1$: $\sum_{n=1}^{\infty} \frac{(-1)^{n}}{n}$ konvergiert.
		\item $\sum \frac{x^{n}}{n^{2}}$; $a_{0} = 0, a_{n} = \frac{1}{n^{2}} ~(n \geq 1)$, $x_{0} = 0$; $\sqrt[n]{|a_{n}|} \rightarrow 1 \Rightarrow \rho = 1 \Rightarrow r = 1$. Die Potenzreihe konvergiert absolut für $|x|< 1$, sie divergiert für $|x| > 1$. $x = 1$: $\sum \frac{1}{n^{2}}$ konvergiert absolut; $x = -1$: $\sum \frac{(-1)^{n}}{n^{2}}$ konvergiert absolut.
	\end{enumerate}	
\end{beispiele}

\index{Cosinus}
\begin{namedtheorem}[Cosinus] \label{4.2:prop-Cosinus}
	Betrachte die Reihe
	$$ \sum_{n=0}^{\infty} (-1)^{n} \frac{x^{2n}}{(2n)!} = 1 - \frac{x^{2}}{2!} + \frac{x^{4}}{4!} + \dotsc $$
	hier: $x_{0} = 0, a_{2n + 1} = 0, a_{2n} = \frac{(-1)^{n}}{(2n)!} ~(n \in \N)$. Mit $\sqrt[n]{|a_{n}|} \leq \frac{1}{\sqrt[n]{n!}}$ folgt
	$$ \sqrt[2n]{|a_{2n}|} = \frac{1}{\sqrt[2n]{(2n)!}} \rightarrow 0 \quad \text{Folgerung nach } \ref{4.1:satz} $$ 
	Also $H(\sqrt[n]{|a_{n}|}) = \{ 0 \}$. Also: $\limsup \sqrt[n]{|a_{n}|} = 0 \xRightarrow[]{\ref{4.1:satz}}$ obige Potenzreihe hat den Konvergenzradius $r = \infty$, konvergiert also absolut in jedem $x \in \R$
	$$ \textbf{Cosinus: } \begin{cases} \cos \colon \R \rightarrow \R \\ \cos x \coloneqq \sum_{n=0}^{\infty} (-1)^{n} \frac{x^{2n}}{(2n)!} \end{cases} $$
\end{namedtheorem}

\index{Sinus}
\begin{namedtheorem}[Sinus] \label{4.3:prop-Sinus}
	Ähnlich wie bei \ref{4.2:prop-Cosinus} sieht man: die Potenzreihe 
	$$ \sum_{n=0}^{\infty} (-1)^{n} \frac{x^{2n+1}}{(2n+1)!} = x - \frac{x^{3}}{3!} + \frac{x^{5}}{5!} +- \dotsc $$
	konvergiert absolut für jedes $x \in \R$.
	$$ \textbf{Sinus: } \begin{cases} \sin \colon \R \rightarrow \R \\ \sin x \coloneqq \sum_{n=0}^{\infty} (-1)^{n} \frac{x^{2n+1}}{(2n+1)!} \end{cases} $$	
\end{namedtheorem}

Klar: $\sin 0 = 0, \cos 0 = 1$, $\sin (-x) = - \sin(x), \cos(-x) = \cos(x) ~\forall x \in \R$.

\index{Additionstheoreme}
Ähnlich wie in \ref{3.12:prop-Exponentialfunktion} zeigt man (mit Cauchyprodukt) die \textbf{Additiostheoreme} $\forall x,y \in \R$:
\begin{align*}
	\sin(x+y) & = \sin x \cos y + \cos x \sin y \\
	\cos(x+y) & = \cos x \cos y - \sin x \sin y
\end{align*}

Für $x \in \R$:
	$$ 1 = \cos(0) = \cos( x + (-x) )= \cos x \cos(-x) - \sin x \sin(-x) = \cos^{2} x + \sin^{2} x $$

Für alle $x \in \R$ ist $\cos^{2} x \leq \cos^{2} x + \sin^{2} x = 1$, also $|\cos x | \leq 1$ und $sin^{2} x \leq cos^{2} x + \sin^{2} x = 1$, also $|\sin x | \leq 1$.


\begin{satz} \label{4.4:satz}
	Es ist $a_{n} \neq 0$ ffa $n \in \N$, die Folge $\left( \left| \frac{a_{n}}{a_{n+1}} \right| \right)$ sei konvergent und $L \coloneqq \lim \left| \frac{a_{n}}{a_{n+1}} \right|$. Dann hat die Potenzreihe $\sum_{n=0}^{\infty} a_{n} (x - x_{0})^{n}$ den Konvergenzradius $L$.
\end{satz}

\begin{proof}
	Sei $x \in \R$, $x \neq x_{0}$ und $b_{n} \coloneqq a_{n} (x - x_{0})^{n}$. Dann:
	\[ \left| \frac{b_{n+1}}{b_{n}} \right| = \left| \frac{a_{n+1}}{a_{n}} \right| |x - x_{0}| \tag*{$(*)$} \]
	Fall 1: $L = 0$. $|x - x_{0}| > 0 \Rightarrow \frac{|a_{n}|}{|a_{n+1}|} \leq | x - x_{0}|$ ffa $n$
	$$ \xRightarrow[]{(*)} \left| \frac{b_{n+1}}{b_{n}} \right| \geq \text{ ffa } n \xRightarrow[]{\ref{3.7:prop-Quotientenkriterium}} \sum b_{n} \text{ divergiert} $$
	Die Potenzreihe konvergiert also nur für $x = x_{0}$, also $r = 0 = L$. \\
	Fall 2: $L > 0$. $\xRightarrow[]{(*)} \lim \left| \frac{b_{n+1}}{b_{n}} \right| = \frac{1}{L} |x - x_{0}|$
	$$ \xRightarrow[]{\ref{3.8:folg}} \begin{cases}
		\text{Die Potenzreihe konv. absolut für } |x - x_{0}| < L \\
		\text{Die Potenzreihe divergiert für } |x - x_{0}| > L
	\end{cases} $$
	$\xRightarrow[]{\ref{4.1:satz}} r = L$.
\end{proof}


\newpage


\section{q-adische Entwicklung}

\begin{definition}
	Sei $x \in \R$. Dann existiert genau eine Zahl $k \in \Z$: $k \leq x < k + 1$; 
		$$ [x] \coloneqq k = \text{ grö{\ss}te ganze Zahl } \leq x $$
	% todo image
\end{definition}


\begin{vereinbarung}
	In diesem $\S$ sei stets $a \geq 0, q \in \N$ und $q > 1$.
\end{vereinbarung}


Setze $z_{0} \coloneqq [a]$, dann: $z_{0} \leq a < z_{0} + 1$. \\
Setze $z_{1} \coloneqq [(a-z_{0})q]$, dann: $z_{1} \leq aq - z_{0}q < z_{1} + 1$. \\
Also
	$$ z_{0} + \frac{z_{1}}{q} \leq a < z_{0} + \frac{z_{1}}{q} + \frac{1}{q} $$
Es ist $z_{1} \in \N_{0}$: Annahme: $z_{1} \geq 1 \Rightarrow \frac{z_{1}}{q} \geq 1$
	$$ \Rightarrow z_{0} + 1 \leq z_{0} + \frac{z_{1}}{q} \leq a < z_{0} +1 $$
Widerspruch! Also: $z_{1} \in \{ 0, 1, \dotsc, q - 1 \}$. \\
Setze  $z_{2} \coloneqq [(a-z_{0}-\frac{z_{1}}{q})q^{2}]$, dann (wie oben)
	$$ z_{0} + \frac{z_{1}}{q} + \frac{z_{2}}{q^{2}} \leq a < z_{0} + \frac{z_{1}}{q} + \frac{z_{2}}{q^{2}} + \frac{1}{q^{2}} $$
und $z_{2} \in \{ 0, 1, \dotsc, q - 1 \}$. \\
Allgemein (induktiv): sind $z_{0}, \dotsc, z_{n}$ schon definiert, so setze
	$$ z_{n+1} \coloneqq [(a - z_{0} - \frac{z_{1}}{q} - \dotsc - \frac{z_{n}}{q^{n}}) q^{n+1}] $$
Wir erhalten so eine Folge $(z_{n})_{n=0}^{\infty}$ mit:
	$$ (*) \begin{cases} ~ z_{0} \in \N_{0}, z_{n} \in \{ 0, 1, \dotsc, q - 1 \} ~\forall n \geq 1 \\ \text{ und} \\ ~\underbrace{z_{0} + \frac{z_{1}}{q} + \dotsc + \frac{z_{n}}{q^{n}}}_{\eqqcolon S_{n}} \leq a < \underbrace{ z_{0} + \frac{z_{1}}{q} + \dotsc + \frac{z_{n}}{q^{n}} + \frac{1}{q^{n}}}_{= S_{n} + \frac{1}{q^{n}}} \end{cases} $$


In den gro{\ss}en Übungen wird gezeigt:

\begin{satz} \label{5.1:satz}
	Ist $(\tilde{z}_{n})_{0}^{\infty}$ eine weitere Folge mit den Eigenschaften in $(*)$, so gilt: 
	$$ z_{n} = \tilde{z}_{n} \quad \forall n \in \N_{0} $$
\end{satz}

\index{q-adische Entwicklung}
Es ist
	$$ 0 \leq \frac{z_{n}}{q^{n}} \leq \frac{q - 1}{q^n} ~\forall n \in \N \quad \text{und} \quad \sum_{n=1}^{\infty} \frac{q - 1}{q^{n}} \text{ konvergiert}. $$
$\xRightarrow[]{\ref{3.5.a:satz}} \sum_{n=0}^{\infty}\sqrt{z_{n}}{q^{n}}$ konvergiert. Also ist $(s_{n})$ konvergent.
	$$ \xRightarrow[]{(*)} a = \lim s_{n} = \sum_{n=0}^{\infty} \frac{z_{n}}{q^{n}} $$
Dafür schreibt man: $a = z_{0}, z_{1} z_{2} z_{3} \dotsc$ (\textbf{q-adische Entwicklung von }$a$)

$q = 10$: Dezimalentwicklung; $q = 2$: Dualentwicklung; \\
(Gilt mit einem $m \in \N$: $z_{n} = 0 ~\forall n > m$, so schreibt man auch: $a = z_{0}, z_{1} \dotsc z_{m}$).


\begin{beispiele} ~\
	\begin{enumerate}
		\item $q = 10, a = 1$. $z_{0} = 1, z_{1} = [(a - z_{0})q] = 0$; \\
			$z_{2} = [(a - z_{0} - \frac{z_{1}}{q})q^{2}] = 0, \dotsc$ allg.: $z_{n} = 0 ~\forall n \geq 1$. \\
			Also $1 = 1,000\dotsc$
		\item $q = 10, a = \frac{1}{2}$. $z_{0} = 0, z_{1} = [(a - z_{0})q] = [\frac{1}{2} 10] = 5$; \\
			$z_{2} = [(a - z_{0} - \frac{z_{1}}{q})q^{2}] = [(\frac{1}{2} - \frac{5}{10}) 100] = 0, \dotsc$ allg.: $z_{n} = 0 ~\forall n \geq 2$. \\
			Also $\frac{1}{2} = 0,5000\dotsc = 0,5$
	\end{enumerate}
\end{beispiele}


\begin{definition}
	Sei $b \in \R$ und $b < 0$. Weiter sei
		$$ -b = z_{0}, z_{1} z_{2} \dotsc $$
	die q-adische Entwicklung von $-b$. Dann ist $b = - z_{0}, z_{1} z_{2} \dotsc$ die q-adische Entwicklung von $b$.
\end{definition}


\begin{satz} \label{5.2:satz}
	Sei $a = z_{0}, z_{1} z_{2} z_{3} \dotsc$ die q-adische Entwicklung von $a$. Dann ist $z_{n} = q -$ ffa $n \in \N$ nicht möglich.
\end{satz}

\begin{proof}
	Annahme: $\exists m \in \N$: $z_{n} = q - 1 ~\forall n \geq m$. Dann 
		$$ a = \sum_{n=0}^{\infty} \frac{z_{n}}{q^{n}} = \underbrace{\sum_{n=0}^{m-1} \frac{z_{n}}{q^{n}}}_{= S_{m-1}} + \sum_{n=m}^{\infty} \frac{q-1}{q^{n}} $$
	und damit
	\begin{align*}
		\sum_{n=m}^{\infty} \frac{q-1}{q^{m}} & = (q-1) \left( \frac{1}{q^{m}} + \frac{1}{q^{m+1}} + \dotsc \right) \\
			& = \frac{q - 1}{q^{m}} (1 + \frac{1}{q} + \frac{1}{q^{2}} + \dotsc) \\
			& = \frac{q - 1}{q^{m}} \frac{1}{1 - \frac{1}{q}} = \frac{1}{q^{m-1}}
	\end{align*} 
	Also $a = S_{m-1} + \frac{1}{q^{m-1}} \overset{(*)}{>} a$. Widerspruch!
\end{proof}


\begin{satz} \label{5.3:satz}
	$\R$ ist überabzählbar.
\end{satz}

\begin{proof}
	es genügt zu zeigen: $[0, 1)$ ist überabzählbar. Annahme $[0, 1)$ abzählbar, also $[0, 1) = \{ a_{1}, a_{2}, \dotsc \}$. Für $j \in \N$ sei
		$$ a_{j} = 0, z_{1}^{(j)} z_{2}^{(j)} z_{3}^{(j)} \dotsc $$
	die 3-adische Entwicklung von $a_{j}$, also $z_{k}^{(j)} \in \{ 0, 1, 2 \}$. Setze
		$$ z_{k} \coloneqq \begin{cases} 1, & \text{falls } z_{k}^{(k)} = 0 \text{ oder } z_{k}^{(k)} = 2 \\ 0, & \text{falls } z_{k}^{(k)} = 1 \end{cases} $$
	Dann: $z_{k} \neq z_{k}^{k} ~\forall k \in \N$ $(**)$ \\
	Setze $a \coloneqq \sum_{n=1}^{\infty} \frac{z_{n}}{3^{n}}$. Dann:
		$$ 0 \leq a \leq \sum_{n=1}^{\infty} \frac{1}{3^{n}} = \frac{1}{2}, \text{ also } a \in [0, 1) $$
	Übung: $0, z_{1} z_{2} z_{3} \dotsc$ ist die 3-adische Entwicklung von $a$.$a \in [0, 1) \Rightarrow \exists m \in \N: a = a_{m}$, also
		$$ 0, z_{1} z_{2} z_{3} \dotsc = 0, z_{1}^{(m)} z_{2}^{(m)} \dotsc $$
	und $z_{j} = z_{j}^{(m)} ~\forall j \in \N \xRightarrow[]{j = m} z_{m} = z_{m}^{(m)}$. Widerspruch zu $(**)$.
\end{proof}


\newpage


\section{Grenzwerte bei Funktionen}

\index{Häufungspunkt}
\begin{definition}
	Sei $D \subseteq \R$ und $x_{0} \in \R$. $x_{0}$ hei{\ss}t ein \textbf{Häufungspunkt} (HP) von $D \iff \exists$ Folge $(x_{n})$ in $D \setminus \{ x_{0} \}$ mit $x_{n} \rightarrow x_{0}$.
\end{definition}


\begin{beispiele}
	\begin{enumerate}
		\item $D = (0, 1]$ \\ % todo image
			$x_{0}$ ist Häufungspunkt von $D \iff x_{0} \in [0 1]$
		\item $D = \{ \frac{1}{n} : n \in \N \}$ \\ % todo image
			$D$ hat genau einen Häufungspunkt: $x_{0} = 0$.
		\item Ist $D$ endlich, so hat $D$ keine Häufungspunkte. 
	\end{enumerate}	
\end{beispiele}


\begin{hilfssatz} \label{6.1:hsatz}
	Sei $D \subseteq \R$ und $x_{0} \in \R$. $x_{0}$ ist Häufungspunkt von $D \iff \forall \epsilon > 0: U_{\epsilon}(x_{0}) \cap (D \setminus \{ x_{0} \}) \neq \emptyset$.
\end{hilfssatz}

\begin{proof} ~\\
	$"'\Rightarrow"'"$ $\exists$ Folge $(x_{n})$ in $D \setminus \{ x_{0} \} : x_{n} \rightarrow x_{0}$. Sei $\epsilon > 0$: $\exists n_{0} \in \N: x_{n} \in U_{\epsilon}(x_{0}) \cap (D \setminus \{ x_{0} \})$ $\forall n \geq n_{0}$ \\
	$"'\Leftarrow"'"$ $\exists x_{1} \in U_{1}(x_{0}) \cap (D \setminus \{ x_{0} \})$, also $|x_{1} - x_{0}| < 1$ $\exists x_{2} \in U_{\frac{1}{2}}(x_{0} \cap (D \setminus \{ x_{0} \})$, also $|x_{2} - x_{0}| < \frac{1}{2}$, etc. \\
	Wir erhalten eine Folge $(x_{n})$ in $D \setminus \{ x_{0} \}$ mit $|x_{n} - x_{0}| < \frac{1}{n} ~\forall n$, also $x_{n} \rightarrow x_{0}$.
\end{proof}


\begin{vereinbarung}
	ab jetzt sei stets in den $\S$en:
	\begin{itemize}
		\item $\emptyset \neq D \subseteq \R, x_{0}$ ein Häufungspunkt von $D$ und
		\item $f \coloneqq D \rightarrow \R$ eine Funktion	
	\end{itemize}
\end{vereinbarung}


\begin{bezeichnung} ~\
	\begin{enumerate}
		\item $D_{\delta}(x_{0}) \coloneqq U_{\delta}(x_{0}) \cap (D \setminus \{ x_{0} \})$
		\item Sei $M \subseteq D$ und $g \colon D \rightarrow \R$ eine weitere Funktion mit $f \leq g$ auf $M \iff f(x) \leq g(x) ~\forall x \in M$.
	\end{enumerate}
\end{bezeichnung}


\begin{definition}
	$\lim_{x \rightarrow x_{0}} f(x)$ existiert $\iff \exists a \in \R$: für jede Folge $(x_{n})$ in $D \in \{ x_{0} \}$ mit $x_{n} \rightarrow x_{0}$ gilt: $f(x_{n}) \rightarrow a$.
	In diesem Fall ist $a$ eindeutig bestimmt und wir schreiben:
		$$ \lim_{x \rightarrow x_{0}} f(x) = a \text{ oder } f(x) \rightarrow a ~(x \rightarrow x_{0}) $$
\end{definition}


\begin{bemerkung}
	sollte $x_{0} \in D$ sein, so ist der Wert $f(x_{0})$ in obiger Definition nicht relevant. Relevant ist allein das Verfahren von $f$ in das $\glqq$Nähe$\grqq$ von $x_{0}$.	
\end{bemerkung}

\index{Grenzwert!linksseitiger} \index{Grenzwert!rechtsseitiger}
\begin{beispiele} ~\
	\begin{enumerate}
		\item $D \coloneqq [0, \infty), p \in \N; f(x) \coloneqq \sqrt[p]{x}$. Sei $x_{0} \in D$ (dann ist $x_{0}$ eine Häufungspunkt von $D$). Sei $(x_{n})$ eine Folge in $D \setminus \{ x_{0} \}$ und $x_{n} \rightarrow x_{0} \xRightarrow[]{\ref{2.4:bsp}} \sqrt[p]{x_{n}} \rightarrow \sqrt[p]{x_{0}}$. Also
			$$ \lim_{x \rightarrow x_{0}} f(x) = \sqrt[p]{x_{0}} $$
		\item $D = (0, 1]$
			$$ f(x) \coloneqq \begin{cases} x^{2}, & 0 < x < \frac{1}{2} \\ \frac{1}{2}, & x = \frac{1}{2} \\ 1, & \frac{1}{2} < x < 1 \\ 0, & x = 1 \end{cases} $$
			Klar: $\lim_{x \rightarrow 0} f(x), \lim_{x \rightarrow 1} f(x) = 1$
			$$ x_{n} \coloneqq \frac{1}{2} - \frac{1}{n}, z_{n} \coloneqq \frac{1}{2} + \frac{1}{n} ~(n \geq 3) $$
			Dann: $x_{n} \rightarrow \frac{1}{2}, z_{n} \rightarrow \frac{1}{2}$, aber $f(x) = \left( \frac{1}{2} - \frac{1}{n} \right)^{2} \rightarrow \frac{1}{4} \neq 1 \leftarrow f(z_{n})$. \\
			D.h. $\lim_{x \rightarrow \frac{1}{2}} f(x)$ existiert nicht, aber $\lim_{\underset{x \in (0, \frac{1}{2})}{x \rightarrow \frac{1}{2}}} f(x) = \frac{1}{4}$, dafür schreibt man $\lim_{x \rightarrow \frac{1}{2}-0} f(x) = \frac{1}{4}$ (linksseitiger Grenzwert) \\
			Analog: $\lim_{\underset{x \in (\frac{1}{2}, \infty)}{x \rightarrow \frac{1}{2}}} f(x) = 1$ (rechtsseitiger Grenzwert)
		\item $D = \R$, $f = E$, also $f(x) = \sum_{n=0}^{\infty} \frac{x^{n}}{n!}; x_{0} = 0$. Sei $|x| \leq 1$
			\begin{align*}
				|E(x) - E(0)| & = |E(x) - 1| = |x + \frac{x^{2}}{2!} + \frac{x^{3}}{3!} + \dotsc | \\
				& = |x| \left| 1 + \frac{x}{2!} + \frac{x^{2}}{3!} + \dotsc \right| \\
				& \leq |x| \left( 1 + \frac{|x|}{2!} + \frac{|x|^{2}}{3!} + \dotsc \right) \\
				& \leq |x| \left( 1 + \frac{1}{2!} + \frac{1}{3!} + \dotsc \right) \\
				& = |x| ( e - 1 )
			\end{align*}
			Sei $(x_{n})$ Folge in $\R$: $x_{n} \rightarrow 0$. $\exists n_{0} \in \N: |x_{n}| \leq 1 ~\forall n \geq n_{0} \Rightarrow |E(x_{n}) - | \leq |x_{n}|(e-1) ~\forall n \geq n_{0} \Rightarrow E(x_{n}) \rightarrow 1$ \\
			Also: $\lim_{x \rightarrow 0} E(x) = 1 = E(0)$. Somit: $\lim_{x \rightarrow 0} \sum_{n=0}^{\infty} \frac{x^{n}}{n!} = \sum_{n=0}^{\infty} \left( \lim_{x \rightarrow 0} \frac{x1{n}}{n!} \right)$.
	\end{enumerate}	
\end{beispiele}

\index{Konvergenzkriterium!Cauchy}
\begin{satz} \label{6.2:satz} ~\
	\begin{enumerate}
		\item $\lim_{x \rightarrow x_{0}} f(x) = a \iff \forall \epsilon > 0 ~\exists \delta > 0 (\delta = \delta(\epsilon)):$
			$$ (*) |f(x)- a| < \epsilon ~\forall x \in D_{\delta}(x_{0}) $$ \label{6.2.a:satz}
		\item $\lim_{x \rightarrow x_{0}} f(x)$ existiert
			$$ \iff \text{ für jede Folge } (x_{n}) \text{ in } D \setminus \{ x_{0} \} \text{ mit } x_{n} \rightarrow x_{0} \text{ ist } (f(x_{n})) \text{ konvergent}. $$ \label{6.2.b:satz} 
		\item \textbf{Cauchykriterium}: $\lim_{x \rightarrow x_{0}} f(x)$ existiert
			$$ \iff \forall \epsilon > 0 ~\exists \delta = \delta(\epsilon) > 0 : |f(x_{1}) - f(x_{2})| < \epsilon ~\forall x_{1}, x_{2} \in D_{\delta}(x_{0}) $$ \label{6.2.c:satz}
	\end{enumerate}
\end{satz}


\begin{satz} \label{6.3:satz}
	$f, g, h \colon D \rightarrow \R$ seien Funktionen. (Erinnerung: $D_{\delta}(x_{0}) \coloneqq U_{\delta}(x_{0}) \cap (D \setminus \{ x_{0} \})$). Es seien $a, b \in \R$ und es gelte $f(x) \rightarrow a, g(x) \rightarrow b ~(x \rightarrow x_{0})$. Dann:
	\begin{enumerate}
		\item $\alpha f(x) + \beta g(x) \rightarrow \alpha a + \beta b$; $f(x) g(x) \rightarrow a b$, \\
			$|f(x)| \rightarrow |a| ~(x \rightarrow x_{0})$ \label{6.3.a:satz}
		\item Ist $a \neq 0$, so existiert ein $\delta > 0$: $f(x) \neq 0 ~\forall x \in D_{\delta}(x_{0})$. Für $\frac{1}{f} \colon D_{\delta}(x_{0}) \rightarrow \R$ gilt: $\frac{1}{f(x)} \rightarrow \frac{1}{a} ~(x \rightarrow x_{0})$. \label{6.3.b:satz}
		\item Für ein $\delta > 0$ gelte: $f \leq g$ auf $D_{\delta}(x_{0})$. Dann: $a \leq b$ \label{6.3.c:satz}
		\item Für ein $\delta > 0$ gelte: $f \leq h \leq g$ auf $D_{\delta}(x_{0})$. Ist $a = b$, so gilt: $h(x) \rightarrow a ~(x \rightarrow x_{0})$. \label{6.3.d:satz}
	\end{enumerate}
\end{satz}

\begin{proof}
	z. B.: c) Sei $(x_{n})$ eine Folge in $D \setminus \{ x_{0} \}$ und $x_{n} \rightarrow x_{0}$. $\exists n_{0} \in \N$: 
		$$ x_{n} \in D_{\delta}(x_{0}) ~\forall n \geq n_{0} $$ 
	Dann: $f(x_{n}) \leq g(x_{n}) ~\forall n \geq n_{0} \xRightarrow[]{\ref{2.2:satz}} a = \lim f(x_{n}) \leq \lim g(x_{n}) = b$.	
\end{proof}


\begin{definition} ~\
	\begin{enumerate}
		\item Sei $(x_{n})$ eine Folge in $\R$.
			\begin{align*}
				x_{n} \rightarrow \infty & \iff \forall c > 0 ~\exists n_{0} = n_{0}(c) \in \N: x_{n} > c ~\forall n \geq n_{0} \\
				x_{n} \rightarrow - \infty & \iff \forall c < 0 ~\exists n_{0} = n_{0}(c) \in \N: x_{n} < c ~\forall n \geq n_{0} 
			\end{align*}
			Übung: $x_{n} \rightarrow \infty \iff x_{n} > 0 \text{ ffa } n \in \N \text{ und } \frac{1}{x_{n}} \rightarrow 0$ und \\
				$x_{n} \rightarrow -\infty\iff x_{n} < 0 \text{ ffa } n \in \N \text{ und } \frac{1}{x_{n}} \rightarrow 0$
		\item Sei $D \subseteq \R$, $x_{0}$ Häufungspunkt von $D$ und $f \colon D \rightarrow \R$ eine Funktion
			\begin{align*}
				\lim_{x \rightarrow x_{0}} f(x) = \infty \iff & \forall (x_{n}) \text{ in } D \setminus \{ x_{0} \} \text{ mit } x_{n} \rightarrow x_{0} \text{ gilt: } f(x_{n}) \rightarrow \infty \\
				\lim_{x \rightarrow x_{0}} f(x) = \infty \iff & \forall (x_{n}) \text{ in } D \setminus \{ x_{0} \} \text{ mit } x_{n} \rightarrow x_{0} \text{ gilt: } f(x_{n}) \rightarrow -\infty
			\end{align*}
		\item $D$ sei nicht nach oben beschränkt, $f \colon D \rightarrow \R$ sei eine Funktion und $a \in \R \cup \{ \infty, - \infty \}$.
			$$ \lim_{x \rightarrow \infty} f(x) = a \iff \forall (x_{n}) \text{ in } D \text{ mit } x_{n} \rightarrow \infty \text{ gilt: } f(x_{n}) \rightarrow a $$
		\item $D$ sei nicht nach unten beschränkt, $f \colon D \rightarrow \R$ sei eine Funktion und $a \in \R \cup \{ \infty, - \infty \}$.
			$$ \lim_{x \rightarrow -\infty} f(x) = a \iff \forall (x_{n}) \text{ in } D \text{ mit } x_{n} \rightarrow -\infty \text{ gilt: } f(x_{n}) \rightarrow a $$		
	\end{enumerate}
\end{definition}


\begin{beispiel}
	$\frac{1}{x} \rightarrow \infty ~(x \rightarrow 0 + 0)$, $\frac{1}{x} \rightarrow -\infty ~(x \rightarrow 0 - 0)$, $\frac{1}{x} \rightarrow 0 ~(x \rightarrow \pm \infty)$
\end{beispiel}


\begin{namedtheorem}[Exponentialfunktionen] \label{6.4:prop-Exponentialfunktionen}
	$E(x) = \sum_{n=0}^{\infty} \frac{x^{n}}{n!} = 1 + x + \frac{x^{2}}{2!} + \frac{x^{3}}{3!} + \dotsc$ \\
	Sei $p \in \N_{0}$: für $x > 0$:
		$$ E(x) = 1 + x + \frac{x^{2}}{2!} + \dotsc + \frac{x^{p+1}}{(p+1)!} + \dotsc \geq \frac{x^{p+1}}{(p+1)!} $$
	Also: $\forall x > 0$: $\frac{E(x)}{x^{p}} > \frac{x}{(p+1)!}$.
	Somit:
		$$ \frac{E(x)}{x^{p}} \rightarrow \infty ~(x \rightarrow \infty) $$
	Insbes. ($p = 0$): $E(x) \rightarrow \infty ~(x \rightarrow \infty)$. Es ist $E(-x) = \frac{1}{E(x)} \rightarrow 0 ~(x \rightarrow \infty)$, also: $(x) \rightarrow  0 ~(x \rightarrow -\infty)$.
	\begin{figure*}[!ht] \centering
		\begin{tikzpicture}
     		\draw[->] (-3,0) -- (4.2,0) node[right] {$x$};
      		\draw[->] (0,-3) -- (0,4.2) node[above] {$y$};
      		\draw[scale=0.5,domain=-10:2,smooth,variable=\x,blue] plot ({\x},{e^\x});
    	\end{tikzpicture}
		\caption{Exponentialfunktion.}	
	\end{figure*}
\end{namedtheorem}


\newpage


\section{Stetigkeit}

\index{stetig}
\begin{definition}
	Sei $D \subseteq \R$, $f \colon D \rightarrow \R$ eine Funktion und $x_{0} \in D$. 
	\begin{enumerate}
		\item $f$ hei{\ss}t $\textbf{in }  x_{0} \textbf{ stetig} \iff$ für jede Folge $(x_{n})$ in $D$ mit $x_{n} \rightarrow x_{0}$ gilt: $f(x_{n}) \rightarrow f(x_{0})$.
		\item $f$ hei{\ss}t \textbf{auf D stetig} $\iff f$ ist in jedem $x \in D$ stetig.
	\end{enumerate}
\end{definition}


\begin{beispiele}
	\begin{enumerate}
		\item $D = [0 \infty), p \in \N, f(x) = \sqrt[p]{x}$. Bekannt: ist $(x_{n})$ eine Folge in $D$ it $x_{n} \rightarrow x_{0} \in D$, so gilt $f(x_{n}) \rightarrow f(x_{0})$. Also: $f \in C\left([0, \infty]\right)$
		\item $D = [0, 1] \cup \{ 2 \}$, $ f(x) \coloneqq \begin{cases}
				x^{2}, & 0 \leq x < 1 \\ 0, & x = 1 \\ 1, & x = 2
			\end{cases}$ \\
			% todo image
			Klar: $f$ ist stetig in jedem $x \in [0, 1)$. 
			\begin{itemize}
				\item $x_{0} \coloneqq 1, x_{n} \coloneqq 1 - \frac{1}{n}$. Dann ist $(x_{n})$ eine Folge in $D$ mit $x_{n} \rightarrow 1$, aber $f(x_{n}) = x_{n}^{2} \rightarrow 1 \neq 0 = f(1)$. $f$ ist also in $x_{0} = 1$ nicht stetig.
				\item $x_{0} \coloneqq 2$, sei $(x_{n})$ eine Folge in $D$ mit $x_{n} \rightarrow 2$. Dann: $x_{n} = 2$ ffa $n$, also $f(x_{n}) = 1$ ffa $n$. Somit: $f(x_{n}) \rightarrow 1 = f(2)$. $f$ ist also in $x_{0} = 2$ stetig.
			\end{itemize}	
	\end{enumerate}	
\end{beispiele}


\begin{satz} \label{7.1:satz}
	$D \subseteq \R, f \colon D \rightarrow \R$ eine Funktion, $x_{0} \in D$.
	\begin{enumerate}
		\item $f$ ist in $x_{0}$ stetig $\iff \forall \epsilon > 0 ~\exists \delta = \delta(\epsilon) > 0$:
			$$ \forall x\in D_{\delta}(x_{0}): \quad |f(x) - f(x_{0})| < \epsilon $$
		\item Ist $x_{0}$ Häufungspunkt von $D$, so gilt:
			$$ f \text{ ist in } x_{0} \text{ stetig } \iff \lim_{x \rightarrow x_{0}} f(x) = f(x_{0}) $$
	\end{enumerate}
\end{satz}

\begin{proof} \
	\begin{enumerate}
		\item fast wörtlich wie bei \ref{6.2:satz}.
		\item Übung.
	\end{enumerate}
\end{proof}


\begin{satz} ~\ \label{7.2:satz} 
	\begin{enumerate}
		\item $f, g \colon D \rightarrow \R$ seien stetig in $x_{0} \in D$ und es seien $\alpha, \beta \in \R$, Dann sind stetig in $x_{0}$;
			$$ \alpha f + \beta g, fg \text{ und } |f| $$
			Ist $x_{0} \in \tilde{D} \coloneqq \{ x \in D : f(x) \neq 0 \}$, so ist $\frac{1}{f} \colon \tilde{D} \rightarrow \R$ stetig in $x_{0}$.
		\item Sind $f, g \in C(D)$ und $\alpha, \beta \in \R$, so gilt:
			$$ \alpha f + \beta g, fg \text{ und } |f| \in C(D)$$
	\end{enumerate}
\end{satz}

\begin{proof} a) Mit \ref{2.2:satz}; b) folgt aus a). \end{proof}


\begin{satz} \label{7.3:satz}
	Es seien $D, D_{0} \subseteq \R, f \colon D \rightarrow \R, g \colon D_{0} \rightarrow \R$ Funktionen, $f(D) \subseteq D_{0}, x_{0} \in D$ und $y_{0} \coloneqq(x_{0})$. Ist $f$ in $x_{0}$ stetig und ist $g$ in $y_{0}$ stetig, so ist
			$$ g \circ f \colon D \rightarrow \R $$
		stetig in $x_{0}$, wobei $(g \circ f)(x) \coloneqq g(f(x))$.
\end{satz}

\begin{proof}
	Sei $(x_{n})$ eine Folge in $D$ mit $x_{n} \rightarrow x_{0}$. $f$ stetig in $x_{0} \Rightarrow f(x_{n}) \rightarrow f(x_{0}) = y_{0}$. $g$ stetig in $y_{0} \Rightarrow \underbrace{g(f(x_{n}))}_{= (g \circ f)(x_{n})} \rightarrow g(x_{0}) = g(f(x_{0})) = (g \circ f)(x_{0})$
\end{proof}


\begin{satz} \label{7.4:satz}
	$\sum_{n=0}^{\infty} a_{n} (x - x_{0})^{n}$ sei eine Potenzreihe mit Konvergenzradius $r > 0$. Es sei $D \coloneqq (x_{0} - r, x_{0} + r)$. falls $r < \infty$ und $D \coloneqq \R$ falls $r = \infty$. Weiter sei
		$$ f(x) \coloneqq \sum_{n=0}^{\infty} a_{n} (x - x_{0})^{n} ~(x \in D) $$
	Dann: $f \in C(D)$.	
\end{satz}

\begin{proof}
	später, nach \ref{8.3:satz}.
\end{proof}


\begin{beispiele}
	Exponentialfunktionen, Sinus und Cosinus sind also auf $\R$ stetig.	
\end{beispiele}


\begin{beispiel} \label{7.5:bsp}
	Beh.: $\lim_{x \rightarrow 0} \frac{\sin x}{x} = 1$.
\end{beispiel}

\begin{proof}
	Für $x \neq 0$: 
	$$ \frac{\sin x}{x} = \frac{1}{x} \left( x - \frac{x^{3}}{3!} + \frac{x^{5}}{5!} -+ \dotsc \right) = \underbrace{1 - \frac{x^{2}}{3!} + \frac{x^{4}}{5!} -+ \dotsc}_{\text{PR mit KR } r = \infty} \xrightarrow[]{\ref{7.4:satz}} 1 ~(x \rightarrow 0) $$
\end{proof}


\begin{beispiel} \label{7.6:bsp}
	Beh.: $\lim_{x \rightarrow 0} \frac{E(x) - 1}{x} = 1$.
\end{beispiel}

\begin{proof}
	Für $x \neq 0$:
	$$ \frac{E(x) - 1}{x} = \frac{1}{x} \left( ( 1+ x + \frac{x^{2}}{2!} + \dotsc) - 1 \right) = \underbrace{1 + \frac{x}{2!} + \frac{x^{2}}{3!} + \dotsc}_{\text{PR mit KR } r = \infty} \xrightarrow[]{\ref{7.4:satz}} 1 ~(x \rightarrow 0) $$
\end{proof}


\begin{folgerung}
	$\forall x_{0} \in \R$: $\lim_{h \rightarrow 0} \frac{E(x_{0} + h) - E(x_{0})}{h} = E(x_{0})$
\end{folgerung}

\begin{proof}
	$\frac{E(x_{0} + h) - E(x_{0})}{h} =  \frac{E(x_{0}) E(h) - E(x_{0})}{h} =  E(x_{0}) \frac{E(h) - 1}{h} \xrightarrow[]{\ref{7.6:bsp}} E(x_{0}) ~(h \rightarrow 0)$
\end{proof}

\index{Zwischenwertsatz}
\begin{namedtheorem}[Zwischenwertsatz] \label{7.7:prop-Zwischenwertsatz}
	Seien $a, b \in \R, a < b, f \in C\left([a,b]\right)$ und $y_{0}$ zwischen $f(a)$ und $f(b)$. \\
	% todo image
	Dann existiert ein $x_{0} \in [a, b]: f(x_{0}) = y_{0}$.
\end{namedtheorem}

\begin{proof}
	Fall 1: $f(a) = y_{0}$ oder $f(b) = y_{0}$, fertig. \\
	Fall 2: $f(a) \neq y_{0} \neq f(b)$. o.B.d.A.: $f(a) < f(b)$, also $f(a) < y_{0} < f(b)$.
	$$ M \coloneqq \{ x \in [a, b]: f(x) \leq y_{0} \}; a \in M \Rightarrow M \neq \emptyset; $$
	$M \subseteq [a, b] \Rightarrow M$ ist beschränkt $\xRightarrow[]{\hyperref[v.axiom-a15]{(A15)}} \exists x_{0} \coloneqq \sup M \in [a, b]$.  \\
	Ist $n \in \N$, so ist $x_{0} - \frac{1}{n}$ keine obere Schranke von $M$, also ex. $x_{n} \in M: x_{n} > x_{0} - \frac{1}{n}$. \\
	Also: $\forall n \in \N$: $x_{0} - \frac{1}{n} < x_{n} \leq x_{n}$. Somit $x_{n} \rightarrow x_{0}$, $f$ stetig in $x_{0} \Rightarrow f(x_{n}) \rightarrow f(x_{0}) \xRightarrow[]{\text{Def. von }M} \forall n \in \N: f(x_{n}) \leq y_{0} \Rightarrow f(x_{0}) \leq y_{0}$. \\
	Es ist $x_{0} < b$ (andernfalls: $x_{0} = b \Rightarrow f(b) = f(x_{0}) \leq y_{0} < f(b)$, Wid!). \\
	% todo image
	$z_{n} \coloneqq x_{0} + \frac{1}{n}$. Es gilt $z_{n} \in [a, b]$ ffa $n \in \N$. $z_{n} > x_{0} \Rightarrow z_{n} \notin M \Rightarrow f(z_{n}) > y_{0}$. $z_{n} \rightarrow x_{0}$, $f$ stetig $\Rightarrow f(z_{n}) \rightarrow f(x_{0}) \Rightarrow f(x_{0}) \geq y_{0}$.
\end{proof}


\begin{folgerung}[vgl. \ref{1.6:satz}]
	Ist $\alpha > 0$ und $n \in \N$, so existiert ein $x_{0} > 0$: $x_{0}^{n} = \alpha$.	
\end{folgerung}

\begin{proof}
	$b \coloneqq 1 + \alpha, f(x) \coloneqq x^{n} ~(x \in [a, b])$. \\
	Dann: $f \in C[a, b], f(0) = 0 < \alpha$, $f(b) = (1 + \alpha)^{n} \overset{BK}{\geq} 1 + n \alpha > \alpha \xRightarrow[]{\ref{7.7:prop-Zwischenwertsatz}} \exists x_{0} \in [a, b]: f(x_{0}) = \alpha$, also $x_{0}^{n} = \alpha$. Klar: $x_{0} > 0$, denn $\alpha > 0$.
\end{proof}

\begin{bemerkung}
	Erst jetzt ist \ref{1.6:satz} vollständig bewiesen!
\end{bemerkung}


Aus \ref{7.7:prop-Zwischenwertsatz} folgt mit $y_{0} = 0$:

\index{Nullstellensatz}
\begin{namedtheorem}[Nullstellensatz von Bolzano] \label{7.8:prop-NullstellensatzVonBolzano}
	Ist $f \in C\left([a, b]\right)$ und $f(a)f(b) \leq 0$, so existiert ein $x_{0} \in [a, b]$: $f(x_{0}) = 0$.
	% todo image
\end{namedtheorem}


\begin{namedtheorem}[Exponentialfunktion]
	$E(x) = \sum_{n=0}^{\infty} \frac{x^{n}}{n!}$. Beh.: $E(\R) = (0, \infty)$.	
\end{namedtheorem}

\begin{proof}
	$\xRightarrow[]{\ref{3.12:prop-Exponentialfunktion}} \forall x \in \R E(x) > 0$, also $E(\R) \subseteq (0, \infty)$. Sei $y_{0} \in (0, \infty)$.
	$$ \xRightarrow[]{\ref{6.4:prop-Exponentialfunktionen}} E(x) \rightarrow \infty (x \rightarrow \infty) \Rightarrow \exists b > 0: E(b) > y_{0} $$	
	$$ \xRightarrow[]{\ref{6.4:prop-Exponentialfunktionen}} E(x) \rightarrow 0 (x \rightarrow -\infty) \Rightarrow \exists a < 0: E(a) < y_{0} $$
	$\xRightarrow[]{\ref{7.7:prop-Zwischenwertsatz}} \exists x_{0} \in [a, b]: E(x_{0}) = y_{0}$. Also: $y_{0} \in E(\R)$. Somit: $(0, \infty) \in E(\R)$.
\end{proof}

\index{abgeschlossen} \index{kompakt}
\begin{definition}
	Sei $D \subseteq \R$.
	\begin{enumerate}
		\item $D$ hei{\ss}t \textbf{abgeschlossen} $\iff$ für jede konvergente Folge $(x_{n})$ in $D$ gilt $\lim x_{n} \in D$.
		\item $D$ hei{\ss}t \textbf{kompakt} $\iff$ jede Folge $(x_{n})$ in $D$ enthält eine konvergente Teilfolge $(x_{n_{k}})$ mit $\lim_{k \rightarrow \infty} x_{n_{k}} \in D$.
	\end{enumerate}
\end{definition}


\begin{satz} \label{7.10:satz}
	Sei $D \subseteq \R$.
	\begin{enumerate}
		\item $D$ ist abgeschlossen $\iff$ jeder Häufungspunkt von $D$ gehört zu $D$.
		\item $D$ ist kompakt $\iff$ $D$ ist beschränkt und abgeschlossen.
		\item Ist $D$ kompakt und $D \neq \emptyset$, so existieren $\max D$ und $\min D$.
	\end{enumerate}
\end{satz}


\begin{beispiele} ~\
	\begin{enumerate}
		\item $[a, b]$ ist kompakt, also auch abgeschlossen.	
		\item endliche Mengen sind kompakt.
		\item $[a, \infty), (-\infty, a], \R$ sind abgeschlossen, aber nicht kompakt.
		\item $\emptyset$ ist abgeschlossen.
		\item $(a, b], [a, b), (a, b)$ sind nicht abgeschlossen.
	\end{enumerate}
\end{beispiele}


\begin{proof}(7.10) \\
	\begin{enumerate}
		\item i. d. gro{\ss}en Übung.
		\item $"'\Leftarrow"'$ Folgt direkt aus \ref{2.12:satz-BolzanoWeierstrass}, $"'\Rightarrow"'$ Übung.
		\item Sei$s \coloneqq \sup D$. $\forall n \in \N ~\exists x_{n} \in D: x_{n} > s - \frac{1}{n}$, also $\forall n \in \N$ $s - \frac{1}{n} < x_{n} \leq s$. Somit: $x_{n} \rightarrow s \xRightarrow[]{b)} D$ ist abgeschlossen $\Rightarrow s \in D \Rightarrow s = \max D$. Analog zeigt man: $\inf D \in D$.
	\end{enumerate}	
\end{proof}

\index{beschränkt}
\begin{definition}
	$f \colon D \rightarrow \R$ hei{\ss}t beschränkt $\iff f(D)$ ist \textbf{beschränkt} $$ (\iff \exists x \geq : |f(x) \leq c ~\forall x \in D) $$
\end{definition}


\begin{satz} \label{7.11:satz}
	Sei $D \subseteq \R$ kompakt und $f \in C(D)$. Dann ist $f(D)$ kompakt. Insbesondere ist $f$ beschränkt	und es ex. $x_{1}, x_{2} \in D$: 
		$$ f(x_{1}) \leq f(x) \leq f(x_{0}) ~\forall x \in D $$
\end{satz}

\begin{proof}
	Sei $(y_{n})$ eine Folge in $f(D)$. $\exists$ Folge $(x_{n})$ in $D$: $\forall n \in \N$ $f(x_{n}) = y_{n}$. $D$ kompakt $\Rightarrow (x_{n})$ enthält eine konvergente Teilfolge $(x_{n_{k}})$ mit $x_{0} \coloneqq \lim_{k \rightarrow \infty} x_{n_{k}} \in D$. $f$ stetig 
	$$ \Rightarrow y_{n_{k}} = f(x_{n_{k}}) \rightarrow f(x_{0}) \in f(D) $$
\end{proof}


\begin{satz} ~\ \label{7.12:satz}
	\begin{enumerate}
		\item Ist $I \subseteq \R$ ein Intervall ($I = \R$ ist zugelassen!) und $f \in C(I)$, so ist $f(I)$ ein Intervall.
		\item Sei $f \in C([a, b]), A \coloneqq \lim f([a, b])$ und $B \coloneqq \max f([a, b])$ (\ref{7.11:satz}!), so ist $f[a, b]) = [A, b]$.
	\end{enumerate}
\end{satz}

\begin{proof}
	a) ohne Beweis. b) folgt aus a) und \ref{7.7:prop-Zwischenwertsatz}.
\end{proof}

 \index{monoton! wachsend}   \index{monoton! streng wachsend} \index{monoton! streng fallend} \index{monoton! fallend}
\begin{definition} ~\
	\begin{enumerate}
		\item $f \colon D \rightarrow \R$ hei{\ss}t \textbf{monoton wachsend} $\iff$ aus $x_{1}, x_{2} \in D$ und $x_{1} < x_{2}$ folgt stets $f(x_{1}) \leq f(x_{2})$ \\
			$f \colon D \rightarrow \R$ hei{\ss}t \textbf{streng monoton wachsend} $\iff$ aus $x_{1}, x_{2} \in D$ und $x_{1} < x_{2}$ folgt stets $f(x_{1}) < f(x_{2})$
		\item Entsprechend definiert man \textbf{(streng) monoton fallend}.
		\item $f$ hei{\ss}t (streng) monoton $\iff f$ ist (streng) monoton wachsend oder (streng) monoton fallend.
	\end{enumerate}
\end{definition}


Sei $I \subseteq \R$ ein Intervall ($I = \R$ ist zugelassen) und $f \colon I \Rightarrow \R$ streng monoton wachsend bzw. fallend. Dann ist $f$ auf $I$ injektiv, es existiert also die \textbf{Umkehrfunktion} $f^{-1} \colon F(I) \Rightarrow I \subseteq \R$ und $f^{-1}$ ist streng monoton wachsend bzw. fallend.

Es gilt $\forall x \in I$: $f^{-1}(f(x)) = x$ und $\forall y \in F(I)$: $f(f^{-1}) =y$

Bem.: $f(I)$ ist i.A. kein Intervall % todo image


\begin{satz} \label{7.13:satz}
	Sei $I \subseteq \R$ ein Intervall, $f \in C(I)$ und $f$ sei auf $I$ streng monoton. Dann:
	$$ f^{-1} \in C \left( f(I) \right) $$	
\end{satz}

\index{Logarithmus}
\begin{namedtheorem}[Der Logarithmus] \label{7.14:prop-Logarithmus}
	Bekannt: $E$ ist auf $\R$ streng monoton wachsend und $E(\R) = (0, \infty)$. Es existiert also $E^{-1} \colon (0, \infty) \rightarrow \R$.
		$$ \log x \coloneqq \ln x \coloneqq E^{-1}(x) \quad (x \in (0, \infty)) $$
	\textbf{Logarithmus}.
\end{namedtheorem}


\begin{eigenschaften} ~\
	\begin{enumerate}
		\item $\log 1 = 0$, $\log e = 1$
		\item $\log \colon (0, \infty) \rightarrow \R$ ist stetig und streng monoton wachsend
		\item $\log \left( (0, \infty) \right) = \R$
		\item $\log x \rightarrow \infty ~(x \Rightarrow \infty)$, $\log x \rightarrow -\infty ~(x \rightarrow 0)$
		\item $\log x + \log y = \log(xy) ~\forall x, y > 0$
		\item $\log\left(\frac{x}{y}\right) = \log x - \log y ~\forall x, y > 0$
	\end{enumerate}	
\end{eigenschaften}

\begin{proof} ~\
	\begin{enumerate}
		\item klar
		\item folgt aus \ref{7.13:satz}
		\item $E(\R) = (0, \infty) \Rightarrow$ Beh.
		\item folgt aus $E(x) \rightarrow \infty ~(x \rightarrow \infty)$ bzw. $E(x) \rightarrow 0 ~(x \rightarrow -\infty)$.
		\item $z \coloneqq \log x + \log y$. Dann: $E(z) =E(\log x + \log y) = E(\log x) E(\log y) = x y \Rightarrow \log(xy) = \log E(z) = z \Rightarrow$ Beh.
		\item Analog.
	\end{enumerate}
\end{proof}


\textbf{Motivation}: $\xRightarrow[]{\ref{3.12:prop-Exponentialfunktion}} E(rx) = E(x)^{r} ~\forall x \in \R ~\forall r \in \Q$. \\
	Sei $a > 0$. Mit $x \coloneqq \log$:
		$$ \forall r \in \Q: \quad E(r \log a) = E(\log a)^{r} = a^{r} $$

\begin{namedtheorem}[Die allgemeine Potenz] \label{7.15:prop-AllgPotenz}
	Sei $a > 0$:
	$$ a^{x} \coloneqq E(x \log a) ~(x \in \R) $$
	Ist speziell $a = e$: $e^{x} = E(x \log e) = E(x) ~\forall x \in \R$. Also:
	$$ a^{x} = e^{x \log a} ~(x \in \R, a > 0). $$ 
\end{namedtheorem}


\begin{eigenschaften}
	Sei $a > 0$ und $x, y \in \R$.
	\begin{enumerate}
		\item $a^{x} > 0 ~\forall x \in \R$.
		\item Die Funktion $x \mapsto a^{x}$ ist auf $\R$ stetig.
		\item $a^{x+y} = e^{(x+y)\log a} = e^{x\log a + y \log a} = e^{x \log a} e^{y \log a} = a^{x} a^{y}$
		\item $a^{-x} = e^{-x\log a} = \frac{1}{e^{x} \log a} = \frac{1}{a^{x}}$
		\item $\log(a^{x}) = \log ( e^{x \log a}) = x \log a$
		\item $(a^{x})^{y} = e^{y \log a^{x}} \overset{e)}{=} e^{xy \log a} = a^{xy}$.
		\item Ist $x > 0$, so ist $a^{x^{y}} \coloneqq a^{\left(x^{y}\right)}$.
	\end{enumerate}	
\end{eigenschaften}


\textbf{Erinnerung an} \ref{7.1:satz}: Sei $f \in C(D)$, $x_{0} \in D$ und $\epsilon > 0$. Dann ex. $\delta = \delta(\epsilon, x_{0}) > 0$:
	$$ |f(x) - f(x_{0})| < \epsilon ~\forall x \in D \text{ mit } |x - x_{0}| < \delta $$
	$\delta$ hängt i.A. von $\epsilon$ und $x_{0}$ ab!

\index{stetig!gleichmä{\ss}ig}
\begin{definition}
	$f \colon D \rightarrow \R$ hei{\ss}t auf $D$ \textbf{gleichmä{\ss}ig stetig} $\iff \forall \epsilon > 0 ~\exists \delta = \delta(\epsilon) > 0$: $|f(x) - f(z)| < \epsilon ~\forall x, z \in D: |x - z| < \delta$.
\end{definition}

Klar: $f$ gleichmä{\ss}ig stetig $\Rightarrow f$ stetig. ($"'\Leftarrow "'$ ist i.A. falsch!). \\
Ohne Beweis.

\begin{satz} \label{7.16:satz}
	Ist $D \subseteq \R$ kompakt und $f \in C(D)$, so ist $f$ auf $D$ gleichmä{\ss}ig stetig.
\end{satz}


\begin{definition}
	$f \colon D \rightarrow \R$ hei{\ss}t auf $D$ \textbf{Lipschitz-stetig} $\iff \exists L \geq 0$:
	$$ | f(x) - f(y) | \leq L |x - y| ~\forall x, y \in D  $$
\end{definition}

Übung: $f$ Lips.-stetig $\Rightarrow f$ glm. stetig.


\begin{beispiel*}
	$D = [0, 1], f(x) = x^{2}$.
	\begin{align*}
		|f(x) - f(y) | & = |x^{2} - y^{y}| = |(x + y) (x - y)| = |x+y| |x-y| \\
			& \leq (|x| + |y|) |x - y| \leq 2 |x - y| \quad \forall x, y \in [0, 1]
	\end{align*}
\end{beispiel*}


\begin{bemerkung}
	$g \colon \R \rightarrow \R, g(x) = x^{2}$ ist nicht glm. stetig, insbesondere nicht Lips.-stetig.	
\end{bemerkung}

\newpage

\section{Funktionenfolge und -reihen}
I. d. $\S$en sei stets: $\emptyset \neq D \subseteq \R$, $(f_{n})$ eine Folge von Funktionen $f_{n} \colon D \rightarrow \R$ und $s_{n} \coloneqq f_{1} + f_{n} + \cdots + f_{n}$ $(n /in \N)$

\index{konvergent!punktweise} \index{Grenzfunktion} \index{Summenfunktion}
\begin{definition} ~\
 	\begin{enumerate}
		\item Die Funktionenfolge $(f_{n})$ hei{\ss}t \textbf{auf D punktweise konvergent} $\iff$ für jedes $x \in D$ ist die Folge $(f_{n}(x))$ konvergent. \\
			In diesem Fall setze $f(x) \coloneqq \lim f_{n}(x)$ $(x \in D)$. Die Funktion $f \colon D \rightarrow \R$ hei{\ss}t die \textbf{Grenzfunktion} von $(f_{n})$.
		\item Die Funktionenreihe $\sum_{n=1}^{\infty} f_{n}$ hei{\ss}t  \textbf{auf D punktweise konvergent} $\iff$ für jedes $x \in D$ ist die Folge $(s_{n}(x))$ konvergent. \\
			In diesem Fall setze $f(x) \coloneqq \sum_{n=1}^{n\rightarrow\infty} f_{n}(x)$ $(x \in D)$. Die Funktion $f \colon D \rightarrow \R$ hei{\ss}t die \textbf{Summenfunktion} von $(f_{n})$.
	\end{enumerate}
\end{definition}


\begin{beispiele} ~\
	\begin{enumerate}
		\item $D = [0,1]$, $f_{n}(x) \coloneqq x^{n}$ %todo image
			$$ f(x) \coloneqq \lim_{n\rightarrow\infty} f_{n}(x) = \begin{cases} 0, & 0 \leq x < 1 \\  1, & x = 1 \end{cases} $$
			$(f_{n})$ konvergiert auf $[0, 1]$ punktweise gegen $f$.
		\item Sei $\sum_{n=0}^{\infty} a_{n} (x-x_{0})^{n}$ eine Potenzreihe mit dem Konvergenzradius $r > 0$ und $D \coloneqq (x_{0} - r, x_{0} + r)$ $(D \Coloneqq \R$, falls $r = \infty$). Hier: $f_{n}(x) = a_{n} (x - x_{0})^{n} \xRightarrow[]{\ref{4.1:satz}} \sum_{n=0}^{\infty} f_{n}$ konvergiert auf $D$ punktweise gegen $f(x) \coloneqq \sum_{n=0}^{\infty} a_{n} (x-x_{0})^{n}$.
		\item $D = [0, \infty$, $f_{n}(x) \coloneqq \frac{nx}{1+n^{2}x^{2}} = \frac{\frac{x}{n}}{\frac{1}{n} + x^{2}} \rightarrow 0$ $(n \rightarrow \infty)$. Also konvergiert $(f_{n})$ auf $D$ punktweise gegen $f \equiv 0$. Es ist $f_{n}(\frac{1}{n}) = \frac{1}{2}$ \\
			% todo image
			Punktweise Konvergenz von $(f_{n})$ auf $D$ gegen $f$ bedeutet: ist $\epsilon > 0$ und $x \in D$, so existiert eine $n_{0} 0 n_{0}(\epsilon, x) \in \N$:
			$$ |f_{n}(x) - f(x)| < \epsilon \quad \forall n \geq n_{0} $$
	\end{enumerate}
\end{beispiele}


\begin{definition} ~\
	\begin{enumerate}
		\item $(f_{n})$ konvergiert \text{auf D gleichmä{\ss}ig}(glm) gegen $f \colon D \rightarrow \R \iff \forall \epsilon > 0 ~\exists n_{0} = n_{0}(\epsilon) \in \N$:
			$$ |f_{n}(x) - f(x)|< \epsilon \quad \forall n \geq n_{0} \text{ und } \forall x \in D. $$
		\item $\sum_{n=1}^{\infty} f_{n}$  konvergiert \text{auf D gleichmä{\ss}ig}(glm) gegen $f \colon D \rightarrow \R \iff \forall \epsilon > 0 ~\exists n_{0} = n_{0}(\epsilon) \in \N$:
			$$ |s_{n}(x) - s(x)|< \epsilon \quad \forall n \geq n_{0} \text{ und } \forall x \in D. $$
	\end{enumerate}
\end{definition}

Klar: gleichmä{\ss}ige Konvergenz $\Rightarrow$ punktweise Konvergenz $("'\Leftarrow"'$ ist i. A. falsch, siehe Beispiele unten). \\

Anschaulich: $(f_{n})$ konvergiert auf $D$ gleichmä{\ss}ig gegen $f$ bedeutet: zu $\epsilon > 0$ existiert ein $n_{0} = n_{0}(\epsilon) \in \N$: für $n \geq n_{0}$ liegt der Graph von $f_{n}$ im $"'\epsilon$-Schlauch$"'$ um $f$.
%todo image

\begin{beispiele} ~\
	\begin{enumerate}
		\item Sei $D = [0, 1], f_{n}(x) = x^{n}$. $(f_{n})$ konvergiert punktweise gegen 
			$$ f(x) = \begin{cases} 0, & \text{falls } x \in [0, 1], \\ 1, & \text{falls } x = 1 \end{cases} $$
			Sei $0 < \epsilon < \frac{1}{2}$. Es ist $f_{n}(x) = \frac{1}{2} \iff x = \frac{1}{\sqrt[n]{2}}$ und damit
				$$ |f_{n}(\frac{1}{\sqrt[n]{2}}) - f(\frac{1}{\sqrt[n]{2}}) = \frac{1}{2} > \epsilon \quad \forall n \in \N $$
			$(f_{n})$ konvergiert also nicht gleichmä{\ss}ig auf $[0, 1]$ gegen $f$.
		\item $\sum_{n=0}^{\infty} x^{n}$, $D = (-1, 1)$, 
			$$ s_{n}(x) = 1 + x + \dotsc + x^{n} = \frac{1 - x^{n+1}}{1 - x} \rightarrow \frac{1}{1 - x} \eqqcolon f(x) \quad \forall x \in D. $$
			Beh.: $\sum_{n=0}^{\infty} x^{n}$ konvergiert auf $D$ nicht gleichmä{\ss}ig gegen $f$. \\
			Bew.: Annahme: $\sum_{n=0}^{\infty} x^{n}$, also $(s_{n})$ konvergiert auf $D$ gleichmä{\ss}ig gegen $f$. Zu $\epsilon = 1$ existiert dann ein $n_{0} \in \N$:
			$$ | s_{n}(x) - f(x) | = \frac{|x|^{n+1}}{1  x} < 1 \quad n \geq n_{n}, \quad \forall x \in D $$
			Aber: $\frac{|x|^{n+1}}{1 - x} \rightarrow \infty ~(x \rightarrow 1-)$, Widerspruch!
		\item $D = [0, \infty)$, $f_{n}(x) = \frac{nx}{1 + n^{2} x^{2}}$. $f_{n}(x) \rightarrow 0 \coloneqq f(x) ~(n \rightarrow \infty)$. Sei $0 < \epsilon < \frac{1}{2}$: 
			$$ |f_{n}(\frac{1}{2}) - f(\frac{1}{2})| = \frac{1}{2} ~\forall n \in \N. $$
			$(f_{n})$ konvergiert also auf $D$ nicht gleichmäßig gegen $f$.
	\end{enumerate}	
\end{beispiele}

\index{Konvergenzkriterium!Funktionen!Weierstra{\ss}}
\begin{satz} ~\ \label{8.1:satz}
	\begin{enumerate}
		\item $(f_{n})$ konvergiere auf $D$ punktweise gegen $f \colon D \rightarrow \R$. Weiter sei $(\alpha_{n})$ eine Folge, $\alpha_{n} \rightarrow 0$, $m \in \N$ und
			$$ |f_{n}(x) - f(x) | \leq \alpha_{n} \quad \forall n \geq m ~\forall x \in D $$
			Dann konvergiert $(f_{n})$ auf $D$ gleichmä{\ss}ig gegen $f$. \label{8.1.a:satz}
		\item \textbf{Kriterium von Weierstra{\ss}}: Sei $m \in \N$, $(c_{n})$ eine Folge in $[0, \infty)$, $\sum_{n=1}^{\infty} c_{n}$ konvergent und
			$$ | f_{n}(x) | \leq c_{n} \quad \forall n \geq m ~\forall x \in D $$
			Dann konvergiert $\sum_{n=1}^{\infty} f_{n}$ auf $D$ gleichmä{\ss}ig. \label{8.1.b:satz}
	\end{enumerate}
\end{satz}

\begin{proof} ~\
	\begin{enumerate}
		\item Sei $\epsilon > 0 ~\exists n_{0} = n_{0}(\epsilon)\geq m$: $\alpha_{n} < \epsilon ~\forall n \geq n_{0}$. Dann:
			$$ | f_{n}(x) - f(x) | < \epsilon \quad \forall n \geq n_{0} ~\forall x \in D $$
		\item Sei $x \in D$. $|f_{n}(x)| \leq c_{n} ~\forall n \geq m \xRightarrow[]{\ref{3.5.a:satz}} \sum_{n=1}^{\infty} f_{n}(x)$ konvergiert (absolut).
	\end{enumerate}
\end{proof}


\begin{satz} \label{8.2:satz}
	$\sum_{n=0}^{\infty} a_{n} (x - x_{0})^{n}$ sei eine Potenzreihe mit Konvergenzradius $r > 0$, es sei $D \coloneqq (x_{0} - r, x_{0} + r) ~(D \coloneqq \R$, falls $r = \infty$). \\
	Ist $[a, b] \subseteq D$, so konvergiert die Potenzreihe auf $[a, b]$ gleichmä{\ss}ig
\end{satz}

\begin{proof}
	Sei o. B. d. A. $x_{0} = 0$ \\	% todo image
	Wähle $\delta > 0$ so, dass $-r < -\delta <a < b < \delta < r$. Sei $x \in [a, b]$. Dann $|x| \leq \delta$, also
		\[ \left| a_{n} x^{n} \right| = |a_{n}| |x|^{n} \leq |a_{n}| \delta^{n} \coloneqq c_{n} \tag*{$(*)$} \]
	$\xRightarrow[]{\ref{4.1:satz}} \sum_{n=0}^{\infty} a_{n} \delta^{n}$ konvergiert absolut; also ist $\sum_{n=0}^{\infty} c_{n}$ konvergent. Aus $(*)$ und $\ref{8.1.b:satz}$ folgt die Behauptung.
\end{proof}


\begin{satz} \label{8.3:satz}
	$(f_{n})$ bzw. $\sum_{n=1}^{\infty} f_{n}$ konvergiere auf $D$ gleichmä{\ss}ig gegen $f \colon D \rightarrow \R$
	\begin{enumerate}
		\item Sind alle $f_{n}$ in $x_{0} \in D$ stetig, so ist $f$ in $x_{0}$ stetig. \label{8.3.a:satz}
		\item Sind alle $f_{n} \in C(D)$, so ist $f \in C(D)$. \label{8.3.b:satz}
	\end{enumerate}
\end{satz}

\begin{folgerungen} ~\
	\begin{enumerate}
		\item Konvergiert $(f_{n})$ auf $D$ punktweise gegen $f \colon D \rightarrow \R$ und gilt $f_{n} \in C(D) ~\forall n$ aber $f \notin C(D)$, so ist Konvergenz nicht gleichmä{\ss}ig!
		\item Voraussetzung wie in $\ref{8.3.a:satz}$; $x_{0}$ sei Häufungspunkt von $D$. Dann:
			\begin{align*}
				\lim_{x \rightarrow x_{0}} \left( \lim_{n \rightarrow \infty} f_{n}(x) \right) & = \lim_{x \rightarrow x_{0}} f(x) \overset{\ref{8.3.a:satz}}{=} f(x_{0}) = \lim_{n \rightarrow \infty} f_{n}(x_{0}) \\
					& = \lim_{n \rightarrow \infty} \left( \lim_{x \rightarrow x_{0}} f_{n}(x) \right)  
			\end{align*}
	\end{enumerate}	
\end{folgerungen}

\begin{proof}(von \ref{8.3:satz}) ~ b) folgt aus a)
	\begin{enumerate}
		\item Sei $\epsilon > 0$. $\exists m \in \N: |f_{m}(x) - f(x)| < \frac{\epsilon}{3} ~\forall x \in D$. $f_{m}$ stetig in $x_{0} \xRightarrow[]{\ref{7.1:satz}} \exists \delta > 0 : |f_{m}(x) - f_{m}(x_{0})| < \frac{\epsilon}{3} ~\forall x \in D \cap U_{\delta}(x_{0})$. Für $x \in U_{\delta}(x_{0}) \cap D$:
			\begin{align*}
				|f(x) - f(x_{0})| & = |f(x) - f_{m}(x) + f_{m}(x) - f_{m}(x_{0}) + f_{m}(x_{0}) - f(x_{0})| \\ 
					& \leq |f(x) - f_{m}(x) | + | f_{m}(x) - f_{m}(x_{0})| + | f_{m}(x_{0}) - f(x_{0})| \\
					< \frac{\epsilon}{3} +  \frac{\epsilon}{3} + \frac{\epsilon}{3} = \epsilon
			\end{align*}
			$\xRightarrow[]{\ref{7.1:satz}}$ Beh.
	\end{enumerate}
\end{proof}


\begin{proof}(von \ref{7.4:satz}) ~ $\sum_{n=0}^{\infty}a_{n}(x - x_{0})^{n}$ sei eine Potenzreihe mit Konvergenzradius $r > 0$, $D \coloneqq$ ($x_{0} - r, x_{0} + r) ~(D = \R$, falls $r = \infty$) und $f(x) \coloneqq \sum_{n=0}^{\infty} a_{n}(x-x_{0})^{n} ~(x \in D)$. \\ % todo image
	Sei $z_{0} \in D$. Wähle $a, b \in \R$ so, dass $z_{0} \in (a, b) \subseteq [a, b] \subseteq D \xRightarrow[]{\ref{8.2:satz}}$ die Potenzreihe konvergiert auf $[a, b]$ gleichmä{\ss}ig $\xRightarrow[]{\ref{8.3:satz}} f \in C([a, b])$. $f$ ist also in $z_{0}$ stetig. $z_{0} \in D$ beliebig $\Rightarrow f \in C(D)$.
\end{proof}

\index{Identitätssatz für Potenzreihen}
\begin{namedtheorem}[Identitätssatz für Potenzreihen] \label{8.4:prop-IdentitätssatzFürPotenzreihe}
	$\sum_{n=0}^{\infty} a_{n} (x - x_{0})^{n}$ sei eine Potenzreihe mit Konvergenzradius $r > 0$, $D \coloneqq (x_{0} - r, x_{0} + r)$ ($D \coloneqq \R$, falls $r = \infty$) und $f(x) \coloneqq \sum_{n=0}^{\infty} a_{n} (x - x_{0})^{n} ~(x \in D)$. \\
	Weiter sei $(x_{k})$ eine Folge in $D \setminus \{ x_{0} \}$ mit $x_{k} \rightarrow x_{0}$ und $f(x_{k}) = 0 ~\forall k \in \N$. Dann:
	$$ a_{n} = 0 \quad \forall n \in \N_{0}. $$ 
\end{namedtheorem}


\newpage


\section{Differentialrechnung}

I.d. $\S$en sei $I \subseteq \R$ ein Intervall und $f \colon I \rightarrow \R$ eine Funktion. 
% todo image

\index{differenzierbar} \index{Ableitung}
\begin{definition}
	$f$ hei{\ss}t \textbf{in $x_{0} \in I$ differenzierbar} (db) $\iff$ es existiert
		$$\lim_{x \rightarrow x_{0}} \frac{f(x) - f(x_{0})}{x - x_{0}} $$
	und ist $\in \R$ ($\iff$ es existiert $\lim_{h \rightarrow 0} \frac{f(x_{0} + h) - f(x_{0})}{h}$ und ist $\in \R$). \\
	I.d. Fall hei{\ss}t obiger Grenzwert die \textbf{Ableitung von $f$ in $x_{0}$} und wir mit $f'(x_{0})$ bezeichnet. \\
	Ist $f$ in jedem $x \in I$ differenzierbar, so hei{\ss}t $f$ \textbf{auf $I$ differenzierbar} und die \textbf{Ableitung $f'$ von $f$} auf $I$ gegeben durch $x \mapsto f'(x)$.
\end{definition}


\begin{beispiele} ~\
	\begin{enumerate}
		\item Sei $x \in \R$ und $f(x) \coloneqq c ~(x \in \R)$. Dann ist $f$ auf $\R$ differenzierbar und $f' \equiv 0$.
		\item $I = \R$, $f(x) = |x|, x_{0} = 0$
			$$ \frac{f(x) - f(x_{0})}{x - x_{0}} = \frac{|x|}{x} = \begin{cases} ~1, & x > 0 \\ -1, & x < 0 \end{cases} $$ % todo image
			$f$ ist also in $x_{0} = 0$ nicht differenzierbar.
		\item $I = \R$, $f(x) = x^{n} ~(x \in \N)$. Sei $x_{0} \in \R, x \neq x_{0}$.
			\begin{align*}
				\frac{f(x) - f(x_{0})}{x - x_{0}} & = \frac{x^{n} - x_{0}^{n}}{x - x_{0}} \\
					& \overset{\S}{=} \frac{(x - x_{0}) (x^{n-1} + x^{n-2} x_{0} + \dotsc + x x_{0}^{n-2} + x_{0}^{n-1}}{x - x_{0}} \\
					& = x^{n-1} + x^{n-2} x_{0} + \dotsc + x x_{0}^{n-2} + x_{0}^{n-1} \rightarrow n x_{0}^{n-1} ~(x \rightarrow x_{0}) 
			\end{align*} 
			Also $f$ ist auf $\R$ differenzierbar und $f'(x) = n x^{n-1}$, kurz:
				$$ (x^{n})' = n x^{n-1} \text{ auf } \R. $$
		\item $I = \R$, $f(x) = e^{x}$. Sei $x_{0} \in \R$ und $x \neq x_{0}$. 
		 	$$ \frac{f(x_{0} + h) - f(x_{0})}{h} = \frac{e^{x_{0} + h} - e^{x_{0}}}{h} \xrightarrow[]{\ref{7.6:bsp}} e^{x_{0}} ~(h \rightarrow 0). $$ 
		 	Also: $f$ ist auf $\R$ differenzierbar und $f'(x) = e^{x}$, kurz:
		 		$$ (e^{x})' = e^{x} \text{ auf } \R $$
	\end{enumerate}	
\end{beispiele}


\begin{satz} \label{9.1:satz}
	Ist $f$ in $x_{0} \in I$ differenzierbar, so ist $f$ in $x_{0}$ stetig.
\end{satz}

\begin{proof}
	Sei $x \in I$, $x \neq x_{0}$
		$$ f(x) - f(x_{0}) = \frac{f(x) - f(x_{0})}{x - x_{0}} (x - x_{0}) \rightarrow f'(x) \cdot 0 = 0 ~(x \rightarrow x_{0}) $$
	Also: $\lim_{x \rightarrow x_{0}} f(x) = f(x_{0})$.
\end{proof}


\begin{namedtheorem}[Differentiationsregeln]
	$g \colon I \rightarrow \R$ sei eine weitere Funktion. $f, g$ seien differenzierbar in $x_{0} \in I$.
	\begin{enumerate}
		\item Für $\alpha, \beta \in \R$ ist $\alpha f + \beta g$ differenzierbar in $x_{0}$ und
			$$ (\alpha f + \beta g)'(x_{0}) = \alpha f'(x_{0}) + \beta g'(x_{0}) $$
		\item $f g$ ist differenzierbar in $x_{0}$ und
			$$ (f g)'(x_{0}) = f'(x_{0})g(x_{0})+ f(x_{0})g'(x_{0}) $$
		\item Ist $g(x_{0}) \neq 0$, so existiert ein $\delta > 0$ mit $g(x) \neq 0 ~(x \in J \coloneqq I \cap U_{\delta}(x_{0}))$. Die Funktion $\frac{f}{g} \colon J \rightarrow \R$ ist differenzierbar in $x_{0}$ und 
			$$ (\frac{f}{g})'(x_{0}) = \frac{f'(x_{0}) g(x_{0}) - f(x_{0})g'(x_{0})}{g(x_{0})^{2}}. $$
	\end{enumerate}		
\end{namedtheorem}

\begin{proof} ~\
	\begin{enumerate}
		\item leichte Übung.
		\item Übung (man orientiere sich an c)).
		\item $g$ stetig in $x_{0}$ (s. \ref{9.1:satz}). $g(x_{0}) \neq 0 \xRightarrow[]{\ref{6.3.b:satz}} \exists \delta > 0$: 
			$$ g(x) \neq 0 \quad \forall x \in I \cap U_{\delta}(x_{0}) \eqqcolon J. $$
			$h \coloneqq \frac{f}{g}$. Für $x \neq x_{0}$ mit $x \rightarrow x_{0}$:
			\begin{align*}
				\frac{h(x) - h(x_{0})}{x - x_{0}} & = \frac{f(x) - f(x_{0})}{x- x_{0}} \frac{1}{g(x)} - f(x_{0}) \frac{\frac{1}{g(x_{0})} - \frac{1}{g(x)}}{x - x_{0}} \\
					& = \underbrace{\frac{1}{g(x)g(x_{0})}}_{\rightarrow \frac{1}{g(x_{0})^{2}}} \bigg( \underbrace{\frac{f(x) - f(x_{0})}{x - x_{0}}}_{\rightarrow f'(x_{0})} g(x_{0}) - f(x_{0}) \underbrace{\frac{g(x) - g(x_{0})}{x - x_{0}}}_{\rightarrow g'(x_{0})} \bigg)
			\end{align*} 
	\end{enumerate}
\end{proof}


\begin{satz} \label{9.3:satz}
	Es sei $f \in C(I)$ streng monoton, in $x_{0} \in I$ differenzierbar und es sei $f'(x_{0}) \neq 0$. Dann ist
	$$ f^{-1} \colon f(I) \rightarrow \R \text{ differenzierbar in } y_{0} \coloneqq f(x_{0}) $$
	und
	$$ (f^{-1})'(y_{0}) = \frac{1}{f'(f^{-1}(y_{0}))} = \frac{1}{f'(x_{0})} $$
\end{satz}

\begin{proof} $\xRightarrow[]{\ref{7.12:satz}} f(I)$ ist ein Intervall; sei $(y_{n})$ eine Folge in $f(I)$ mit $y_{n} \rightarrow y_{0}$ und $y_{n} \neq y_{0} ~\forall n$. $x_{n} \coloneqq f^{-1}(y_{n}) \xRightarrow[]{\ref{7.13:satz}} x_{n} \rightarrow x_{0} = f^{-1}(y_{0})$, also
	$$ \frac{f^{-1}(y_{n}) - f^{-1}(y_{0})}{y_{n} - y_{0}} = \frac{x_{n} - x_{0}}{f(x_{n}) - f(x_{0})} \rightarrow \frac{1}{f'(x_{0})} \quad (x \rightarrow x_{0}) $$
\end{proof}

\index{Kettenregel}
\begin{namedtheorem}[Kettenregel] \label{9.4:prop-Kettenregel}
	$J \subseteq \R$ sei ein weiteres Intervall, $g \colon J \rightarrow \R$ eine Funktion und $f(I) \subseteq J$. $f$ sei in $x_{0} \in I$ differenzierbar und $g$ sei in $y_{0} \coloneqq f(x_{0})$ differenzierbar. Dann ist
		$$ g \circ f \colon I \rightarrow \R \text{ differenzierbar in } x_{0} $$
		und
		$$ (g \circ f)'(x_{0}) = g'(f(x_{0})) f'(x_{0}). $$
\end{namedtheorem}

\begin{proof}
	Für $y \in J$:
		$$ \tilde{g}(y) \coloneqq \begin{cases} \frac{g(y) - g(y_{0})}{y - y_{0}}, & y \neq y_{0} \\ g'(y_{0}), & y = y_{0} \end{cases} $$
	$g$ ist differenzierbar in $y_{0} \Rightarrow \tilde{g}$ ist stetig in $y_{0}$ d.h. $\tilde{g}(y) \rightarrow \tilde{g}(y_{0}) = g'(y_{0}) = g'(f(x_{0})) ~(y \rightarrow y_{0})$
		$$ \Rightarrow \tilde{g}(f(x)) \rightarrow g'(f(x_{0})) \quad (x \rightarrow x_{0}). $$
	Es ist $g(y) - g(y_{0}) = \tilde{g}(y) (y - y_{0}) ~\forall y \in J$, daraus folgt:
		$$ \frac{g(f(x)) - g(f(x_{0}))}{x - x_{0}} = \tilde{g}(f(x)) \frac{f(x) - f(x_{0})}{x - x_{0}} \rightarrow g'(f(x_{0})) f'(x_{0}) \quad (x \rightarrow x_{0}) $$
\end{proof}


\begin{beispiele} ~\
	\begin{enumerate}
		\item Sei $a > 0$ und $h(x) \coloneqq a^{x} = e^{x \log a} = g(f(x))$, wobei $g(x) = e^{x}$ und $f(x) = x \log a$. Dann: $h'(x) = g'(f(x)) f'(x) = e^{x \log a} \cdot \log a = a^{x} \log a$. Kurz:
			$$ (a^{x})' = a^{x} \log a \quad (x \in \R) $$
		\item $f(x) = e^{x}$, $f^{-1}(y) = \log y ~(y > 0) \xRightarrow[]{\ref{9.3:satz}} f^{-1}$ ist auf $(0, \infty)$ differenzierbar und
			$$ (f^{-1})'(y) = \frac{1}{f'(x)} = \frac{1}{e^{x}} = \frac{1}{y} $$
			Kurz: $(\log x)' = \frac{1}{x}$ auf $(0, \infty)$.
		\item Sei $\alpha \in \R$ und $f(x) = x^{\alpha} = e^{\alpha \log x} ~(x > 0)$.
			$$ f'(x) = e^{\alpha \log x} (\alpha \log x)' = x^{\alpha} \alpha \frac{1}{x} = \alpha x^{\alpha - 1} $$
			Kurz: $(x^{alpha})' = \alpha x^{\alpha - 1}$ auf $(0, \infty)$.
		\item aus Bsp. c): $(\sqrt{x})' = \frac{1}{2 \sqrt{x}}$ auf $(0, \infty)$.
	\end{enumerate}
\end{beispiele}


\begin{anwendung} \label{9.5:anwendung}
	Sei $a \in \R$ und o.B.d.A. $a \neq 0$. $f(t) \coloneqq \log(1 + t) ~(t > -1)$. Dann: $f'(t) = \frac{1}{1 + t}$
	$$ \lim_{t \rightarrow 0} \frac{\log(1+t)}{t} \lim_{t \rightarrow 0} \frac{f(t) - f(0)}{t - 0} = f'(0) = 1 $$
	$$ \Rightarrow 1 = \lim_{x \rightarrow \infty} \frac{\log(1 + \frac{a}{x}}{\frac{a}{x}} = \frac{1}{a} x \log(1 + \frac{a}{x}) = \frac{1}{a} \log (1 + \frac{a}{x})^{x} $$
	$\Rightarrow \log (1 + \frac{a}{x})^{x} \rightarrow a ~(x \rightarrow \infty) \Rightarrow \lim_{x \rightarrow \infty} (1 + \frac{a}{x})^{x} = e^{a}$
\end{anwendung}


\begin{definition}
	Sei $\emptyset \neq M \subseteq \R$ und $g \colon M \rightarrow \R$ eine Funktion
	\begin{enumerate}
		\item $x_{0} \in M$ hei{\ss}t ein \textbf{innerer Punkt von M} $\iff \exists > 0: U_{\delta}(x_{0}) \subseteq M$
		\item $g$ hat in $x_{0} \in M$ eine \textbf{relatives Maximum} $\iff \exists \delta > 0$: $g(x) \leq g(x_{0}) ~\forall x \in U_{\delta}(x_{0}) \cap M$.
		\item $g$ hat in $x_{0} \in M$ eine \textbf{relatives Minimum} $\iff \exists \delta > 0$: $g(x) \geq g(x_{0}) ~\forall x \in U_{\delta}(x_{0}) \cap M$.
	\end{enumerate}
	% todo image
	relatives Extremum = relative Max. oder Min.
\end{definition}


\begin{satz} \label{9.6:satz}
	$f \colon I \rightarrow \R$ habe in $x_{0}$ ein relatives Extremum und sei in $x_{0} \in I$ differenzierbar. Ist $x_{0}$ ein innerer Punkt von $I$, so ist $f'(x_{0}) = 0$.
\end{satz}

\begin{proof}
	Sei o.B.d.A. $x_{0}$ ein relatives Maximum von Funktion $f$. $\exists \delta > 0: U_{\delta}(x_{0}) \subseteq I$ und $f(x) \leq f(x_{0}) ~(x \in U_{\delta}(x_{0}))$
	$$ D(x) \coloneqq \frac{f(x) - f(x_{0})}{x - x_{0}} \begin{cases} \leq 0, x > x_{0} \\ \geq 0, x < x_{0} \end{cases} (x \in U_{\delta}(x_{0}) \setminus \{ x_{0} \} $$
	Also: $f'(x_{0}) = \lim_{x \rightarrow x_{0} +} D(x) \leq 0$ und $f'(x_{0}) = \lim_{x \rightarrow x_{0} -} D(x) \geq 0$.
\end{proof}

\index{Mittelwertsatz}
\begin{namedtheorem}[Mittelwertsatz] \label{9.7:prop-Mittelwertsatz}
	(MWS) der Differentialrechnung. \\
	Es sei $f \in C[a, b]$ und $f$ sei auf $(a, b)$ differenzierbar. Dann existiert ein $\xi \in (a, b)$:
	$$ \frac{f(b) - f(a)}{b - a} = f'(\xi). $$
	% todo image	
\end{namedtheorem}

\begin{proof}
	$g(x) \coloneqq f(x) - f(a) - \frac{f(b) - f(a)}{b - a} (x - a) ~(x \in [a, b])$. Dann: $g \in C[a, b]$, $g$ ist differenzierbar auf $(a, b)$, $g(a) = g(b) = 0$ und 
		$$ g'(x) = f'(x) - \frac{f(b) - f(a)}{b - a} ~(x \in (a, b)). $$
	Z.z.: $\exists \xi \in (a, b)$: $g'(\xi) = 0$. \\
	Fall 1: $g \equiv 0$ \checkmark \\
	Fall 2: $g \not\equiv 0 \xRightarrow[]{\ref{7.11:satz}} \exists x_{1}, x_{2} \in [a, b]: g(x_{1}) \leq g(x) \leq g(x_{2}) ~(x \in [a, b])$. \\
	Da $g \not\equiv 0: x_{1} \in (a, b)$ oder $x_{2} \in (a, b) \xRightarrow[]{\ref{9.6:satz}} g'(x_{1}) = 0$ oder $g'(x_{2}) = 0$.
\end{proof}


\begin{folg} \label{9.8:folg}
	$f \colon I \rightarrow \R$ sei differenzierbar auf $I$. $f$ ist auf $I$ konstant $\iff f' \equiv 0$ auf $I$.	
\end{folg}

\begin{proof}
	$"'\Rightarrow"'$ \checkmark, $"'\Leftarrow"'$ Seien $x_{1}, x_{2} \in I, x_{1} < x_{2} \xRightarrow[]{\hyperref[9.7:prop-Mittelwertsatz]{MWS}} \exists \xi \in (x_{1}, x_{2})$: 
		$$ f(x_{2}) - f(x_{1}) = f'(\xi) (x_{2} - x_{1}) = 0, $$ 
	also $f(x_{1}) = f(x_{2})$.
\end{proof}


\begin{anwendung} \label{9.9:anwendung}
	$f \colon I \rightarrow \R$ sei differenzierbar. Dann: 
		$$ f' = f \text{ auf } I \iff \exists c \in \R: f(x) = c e^{x} ~(x \in I) $$
\end{anwendung}

\begin{proof}
	$"'\Rightarrow"'$ \checkmark, $"'\Leftarrow"'$ $g(x) \coloneqq \frac{f(x)}{e^{x}}$. Dann:
	$$ g'(x) = \frac{f'(x) e^{x} - e^{x} f(x)}{e^{2x}} = 0 \quad \forall x \in I. $$
	$\xRightarrow[]{\ref{9.8:folg}} \exists c \in \R: g(x) = c ~\forall x \in I \Rightarrow$ Beh.
\end{proof}


\begin{satz} \label{9.10:satz}
	$f, g \colon I \rightarrow \R$ seien auf $I$ differenzierbar.
	\begin{enumerate}
		\item Ist $f' = g'$ auf $I$, so $\exists c \in \R: f = g + c$ auf $I$.
		\item Ist $' \geq 0$ auf $I$, so ist $f$ monoton wachsend auf $I$. \\
				Ist $f' > 0$ auf $I$, so ist $f$ streng monoton wachsend auf $I$.
		\item Ist $' \leq 0$ auf $I$, so ist $f$ monoton fallend auf $I$. \\
				Ist $f' < 0$ auf $I$, so ist $f$ streng monoton wachsend auf $I$.
	\end{enumerate}
\end{satz}

\begin{proof}  ~\
	\begin{enumerate}
		\item $(f - g')' = 0$ auf $I \xRightarrow[]{\ref{9.8:folg}}$ Beh.
		\item Sei $g' \geq 0$ auf $I$. Seien $x_{1}, x_{2} \in I$ und $x_{1} < x_{2} \xRightarrow[]{\hyperref[9.7:prop-Mittelwertsatz]{MWS}} \exists \xi \in (x_{1}, x_{2})$: $f(x_{2}) - f(x_{1}) = \underbrace{f'(\xi)}_{\geq 0} (x_{2} - x_{1}) \geq 0$, also $f(x_{1}) \leq f(x_{2})$.
		\item Analog zur b).
	\end{enumerate}
\end{proof}

Ohne Beweis:
\begin{namedtheorem}[Die Regeln von de l'Hospital] \label{9.11:prop:lHopital}
	Es sei $I = (a, b)$, wobei $a = -\infty$ oder $b = \infty$ zugelassen ist. $f, g \colon I \rightarrow \R$ seien auf $I$ differenzierbar und $g'(x) \neq 0 ~\forall x \in I$. Es existiere
	$$ L \coloneqq \lim_{x \rightarrow {a \above 0pt b}} \frac{f'(x)}{g'(x)} \quad  (L \in \R \cup \{ - \infty, \infty \}) $$
	Gilt (I) $\lim_{x \rightarrow {a \above 0pt b}} f(x) = \lim_{x \rightarrow {a \above 0pt b}} g(x) = 0$ \\
	oder (II) $\lim_{x \rightarrow {a \above 0pt b}} g(x) = \pm \infty$, \\
	so ist 
	$$ \lim_{x \rightarrow {a \above 0pt b}} \frac{f(x)}{g(x)} = L. $$
\end{namedtheorem}

\begin{beispiele} ~\
	\begin{enumerate}
		\item $a, b > 0$. $\lim_{x \rightarrow 0} \frac{a^{x} - b^{x}}{x} = \lim_{x \rightarrow 0} \frac{a^{x} \log a - b^{x} \log b}{1} = \log a - \log b$.
		\item $\lim_{x \rightarrow \infty} \frac{\log x}{x} = \lim_{x \rightarrow \infty} \frac{\frac{1}{x}}{1} = 0$.
		\item $\lim_{x \rightarrow 0} x \log x = \lim_{x \rightarrow 0} \frac{\log x}{\frac{1}{x}} = \lim_{x \rightarrow 0} \frac{\frac{1}{x}}{\frac{-1}{x^{2}}} = \lim_{x \rightarrow 0} (-x) = \epsilon$.
		\item $\lim_{x \rightarrow 0 + 0} x^{x} = \lim_{x \rightarrow 0} e^{x \log x} \overset{c)}{=} e^{0} = 1$.
	\end{enumerate}
\end{beispiele}


\begin{satz} \label{9.12:satz}
	Es sei $\sum_{n=0}^{\infty} a_{n} (x - x_{0})^{n}$ eine Potenzreihe mit Konvergenzradius $r > 0$, $I = (x_{0}- r, x_{0} + r)$ ($I = \R$, falls $r = \infty$) und $f(x) \coloneqq \sum_{n = 0}^{\infty} a_{n} (x - x_{0})^{n} ~(x \in I)$.
	\begin{enumerate}
		\item Die Potenzreihe $\sum_{n=1}^{\infty} n a_{n} (x - x_{0})^{n-1}$ hat den Konvergenzradius $r$.
		\item $f$ ist auf $I$ differenzierbar und
			$$ f'(x) = \sum_{n=1}^{\infty} n a_{n} (x - x_{0})^{n-1} \quad \forall x \in I $$
	\end{enumerate}
\end{satz}

\begin{proof} ~\
	\begin{enumerate}
		\item $\sum[n]{n |a_{n}|} = \underbrace{\sqrt[n]{n}}_{\rightarrow 1} \sqrt[n]{|a_{n}|}$; $r > 0 \xRightarrow[]{\ref{4.1:satz}} \sqrt[n]{|a_{n}|}$ ist beschränkt $\Rightarrow \sqrt[n]{n|a_{n}|}$ ist beschränkt $\Rightarrow \limsup \sqrt[n]{n|a_{n}|} = \limsup \sqrt[n]{|a_{n}|}$, da $H(\sqrt[n]{|a_{n}|}) = H(\sqrt[n]{|n a_{n}|}) \Rightarrow$ Beh.
		\item später, nach \ref{10.17:satz}.
	\end{enumerate}
\end{proof}

\begin{namedtheorem}[Sinus/Cosinus] \label{9.13:prop-SinusCosinus}
$\sin x =\sum_{n=0}^{\infty} (-1)^{n} \frac{x^{2n+1}}{(2n+1)!} \xRightarrow[]{\ref{9.12:satz}} \sin$ ist auf $\R$ differenzierbar und 
	$$ (\sin x)' = \sum_{n=0}^{\infty} (-1)^{n} \frac{(2n+1) x^{2n}}{(2n + 1)!} = \sum_{n=0}^{\infty} (-1)^{n} \frac{x^{2n}}{(2n)!} = \cos x $$
	Analog: $\cos$ ist auf $\R$ differenzierbar und $(\cos x)' = - \sin x$.
\end{namedtheorem}
	
\begin{namedtheorem}[Definition von $\pi$] ~\ \label{9.14:prop-DefPi}
	\begin{enumerate}
		\item Für $x \in (0, 2)$ ist
			$$ \sin x = \underbrace{(x - \frac{x^{3}}{3!})}_{> 0} + \underbrace{(\frac{x^{5}}{5!} - \frac{x^{7}}{7!}}_{> 0} + \underbrace{(\frac{x^{9}}{9!} - \frac{x^{11}}{11!}}_{> 0} + \dotsc > x - \frac{x^{3}}{3!} > 0 $$
			Speziell: $\sin 1 > 1 - \frac{1}{6} = \frac{5}{6}$.
		\item $\exists \xi_{0} \in (0, 2)$: $\cos \xi_{0} = 0$ und $\cos x > 0 ~\forall x \in [0, \xi_{0})$
		\begin{proof}
			$\cos 0 = 1 > 0$; 
			\begin{align*}
				\cos 2 & = \cos (1 + 1) \overset{\ref{4.3:prop-Sinus}}{=} \cos^{2} 1 - \sin^{2} 1 = \cos^{2} 1 + \sin^{2} - 2 \sin^{2} 1 = 1 - 2 \sin^{2} 1 \\
					& \leq 1 - 2 \frac{25}{36} < 0
			\end{align*} 
			$\xRightarrow[]{\ref{7.7:prop-Zwischenwertsatz}} \exists \xi_{0} \in (0, 2): \cos \xi_{0} = 0$. Auf $(0, 2)$:
			$$ (\cos x)' = - \sin x \overset{a)}< 0 \Rightarrow \cos x > 0 ~\forall x \in [0, \xi_{0}) $$
		\end{proof}
		\item Sei $\xi_{0}$ wie in b). $\pi \coloneqq 2 \xi_{0}$ (Pi). 
		$$ \xi_{0} \in (0, 2) \Rightarrow \pi \in (0, 4) \quad (\pi \approx 3,14\dotsc) $$
		$\frac{\pi}{2} = \xi_{0}$, also $\cos \frac{\pi}{2} = 0$.
		$$ \sin^{2} \frac{\pi}{2}= 1 - \cos^{2} \frac{\pi}{2} = 1 \Rightarrow | \sin \frac{\pi}{2} | = 1 \xRightarrow[]{a)} \sin \frac{\pi}{2} = 1 $$
		$\cos$ hat in $[0, \frac{\pi}{2}]$ genau eine Nullstelle.
	\end{enumerate}	 
\end{namedtheorem}


\begin{figure*}[!ht] \centering
	\begin{tikzpicture}
     	\draw[->] (-2,0) -- (4.5,0) node[right] {$x$};
      	\draw[->] (0,-2) -- (0,2) node[above] {$y$};
      	\draw[scale=0.1,domain=-10:29,smooth,variable=\x] plot ({\x},{12.5*sin(deg(\x/6))}) node[below] {$\sin x$};
      	\draw[scale=0.1,domain=-10:29,smooth,variable=\x] plot ({\x},{12.5*cos(deg(\x/6))}) node[above] {$\cos x$};
    \end{tikzpicture}
	\caption{Sinus und Cosinus.}	
\end{figure*}

\begin{namedtheorem}[Weitere Eigenschaften von Sinus und Cosinus] \label{9.15:prop-EigSinusCosinus}
	\begin{enumerate}
		\item Aus \ref{4.3:prop-Sinus}:
			\begin{align*}
				\sin(x + \frac{\pi}{2}) & = \sin x \cos \frac{\pi}{2} + \cos x \sin \frac{\pi}{2} = \cos x 
				\intertext{Analog:}
				\cos(x + \frac{\pi}{2}) & = - \sin x \\
				\sin(x + \pi) & = - \sin x , \cos(x + \pi) = - \cos x \\
				\sin(x + 2\pi) &  = \sin x, \cos(x + 2\pi) = \cos 2
			\end{align*}
		\item $\cos$ hat in $[0, \pi]$ genau eine Nullstelle.
		\item I. d. gr. Übungen:
			\begin{align*}
				\cos x = 0 & \iff x \in \{ (2k + 1) \frac{\pi}{2} : k \in \Z \} \\
				\sin x = 0 & \iff x \in \{ k \pi : k \in \Z \}
			\end{align*}
	\end{enumerate}
	% todo image
\end{namedtheorem}

\index{Tangens}
\begin{definition}
	$\tan \colon \R \setminus \{ (2k + 1) \frac{\pi}{2} \colon k \in \Z \} \rightarrow \R$; $\tan x \coloneqq \frac{\sin x}{\cos x}$ \textbf{Tangens}
	\[ (\tan x)' = \frac{\cos^{2} x + \sin^{2} x}{\cos^{2} x} = \frac{1}{\cos^{2} x} > 0 \]
	$\tan$ ist also auf $(-\frac{\pi}{2}, \frac{\pi}{2})$ streng wachsend.
	% todo image
\end{definition}

\index{Arkustangens}
\begin{uebung}
	$\tan((-\frac{\pi}{2}, \frac{\pi}{2})) = \R$. Es ex. also
	\[ \arctan \coloneqq \tan^{-1} \colon \R \rightarrow (-\frac{\pi}{2},\frac{\pi}{2}) \]
	\textbf{Arkustangens}. \\
	% todo image	
	I.d. Übungen wird gezeigt: $(\arctan x)' = \frac{1}{1 + x^{2}}$ auf $\R$.
\end{uebung}

Ohne Beweis: \index{Abelscher Grenzwertsatz}
\begin{namedtheorem}[Abelscher Grenzwertsatz] \label{9.16:prop-AbelscherGrenzwertsatz}
$\sum_{n=0}^{\infty} a_{n} (x - x_{0})^{n}$ sei eine Potenzreihe mit Konvergenzradius $r > 0$ und $r < \infty$. Die Potenzreihe konvergiere auch noch in ${x_{0} + r \above 0pt x_{0} - r}$. Es sei
	$$ f(x) \coloneqq \sum_{n=0}^{\infty} a_{n} (x - x_{0})^{n} \text{ für } x \in {(x_{0} - r, x_{0} + r] \above 0pt [x_{0} - r, x_{0} + r)} $$
	Dann ist f stetig in ${x_{0} + r \above 0pt x_{0} - r}$.
\end{namedtheorem}


% todo punctionation stop

\begin{anwendungen} ~\ \label{9.17:anwendungen}
	\begin{enumerate}
		\item $f(x) \coloneqq \log(1+x)$ für $x \in (-1, 1) \eqqcolon I$. Dann: \label{9.17.a:anwendungen}
			$$ f'(x) = \frac{1}{1+x} = \frac{1}{1 - (-x)} = \sum_{n=0}^{\infty} (-x)^{n} = \sum_{n=0}^{\infty} (-1)^{n} x^{n} \quad \forall x \in I. $$
			$g \coloneqq \sum_{n=1}^{\infty} (-1)^{n+1} \frac{x^{n}}{n}$ für $x \in (-1, 1] \xRightarrow[]{\ref{9.12:satz}} g$ ist differenzierbar auf $I$ und 
			$$ g'(x) = \sum_{n=1}^{\infty} (-1)^{n+1} x^{n-1} = \sum_{n=1}^{\infty} (-1)^{n-1} x^{n-1} = f'(x) \quad \forall x \in I $$
			Damit ex. $c \in \R$: $f(x) = g(x) + c ~\forall x \in I$. Mit $x = 0$ folgt: $c = 0$. Also
			$$ f(x) = g(x) \quad \forall x \in I $$
			$\xRightarrow[x \rightarrow 1]{\ref{9.16:prop-AbelscherGrenzwertsatz}} f(x) = g(x) ~\forall x \in (-1, 1]$. Fazit:
			$$ \log(1+x) = \sum_{n=1}^{\infty} (-1)^{n+1} \frac{x^{n}}{n} \quad \forall x \in (-1 , 1] $$
			Für $x 0 1$: $\log(2) = \sum_{n=1}^{\infty} \frac{(-1)^{n+1}}{n}$. 
		\item Vgl. Übungsblatt: \label{9.17.b:anwendungen}
			$$ \arctan x = \sum_{n=0}^{\infty} (-1)^{n} \frac{x^{2n +1}}{2n + 1} \quad \forall x \in [-1, 1] $$
			Speziell: $\arctan 1 = \sum_{n=0}^{\infty} \frac{(-1)^{n}}{2n + 1} = 1 - \frac{1}{3} + \frac{1}{5} - \frac{1}{7} +- \dotsc$
		$$ \cos \frac{\pi}{4} = \cos(-\frac{\pi}{4}) \overset{\ref{9.15:prop-EigSinusCosinus}}{=} \sin( \frac{\pi}{2} - \frac{\pi}{4}) = \sin \frac{\pi}{4} \Rightarrow \tan \frac{\pi}{4} = 1 \Rightarrow \arctan 1 = \frac{\pi}{4} $$
		$\Rightarrow \frac{\pi}{4} = 1 - \frac{1}{3} + \frac{1}{5} - \frac{1}{7} +- \dotsc$
	\end{enumerate}
\end{anwendungen}

\index{differenzierbar!stetig} \index{differenzierbar!n-mal}  \index{Ableitung!n-te}
\begin{definition} ~\
	\begin{enumerate}
		\item $f \colon I \rightarrow \R$ sei auf $I$ differenzierbar. Ist $f'$ in $x_{0} \in I$ differenzierbar, so hei{\ss}t f \textbf{in $x_{0}$ zweimal differenzierbar} und
		$$ f''(x_{0}) \coloneqq (f')'(x_{0}) \text{ die 2. Ableitung von $f$ in $x_{0}$.} $$
		\item Ist $f'$ auf $I$ differenzierbar, so hei{\ss}t $f$ \textbf{auf $I$ zweimal differenzierbar} und $'' \coloneqq (f')'$ die 2. Ableitung von $f$ auf $I$. \\
			Entsprechend definiert man, falls vorhanden: $f'''(x_{0}), f^{(4)}(x_{0}), \dotsc$ und $f''', f^{(4)}, \dotsc$ 
		\item $C^{0}(I) \coloneqq C(I); f^{0} \coloneqq f$; Sei $n \in \N$. $f$ hei{\ss}t \textbf{auf $I$ n-mal stetig differenzierbar} $\iff f$ ist auf $I$ n-mal differenzierbar $f, f', \dotsc, f^{(n)} \in C(I)$.
			$$ C^{\infty} \coloneqq \bigcap_{n \geq 0} C^{n}(I). $$
	\end{enumerate}
\end{definition}


\begin{beispiele} ~\
	\begin{enumerate}
		\item $(e^{x})'' = e^{x}$, $(\sin x)'' = (\cos x)' = - \sin x$
		\item $f(x) \coloneqq x |x|$ % todo image
			\begin{description}
				\item Für $x > 0$: $f(x) = x^{2}, f'(x) = 2x$
				\item Für $x < 0$: $f(x) = -x^{2}, f'(x) = -2x$
				\item Für $x = 0$: $\frac{f(x) - f(0)}{x - 0} = |x| \rightarrow 0 ~(x \rightarrow 0)$
			\end{description}
			$f$ ist also auf $\R$ differenzierbar und $f'(x) = 2 |x| ~(x \in \R)$. $f$ ist also in $x_{0} = 0$ nicht zweimal differenzierbar.
	\end{enumerate}	
\end{beispiele}


\begin{beispiel} \label{9.18:bsp}
	$f(x) \coloneqq \begin{cases} x^{\frac{3}{2}} \sin \frac{1}{x}, & x \in (0, 1] \\0, & x = 0 \end{cases}$ \\
	Auf $(0, 1]$: 
	$$ f'(x) = \frac{3}{2} \sqrt{x} \sin \frac{1}{x} x^{\frac{3}{2}} \cos \frac{1}{x} (-\frac{1}{x^{2}}) $$
	Für $x_{0} = 0$ betrachte:
	$$ \frac{f(x) - f(0)}{x - 0} = \underbrace{\sqrt{x}}_{\rightarrow 0} \underbrace{\sin \frac{1}{x}}_{\text{beschr.}} \rightarrow 0 $$
	$f$ ist also auf $[0, 1]$ differenzierbar ($f'(0) = 0$). $x_{n} \coloneqq \frac{1}{n \pi}$. Dann: $x_{n} \rightarrow 0$, 
		$$ f'(x_{n}) = \sqrt{n \pi} \cos(n \pi) = \sqrt{n \pi} (-1)^{n+1} \not\rightarrow 0 = f'(0). $$ 
	D.h.: $f$ ist auf $[0, 1]$ differenzierbar, aber $f \notin C^{1}[0, 1]$. $|f'(x_{n})| = \sqrt{n \pi} \Rightarrow f'$ ist auf $[0, 1]$ nicht beschränkt.
\end{beispiel}

Aus \ref{9.12:satz} folgt (induktiv):

\begin{satz} \label{9.19:satz}
	Sei $\sum_{n=0}^{\infty} a_{n} (x - x_{0})^{n}$ eine Potenzreihe mit Konvergenzradius $r > 0$, $I \coloneqq (x_{0} - r, x_{0} + r)$ ($I = \R$, falls $r = \infty$) und $f(x) \coloneqq \sum_{n=0}^{\infty} a_{n} (x - x_{0})^{n} ~(x \in I)$. Dann: $f \in C^{\infty}(I)$ und 
	$$ f^{(k)}(x) = \sum_{n=k}^{\infty} n(n-1) \cdots (n-k+1) \cdot a_{n} (x - x_{0})^{n-k} \quad \forall x \in I ~\forall k \in \N_{0} $$
	Insbes.: $(x = x_{0})$: $f^{(k)}(x_{0}) = k! \cdot a_{k}$, also
		$$ a_{k} = \frac{f^{(k)}(x_{0})}{k!} \quad \forall k \in \N_{0} $$
\end{satz}

Ohne Beweis:

\begin{satz} \label{9.20:satz-Taylor}
	Sei $n \in \N_{0}$ und sei $f$ auf $I$ $(n+1)$-mal differenzierbar (insb. $f \in C^{n}(I)$), $x, x_{0} \in I$ und $x \neq x_{0}$. Dann existiert ein $\xi$ zwischen $x$ und $x_{0}$, $x \neq \xi \neq x_{0}$:
		$$ f(x) = f(x_{0}) + \frac{f'(x_{0})}{1!} (x - x_{0}) + \dotsc + \frac{f^{(n)}(x_{0})}{n!} (x - x_{0})^{n} + \frac{f^{(n+1)}(\xi)}{(n+1)!} (x - x_{0})^{n+1} $$
\end{satz}

\begin{bemerkung}
	Im Fall $n = 0$ vgl. \hyperref[9.7:prop-Mittelwertsatz]{MWS}.	
\end{bemerkung}


\begin{satz} \label{9.21:satz}
	Sei $n \geq 2$, $f \in C^{n}(I)$, $x_{0} \in I$ sei ein innerer Pinkt von $I$ und
	$$ f'(x_{0} = f''(x_{0} = \dotsc = f^{(n-1)}(x_{0}) = 0 \text{ und } f^{(n)}(x_{0}) \neq 0 $$
	\begin{enumerate}
		\item Ist $n$ gerade und $f^{(n)}(x_{0}) < 0$, so hat $f$ in $x_{0}$ ein relatives Maximum.
		\item Ist $n$ gerade und $f^{(n)}(x_{0}) > 0$, so hat $f$ in $x_{0}$ ein relatives Minimum.
		\item Ist $n$ ungerade, so hat $f$ in $x_{0}$ kein relatives Extremum. 
	\end{enumerate}	
\end{satz}

\begin{proof}
	$f^{(n)}(x_{0}) \neq 0$, $f^{(n)}$ stetig $\Rightarrow \exists \delta > 0: U_{\delta}(x_{0}) \subseteq I$. \\
	Sei $x \in U_{\delta}(x_{0}) \setminus \{ x_{0} \} \xRightarrow[]{\ref{9.20:satz-Taylor}} \exists \xi$ zwischen $x$ und $x_{0}$:
	$$ f(x) = \underbrace{\sum_{k=0}^{n-1} \frac{f^{(k)}(x_{0})}{k!} (x - x_{0})^{k}}_{= f(x_{0})} + \underbrace{\frac{f^{(n)}(\xi)}{n!} (x - x_{0})^{n}}_{\eqqcolon R(x)} $$
	\begin{enumerate}
		\item Sei $f^{(n)}(x_{0}) < 0 \xRightarrow[]{(*)} f^{n}(\xi) < 0$, $n$ gerade $\Rightarrow (x - x_{0})^{n} > 0$. Also: $R(x) < 0$; somit: $f(x) < f(x_{0})$.
		\item Sei $f^{(n)}(x_{0}) > 0 \xRightarrow[]{(*)} f^{n}(\xi) > 0$, $n$ gerade $\Rightarrow (x - x_{0})^{n} > 0$. Also: $R(x) > 0$; somit: $f(x) > f(x_{0})$.
		\item Sei o.B.d.A. $f^{(n)}(x_{0} > 0 \xRightarrow[]{(*)} f^{(n)}(\xi) > 0$, $n$ ungerade
			$$ \Rightarrow (x - x_{0})^{n} \begin{cases} > 0, & x > x_{0} \\ < 0, & x < x_{0} \end{cases} $$
			$\Rightarrow R(x) \begin{cases} > 0, & x > x_{0} \\ < 0, & x < x_{0} \end{cases} \Rightarrow f(x) \begin{cases} > f(x_{0}), & x > x_{0} \\ < f(x_{0}), & x < x_{0} \end{cases}$
			% todo image
	\end{enumerate}
\end{proof}


\newpage


\section{Das Riemann-Integral}

\index{Zerlegung}
\begin{vereinbarung}
I. d. $\S$en sei stets $a < b$, $f \colon [a, b] \rightarrow \R$ eine Funktion und $f$ beschränkt auf $[a, b]$; $m \coloneqq \inf f([a, b]), M \coloneqq \sup f([a, b])$. \\
\end{vereinbarung}


$Z \coloneqq \{ x_{0}, x_{1}, \dotsc, x_{n} \}$ hei{\ss}t eine \textbf{Zerlegung} von $[a, b] \iff a = x_{0} < x_{1} < \dotsc < x_{n} = b$. $\mathcal{J} \coloneqq \{ Z: Z$ ist eine Zerlegung von $[a, b] \}$. 
% todo image
Sei $Z = \{ x_{0}, \dotsc, x_{n} \} \in \mathcal{J}$. Definiere $I_{j} \coloneqq [x_{j-1} , x_{j}], |I_{j}| \coloneqq x_{j} - x_{j-1}$ "'Länge"' von $I_{j}$ und $m_{j} \coloneqq \inf f(I_{j}), M_{j} \coloneqq \sup f(I_{j}) ~(j = 1, \dotsc, n)$
	\begin{align*}
		s_{f}(Z) & \coloneqq \sum_{j=1}^{n} m_{j} |I_{j}| \text{\textbf{Untersumme} von $f$ bzgl. $Z$.} \\
		S_{f}(Z) & \coloneqq \sum_{j=1}^{n} M_{j} |I_{j}| \text{\textbf{Obersumme} von $f$ bzgl. $Z$.}
	\end{align*}
Es ist $m \leq m_{j} \leq M_{j} \leq M$, also $m |I_{j}| \leq m_{j} |I_{j}| \leq M_{j} |I_{j}| \leq M |I_{j}|$, somit
	\[ m \underbrace{\sum_{j=1}^{n} |I_{j}|}_{=b-a} \leq s_{f}(Z) \leq S_{f}(Z) \leq M \sum_{j=1}^{n} |I_{j}| = M (b - a) \tag*{$(*)$} \]


\begin{definition}
	Seien $Z_{1}, Z_{2} \in \mathcal{J}$. $Z_{2}$ hei{\ss}t eine Verfeinerung von $Z_{1} \iff Z_{1} \subseteq Z_{2}$.
\end{definition}

Ohne Beweis:

\begin{satz} \label{10.1:satz}
Seien $Z_{1}, Z_{2} \in \mathcal{J}$.
	\begin{enumerate}
		\item $s_{f}(Z_{1}) \leq S_{f}(Z_{2})$ \label{10.1.a:satz}
		\item Ist $Z_{1} \leq Z_{2}$ so gilt: $s_{f}(Z_{1}) \leq s_{f}(Z_{2})$ und $S_{f}(Z_{1} \geq S_{f}(Z_{2})$. Aus $(*)$ folgt: es existieren
			\begin{align*}
				 s_{f} & \coloneqq \sup \{ s_{f}(Z) \colon Z \in \mathcal{J} \}
				 \intertext{und}
				 S_{f} & \coloneqq \inf \{ S_{f}(Z) \colon Z \in \mathcal{J} \}.
			\end{align*} \label{10.1.b:satz}
	\end{enumerate}
\end{satz}

Aus $(*)$ und \ref{10.1.a:satz} mit $Z \in \mathcal{J}$:
	\[ m (b - a) \leq s_{f} \leq S_{f}(Z) \leq S_{f} \leq M (b - a). \tag*{$(**)$} \]

\index{integrierbar} \index{integrierbar!Riemann} \index{Integral} \index{Integral!Riemann}
\begin{definition}
	$f$ hei{\ss}t (Riemann-)\textbf{integrierbar} (ib) über $[a, b] \iff s_{f} = S_{f}$. I. d. Fall hei{\ss}t
		\[ \int_{a}^{b} f dx \coloneqq \int_{a}^{b} f(x) dx \coloneqq S_{f} (= s_{f}) \]
	das (Riemann-)\textbf{Integral}	von $f$ über $[a, b]$ und wir schreiben: $f \in R[a, b]$.
\end{definition}


\begin{beispiele} ~\
	\begin{enumerate}
		\item Sei $c \in \R$ und $f(x) = c ~(x \in [a, b]) \xRightarrow[]{(**)} c (b - a) \leq s_{f} \leq S_{f} \leq c(b- a) \Rightarrow f \in R[a, b]$ und $\int_{a}^{b}c dx = c(b - a)$.
		\item Sei $Z = \{ x_{0}, \dotsc, x_{n} \}$ eine Zerlegung von $[0, 1]$. definiere
			$$ f(x) \coloneqq \begin{cases} 1, & x \in [0, 1] \cap \Q \\ 0, & x \in [0, 1] \setminus \Q \end{cases}. $$  
			$m_{j}, M_{j}$ seien wie immer, dann: $m_{j} = \inf f(I_{j}) = 0, M_{j} = \sup f(I_{j}) = 1$; $s_{f}(Z) = 0, S_{f}(Z) = 1$. Also: $s_{f} = 0 \neq 1 = S_{f}$; $f \notin R[0, 1]$.
	\end{enumerate}	
\end{beispiele}

\begin{satz} \label{10.2:satz}
	Es seien $f, g \in R[a, b]$.
	\begin{enumerate}
		\item Ist $f \leq g$ auf $[a, b]$, so ist $\int_{a}^{b} f dx \leq \int_{a}^{b} g dx$.
		\item Für $\alpha, \beta \in \R$ ist $\alpha f + \beta g \in R[a, b]$ und
			$$ \int_{a}^{b} (\alpha f + \beta g) dx = \alpha \int_{a}^{b} f dx + \beta \int_{a}^{b} g dx. $$
	\end{enumerate}
\end{satz}

\begin{proof}
	nur a) ( b) Übung): Sei $Z = \{ x_{0}, \dotsc x_{n} \} \in \mathcal{J}$, $I_{j}$ und $m_{j}$ wie immer. $\tilde{m_{j}} \coloneqq \int g(I_{j}) ~(j = 1, \dotsc, n)$. $f \leq g$ auf $I_{j}$
		$$ \Rightarrow m_{j} \leq \tilde{m_{j}} \Rightarrow s_{f}(Z) \leq s_{g}(Z) \leq s_{g} \overset{Vor.}{=} \int_{a}^{b} g dx $$ 
		$\Rightarrow s_{f} \leq \int_{a}^{b} g dx \xRightarrow[]{Vor.}$ Beh.
\end{proof}

\index{Riemannsches Integrabilitätskriterium}
\begin{namedtheorem}[Riemannsches Integrabilitätskriterium] \label{10.3:prop-RiemannschesIntegrabilitaetskriterium}
	$f \in R[a, b]$
	$$ \iff \forall \epsilon > 0 ~\exists Z = Z(\epsilon) \in \mathcal{J}: S_{f}(Z) - s_{f}(Z) < \epsilon. $$	
\end{namedtheorem}

\begin{satz} \label{10.4:satz}
	Ist $f$ auf $[a, b]$ monoton, so ist $f \in R[a, b]$.
\end{satz}

\begin{proof}
	Sei $\epsilon > 0$. Wähle $n \in \N$ so, dass 
	$$ \frac{b - a}{n} (f(b) - f(a)) < \epsilon. $$
	Für $j = 0, \dotsc, n$ sei $x_{j} \coloneqq a + j \frac{b - a}{n}$ und $Z \coloneqq \{ x_{0}, \dotsc, x_{n} \} \in \mathcal{J}$. Seien $I_{j}, m_{j}$ und $M_{j}$ wie immer, dann $|I_{j}| = \frac{b - a}{n}, m_{j} = f(x_{j-1}), M_{j} = f(x_{j})$. Also: \\ % todo image 
	\begin{align*}
		S_{f}(Z) - s_{f}(Z) & = \sum_{j=1}^{n} (M_{j} - m_{j})|I_{j}| \\
		& = \frac{b - a}{n} \sum_{j=1}^{n} ( f(x_{j}) - f(x_{j-1}) ) \\
		& = \frac{b - a}{n} ( f(b) - f(a) ) < \epsilon
	\end{align*} 
	$\xRightarrow[]{\ref{10.3:prop-RiemannschesIntegrabilitaetskriterium}}$ Beh.
\end{proof}


\begin{satz} \label{10.5:satz}
	$C[a, b] \subseteq R[a, b]$.
\end{satz}

\begin{proof}
	Sei $f \in C[a, b]$ und $\epsilon > 0 \xRightarrow[]{\ref{7.16:satz}} \exists \delta > 0$:
	\[ |f(t) - f(s)| < \frac{\epsilon}{b - a} \quad \forall t, s \in [a, b] \text{ mit } |t - s| < \delta. \tag*{$(*)$} \]
	Sei $Z = \{ x_{0}, \dotsc, x_{n} \} \in \mathcal{J}$, $I_{j}, M_{j}, m_{j}$ wie immer und $Z$ sei so gewählt, dass $|I_{j}| < \delta ~(j = 1, \dotsc, n)$. Betrachte $I_{j}$: % todo image
	$$ \xRightarrow[]{\ref{7.11:satz}} \exists \xi, \eta \in I_{j}: \quad f(\xi) = m_{j}, f(\eta) = M_{j} $$
	$|I_{j}| < \delta \Rightarrow |\xi - \eta| < \delta \xRightarrow[]{(*)}$
	$$ M_{j} - m_{j} = f(\eta) - f(\xi) = |f(\eta) - f(\xi)| < \frac{\epsilon}{b - a} $$
	Dann: $S_{f}(Z) - s_{f}(Z) = \sum_{j=1}^{n} (M_{j} - m_{j}) |I_{j}| < \frac{\epsilon}{b - a} \sum_{j=1}^{n} |I_{j}| = \epsilon \xRightarrow[]{\ref{10.3:prop-RiemannschesIntegrabilitaetskriterium}}$ Beh.
\end{proof}

\index{Stammfunktion}
\begin{definition}
	$I \subseteq \R$ sei ein Intervall und $G, g \colon I \rightarrow \R$ Funktionen. $G$ hei{\ss}t eine \textbf{Stammfunktion} von $g$ auf $I$ $\iff$ $G$ ist auf $I$ differenzierbar und $G' = g$ auf $I$.
\end{definition}


Beachte: Sind $G, H$ Stammfunktionen von $g$ auf $I$ $\Rightarrow G' = g = H'$ auf $I$ $\xRightarrow[]{\ref{9.10:satz}} \exists c \in \R: G = H + c$ auf $I$.

\index{Hauptsätze der Diff.- und Integralrechnung!1. Hauptsatz}
\begin{namedtheorem}[1. Hauptsatz der Differential- und Integralrechnung]\label{10.6:prop-1Hauptsatz}
	Ist $f \in R[a, b]$ und besitzt $f$ auf $[a, b]$ eine Stammfunktion $F$, so ist 
	$$ \int_{a}^{b} f(x) dx = F(b) - F(a) \eqqcolon F(x) \Big|_{a}^{b} \eqqcolon \left[ F(x) \right]_{a}^{b} $$
\end{namedtheorem}

\begin{proof}
	Sei $Z = \{ x_{0}, \dotsc, x_{n} \} \in \mathcal{J}$, $I_{j}, m_{j}, M_{j}$ wie immer.
	$$ F(x_{j}) - f(x_{j-1}) \overset{\hyperref[9.7:prop-Mittelwertsatz]{MWS}}{=} F'(\xi_{j}) (x_{j} - x_{j-1}) = f(\xi_{j}) \underbrace{(x_{j} - x_{j-1})}_{ = |I_{j}|} $$
	mit $\xi_{j} \in (x_{j-1}, x_{j})$. $m_{j} \leq f(\xi_{j}) \leq M_{j} \Rightarrow m_{j} |I_{j}| \leq f(\xi_{j}) |I_{j}| \leq M_{j} |I_{j}| \Rightarrow$
	\begin{align*}
		s_{f}(Z) & \leq \sum_{j=1}^{n} f(\xi_{j}) |I_{j}| \\
			& = \sum_{j=1}^{n} \left( F(x_{j}) - F(x_{j-1}) \right) \\
			& = F(b) - F(a)  \\
			& \leq S_{f}(Z)
	\end{align*}
	Also: $s_{f}(Z) \leq F(b) - F(a) \leq S_{f}(Z) ~\forall Z \in \mathcal{J}$. \\
	$f \in R[a, b] \Rightarrow \int_{a}^{b} f dx = s_{f} \leq F(b) - F(a) \leq S_{f} = \int_{a}^{b} f dx$.
\end{proof}


\begin{beispiele} ~\
	\begin{enumerate}
		\item $0 < a < b$, $f(x) \coloneqq \frac{1}{x}$; $f \in C[a, b] \xRightarrow[]{\ref{10.5:satz}} f \in R[a, b]$. $F(x) \coloneqq \log x$; $F$ ist eine Stammfunktion von $f$ auf $[a, b]$.
			$$ \xRightarrow[]{\ref{10.6:prop-1Hauptsatz}} \int_{a}^{b} \frac{1}{x} dx = \log x \Big|_{a}^{b} = \log b - \log a. $$
		\item Sei $a < b$. $\int_{a}^{b} \cos x dx = \sin b - \sin a$.
	\end{enumerate}
\end{beispiele}

Warnungen:
\begin{enumerate}
	\item Es gibt integrierbare Funktionen, die keine Stammfunktion besitzen!
	\item Es gib nicht integrierbare Funktionen, die Stammfunktionen besitzen!
\end{enumerate}

\begin{beispiele} ~\
	\begin{enumerate}
		\item $f(x) = \begin{cases} 1, & x \in (0, 1] \\ 0, & x = 0 \end{cases}$. % todo image
			$f$ ist monoton $\xRightarrow[]{\ref{10.4:satz}} f \in R[0, 1]$. \\
			Annahme: $f$ besitzt auf $[0, 1]$ eine Stammfunktion $F$. Dann:
			$$ F'(x) = f(x) ~\forall x \in [0, 1] \Rightarrow F'(x) = 1 = (x)' \text{ auf } (0, 1] $$
			$\xRightarrow[]{\ref{9.10:satz}} \exists c \in \R: F(x) = x + c$ auf $(0, 1]$. $F$ ist differenzierbar in $0 \Rightarrow F$ stetig in $0 \xRightarrow[]{x \rightarrow 0} F(0) = c$. Also: $F(x) = x ü c ~\forall x \in [0, 1]$. Aber:
			$$ 0 = f(0) = F'(0) = \lim_{x \rightarrow 0} \frac{F(x) - F(0)}{x - 0} = \lim_{x \rightarrow 0} \frac{x + c - c}{x} = 1, $$
			Widerspruch!
		\item $F(x) \coloneqq \begin{cases} x^{\frac{3}{2}} \sin \frac{1}{x}, & x \in (0, 1] \\ 0, & x = 0 \end{cases} \xRightarrow[]{\ref{9.18:bsp}} F$ ist auf $[0, 1]$ differenzierbar; $f \coloneqq F'$. Dann ist $F$ eine Stammfunktion von $f$ auf $[0, 1] \xRightarrow[]{\ref{9.18:bsp}} f$ ist auf $[0, 1]$ nicht beschränkt, also $f \notin R[a, b]$. 
	\end{enumerate}	
\end{beispiele}

Ohne Beweis:

\begin{satz} \label{10.7:satz}
	Sei $c \in (a, b)$.
	$$  f \in R[a, b] \iff f \in R[a, c] \text{ und } f \in R[c, b] $$	
	I. d. Fall: $\int_{a}^{b} f dx = \int_{a}^{c} f dx + \int_{c}^{b} f dx$. % todo image
\end{satz}


\textbf{Motivation}
Für	$n \geq 2$ sei $f_{n} \colon [0, 1] \rightarrow \R$, $f_{n}(x) = \begin{cases} n^{2} x, & x \in [0, \frac{1}{n}), \\ n - (x - \frac{1}{n}) n^{2}, & x \in [\frac{1}{n}, \frac{2}{n}) \\ 0, & x \in [\frac{2}{n}, 1] \end{cases} $


\begin{figure*}[!ht] \centering
	\begin{tikzpicture}
		\begin{axis}
			\addplot[domain=0:0.2] {x*5*5};
			\addplot[domain=0.2:0.4] {10-x*5*5};
			\addplot[domain=0.4:1] {0*x};
		\end{axis}
	\end{tikzpicture}	
	\caption{$f_{n}$ für $n = 5$.}	
\end{figure*}

$f_{n} \in C[0, 1] \xRightarrow[]{\ref{10.5:satz}} f_{n} \in R[0, 1] \xRightarrow[\ref{10.7:satz}]{\ref{10.6:prop-1Hauptsatz}} \int_{0}^{1} f_{n} dx = 1 ~\forall n \geq 2$. \\
Übung: $(f_{n})$ konvergiert auf $[0, 1]$ punktweise gegen $f \equiv 0$. Aber: 
	$$ \lim_{n \rightarrow 0} \int_{0}^{1} f_{n} dx = 1 \neq 0 \int_{0}^{1} f dx = \int_{0}^{1} \left( \lim_{n \rightarrow \infty} f_{n}(x) \right) dx. $$

Ohne Beweis:

\begin{satz} \label{10.8:satz}
	Sei $(f_{n})$ eine Folge in $R[a, b]$ und $f_{n}$ konvergiert auf $[a, b]$ gleichmä{\ss}ig gegen $f \colon [a, b] \rightarrow \R$. Dann: $f \in R[a, b]$ und
	$$ \lim_{n \rightarrow \infty} \int_{a}^{b} f_{n}(x) dx = \int_{a}^{b} \left( \lim_{n \rightarrow \infty} f_{n}(x) \right) dx. $$
\end{satz}


\begin{satz} \label{10.9:satz}
	$\sum_{n=0}^{\infty} a_{n} (x - x_{0})^{n}$ sei eine Potenzreihe mit Konvergenzradius $r > 0$, $I \coloneqq (x_{0} - r, x_{0} + r)$ ($I \coloneqq \R$, falls $r = \infty$) und
	$$ g(x) \coloneqq \sum_{n=0}^{\infty} a_{n} (x - x_{0})^{n} \quad (x \in I) $$
	Dann hat die Potenzreihe $\sum_{n=0}^{\infty} \frac{a_{n}}{n + 1} (x - x_{0})^{n+1}$ den Konvergenzradius $r$ und für
	\[ G(x) \coloneqq \sum_{n=0}^{\infty} \frac{a_{n}}{n + 1} (x - x_{0})^{n+1} \quad (x \in I) \tag*{$(*)$} \]
	gilt $G' = g$ auf $I$.
\end{satz}

\begin{proof}
	Sei $\tilde{r}$ der Konvergenzradius der Potenzreihe in $(*) \xRightarrow[]{\ref{9.12:satz}} r = \tilde{r}$ und $G' = g$ auf $I$.	
\end{proof}


\begin{satz} \label{10.10:satz}
	Es seien $f, g \in R[a, b]$.
	\begin{enumerate}
		\item Sei $D \coloneqq f([a, b])$ und mit einem $L \geq 0$ gelte für $h \colon D \rightarrow \R$:
			$$ |h(s) - h(t)| \leq L |s - t| \quad \forall t, s \in D $$
			Dann: $h \circ f \in R[a, b]$.
		\item $|f| \in R[a, b]$ und $|\int_{a}^{b} f(x) dx| \leq \int_{a}^{b} |f(x)| dx$ ($\triangle$-Ungleichung für Integrale).
		\item $fg \in R[a, b]$.
		\item Ist $g(x) \neq 0 ~\forall x \in [a, b]$ und $\frac{1}{g}$ auf $[a, b]$ beschränkt, so ist $\frac{1}{g} \in R[a, b]$.
	\end{enumerate}
\end{satz}

\begin{proof} ~\
	\begin{enumerate}
		\item c) und d) ohne Beweis.
		\item $D \coloneqq f([a, b])$; $h(t) \coloneqq |t| ~(t \in D)$. Dann: $|f| = h \circ f$. Für $t, s \in D$: 
			$$ |h(t) - h(s)| = \left| |t| - |s| \right| \overset{\S 1}{=} |t - s|. $$
			Aus a): $|f| \in R[a, b]$. Es ist $\pm f \leq |f|$ auf $[a, b]$
			$$ \xRightarrow[]{\ref{10.2:satz}} \pm \int_{a}^{b} f dx \leq \int_{a}^{b} |f| dx$$
			$\Rightarrow \triangle$-Ungleichung.
	\end{enumerate}
\end{proof}


\begin{definition}
	Sei $f \in R[a, b]$ und $\alpha, \beta \in [a, b]$. $\int_{\alpha}^{\alpha} f(x) dx \eqqcolon 0$. Sei $\alpha < \beta \xRightarrow[]{\ref{10.7:satz}} f \in R[\alpha, \beta]$.
	$$ \int_{\beta}^{\alpha} f(x) dx \coloneqq - \int_{\alpha}^{\beta} f(x) dx. $$
\end{definition}

\index{Hauptsätze der Diff.- und Integralrechnung!2. Hauptsatz}
\begin{namedtheorem}[2. Hauptsatz der Differential- und Integralrechnung]\label{10.11:2Hauptsatz}
	Sei $f \in R[a, b]$ und
	$$ F(x) \coloneqq \int_{a}^{x} f(t) dt \quad \left( x \in [a, b] \right). $$
	Dann gilt:
	\begin{enumerate}
		\item $F(y) - F(x) = \int_{x}^{y} f(t) dt ~\forall x, y \in [a, b]$.
		\item $F \in C[a, b]$.
		\item Ist $f \in C[a, b]$, so ist $F \in C^{1}[a, b]$ und $F' = f$ auf $[a, b]$.
	\end{enumerate}
\end{namedtheorem}

\begin{proof} ~\
	\begin{enumerate}
		\item Seien $x, y \in [a, b]$. Fall 1: $x \leq y$ % todo image
			\begin{align*}
				F(y) - F(x) & ~ = \int_{a}^{y} f(t) dt - \int_{a}^{x} f(t) dt \\
				& \overset{\ref{10.7:satz}}{=} \int_{a}^{x} f(t) dt + \int_{x}^{y} f(t) - \int_{a}^{x} f(t) dt \\
				& ~ = \int_{x}^{y} f(t) dt
			\end{align*}
			Fall 2: $x > y$: 
			$$ F(y) - F(x) = - (F(x) - F(y)) \overset{Fall 1}{=} - \int_{y}^{x} f(t) dt = \int_{x}^{y} f(t) dt. $$
		\item $L \coloneqq \sup \{ |f(t)| : t \in [a, b] \}$. Seien $x, y \in [a, b]$. O.B.d.A.: $x \leq y$, dann:
			\begin{align*}
				|F(y) - F(x)| & \overset{a)}{=} | \int_{x}^{y} f(t) dt | \overset{\ref{10.10:satz}}{\leq} \int_{x}^{<} |f(t)| dt \overset{\ref{10.2:satz}}{\leq} \int_{x}^{y} L dt \\
				& = L (y - x) = L |y - x|.
			\end{align*}
		\item Wir zeigen für $x_{0} \in [a, b)$:
			$$ \lim_{h \rightarrow 0+0} \frac{F(x_{0} + h) - F(x_{0})}{h} = f(x_{0}), $$
			(analog zeigt man für $x_{0} \in (a, b]$: $\lim_{h \rightarrow 0 - 0} \frac{F(x_{0} + h) - F(x_{0})}{h} = f(x_{0})$). \\
			Sei also $x_{0} \in [a, b), h > 0$ und $x_{0} + h \in [a, b]$. Es ist
			$$ \frac{1}{h} \int_{x_{0}}^{x_{0}+h} f(x_{0}) dt = f(x_{0}) $$
			und
			$$ \frac{F(x_{0} + h) - F(x_{0})}{h} \overset{a)}{=} \frac{1}{h} \int_{x_{0}}^{x_{0}+h} f(t) dt. $$
			Dann: 
			\begin{align*} 
				L(h) & \coloneqq \left| \frac{F(x_{0} + h) - F(x_{0})}{h} - f(x_{0}) \right| \\
					 & = \frac{1}{h} \left| \int_{x_{0}}^{x_{0}+h} ( f(t) - f(x_{0}) ) dt \right| \\
					 & \overset{\ref{10.10:satz}}{\leq} \frac{1}{h} \int_{x_{0}}^{x_{0}+h} | f(t) - f(x_{0})| dt. 
			\end{align*} 
			$\xRightarrow[]{\ref{7.11:satz}} \exists \xi_{h} \in [x_{0}, x_{0} + h]$: $|f(t) - f(x_{0})| \leq | f(\xi_{h}) - f(x_{0})| ~\forall t \in [x_{0}, x_{0} + h]$. Dann: 
			$$ L(h) \leq \frac{1}{h} \int_{x_{0}}^{x_{0}+ h} |f(\xi_{h}) - f(x_{0})| dt = |f(\xi_{h}) - f(x_{0})|. $$ 
			Für $h \rightarrow 0$: $\xi_{h} \rightarrow x_{0}$; $f$ stetig $\Rightarrow f(\xi_{h}) \rightarrow f(x_{0})$. Also: $L(h) \rightarrow 0 ~(h \rightarrow 0)$. 
	\end{enumerate}
\end{proof}


Aus \ref{10.10:satz} folgt (Übung):

\begin{folg} \label{10.12:folg}
	Sei $I \subseteq \R$ ein Intervall, $g \in C(I)$ und $\xi \in I$ (fest). Definiere $G \colon I \rightarrow \R$ durch
	$$ G(x) \coloneqq \int_{\xi}^{x} f(t) dt. $$
	Dann: $G \in C^{1}(I)$ und $G' = g$ auf $I$.
\end{folg}

\index{Integral!unbestimmtes}
\begin{definition}
	Sei $I \subseteq \R$ ein Intervall. Besitzt $g \colon I \rightarrow \R$ auf $I$ eine Stammfunktion, so schreibt man für eine solche auch $\int g(x) dx$ (\textbf{unbestimmtes Integral}).
\end{definition}

\begin{beispiel} 
	$\int \cos x dx = \sin x ( + c) ~(c \in \R)$.
\end{beispiel}

\index{Partielle Integration}
\begin{namedtheorem}[Partielle Integration]
	Sei $I \subseteq \R$ ein Intervall und $f, g \in C^{1}(I)$. Dann:
	\begin{enumerate}
		\item $\int f' g dx = f g - \int f g' dx$ auf $I$.
		\item Ist $I = [a, b]$, $\int_{a}^{b} f' g dx = f g \Big|_{a}^{b} - \int_{a}^{b} f g' dx$.
	\end{enumerate}
\end{namedtheorem}

\begin{proof}
	$(f g)' = f' g + f g' \Rightarrow f' g = (fg)' - fg'$ \\
	$\Rightarrow$ a), sowie \\ 
	$$ \int_{a}^{b} f' g dx = \int_{a}^{b} (fg)' dx - \int_{a}^{b} fg' dx \overset{\ref{10.6:prop-1Hauptsatz}}{\underset{1. HS}{=}} fg \Big|_{a}^{b} - \int_{a}^{b} fg' dx. $$ 
\end{proof}


\begin{beispiele} ~\
	\begin{enumerate}
		\item $\int \sin^{2} x dx = \int \underbrace{\sin x}_{ f' } \underbrace{\sin x}_{ g } dx = - \cos x \sin x - \int - \cos^{2} x dx$
			\begin{align*}
				& = - \cos x \sin x + \int \cos^{2} x dx \\
				& = - \cos x \sin x + \int (1 - \sin^{2} x) dx \\
				& = x - \cos x \sin x - \int \sin^{2} x dx 
			\end{align*}
			$\Rightarrow \int \sin^{2} x dx = \frac{1}{2} (x - \cos x \sin x)$.
		\item $\int \underbrace{x}_{f'} \underbrace{e^{x}}_{g} dx = \frac{1}{2} x^{x} e^{x} - \int \frac{1}{2} x^{2} e^{x} dx \rightarrow$ komplizierter \\
			$\int \underbrace{x}_{g} \underbrace{e^{x}}_{f'} = x e^{x} - \int e^{x} dx = x e^{x}- e^{x}$.
		\item $\int \log x dx = \int \underbrace{1}_{f'} \underbrace{\log x}_{g} dx = x \log x - \int x \frac{1}{x} dx = x \log x - x$.
	\end{enumerate}	
\end{beispiele}


\textbf{Bez.}: Seien $\alpha, \beta \in \R$ und $\alpha \neq \beta$.
	$$ \langle \alpha, \beta \rangle \coloneqq \begin{cases} [\alpha, \beta], & \text{ falls } \alpha < \beta \\ [\beta, \alpha], & \text{ falls } \alpha > \beta \end{cases} $$

\index{Substitutionsregeln}
\begin{namedtheorem}[Substitutionsregeln] \label{10.14:prop-Substitutionsregeln}
	$I$ und $J$ seien Intervalle in $\R$, es sei $f \in C(I)$, $g \in C^{1}(J)$ und $g(J) \subseteq I$.
	\begin{enumerate}
		\item $\int f(g(t)) g'(t) dt = \int f(x) dx \Big|_{x = g(t)}$ auf $J$.
		\item Sei $g'(t) \neq 0 ~\forall t \in J$ ($\Rightarrow g' > 0$ auf $J$ oder $g' < 0$ auf $J \Rightarrow g$ ist streng monoton). Dann:
			$$ \int f(x) dx = \int f(g(t)) g'(t) dt \Big|_{t=g^{-1}(x)} \text{ auf } I $$
		\item Ist $I = \langle a, b \rangle$, $J = \langle \alpha, \beta \rangle$, $g(\alpha) = a$ und $g(\beta) = b$, so gilt
			$$ \int_{a}^{b} f(x) dx = \int_{\alpha}^{\beta} f(g(t)) g'(t) dt. $$
	\end{enumerate}
\end{namedtheorem}

\begin{proof}
	$\xRightarrow[]{\ref{10.2:satz}} f$ hat auf $I$ eine Stammfunktion $F$. $G(t) \coloneqq F(g(t)) ~ (t \in J)$. Kettenregel $\Rightarrow G \in C^{1}(J)$ und
	$$ G'(t) = F'(g(t)) g'(t) = f(g(t)) g'(t) \quad \forall t \in J. $$
	\begin{enumerate}
		\item $\int f(g(t)) g'(t) dt = \int G'(t) dt = G(t) = F(x) \Big|_{x=g(t)}$.
		\item $\int f(g(t)) g'(t) dt\Big|_{t = g^{-1}(x)} = G(g^{-1}(x)) = F(g(g^{-1}(x))) = F(x)$.
		\item $\int_{\alpha}^{\beta} f(g(t)) g'(t) dt \overset{\ref{10.6:prop-1Hauptsatz}}{=} G(\beta) - G(\alpha) = F(g(\beta)) - F(g(\alpha))$ 
		$$ = F(b) - F(a) \overset{\ref{10.6:prop-1Hauptsatz}}{=} \int_{a}^{b} f(x) dx. $$ 
	\end{enumerate}
\end{proof}


\textbf{Merkregel}:  Ist $y = y(x)$ eine differenzierbare Funktion, so schreibt man für $y'$ auch $\frac{dy}{dx}$.

Zu \ref{10.14:prop-Substitutionsregeln}: substituiere $x = g(t)$, fasse also $x$ als Funktion von $t$ auf. Dann: $\frac{dx}{dt} = g'(t)$, also
	$$ "' dx = g'(t) dt "' $$


\begin{beispiele} ~\
	\begin{enumerate}
		\item $\int_{0}^{1} \frac{e^{2x} + 1}{e^{x}} dx = \begin{cases} x = \log t, e^{x} = t \\ \frac{dx}{dt} = \frac{1}{t}, dx = \frac{1}{t} dt \\ x = 0 \Rightarrow t = 1, x = 1 \Rightarrow t = e \end{cases} $
			\begin{align*}
				& = \int_{1}^{e} \frac{t^{2} + 1}{t} \cdot \frac{1}{t} dt = \int_{1}^{e} \frac{t^{2} + 1}{t^{2}} = \int_{1}^{e} (1 + \frac{1}{t^{2}}) dt \\
				& = \left[ t - \frac{1}{t} \right]_{1}^{e} = e - \frac{1}{e} - (1 - 1) = e - \frac{
				1}{e}.
			\end{align*}
		\item Berechne $\beta \coloneqq \int_{0}^{1} \sqrt{1 - x^{2}} dx = \begin{cases} x = \sin t, t \in [0, \frac{\pi}{2}] \\ \frac{dx}{dt} = \cos t, dx = \cos t dt \end{cases}$
			% todo image
			\begin{align*}
				& = \int_{0}^{\frac{\pi}{2}} \sqrt{1 - \sin^{2}t} \cos t dt = \int_{0}^{\frac{\pi}{2}} \sqrt{\cos^{2} t} \cos t dt \\
				& = \int_{0}^{\frac{\pi}{2}} |\cos t| \cos t dt = \int_{0}^{\frac{\pi}{2}} \cos^{2} t dt = \int_{0}^{\frac{\pi}{2}} (1 - \sin^{2} t) dt \\
				& \overset{s.o.}{=} \left[ t - \frac{1}{2} (t - \cos t \sin t) \right]_{0}^{\frac{\pi}{2}} = \frac{\pi}{4}.
			\end{align*}
	\end{enumerate}
\end{beispiele}

Ohne Beweis:

\begin{satz} \label{10.15:satz}
	$f, g \colon [a, b] \rightarrow \R$ seien beschränkt.
	\begin{enumerate}
		\item Ist $\{ x \in [a, b]: f$ ist in $x$ nicht stetig $\}$ höchstens endlich, so ist $f \in R[a, b]$.
		\item Sei $f \in R[a, b]$ und $M \coloneqq \{ x \in [a, b] : f(x) \neq g(x) \}$ höchstens endlich, so ist $g \in R[a, b]$ und 
			$$ \int_{a}^{b} f(x) = \int_{a}^{b} g(x) dx. $$
	\end{enumerate}	
\end{satz}


\begin{satz} \label{10.16:satz-Mittelwertsätze}
	Es seien $f, g \in R[a, b], g \geq 0$ auf $[a, b]$, $m \coloneqq \inf f([a, b])$ und $M \coloneqq \sup f([a, b])$.
	\begin{enumerate}
		\item $\exists \mu \in [m, M]$: $\int_{a}^{b} fg dx = \mu \int_{a}^{b} g dx$.
		\item $\exists \mu \in [m, M]$: $\int_{a}^{b} f dx = \mu (b - a)$.
	\end{enumerate}
	Ist $f \in C[a, b]$, so existiert ein $\xi \in [a, b]$: für die Zahl $\mu$ in a) bzw. b): $\mu = f(\xi)$.
\end{satz}

\begin{proof} ~\
	\begin{enumerate}
		\item $g \geq 0$ auf $[a, b] \Rightarrow mg \leq fg \leq Mg$ auf $[a, b]$
			$$ \xRightarrow[]{\ref{10.2:satz}} m \underbrace{\int_{a}^{b} g dx}_{\eqqcolon A} \leq \underbrace{\int_{a}^{b} fg dx}_{\eqqcolon B} \leq M \int_{a}^{b} g dx. $$
			Also: $m A \leq B \leq M A$. \\
			Fall 1: $A = 0$. Dann ist $B = 0$ und jedes $\mu \in [m, M]$ leistet das Verlangte. \\
			Fall 2: $A \neq 0$. $g \geq 0 \xRightarrow[]{\ref{10.2:satz}} A > 0 \Rightarrow m \leq \frac{B}{A} \leq M$. $\mu \coloneqq \frac{B}{A}$.
		\item folgt aus a) mit $g \equiv 1$. Der Zusatz folgt aus \ref{7.7:prop-Zwischenwertsatz} und \ref{7.11:satz}.
	\end{enumerate}
\end{proof}


\begin{satz} \label{10.17:satz}
	Sei $(f_{n})$ eine Folge mit:
	\begin{enumerate}[label=\roman*\upshape)]
		\item $f_{n} \in C^{1}[a, b] ~\forall n \in \N$,
		\item $(f_{n}(a))$ ist konvergent und
		\item $(f_{n}')$ konvergiert auf $[a, b]$ gleichmä{\ss}ig gegen $g \colon [a, b] \rightarrow \R$.
	\end{enumerate}
	Dann konvergiert $(f_{n})$ auf $[a, b]$ gleichmä{\ss}ig und für $f(x) \coloneqq \lim_{x \rightarrow \infty} f_{n}(x) ~(x \in [a, b])$ gilt:
		$$ f \in C^{1}[a, b] \text{ und } g' = g \text{ auf } [a, b] $$
	Also: $\lim_{n \rightarrow \infty} f_{n}'(x) = g(x) = f'(x) = (\lim_{n \rightarrow \infty} f_{n}(x))'$ auf $[a, b]$.
\end{satz}

\begin{proof}
	$\alpha_{n} \coloneqq \int_{a}^{b} |f_{n}'(t)- g(t)| dt \xRightarrow[]{iii)} (|f_{n}' - g|)$ konvergiert auf $[a, b]$ gleichmä{\ss}ig gegen $0 \xRightarrow[]{\ref{10.8:satz}} \alpha_{n} \rightarrow 0$. o.B.d.A. $(f_{n}(a))$ ist eine Nullfolge. Für $n \in \N$ und $x \in [a, b]$:
	$$ f_{n}(x) \overset{\ref{10.6:prop-1Hauptsatz}}{=} \underbrace{f_{n}(a)}_{\rightarrow 0} + \int_{a}^{x} f_{n}'(t) dt \xrightarrow[]{\ref{10.8:satz}} \int_{a}^{x} g(t) dt \eqqcolon f(x). $$
	Also: $(f_{n})$ konvergiert auf $[a, b]$ punktweise gegen $f$. 
	$$ \xRightarrow[]{\ref{8.3.a:satz}} g \in C[a, b] \xRightarrow[]{\ref{10.11:2Hauptsatz}} f \in C^{1}[a, b] \text{ und } f' = g. $$
	Noch zu zeigen: $(f_{n})$ konvergiert auf $[a, b]$ gleichmä{\ss}ig gegen $f$.
	\begin{align*}
		|f_{n}(x) - f(x)| & = |f_{n}(x) - f_{n}(a) - \int_{a}^{x} g(t) dt + f_{n}(a) | \\
		& \overset{\ref{10.6:prop-1Hauptsatz}}{=} |\int_{a}^{x} (f_{n}'(t) - g(t)) dt + f_{n}(a) | \\
		& \leq \int_{a}^{x} |f_{n}'(t) - g(t)| dt + |f_{n}(a)| \\
		& \leq \int_{a}^{b} |f'_{n}(t) - g(t)| dt + |f_{n}(a)| \\
		& = \underbrace{\alpha_{n} + |f_{n}(a)|}_{\rightarrow 0} \quad \forall x \in [a, b].
	\end{align*}
	$\xRightarrow[]{\ref{8.1:satz}}$ Gleichmä{\ss}ige Konvergenz.
\end{proof}


\begin{bemerkung}
	Der Beweis von \ref{9.12:satz} folgt aus \ref{10.17:satz}.	
\end{bemerkung}


\newpage


\section{Uneigentliche Integrale}


\begin{vereinbarung}
	Ist $I \subseteq \R$ ein Intervall und $f \colon I \rightarrow \R$ eine Funktion, so soll stets gelten: $f \in R(J)$ für jedes kompakte Intervall $J \subseteq I$.	
\end{vereinbarung}

\index{uneigentliche Integral} \index{konvergent}
\begin{definition} ~\
	\begin{enumerate}
		\item Sei $a \in \R, \beta \in \R \cup \{ \infty \}, a < \beta$ und $f \colon [a, \beta) \rightarrow \R$ eine Funktion. Das \textbf{uneigentliche Integral} $\int_{a}^{\beta} f(x) dx$ hei{\ss}t \textbf{konvergent} $\iff$ der Grenzwert
			$$ \lim_{t \rightarrow \beta - 0} \int_{a}^{t} f(x) dx $$
			existiert und ist $\in \R$. In diesem Fall:
			$$ \int_{a}^{\beta} f(x) dx \coloneqq \lim_{t \rightarrow \beta - 0} \int_{a}^{t} f(x) dx. $$
		\item Sei $b \in \R, \alpha \in \R \cup \{ \infty \}, \alpha < b$ und $f \colon (\alpha, b] \rightarrow \R$ eine Funktion. Das \textbf{uneigentliche Integral} $\int_{\alpha}^{b} f(x) dx$ hei{\ss}t \textbf{konvergent} $\iff$ der Grenzwert
			$$ \lim_{t \rightarrow \alpha + 0} \int_{t}^{b} f(x) dx $$
			existiert und ist $\in \R$. In diesem Fall:
			$$ \int_{\alpha}^{b} f(x) dx \coloneqq \lim_{t \rightarrow \alpha + 0} \int_{t}^{b} f(x) dx. $$
	\end{enumerate}
	Ein nicht konvergentes uneigentliches Integral hei{\ss}t divergent.
	% todo image
\end{definition}


\begin{beispiele} ~\
	\begin{enumerate}
		\item $\int_{1}^{\infty} \frac{1}{x^{\gamma}} dx ~(\gamma > 0)$ ($a = 1, \beta = \infty$). Sei $t > 1$, \label{11.0:bsp-obigeBspa}
			$$ \int_{1}^{t} \frac{1}{x^{\gamma}} dx = \begin{cases} \log t, & \text{ falls } \gamma = 1 \\ \frac{1}{1 - \gamma} (t^{1 - \gamma} - 1), & \text{ falls } \gamma \neq 1. \end{cases} $$
			Also: $\int_{1}^{\infty} \frac{1}{x^{\gamma}} dx$ konvergiert $\iff \gamma > 1$. In diesem Fall:
			$$ \int_{1}^{\infty} \frac{1}{x^{\gamma}} dx = \frac{1}{\gamma - 1} $$
		\item $\int_{0}^{\infty} \frac{1}{1 + x^{2}} dx$ ($a = 0, \beta = \infty$). \\
			$$ \int_{0}^{t} \frac{1}{1 + x^{2}} dx = \arctan t \rightarrow \frac{\pi}{2} ~(t \rightarrow \infty). $$ \label{11.0:bsp-obigeBspb}
			Also ist $\int_{0}^{\infty} \frac{1}{1 + x^{2}} dx$ konvergent und $= \frac{\pi}{2}$.
		\item $\int_{0}^{1} \frac{1}{x^{\gamma}} dx ~(\gamma > 0)$ ($\alpha = 0, b = 1$). Wie in Beispiel a) sieht man: \label{11.0:bsp-obigeBspc}
			$$ \int_{0}^{1} \frac{1}{x^{\gamma}} dx \text{ konvergiert} \iff \gamma < 1 $$
		\item $\int_{-\infty}^{0} \frac{1}{1 + x^{2}} dx$ ($\alpha = -\infty, b = 0$). Wie in Beispiel b) sieht man: \label{11.0:bsp-obigeBspd}
			$$ \int_{-\infty}^{0} \frac{1}{1 + x^{2}} dx  \text{ konvergiert und } = \frac{\pi}{2}. $$
		\item $\int_{0}^{\infty} \sin x dx$. Sei $t_{n} \coloneqq (2n + 1) \pi ~(n \in \N)$. \label{11.0:bsp-obigeBspe}
			$$ \int_{0}^{t_{n}} \sin x dx = -\cos \Big|_{0}^{t_{n}} = 1 - \cos t_{n} = 1 - \cos(2n\pi + \pi) = 1 - \cos \pi = 2 $$
			Definiere $s_{n} \coloneqq 2n \pi$, dann:
			$$ \int_{0}^{s_{n}} \sin x dx = - \cos x \Big|_{0}^{s_{n}} = 1 - \cos s_{n} = 0 $$
			$\int_{0}^{\infty} \sin x dx$ ist also divergent.
	\end{enumerate}	
\end{beispiele}

\index{uneigentliche Integral!konvergiert}
\begin{definition}
	Sei $\alpha < \beta$, $\alpha \in \R \cup \{ - \infty \}, \beta \in \R \cup \{ \infty \}$ und $f \colon (\alpha, \beta) \rightarrow \R$ eine Funktion. \\
	Das \textbf{uneigentliche Integral} $\int_{\alpha}^{\beta} f(x) dx$ ist \textbf{konvergent} $\iff \exists c \in (\alpha, \beta)$: $\int_{\alpha}^{c} f(x) dx$ und $\int_{c}^{\beta} f(x) dx$ sind beide konvergent. In diesem Fall:
	$$ \int_{\alpha}^{\beta} f(x) dx \coloneqq \int_{\alpha}^{c} f(x) dx + \int_{c}^{\beta} f(x) dx $$
	(divergent = nicht konvergent).
\end{definition}

\textbf{Übung}: obige Definition ist unabhängig von $c \in (\alpha, \beta)$!

\begin{beispiele} ~\
	\begin{enumerate}
		\item $\int_{-\infty}^{\infty} x dx$ ist divergent, denn $\int_{0}^{\infty} x dx$ ist divergent. (Aber: $\lim_{t \rightarrow \infty} \int_{-t}^{t} x dx = 0$).
		\item Sei $\gamma > 0$. Obige Beispiele \hyperref[11.0:bsp-obigeBspa]{a)} und \hyperref[11.0:bsp-obigeBspc]{c)} zeigen:
			$$ \int_{0}^{\infty} \frac{1}{x^{\gamma}} dx \text{ ist divergent.} $$
		\item Obige Beispiele \hyperref[11.0:bsp-obigeBspb]{b)} und \hyperref[11.0:bsp-obigeBspd]{d)} zeigen:
			$$ \int_{-\infty}^{\infty} \frac{1}{1 + x^{2}} dx \text{ ist konvergent und} = \pi. $$
	\end{enumerate}
\end{beispiele}

Die folgenden Definitionen und Sätze formulieren wir nur für Funktionen $f \colon [a, \beta) \rightarrow \R$ (wie in der ersten Definition dieses Kapitels). Diese Definitionen und Sätze gelten sinngemä{\ss} auch für die beiden anderen Typen uneigentlicher Integrale.

\textbf{Beachte}: Für $t \in (a, \beta)$: $g(t) \coloneqq \int_{a}^{t} f(x) dx$. Dann:
	$$ \int_{a}^{\beta} f(x) dx \text{ konvergiert } \iff \lim_{t \rightarrow \beta} g(t) \text{ existiert.} $$
Aus \ref{6.2.c:satz} folgt:

\index{Cauchykriterium}
\begin{namedtheorem}[Cauchykriterium] \label{11.1:prop-Cauchykriterium}
	$\int_{a}^{\beta} f(x) dx$ konvergiert
	$$ \iff \forall \epsilon > 0 ~\exists c \in (a, \beta):  | \int_{u}^{v} f(x) dx | < \epsilon \quad \forall u, v \in (c, \beta).  $$
\end{namedtheorem}


\begin{beispiel*}
	Beh.: $\int_{1}^{\infty} \frac{\sin x}{x} dx$ konvergiert.	
\end{beispiel*}

\begin{proof}
	Seien $c \leq u < v$.
	\begin{align*}
		|\int_{u}^{v} \frac{\sin x}{x} dx | & = | \int_{u}^{v} \underbrace{\frac{1}{x}}_{g} \underbrace{\sin x}_{f'} dx | \\
			& = | \left[ -\frac{\cos x}{x} \right]_{v}^{u} - \int_{u}^{v} - \frac{1}{x^{2}} (-\cos x) dx | \\
			& = | \frac{\cos v}{v} - \frac{\cos u}{u} - \int_{u}^{v} \frac{\cos x}{x^{2}} dx | \\
			& \leq \frac{1}{v} + \frac{1}{u} + \int_{u}^{v} \frac{1}{x^{2}} dx = \frac{2}{u}
	\end{align*}
	(für $\epsilon > 0$: $\frac{2}{u} < \epsilon \iff u > \frac{2}{\epsilon}$). Sei $\epsilon > 0, c \coloneqq \frac{2}{\epsilon}$. Seien $\frac{2}{\epsilon} < u < v$. Dann:
	$$ | \int_{u}^{v} \frac{\sin x}{x} dx | \leq \frac{2}{u} < \epsilon $$
	$\xRightarrow[]{\ref{11.1:prop-Cauchykriterium}}$ Beh.
\end{proof}

\index{konvergent!absolut}
\begin{definition}
	$\int_{a}^{\beta} f(x) dx$ hei{\ss}t \textbf{absolut konvergent} $\iff \int_{a}^{\beta} |f(x)| dx$ ist konvergent.
\end{definition}

Den folgenden Satz beweist man mit \ref{11.1:prop-Cauchykriterium} ähnlich wie bei Reihen:

\index{Majorantenkriterium} \index{Minorantenkriterium}
\begin{satz} ~\ \label{11.2:satz}
	\begin{enumerate}
		\item Ist $\int_{a}^{\beta} f(x) d$ absolut konvergent, so ist $\int_{a}^{\beta} f(x) dx$ konvergent und
			$$ | \int_{a}^{\beta} f(x) dx | \leq \int_{a}^{\beta} |f(x)| dx. $$
		\item \textbf{Majorantenkriterium}: Ist $|f| \leq h$ auf $[a, \beta)$ und $\int_{a}^{\beta} h(x) dx$ konvergiert, so ist $\int_{a}^{\beta} f(x) dx$ konvergent.
		\item \textbf{Minorantenkriterium}: Ist $f \geq h \geq 0$ auf $[a, \beta)$ und $\int_{a}^{\beta} h(x) dx$ divergiert, so ist $\int_{a}^{\beta} f(x) dx$ divergent.
	\end{enumerate}	
\end{satz}


\begin{beispiele} ~\
	\begin{enumerate}
		\item $\int_{1}^{\infty} \underbrace{\frac{x}{\sqrt{1 + x^{5}}}}_{\eqqcolon f(x)} dx$. $|f(x)| = f(x) \leq \frac{x}{\sqrt{x^{5}}} = \frac{1}{x^{\frac{3}{2}}} \eqqcolon g(x)$. 
			$$ \int_{1}^{\infty} g(x) dx \text{ konvergiert } \Rightarrow \int_{1}^{\infty} f(x) dx \text{ konvergiert}. $$
		\item $\int_{1}^{\infty} \underbrace{\frac{x}{x^{2} + 7x}}_{\eqqcolon f(x)} dx$, $g(x) \coloneqq \frac{1}{x}$; $\frac{f(x)}{g(x)} = \frac{x^{2}}{x^{2} + 7x} \rightarrow 1 ~(x \rightarrow \infty)$.
			$$ \Rightarrow \exists c \geq 1: \frac{f(x)}{g(x)} \geq \frac{1}{2} ~\forall  x \geq c \Rightarrow f(x) \geq \frac{1}{2} g(x) ~\forall x \geq c. $$
			$\int_{c}^{\infty} g(x) dx$ divergiert $\Rightarrow \int_{1}^{\infty} g(x) dx$ divergiert $\Rightarrow \int_{1}^{\infty} f(x) dx$ divergiert.
	\end{enumerate}
\end{beispiele}

\newpage

\section{Die komplexe Exponentialfunktion}

\index{Betrag!einer komplexen Zahl}
Sei $z = x + iy \in \C ~(x, y \in \R)$.
\begin{description}  \addtolength{\itemindent}{0.4cm}
	\item $|z| \coloneqq \sqrt{x^{2} + y^{2}}$ \textbf{Betrag} von $z$.
	\item $\overline{z} \coloneqq x - iy$.
	\item $z \cdot \overline{z} = |z|^{2}$.
	\item $|z \cdot w| = |z| \cdot |w| ~(z, w \in \C)$.
	\item $e^{z} \coloneqq e^{x} (\cos y + i \sin y)$.
\end{description}
Ist $z = x \in \R$: $e^{z} = e^{x}$; ist $z = it ~(t \in \R)$: $e^{it} = \cos t + i \sin t$.


\begin{satz} \label{12.1:satz}
	Es gilt $\sum_{k=0}^{n} z^{k} = \frac{1 - z^{k+1}}{1 - z}, z \neq 1$.
	\begin{enumerate}
		\item $e^{z +w} = e^{z} e^{w} ~\forall z, w \in \C$. \label{12.1.a:satz}
		\item $|e^{it}| = 1 ~\forall t \in \R$, $e^{-it} = \overline{e^{it}} ~\forall t \in \R$. \label{12.1.b:satz}
		\item $e^{i \pi} + 1 = 0$. \label{12.1.c:satz}
		\item $e^{z + 2 k \pi i} = e^{z} ~\forall k \in \Z, z \in \C$. \label{12.1.d:satz}
		\item Für $t \in \R$: $\cos t = \frac{1}{2} \left( e^{it} + e^{-it} \right), \quad \sin t = \frac{1}{2i} \left( e^{it} - e^{-it} \right)$. \label{12.1.e:satz}
	\end{enumerate}
\end{satz}

\begin{proof} ~\
	\begin{enumerate}
		\item Übung (Add. von $e$-Funktionen, $\sin$, $\cos$).
		\item $e^{it} = \cos t + i \sin t \Rightarrow |e^{it} = (\cos^{2} t + \sin^{2} t)^{\frac{1}{2}} = 1$,
			$$ e^{-it} = \cos (-t) + i \sin (-t) = \cos t - i \sin t = \overline{\cos t + i \sin t} = \overline{e^{it}}. $$
		\item $e^{i\pi} = \cos \pi + i \sin \pi = -1$.
		\item $e^{2 k \pi i} = \cos (2 k \pi) + i \sin (2k \pi) = 1 \xRightarrow[]{a)}$ Beh.
		\item $e^{it} + e^{-it} = 1 \cos t$.
	\end{enumerate}
\end{proof}


\begin{definition}
	Für $z \in \C$:
	$$ \cos z \coloneqq \frac{1}{2} \left( e^{iz} + e^{-iz} \right), \quad \sin z \coloneqq \frac{1}{2 i} \left( e^{iz} - e^{-iz} \right). $$
\end{definition}


\begin{uebung}
	$\forall z, w \in \C$:
	\begin{align*}
		\sin (z + w) & = \sin z \cos w + \sin w \cos z \\
		\cos (z + w) & = \cos z \cos w - \sin z \sin w
	\end{align*}	
\end{uebung}


\begin{satz} \label{12.2:satz}
	Sei $z = x + i y \in \C ~(x, y \in \R)$.
	$$ e^{z} = 1 \iff \exists k \in \Z: z = 2 k \pi i $$
\end{satz}

\begin{proof}
	$"' \Leftarrow "'$: $\ref{12.1.d:satz}$. \\
	$"' \Rightarrow "'$ Sei $e^{z} = 1$, also $1 = e^{x} ( \cos y + i \sin y) = e^{x} \cos y + i e^{x} \sin y \Rightarrow e^{x} \cos y = 1, e^{x} \sin y = 0 \Rightarrow \sin y = 0 \Rightarrow \exists j \in \Z: y = j \pi \Rightarrow \cos y = (-1)^{j}$ somit $1 = e^{x} (-1)^{j} \Rightarrow j = 2k ~(k \in \Z)$ und $x = 0$. Also: $z = 2k \pi i$.
\end{proof}

\index{Polarkoordinaten} \index{Argument}
\textbf{Polarkoordinaten}: Sei $z = x + iy \in \C$, $x, y \in \R$ und $z \neq 0$.
	$$ r \coloneqq |z| = (x^{2} + y^{2})^{\frac{1}{2}} $$
	% todo image
	Die Gerade durch $0$ und $z$ schließt mit der positiven x-Achse einen Winkel $\varphi \in (-\pi, \pi]$ ein.
	
	$\varphi$ hei{\ss}t das \textbf{Argument von $z$}; $\phi = \arg z$. Es ist
		$$ \cos \varphi = \frac{x}{r}, \sin \varphi = \frac{y}{r}, $$
	also
		$$  z = x + iy = r \cos \varphi + i r \varphi = r e^{i \varphi} = |z| e^{i \varphi} = |z| e^{i \arg z} $$
	Ist weiter $w \in \C$ und $\psi \coloneqq \arg w$, so gilt:
		$$ z w = |z| e^{i \varphi} |w| e^{i \psi} = |z| |w| e^{i(\varphi + \psi)} $$
	Seien $z, w \in \C$ und $n \in \N \xRightarrow[]{\ref{12.1:satz}} (e^{z})^{n} = e^{nz}$; Es gilt:
	\begin{align*}
		e^{z} = e^{w} & \iff e^{z} e^{-w} = e^{w} e^{-w} \\
			& \iff e^{z-w} = e^{w - w} = e^{0} =1 \\
			& \overset{\ref{12.2:satz}}{\iff} \exists k \in \Z: z = w + 2 k \pi i
	\end{align*}

Ohne Beweis:

\index{Fundamentalsatz der Algebra}
\begin{namedtheorem}[Fundamentalsatz der Algebra] \label{12.3:prop-FundamentalsatzDerAlgebra}
	Sei $p(z) = a_{0} + a_{1} z + \dotsc + a_{n} z^{n}$ ein Polynom mit $n \geq 1$, $a_{0}, \dotsc, a_{n} \in \C$ und $a_{n} \neq 0$. Dann existieren $z_{1}, \dotsc, z_{n} \in \C$ (eind. bestimmt) mit $p(z) = a_{n} (z - z_{1}) \cdot \dotsc \cdot (z - z_{n}) ~(z \in \C)$. $z_{1}, \dotsc, z_{n}$ sind genau die Nullstellen von $p$.
\end{namedtheorem}

\index{Wurzel!n-te}
\begin{definition}
	Sei $a \in \C \setminus \{ 0 \}$ und $n \in \N$. Jedes $z \in \C$ mit $z^{n} = a$ hei{\ss}t eine \textbf{n-te Wurzel aus} $a$.
	$$ \sqrt[n]{a} \text{ bez. eine n-te Wurzel aus } a ~(n = 2 \text{kurz: Wurzel}) $$
\end{definition}


\begin{satz} \label{12.4:satz}
	Sei $a \in \C \setminus \{ 0 \}$, $n \in \N, r \coloneqq |a|$ und $\varphi \coloneqq \arg a$. (also $a = |a| e^{i \varphi} = r e^{i \varphi}$). Für $k = 0, 1, \dotsc, n - 1$ sei
		$$ z_{k} \coloneqq \sqrt[n]{r} e^{i \frac{l4\varphi + 2 k \pi}{n}} $$
	Dann:
	\begin{enumerate}
		\item $z_{j} \neq z_{k}$ für $j \neq k$.
		\item $z$ ist eine n-te Wurzel aus $a \iff z \in \{ z_{0}, z_{1}, \dotsc, z_{k-1} \}$.
	\end{enumerate}
\end{satz}

\begin{proof} ~\
	\begin{enumerate}
		\item Seien $j, k \in \{0, \dotsc, n - 1 \}$, $z_{j} = z_{k}$ und $k \geq j$. Also: $e^{i \frac{\varphi + 2k \pi}{n}} = e^{i \frac{\varphi + 2 j \pi}{n}} \xRightarrow[]{s.o.} \exists l \in \Z$:
			$$ i \frac{\varphi + 2k \pi}{n} = i \frac{\varphi + 2 j \pi}{n} + 2 e \pi i \Rightarrow \frac{\varphi}{2 \pi} + k = \frac{\varphi}{2 \pi} + j + l n $$
			$$ \Rightarrow \frac{k - j}{n} = l \Rightarrow |l| = \frac{|k - j|}{n} = \frac{k - j}{n} \leq \frac{k}{n} \leq \frac{n - 1}{n} = 1 - \frac{1}{n} < 1 $$
			$\Rightarrow l = 0 \Rightarrow k = j$.
		\item $p(z) \coloneqq z^{n} - a$. Dann: $z$ ist eine n-te Wurzel aus $a \iff p(z) = 0$. Es gilt
			$$ z_{k}^{n} \overset{s.o.}{=} r e^{i(\varphi +2 k \pi)} = r e^{i \varphi} e^{2 k \pi i} = r e^{i \varphi} = a. $$
			Also: $p(z_{k}) = 0 ~(k = 0, \dotsc, n - 1)$. Aus a) und \ref{12.3:prop-FundamentalsatzDerAlgebra} folgt die Beh.
	\end{enumerate}
\end{proof}

\index{Einheitswurzeln!n-te}
\begin{bezeichnung}
	Ist $a = 1$, so hei{\ss}en die Zahlen $z_{0}, \dotsc, z_{n-1}$ aus \ref{12.4:satz} die \textbf{n-ten Einheitswurzeln}.	Also $z_{k} = e^{\frac{2 k \pi i}{n}} ~(k = 0, \dotsc, n-1)$.
\end{bezeichnung}


\begin{bemerkung}
	$z^{n} - 1 = \Pi_{k=0}^{n-1} (z - z_{k})$.	
\end{bemerkung}


\begin{beispiele} ~\
	\begin{enumerate}
		\item Im Rahmen ist $\sqrt{4} = 2$; Im Komplexen sind die Wurzeln aus $4$: $2, -2$.
		\item Die 4. Wurzeln aus $16$ sind $2, -2, 2i, -2i$.
		\item Die 4. Einheitswurzel sind $1, -1, i, -i$.
	\end{enumerate}	
\end{beispiele}


\begin{beispiel*}
	Man kann $\sqrt{-3 + 4i}$ mittels verschiedener Ansätze berechnen:
	\begin{description}
		\item 1. Möglichkeit: $w = u + iv$, $w^{2} = u^{2} - v^{2} + 2iuv = -3 + 4i$
			$$ \iff u^{2} - v^{2} = -3, 2uv = 4 $$
			Löse Gleichungssystem.
		\item 2. Möglichkeit: $z = 3 + 4i$. Bestimme $|z|, \varphi = \arg z$. Dann sind
			$$ \pm \sqrt{|z|} = e^{i \frac{\arg z}{z}} \text{ die Wurzeln von z.} $$
		\item 3. Möglichkeit: Ist $z \in (-\infty, 0]$, so sind $w = \pm \sqrt{-z}$ die Wurzeln von $z$. Ist $z \in \C \setminus (-\infty, 0]$, so sind
			$$ w = \pm \sqrt{|z|} \frac{z + |z|}{\left| z + |z| \right|} $$
			die Wurzeln von $z$; Beweis:
			\begin{proof}
				Betrachte:
				\begin{align*}
					 \left( \sqrt{|z|} \frac{z + |z|}{\left| z + |z| \right|} \right)^{2} & = |z| \frac{(z + |z|) (z + |z|)}{(z + |z|) (\overline{z} + |z|)} =  |z| \frac{(z + |z|)}{(\overline{z} + |z|)} \\
					 & = \frac{(|z|z + z \overline{z})}{(\overline{z} + |z|)} = z \frac{(|z| + \overline{z})}{(\overline{z} + |z|)} = z 
				\end{align*}
			\end{proof}
			Also: $\sqrt{-3 + 4i} = \pm \sqrt{5} \frac{-3 + 4i + 5}{|-3 + 4i + 5} = \sqrt{5} \frac{2 + 4i}{\sqrt{20}} = \pm (1 + 2i)$.
	\end{description}
\end{beispiel*}


\begin{satz} \label{12.5:satz}
	Seien $p, q \in \C$. Für $z \in \C$:
	$$ z^{2} + pz + q = 0 \iff z = -\frac{p}{2} \pm \underbrace{\sqrt{\frac{p^{2}}{4} - q}}_{\text{doppeldeutig!}}. $$	
\end{satz}

\begin{proof}
	$"' \Leftarrow "'$ nachrechnen. Rest mit \ref{12.3:prop-FundamentalsatzDerAlgebra}.
\end{proof}


\begin{beispiel}
	Löse $z^{2} + (1 - 2i)z - 2i = 0$ $(*)$.
	\begin{align*}
		z & = \frac{2i - 1}{2} \pm \sqrt{ \frac{(2i - 1)^{2}}{4} + 2i} = i - \frac{1}{2} \pm \sqrt{\frac{-4 - 4i + 1}{4} + 2i} \\
		  & = i - \frac{1}{2} \pm \sqrt{ -3 - 3i + 8i}  = i - \frac{1}{2} \pm \sqrt{1}{2} \sqrt{-3 + 4i}.
	\end{align*}
	Also sind 
	\begin{align*}
		z_{1} & = i - \frac{1}{2} + \frac{1}{2} (1 + 2i) = 2i
		\intertext{und}
		z_{2} & = i - \frac{1}{2} + \frac{1}{2} (-1 - 2i) = -1 
	\end{align*}
	die Lösungen von $(*)$. Es gilt $z^{2} + (1 - 2i) z - 2i = ( z - z_{1})(z - z_{2}) = (z - 2i) (z + 1)$.
\end{beispiel}

\index{Logarithmus}
\begin{definition}
	Sei $w \in \C \setminus \{ 0 \}$. Jedes $z \in \C$ mit $e^{z} = w$ hei{\ss}t ein \textbf{Logarithmus von $w$}.
\end{definition}


\begin{satz} \label{12.6:satz}
	Sei $w \in \C \setminus \{ 0 \}$, $ r \coloneqq |w|$ und $\varphi = \arg w$, also $w = r e^{i \varphi}$. Sei $z \in \C$. 
	$$ z \text{ ist ein Logarithmus von } w \iff \exists k \in \Z: z = \underbrace{\log |w|}_{\log \text{ in } \R} + i \varphi + 2 k \pi i. $$	
\end{satz}

\begin{proof}
	$"' \Leftarrow "'$ $e^{z} = e^{\log |w|} e^{i \varphi} e^{2 k \pi i} = |w| e^{i \varphi} = w$. \\
	$"' \Rightarrow "'$ Sei $z = x + iy ~(x, y \in \R)$ und $w = e^{z} = e^{x} e^{iy} \Rightarrow |w| = e^{x} \Rightarrow x = \log |w|$. Es ist
	$$ |w| e^{i \varphi} = w = e^{z} = e^{x} e^{iy} = |w| e^{iy} $$
	$\Rightarrow e^{i \varphi} = e^{iy} \xRightarrow[]{s.o.} \exists k \in \Z: i y = i \varphi + 2k \pi i \Rightarrow z = \log |w| + i \varphi + 2k \pi i$.
\end{proof}


\begin{beispiele} ~\
	\begin{enumerate}
		\item $w = -1$; $|w| = 1, \arg w = \pi$. Alle Logarithmen von $-1$:
			$$ i \pi + 2 k \pi i \quad (k \in \Z). $$
		\item $w = 1$; $|w| = 1, \arg w = 0$. Alle Logarithmen von $1$:
			$$ 2 k \pi i \quad (k \in \Z). $$
		\item $w = 1 + i$; $|w| = \sqrt{2}, \arg w = \frac{\pi}{4}$. Alle Logarithmen von $1 + i$:
			$$ \log \sqrt{2} + i \frac{\pi}{4} + 2k \pi i \quad (k \in \Z). $$			
	\end{enumerate}
\end{beispiele}


\newpage

\section{Fourierreihen}

I. d. $\S$-en sei $f \colon \R \rightarrow \R$ eine Funktion mit:
	$$ (V) ~ \begin{cases}
				f \in R[-\pi, \pi] \text{ und $f$ ist auf $\R$ $2\pi$-periodisch,} \\
				\text{d.h. } f(x + 2 \pi) = f(x) ~\forall x \in \R.
			\end{cases} $$

\index{trigonometrische Reihe}
\begin{definition}
	Es seien $(a_{n})_{n=0}^{\infty}$ und $(b_{n})_{n=1}^{\infty}$ Folge in $\R$. Eine Reihe der Form
	$$ \frac{a_{0}}{2} + \sum_{n=1}^{\infty} \left( a_{n} \cos(nx) + b_{n} \sin(nx) \right) $$
	hei{\ss}t eine \textbf{trigonometrische Reihe} (TR).
\end{definition}

\textbf{Fragen}: Wann ist $f$ durch eine trigonometrisch Reihe darstellbar? Wie hängt dann $f$ mit $(a_{n})$, $(b_{n})$ zusammen?

\index{Orthogonalitätsrelationen}
\begin{satz} \label{13.1:satz} ~\
	\begin{enumerate}
		\item Ist $a \in \R$, so gilt: $f \in R[a, a + 2\pi]$ und $\int_{a}^{a+ 2\pi} f(x) dx = \int_{-\pi}^{\pi} f(x) dx$.
		\item \textbf{Orthogonalitätsrelationen}: für $k, n \in \N$:
			\begin{align*}
				\int_{-\pi}^{\pi} \sin(nx) \cos(kx) dx & = 0
				\intertext{und}
				\int_{-\pi}^{\pi} \sin(nx) \sin(kx) dx & = \int_{-\pi}^{\pi} \cos(nx) \cos(kx) dx \\
				& = \begin{cases} \pi, & k = n \\ 0, & k \neq n. \end{cases}				
			\end{align*}
	\end{enumerate}	
\end{satz}

\begin{proof}
	Übung.
\end{proof}


\textbf{Motivation}: Es seien $(a_{n})_{n=0}^{\infty}$ und $(b_{n})_{n=1}^{\infty}$ Folgen und es gelte
	$$ f(x) = \frac{a_{0}}{2} + \sum_{n=1}^{\infty} \left( a_{n} \cos(nx) + b_{n} \sin(nx) \right) \quad \forall x \in \R. $$
	Weiter sei diese trigonometrisch Reihe auf $\R$ gleichmä{\ss}ig konvergent. Sei $k \in \N$, dann:
	$$ f(x) \sin(k x) = \frac{a_{0}}{2} \sin(kx) + \sum_{n=1}^{\infty} \left( a_{n} \cos(nx) \sin(kx) + b_{n} \sin(nx) \sin(kx) \right) \quad \forall x \in \R.$$
	Übung: die letzte Reihe konvergiert auf $\R$ ebenfalls gleichmä{\ss}ig.
	\begin{align*}
		\xRightarrow[]{\ref{10.8:satz}} \int_{-\pi}^{\pi} f(x) \sin(kx) dx & = \frac{a_{0}}{2} \underbrace{\int_{-\pi}^{\pi} \sin(kx) dx}_{= 0} \sum_{n = 1}^{\infty} a_{n} \underbrace{\int_{-\pi}^{\pi} \cos(nx)\sin(kx) dx}_{\overset{\ref{13.1:satz}}{=} 0} \\
			& ~\qquad + \sum_{n = 1}^{\infty} b_{n} \underbrace{\int_{-\pi}^{\pi} \sin(nx) \sin(kx) dx}_{\overset{\ref{13.1:satz}}{=} \begin{cases} \pi, & \text{ fals } k = n \\ 0, & \text{ falls } k \neq n \end{cases}} \\
		& = b_{k} \pi
	\end{align*}
	Also:
	$$ b_{k} = \frac{1}{\pi} \int_{-\pi}^{\pi} f(x) \sin(kx) dx \quad \forall k \in \N. $$
	Analog zeigt man:
	$$ a_{k} = \frac{1}{\pi} \int_{-\pi}^{\pi} f(x) \cos(kx) dx \quad \forall k \in \N_{0}. $$
	d.h. $a_{0} = \frac{1}{\pi} \int_{-\pi}^{\pi} f(x) dx = \frac{1}{\pi} \int_{-\pi}^{\pi} \frac{a_{0}}{2} dx$.
	
\index{Fourierkoeffizienten} \index{Fourierreihe}
\begin{definition}
	$f$ erfülle $(V)$. Setze
	\begin{align*}
		a_{n} & \coloneqq \frac{1}{\pi} \int_{-\pi}^{\pi} f(x) \cos(nx) dx \text{ für } n \in \N_{0}.
		\intertext{und}
		b_{n} & \coloneqq \frac{1}{\pi} \int_{-\pi}^{\pi} f(x) \sin(nx) dx \text{ für } n \in \N.
	\end{align*} 
	Die Zahlen $a_{n}$, $b_{n}$ hei{\ss}en die \textbf{Fourierkoeffizienten} (FK) von $f$ und die mit $a_{n}$ und $b_{n}$ gebildete Fourierreihe hei{\ss}t \textbf{die zu $f$ gehörenden Fourierreihe}. Man schreibt:
	$$ f(x) \sim \frac{a_{0}}{2} + \sum_{n=1}^{\infty} \left( a_{n} \cos(nx) + b_{n} \sin(nx) \right). $$
\end{definition}

\textbf{Frage}: wann, bzw. für welche $x \in \R$konvergiert die zu $f$ gehörige Fourierreihe gegen $f(x)$?

\begin{satz} \label{13.2:satz}
	Für $f$ gelte $(V)$.
	\begin{enumerate}
		\item Ist $f$ gerade, also $f(x) = f(-x) ~\forall x \in \R$, so gilt für die Fourierkoeffizienten von $f$:
			$$ a_{n} = \frac{2}{\pi} \int_{0}^{\pi} f(x) \cos(nx) dx \text{ und } b_{n} = 0 $$
		\item Ist $f$ ungerade, also $f(x) = -f(-x) ~\forall x \in \R$, so gilt für die Fourierkoeffizienten von $f$:
			$$ a_{n} = 0 \text{ und } b_{n} = \frac{2}{\pi} \int_{0}^{\pi} f(x) \sin(nx) dx $$
	\end{enumerate}	
\end{satz}

\begin{proof}
	Übung.
\end{proof}

\index{Grenzwert!rechtsseitiger} \index{Grenzwert!linksseitiger}
\begin{definition} ~\
	\begin{enumerate}
		\item Sei $x_{0} \in \R, \delta > 0$ und $g \colon (x_{0}, x_{0} + \delta) \rightarrow \R$ eine Funktion
			$$ g(x_{0}+) \coloneqq \lim_{x \rightarrow x_{0} + 0} g(x), \text{ falls dieser Grenzwert existiert und $\in \R$ ist}. $$
		\item Sei $x_{0} \in \R, \delta > 0$ und $g \colon (x_{0} - \delta, x_{0}) \rightarrow \R$ eine Funktion
			$$ g(x_{0}-) \coloneqq \lim_{x \rightarrow x_{0} - 0} g(x), \text{ falls dieser Grenzwert existiert und $\in \R$ ist}. $$
	\end{enumerate}	
\end{definition}


\begin{definition}
	Für $f$ gelte $(V)$. $f$ hei{\ss}t \textbf{stückweise glatt} $\iff$ es existiert eine Zerlegung $\{ t_{0}, t_{1}, \dotsc, t_{n} \}$ von $[-\pi, \pi]$ (also $-\pi = t_{0} < t_{1} < \dotsc < t_{n-1} < t_{n} = \pi$) mit:
	\begin{enumerate}[label=\roman*\upshape)]
		\item $f \in C^{1} \left( (t_{j-1}, t_{j}) \right) ~(j = 1, \dotsc, n)$.
		\item Es existieren die folgenden Grenzwerte: 
			$$ f(\pi-), ~ f'(\pi-), ~ f(-\pi+), ~ f'(-\pi+) $$ 
			und
			$$ f(t_{j}+), ~f'(t_{j}+), ~f(t_{j}-), ~f'(t_{j}-) \quad (j =1, \dotsc, n - 1) $$
	\end{enumerate}
	% todo image
\end{definition}


\textbf{Beachte}: ~\
\begin{enumerate}
	\item In den Punkten $t_{j}$ muss $f$ nicht stetig sein (aber: $f(t_{0}) = f(-\pi) = f(-\pi + 2\pi) = f(\pi) = f(t_{n})$)
	\item $f$ $2\pi$-per $\Rightarrow f(x-), f(x+)$ existiert in jedem $x \in \R$.
		$$ s_{f}(x) \coloneqq \frac{f(x+) + f(x-)}{2} \quad (x \in \R) $$
\end{enumerate}

Ohne Beweis:

\begin{satz} \label{13.3:satz}
	Für $f$ gelte $(V)$ und $f$ sei stückweise glatt. Dann konvergiert die Fourierreihe von $f$ in jedem $x \in \R$ gegen $s_{f}(x)$. Ist in diesem Fall $f$ in $x \in \R$ stetig, so konvergiert die Fourierreihe von $f$ also gegen $f(x)$.	
\end{satz}


\begin{beispiel} \label{13.4:bsp}
	$f \colon \R \rightarrow \R$ sei $2\pi$-periodisch und auf $(-\pi, \pi]$ definiert durch
	$$ f(x) \coloneqq \begin{cases} x, & x \in (-\pi, \pi) \\ 0, & x = \pi \end{cases} $$
	% todo image
	$\xRightarrow[]{\ref{10.15:satz}} f \in R[-\pi, \pi]$, $f$ erfüllt also $(V)$. $f$ ist stückweise glatt und $s_{f}(x) = f(x) ~\forall x \in \R$. $f$ ist ungerade $\xRightarrow[]{\ref{13.2:satz}} a_{n} = 0 ~\forall n \in \N_{0}$ und
	$$ b_{n} = \frac{2}{\pi} \int_{0}^{\pi} f(x) \sin(nx) dx \overset{\ref{10.15:satz}}{=} \frac{2}{\pi} \int_{0}^{\pi} x \sin(nx) dx \overset{Übung}{=} (-1)^{n+1} \frac{2}{n} \quad \forall n \in \N. $$
	$\xRightarrow[]{\ref{13.3:satz}} f(x) = 2 \sum_{n=1}^{\infty} \frac{(-1)^{n+1}}{n} \sin(nx) ~\forall x \in \R$.
	$$ \Rightarrow \frac{x}{2} = \sum_{n=1}^{\infty} \frac{(-1)^{n+1}}{n} \sin(nx) \quad \forall x \in (-\pi, \pi). $$
	$x = \frac{\pi}{2}$: $\frac{\pi}{4} = 1 - \frac{1}{3} + \frac{1}{5} - \frac{1}{7} +- \dotsc$ (vgl. \ref{9.17.b:anwendungen}.
\end{beispiel}


\begin{beispiel} \label{13.5:bsp}
	$f \colon \R \rightarrow \R$ sei $2\pi$-periodisch und auf $[-\pi, \pi]$ definiert durch $f(x) = x^{2}$. \\
	% todo image
	Klar: $f$ erfüllt $(V)$, $f$ ist stückweise glatt, $f$ ist gerade und $f(x) = s_{f}(x) ~\forall x \in \R$.
	$$ \xRightarrow[]{\ref{13.2:satz}} b_{n} = 0 ~\forall n \in \N, \quad a_{n} = \frac{2}{\pi} \int_{0}^{\pi} x^{2} \cos(nx) dx = \begin{cases} \frac{2 \pi^{2}}{3}, & n = 0 \\ 4 \frac{ (-1)^{n}}{n^{2}}, & \text{sonst} \end{cases}  $$
	Aus der Rechnung in \ref{13.3:satz} folgt:
	$$ f(x) = \frac{\pi^{2}}{3} - 4 \left( \frac{\cos x}{1^{2}} - \frac{\cos(2x)}{2^{2}} + \frac{\cos(3x)}{3^{2}} -+ \dotsc \right) \quad \forall x \in \R. $$
	$\Rightarrow x^{2} = \frac{\pi^{2}}{3} - 4 \left( \frac{\cos x}{1^{2}} - \frac{\cos(2x)}{2^{2}} + \frac{\cos(3x)}{3^{2}} -+ \dotsc \right) ~\forall x \in [-\pi, \pi]$.
	\begin{align}
		& x = 0: \quad \frac{\pi^{2}}{12} = 1 - \frac{1}{x^{2}} + \frac{1}{3^{2}} - \frac{1}{4^{2}} +- \dotsc = \sum_{n=1}^{\infty} \frac{(-1)^{n+1}}{n^{2}} \tag{1} \\
		& x = \pi: \quad \frac{\pi^{2}}{6} = 1 + \frac{1}{x^{2}} + \frac{1}{3^{2}} + \frac{1}{4^{2}} + \dotsc = \sum_{n=1}^{\infty} \frac{1}{n^{2}} \tag{2}
	\end{align}
	Addition von $(1)$, $(2)$: $\frac{\pi^{2}}{8} = 1 + \frac{1}{3^{2}} + \frac{1}{5^{2}} + \dotsc = \sum_{n=0}^{\infty} \frac{1}{(2n +1)^{n}}$.
\end{beispiel}

Ohne Beweis:

\begin{satz} \label{13.6:satz}
	Es gelte $(V)$, es sei $f \in C(\R)$ und $f$ sei stückweise glatt.
	\begin{enumerate}
		\item Die Fourierreihe von $f$ konvergiert in jedem $x \in \R$ absolut.
		\item Die Fourierreihe von $f$ konvergiert auf $\R$ gleichmä{\ss}ig (gegen $f$).
		\item Sind $a_{n}, b_{n}$ die Fourierkoeffizienten von$f$, so konvergieren die Reihen
			$$ \frac{a_{0}}{2} + \sum_{n=1}^{\infty} \left( a_{n} \cos(nx) + b_{n} \sin(nx) \right) $$
			die zu $g$ gehörige Fourierreihe.
	\end{enumerate}	
\end{satz}


\begin{definition}
	Sei $g \in R[-\pi, \pi]$. Setze
	\begin{align*}
		a_{n} & \coloneqq \frac{1}{\pi} \int_{-\pi}^{\pi} g(x) \cos(nx) dx \quad (n \in \N_{0})
		\intertext{und}
		b_{n} & \coloneqq \frac{1}{\pi} \int_{-\pi}^{\pi} g(x) \sin(nx) dx \quad (n \in \N).
	\end{align*} 
	Auch in diesem Fall hei{\ss}en die Zahlen $a_{n}, b_{n}$ die Fourierkoeffizienten von $g$ und die Reihe 
	$$ \frac{a_{0}}{2} + \sum_{n=1}^{\infty} \left( a_{n} \cos(nx) + b_{n} \sin(nx) \right) $$
	die zu $g$ gehörige Fourierreihe.
\end{definition}

\index{Besselsche Ungleichung}
\begin{satz} \label{13.7:satz}
	$g$, $a_{n}$ und $b_{n}$ seien wie in der obigen Definition.
	\begin{enumerate}
		\item $\sum_{n=1}^{\infty} (a_{n}^{2} + b_{n}^{2})$ ist konvergent. \label{13.7.a:satz}
		\item $\frac{a_{0}^{2}}{2} + \sum_{n=1}^{\infty} (a_{n}^{2} + b_{n}^{2}) \leq  \frac{1}{\pi} \int_{-\pi}^{\pi} g(x)^{2} dx$ Besselsche Ungleichung. \label{13.7.b:satz}
		\item $a_{n} \rightarrow 0$, $b_{n} \rightarrow 0$. \label{13.7.c:satz}
	\end{enumerate}	
\end{satz}

\begin{proof}
	Für $n \in \N$ und $x \in [-\pi, \pi]$:
		$$ s_{n}(x) \coloneqq \frac{a_{0}}{2} + \sum_{k=1}^{n} (a_{k} \cos(kx) + b_{k} \sin(kx)) $$
	Dann: 
	\begin{align*}
		0 \leq \int_{-\pi}^{\pi} (g(x) - s_{n}(x))^{2} dx & = \int_{-\pi}^{\pi} (g(x)^{2} - 2g(x) s_{n} + s_{n}(x)^{2} ) dx \\
		  & \overset{\ref{13.1:satz}}{\underset{nachr.}{=}} \int_{-\pi}^{\pi} g(x)^{2} dx - \pi \left( \frac{a_{0}^{2}}{2} + \sum_{k=1}^{n} ( a_{k}^{2} + b_{k}^{2}) \right)
	\end{align*}
	$\Rightarrow \alpha_{n} \coloneqq \frac{a_{0}^{2}}{2} + \underbrace{\sum_{k=1}^{b}(a_{k}^{2} + b_{k}^{2})}_{\eqqcolon \beta_{n}} \leq \frac{1}{\pi} \int_{-\pi}^{\pi} g(x)^{2} dx \eqqcolon \alpha$. \\
	Also ist $(\alpha_{n})$ monoton und beschränkt, somit ist $(\alpha_{n})$ konvergent. Damit ist $(\beta_{n})$ konvergent $\Rightarrow (1)$. \\
	$\alpha_{n} \leq \alpha ~\forall n \in \N \Rightarrow (2)$. \\
	$(3) ~ a_{n}^{2} \leq a_{n}^{2} + b_{n}^{2}$. Aus $(1)$ und \ref{3.1:satz}: $a_{n}^{2} + b_{n}^{2} \rightarrow 0 \Rightarrow a_{n}^{2} \rightarrow 0 \Rightarrow a_{n} \rightarrow 0$. Genauso: $b_{n} \rightarrow 0$.
\end{proof}


\begin{namedtheorem}[Satz von Riemann-Lebesgue] \label{13.8:prop-SatzVonRiemann-Lebesgue}
	Seien $a, b \in \R$ und $a < b$ und $g \in R[a, b]$. Dann:
		$$ \int_{a}^{b} g(x) \sin(nx) dx \rightarrow 0 \text{ und } \int_{a}^{b} g(x) \cos(nx) dx \rightarrow 0 \quad (n \rightarrow \infty) $$
		Ohne Beweis. Für $[a, b] = [-\pi, \pi]$ vgl. $\ref{13.7.c:satz}$.
\end{namedtheorem}

\newpage

\section{Der Raum $\R^{n}$}

Es sei $n \in \N$. $\R^{n} \coloneqq \{ (x_{1}, \dotsc, x_{n}) : x_{1}, \dotsc, x_{n} \in \R \}$. $\R^{n}$ ist mit der bekannten Addition und Skalarmultiplikation ein Vektorraum über $\R$, $\dim \R^{n} = n$.

\index{Einheitsvektoren}
\textbf{Einheitsvektoren}: \\
$e_{1} \coloneqq (1, 0, \dotsc, 0)$, $e_{2} \coloneqq (0, 1, 0, \dotsc, 0)$, $\dotsc$, $e_{n} \coloneqq (0, \dotsc, 0, 1)$ \\
$\{ e_{1}, \dotsc, e_{n} \}$ ist ein Basis des $\R^{n}$. Ist $x = (x_{1}, \dotsc, x_{n}) \in \R^{n}$, so ist
$$ x = x_{1} e_{1} + \dotsc + x_{n} e_{n}. $$

\index{Innenprodukt} \index{Skalarprodukt} \index{Norm} \index{Länge}
\begin{definition}
	Seien $x = (x_{1}, \dotsc, x_{n})$, $y = (y_{1}, \dotsc, y_{n}) \in \R^{n}$.
	\begin{enumerate}
		\item $xy \coloneqq x \dotsc y \coloneqq x_{1} y_{1} + \dotsc + x_{n} y_{n}$ \textbf{Skalarprodukt} oder \textbf{Innenprodukt} von $x$ und $y$. Beachte: $xy \in \R$.
		\item $\| x \| \coloneqq \sqrt{x \cdot x} = (x_{1}^{2} + \dotsc + x_{n}^{2})^{\frac{1}{2}}$ \textbf{Norm} oder \text{Länge} von $x$. Beachte: $\|x\|^{2} = x \cdot x$ (Im Fall $n = 1: \|x\| = |x|$).
		\item $\| x - y \|$ hei{\ss}t Abstand von $x$ und $y$. Beachte: $\| x - y \| = \| y - x \|$.
			% todo image
	\end{enumerate}
\end{definition}


\begin{beispiele} ~\
	\begin{enumerate}
		\item $(1, 2, -1) \cdot (1, 3, 4) = 1 + 6 - 4 = 3$.
		\item $\| (1, 2, -1) \| = (1 + 4 + 1)^{\frac{1}{2}} = \sqrt{6}$.
		\item $\| e_{j} \| = 1 ~(j = 1, \dotsc, n)$.
	\end{enumerate}
\end{beispiele}

\index{Cauchy-Schwarz Ungleichung} \index{Dreiecksungleichung}
\begin{satz} \label{14.1:satz}
	Seien $x = (x_{1}, \dotsc, x_{n})$, $y, z \in \R^{n}$ und $\alpha \in \R$.
	\begin{enumerate}
		\item $(x + y) \cdot z = x \cdot z + y \cdot y$; $x \cdot y = y \cdot x$.
		\item $(\alpha x) \cdot y = \alpha (x \cdot y) = x \cdot (\alpha y)$.
		\item $\| x \| \geq 0$; $\| x \| = 0 \iff x = 0 = (0, \dotsc, 0)$.
		\item $\| \alpha x \| = |\alpha| \| x \|$.
		\item $\| x \cdot y \| \leq \| x \| \| y \|$ \textbf{Cauchy-Schwarz Ungleichung} (CSU).
		\item $\| x + y\| \leq \| x \| + \| y \|$ \textbf{Dreiecksungleichung}.
		\item $\left \|x\| - \| y \| \right| \leq \| x - y \|$.
		\item Für $j \in \{ 1, \dotsc, n \}$: $|x_{j}| \leq \| x \| \leq \sum_{k=1}^{n} |x_{k}|$
	\end{enumerate}
\end{satz}

\begin{proof}
	a) - d): Nachrechnen.
	\begin{enumerate} \setcounter{enumi}{4}
		\item o.B.d.A. $y \neq 0$, also $\| y \| > 0$. $A \coloneqq \| x \|^{2} = x \cdot x$, $B \coloneqq x \cdot y$, $C \coloneqq \| y \|^{2} = y \cdot y$, $\alpha \coloneqq \frac{B}{C}$. Dann:
			\begin{align*}
				0 & \leq \sum_{j=1}^{n} (x_{j} - \alpha y_{j})^{2} = \sum_{j=1}^{n} \left( x_{j}^{2} - 2\alpha x_{j} y_{j} + \alpha^{2} y_{j}^{2} \right) \\
				  & = A - 2 \alpha B + \alpha^{2} C = A - 2 \frac{B^{2}}{C} + \frac{B^{2}}{C} = A - \frac{B^{2}}{C} 	
			\end{align*}
			$\Rightarrow B^{2} \leq AC \Rightarrow (x \cdot y)^{2} \leq \|x\|^{2} \|y\|^{2}$.
		\item $\|x + y \|^{2} = (x + y) \cdot (x + y) = x \cdot x + 2 x \cdot y + y \cdot y = \| x \|^{2} + 2x \cdot y + \| y \|^{2}$. Damit:
			$$ \|x + y \|^{2} \leq \| x \|^{2} + 2 | x \cdot y | + \| y \|^{2} \overset{e)}{=} \| x \|^{2} + 2 \| x \| \| y \| + \| y \|^{2} = \left( \| x \|+ \| y \| \right)^{2}. $$
		\item Übung.
		\item Es ist $|x_{j}|^{2} = x_{j}^{2} \leq x_{1}^{2} + \dotsc + x_{n}^{2} = \|x\|^{2} \Rightarrow |x_{j}| \leq \| x \|$. Es ist 
			$$ x = x_{1} e_{1} + \dotsc + x_{n} e_{n} \Rightarrow \| x \| \overset{d)}{\underset{f)}{=}} |x_{1}| \| e_{1} \| + \dotsc |x_{n}| \| e_{n} \| = |x_{1}| + \dotsc + |x_{n}|. $$
	\end{enumerate}
\end{proof}

\index{Norm!Matrizen}
\begin{definition}
	Seien $l, m, n \in \N$ und $A \coloneqq \begin{pmatrix} a_{11} & \dotsc	& a_{1n} \\ \vdots & & \vdots \\ a_{m1} & \dotsc & a_{mn} \end{pmatrix}$ eine reelle $m \times n$-Matrix.
	$$ \| A \| \coloneqq \left( \sum_{j=1}^{m} \sum_{k=1}^{n} a_{jk}^{2} \right)^{\frac{1}{2}} \textbf{ Norm von $A$}	 $$

	Sei $B$ eine reelle $n \times l$-Matrix, dann existiert $AB$. Übungsblatt:
	\begin{align}
		\|AB\| \leq \|A\| \|B\| \tag{$*$}.	
	\end{align}
	Sei $x = (x_{1}, \dotsc, x_{n}) \in \R^{n}$.
	$$ A x \coloneqq A \cdot x^{T} = A \begin{pmatrix} x_{1} \\ \vdots \\ x_{n} \end{pmatrix} \quad (\textbf{Matrix-Vektorprodukt}) $$
\end{definition}

Aus $(*)$ folgt: $\| A x \| \leq \| A \| \| x \|$.

\index{Kugel} \index{Kugel!offene} \index{Kugel!abgeschlossene} \index{Umgebung}
\begin{definition}
	Sei $x_{0} \in \R^{n}$ und $\epsilon > 0$.
	\begin{enumerate}
		\item $U_{\epsilon}(x_{0}) \coloneqq \{ x \in \R^{n}: \| x - x_{0} \| < \epsilon \}$ hei{\ss}t \textbf{offene Kugel um $x_{0}$ mit Radius $\epsilon$}.
		\item $\overline{U_{\epsilon}(x_{0})} \coloneqq \{ x \in \R^{n}: \| x - x_{0} \| \leq \epsilon \}$ hei{\ss}t \textbf{abgeschlossene Kugel um $x_{0}$ mit Radius $\epsilon$}. 
	\end{enumerate}
	$U_{\epsilon}(x_{0})$ hei{\ss}t auch \textbf{$\epsilon$-Umgebung von $x_{0}$}.
\end{definition}

\index{beschränkt} \index{offen} \index{abgeschlossen} \index{kompakt}
\begin{definition}
	Sei $A \subseteq \R^{n}$.
	\begin{enumerate}
		\item $A$ hei{\ss}t \textbf{beschränkt} $\iff \exists c \geq 0: \| a \| \leq c ~\forall a \in A$.
		\item $A$ hei{\ss}t \textbf{offen} $\iff \forall a \in A ~\exists \epsilon = \epsilon(a) > 0: U_{\epsilon}(a) \subseteq A$. 
		\item $A$ hei{\ss}t \textbf{abgeschlossen} $\iff \R^{n} \setminus A$ ist offen.
		\item $A$ hei{\ss}t \textbf{kompakt} $\iff A$ ist beschränkt und abgeschlossen.
	\end{enumerate}
\end{definition}


\begin{beispiele} ~\
	\begin{enumerate}
		\item Offene Kugeln sind offen, abgeschlossene Kugeln sind nicht offen.
		\item $\R^{n}$ ist offen, $\emptyset$ ist offen, $\R^{n}$ ist abgeschlossen, $\emptyset$ ist abgeschlossen.
		\item Abgeschlossene Kugeln sind kompakt.
		\item $A = \{ (x, y) \in \R^{2}: y = x^{2} \}$. $A$ ist nicht beschränkt, also auch nicht kompakt. $A$ ist nicht offen, aber $A$ ist abgeschlossen.
		\item $A = \{ (x, y) \in \R^{2} : y \geq 0, x > 0 \}$. $A$ ist nicht offen und auch nicht abgeschlossen.
	\end{enumerate}
\end{beispiele}


% Skript - Ende

\appendix 

% Inhaltsverzeichnis

\cleardoublepage \phantomsection \renewcommand{\indexname}{Stichwortverzeichnis} \addcontentsline{toc}{\indexsection}{\indexname}

\printindex


\end{document}