\section{Folgen und Konvergenz}


\begin{definition*} \index{Folge} \index{Folge!reelle}
	Es sei $X$ eine Menge, $X \neq \emptyset$. Eine Funktion $a \colon \N \to X$ hei{\ss}t eine \textbf{Folge in X}. Ist $X = \R$, so hei{\ss}t $a$ eine \textbf{reelle Folge}.
\end{definition*}


\begin{schreibweisen}
$a_{n}$ statt $a(n)$ (n-tes Folgenglied) \\
$(a_{n})$ oder $(a_{n})_{n = 1}^{\infty}$ oder $(a_{1}, a_{2}, \dotsc)$ statt $a$
\end{schreibweisen}


\begin{beispiele*}
	\begin{enumerate}
		\item $a_{n} \coloneqq \frac{1}{n}$ $~(n \in \N)$, also $(a_{n}) = (1, \frac{1}{2}, \frac{1}{3}, \dotsc)$
		\item $a_{2n} \coloneqq 0$, $a_{2n-1} \coloneqq 1$ $~(n \in \N)$, also $(a_{n}) = (1, 0, 1, 0, \dotsc)$
	\end{enumerate}
\end{beispiele*}


\begin{bemerkung*}
	Ist $p \in \Z$ und $a \colon \{ p, p + 1, \dotsc \} \to X$ eine Funktion, so spricht man ebenfalls von einer Folge in $X$. Bez.: $(a_{n})_{n = p}^{\infty}$. Meist $p = 0$ oder $p = 1$.
\end{bemerkung*}


\begin{definition*} \index{abzählbar} \index{uberabzahlbar@überabzählbar}
	Sei $X$ eine Menge, $X \neq \emptyset$.
	\begin{enumerate}
		\item $X$ hei{\ss}t \textbf{abzählbar} $:\gdw \exists$ Folge $(a_{n})$ in $X$: $X = \{ a_{1}, a_{2}, a_{3}, \dotsc \}$
		\item $X$ hei{\ss}t \textbf{überabzählbar} $:\gdw X$ ist nicht abzählbar
	\end{enumerate}
\end{definition*}


\begin{beispiele*}
	\begin{enumerate}
		\item Ist $X$ endlich, so ist $X$ abzählbar.
		\item $\N$ ist abzählbar, denn $\N = \{ a_{1}, a_{2}, a_{3}, \dotsc \}$ mit $a_{n} \coloneqq n$ $(n \in \N)$
		\item $\Z$ ist abzählbar, denn $\Z = \{ a_{1}, a_{2}, a_{3}, \dotsc \}$ mit $a_{1} \coloneqq 0, a_{2} \coloneqq 1, a_{3} \coloneqq -1, a_{4} \coloneqq 2, a_{5} \coloneqq -2, \dotsc$ also
			$$ a_{2n} \coloneqq n, \quad a_{2n + 1} \coloneqq -n \quad (n \in \N) $$
		\item $\Q$ ist abzählbar!
			\begin{center}
 			  \begin{tikzpicture}
				\matrix(m)[matrix of math nodes,column sep=1cm,row sep=1cm]{
    				1 & 2 & 3 & 4 &  5 & 6 & \cdots \\
    				\frac{1}{2} & \frac{2}{2} & \frac{3}{2} & \frac{4}{2} & \frac{5}{2} & \cdots & \cdots \\
    				\frac{1}{3} & \frac{2}{3} & \frac{3}{3} & \frac{4}{3} & \frac{5}{3} & \cdots \\
    				\frac{1}{4} & \frac{2}{4} & \frac{3}{4} & \frac{4}{4} & \cdots \\
    				\frac{1}{5} & \frac{2}{5} & \cdots & \cdots \\
    				\cdots & \cdots &  \\
				};
				\draw[->]
					(m-1-1)edge(m-1-2) (m-1-2)edge(m-2-1) (m-2-1)edge(m-3-1) (m-3-1)edge(m-2-2) (m-2-2)edge(m-1-3) (m-1-3)edge(m-1-4) (m-1-4)edge(m-2-3) (m-2-3)edge(m-3-2) (m-3-2)edge(m-4-1) (m-4-1)edge(m-5-1) (m-5-1)edge(m-4-2) (m-4-2)edge(m-3-3) (m-3-3)edge(m-2-4) (m-2-4)edge(m-1-5) (m-1-5)edge(m-1-6); 
  			  \end{tikzpicture}
			\end{center}
			Durchnummerieren in Pfeilrichtung liefert
				$$ \{ x \in \Q : x > 0 \} = \{ a_{1}, a_{2}, a_{3}, \dotsc \} $$
			$b_{1} \coloneqq 0, b_{2n} \coloneqq a_{n}, b_{2n + 1} \coloneqq - a_{n}$ $(n \in \N)$. Dann:
			$$ \Q = \{ b_{1}, b_{2}, b_{3}, \dotsc \} $$
		\item $\R$ ist überabzählbar (Beweis in \S 5).
	\end{enumerate}	
\end{beispiele*}


\begin{vereinbarung}
	Solange nichts anderes gesagt wird, seien alle vorkommenden Folgen stets Folgen in $\R$. \\                                                                                                  c
	Die folgenden Sätze und Definitionen formulieren wir nur für Folgen der Form $(a_{n})_{n=1}^{\infty}$. Sie gelten sinngemä{\ss} für Folgen der Form $(a_{n})_{n=p}^{\infty}$ $(p \in \Z)$.
\end{vereinbarung}


\begin{definition*} \index{beschränkt!Folge}
	Sei $(a_{n})$ eine Folge und $M \coloneqq \{ a_{1}, a_{2}, \dotsc \}$.
	\begin{enumerate}
		\item$(a_{n})$ hei{\ss}t \textbf{nach oben beschränkt} $:\gdw M$ ist nach oben beschränkt. I.d. Fall: $\sup_{n \in \N} a_{n} \coloneqq \sup_{n = 1}^{\infty} a_{n} \coloneqq \sup M$.
		\item$(a_{n})$ hei{\ss}t \textbf{nach unten beschränkt} $:\gdw M$ ist nach unten beschränkt. I.d. Fall: $\inf_{n \in \N} a_{n} \coloneqq \inf_{n = 1}^{\infty} a_{n} \coloneqq \inf M$.
		\item$(a_{n})$ hei{\ss}t \textbf{beschränkt} $~:\gdw M$ ist beschränkt 
			$$ \gdw \exists c \geq 0: |a_{n}| \leq c ~\forall n \in \N $$
	\end{enumerate}
\end{definition*}


\begin{definition*} \index{für fast alle}
	Sei $A(n)$ eine für jedes $n \in \N$ definierte Aussage. \\
	$A(n)$ gilt \textbf{für fast alle} (ffa) $n \in \N$ $:\gdw \exists n_{0} \in \N: A(n)$ ist wahr $\forall n \geq n_{0}$
\end{definition*}


\begin{definition*} \index{Umgebung}
	Sei $a \in \R$ und $\epsilon > 0$
		$$ U_{\epsilon}(a) \coloneqq (a - \epsilon, a + \epsilon) = \{ x \in \R : | x - a| < \epsilon \} $$
	hei{\ss}t $\epsilon$\textbf{-Umgebung von a}.
\end{definition*}


\begin{definition*} \index{konvergent} \index{Grenzwert} \index{Limes} \index{divergent}
	Eine Folge $(a_{n})$ hei{\ss}t \textbf{konvergent}
	$$ :\gdw \exists a \in \R : \begin{cases} \text{zu jedem } \epsilon > 0 \text{ ex. } n_{0} = n_{0}(\epsilon) \in \N: \\
		|a_{n} - a| < \epsilon ~\forall n \geq n_{0}
	\end{cases} $$
	I. d. Fall hei{\ss}t $a$ \textbf{Grenzwert} (GW) oder \textbf{Limes} von $(a_{n})$ und man schreibt
	$$ 
		a_{n} \rightarrow a ~(n \rightarrow \infty) \text{ oder } a_{n} \rightarrow a \text{ oder } \lim_{n \rightarrow \infty} a_{n} = a
	$$
	Ist $(a_{n})$ nicht konvergent, so hei{\ss}t $(a_{n})$ \textbf{divergent}

	\begin{align*}
		\text{Beachte: } \quad a_{n} \rightarrow a ~(n \rightarrow \infty) & \gdw \forall \epsilon > 0 ~\exists n_{0} \in \N: a_{n} \in U_{\epsilon}(a) ~\forall n \geq n_{0} \\
				& \gdw \forall \epsilon > 0 \text{ gilt: } a_{n} \in U_{\epsilon}(a) \text{ ffa } n \in \N \\
				& \gdw \forall \epsilon > 0 \text{ gilt: } a_{n} \notin U_{\epsilon}(a) \text{ für höchstens endlich viele } n \in \N
	\end{align*}
\end{definition*}


\begin{satz} \label{satz-2.1}
	$(a_{n})$ sei konvergent und $a = \lim a_{n}$
	\begin{enumerate}
		\item Gilt auch noch $a_{n} \rightarrow b$, so sit $a = b$
		\item $(a_{n})$ ist beschränkt
	\end{enumerate}
\end{satz}

\begin{beweis}
	\begin{enumerate}
		\item Annahme $a \neq b$. Dann ist $\epsilon \coloneqq \frac{|a - b|}{2} > 0$.
			$$
			\exists n_{0} \in \N: |a_{n_{0}} - a| < \epsilon \quad \forall n \geq n_{0} \text{ und } \exists n_{1} \in \N: |a_n - b| < \epsilon \quad \forall n \geq n_{1}
			$$
			$N \coloneqq \max \{ n_{0}, n_{1} \}$. Dann:
			$$
				2 \epsilon = |a - b| = | a - a_{N} + a_{N} - b| \leq |a_{N} - a| + |a_{N} - b| < 2 \epsilon
			$$
			Widerspruch! Also $ a = b$
		\item  Zu $\epsilon = 1 ~\exists n_{0} \in \N: |a_{n} - a| < 1 ~\forall n \geq n_{0}$. Dann:
			$$
				|a_{n}| = |a_{n} - a + a| \leq |a_{n} - a| + |a| \leq 1 + |a| \quad \forall n \geq n_{0}
			$$
			$c \coloneqq \max \{ 1 + |a|, |a_{1}|, \dotsc, |a_{n_{0} - 1}| \}$. Dann: $|a_{n}| \leq \epsilon ~\forall n \geq 1$.
	\end{enumerate}
\end{beweis}


\begin{beispiele*}
	\begin{enumerate}
		\item Sei $c \in \R$ und $a_{n} \coloneqq c ~\forall n \in \N$. Dann:
			$$
				| a_{n} - c | = 0 \quad \forall n \in \N
			$$
			Also: $a_{n} \rightarrow c$.
		\item $a_{n} \coloneqq \frac{1}{n} ~(n \in \N)$. Beh: $a_{n} \rightarrow 0 ~(n \rightarrow \infty)$.
			\begin{beweis}
				Sei $\epsilon > 0: |a_{n} - 0 | = |a_{n}| = \frac{1}{n} < \epsilon \gdw n > \frac{1}{\epsilon}$
				$$
						\xRightarrow[]{\hyperref[satz-1.3]{\text{1.3 c)}}} \exists n_{0} \in \N: n_{0} > \frac{1}{\epsilon}
				$$
				Für $n \geq n_{0}$ ist $n > \frac{1}{\epsilon}$, also $\frac{1}{n} < \epsilon$. Somit $|a_{n} - 0| < \epsilon ~\forall n \geq n_{0}$
			\end{beweis}
		\item $a_{n} \coloneqq (-1)^{n}$. Es ist $|a_{n}| = 1 ~\forall n \in \N$, $(a_{n})$ ist also beschränkt. Behauptung: $(a_{n})$ ist divergent.
			\begin{beweis}
				$\forall n \in \N: |a_{n} - a_{n+1}| = |(-1)^{n} - (-1)^{n+1}| = |(-1)^{n}| \left( 1 - (-1) \right) = 2$. \\
				Annahme: $(a_{n})$ konvergiert. Definiere $a \coloneqq \lim a_{n}$, dann 
				$$
					 \exists n_{0} \in \N: ~ |a_{n} - a| < \frac{1}{2} \quad \forall n \geq n_{0}
				$$
				Für $n \geq n_{0}$ gilt dann aber:
				$$
					2 = |a_{n} - a_{n+1}| = |a_{n} - a + a - a_{n + 1}| \leq |a_{n} - a| + |a_{n+1} - a| < \frac{1}{2} + \frac{1}{2} = 1
				$$
				Widerspruch!
			\end{beweis}
		\item $a_{n} \coloneqq n ~(n \in \N)$. $(a_{n})$ ist nicht beschränkt $\xRightarrow[]{\hyperref[satz-2.1]{2.1 b)}} (a_{n})$ ist divergent.
		\item $a_{n} \coloneqq  \frac{1}{\sqrt{n}} (n \in \N)$. Beh.: $a_{n} \rightarrow 0$
			\begin{beweis}
				Sei $\epsilon > 0$.
				$$
					|a_{n} - 0| = \frac{1}{\sqrt{n}} < \epsilon \gdw \sqrt{n} > \frac{1}{n} \gdw n > \frac{1}{\epsilon^{2}}
				$$
				$\xRightarrow[]{\hyperref[satz-1.3]{1.3 c)}} \exists n_{0} \in \N: n_{0} > \frac{1}{\epsilon^{2}}$. Ist $n \geq n_{0} \Rightarrow n > \frac{1}{\epsilon^{2}} \Rightarrow \frac{1}{\sqrt{n}} < \epsilon \Rightarrow |a_{n} - 0 | < \epsilon$ 
			\end{beweis}
		\item $a_{n} \coloneqq \sqrt{n + 1} - \sqrt{n}$. 
			\begin{beweis}
				$$
					a_{n} = \frac{(\sqrt{n + 1} - \sqrt{n})(\sqrt{n + 1} + \sqrt{n})}{\sqrt{n + 1} + \sqrt{n}} = \frac{1}{\sqrt{n + 1} + \sqrt{n}} \leq \frac{1}{\sqrt{n}}
				$$
				$\Rightarrow |a_{n} - 0| \leq \frac{1}{\sqrt{n}} ~\forall n \in \N$. Sei $\epsilon > 0$, nach Beispiel e) folgt:
				$$
					\exists n_{0} \in \N: ~ \frac{1}{\sqrt{n}} < \epsilon \quad \forall n \geq n_{0} \Rightarrow |a_{n} - 0| < \epsilon \quad \forall n \geq n_{0}
				$$
				Also $a_{n} \rightarrow 0$.
			\end{beweis}
	\end{enumerate}
\end{beispiele*}


\begin{definition*}
	$(a_{n})$ und $(b_{n})$ seien Folgen und $\alpha \in \R$
	$$
		(a_{n}) \pm (b_{n}) \coloneqq (a_{n} \pm b_{n}); ~
		\alpha (a_{n}) \coloneqq (\alpha a_{n}); ~
		(a_{n}) (b_{n}) \coloneqq (a_{n} b_{n}) 		
	$$	
	Gilt $b_{n} \neq 0 ~\forall n \geq m$, so ist die Folge $\left( \frac{a_{n}}{b_{n}} \right)_{n = m}^{\infty}$ definiert.
\end{definition*}


\begin{satz} \label{satz-2.2}
	$(a_{n}),  (b_{n}), (c_{n})$ und $(\alpha_{n})$ seien Folge und $a, b, \alpha \in \R$

	\begin{enumerate}
		\item $a_{n} \rightarrow a \gdw |a_{n} - a| \rightarrow 0$
		\item Gilt $|a_{n} - a| \leq \alpha_{n}$ ffa $n \in \N$ und $\alpha_{n} \rightarrow 0$, so gilt $a_{n} \rightarrow a$
		\item Es gelte $a_{n} \rightarrow a$ und $b_{n} \rightarrow b$. Dann:
			\begin{enumerate}
				\item $|a_{n}| \rightarrow |a|$ 
				\item $a_{n} + b_{n} \rightarrow a + b$
				\item $\alpha a_{n} \rightarrow \alpha a$
				\item $a_{n} b_{n} \rightarrow a b$
				\item ist $a \neq 0$, so ex. ein $m \in \N$:
					$$
						a_{n} \neq 0 ~\forall n \geq m \text{ und für die Folge } \left( \frac{1}{a_{n}} \right)_{n = m}^{\infty} \text{ gilt: } \frac{1}{a_{n}} \rightarrow \frac{1}{a}
					$$
			\end{enumerate}
		\item Es gelte $a_{n} \rightarrow a$, $b_{n} \rightarrow b$ und $a_{n} \leq b_{n}$ ffa $n \in \N \Rightarrow a \leq b$
		\item Es gelte $a_{n} \rightarrow a$, $b_{n} \rightarrow a$ und $a_{n} \leq c_{n} \leq b_{n}$ ffa $n \in \N$. Dann $c_{n} \rightarrow a$.
	\end{enumerate}
\end{satz}


\begin{beispiele*}
	\begin{enumerate}
		\item Sei $p \in \N$ und $a_{n} \coloneqq \frac{1}{n^{p}}$. Es ist $n \leq n^{p} ~\forall n \in \N$. \\
			Dann: $0 \leq a_{n} \leq \frac{1}{n} ~\forall n \in \N \xRightarrow[]{\hyperref[satz-2.2]{2.2 e)}} a_{n} \rightarrow 0$, also $\frac{1}{n^{p}} \rightarrow 0$.
		\item $a_{n} \coloneqq \frac{5n^{2} + 3n + 1}{4n^{2} - n + 2} = \frac{5 + \frac{3}{n} + \frac{1}{n{2}}}{4 - \frac{1}{n} + \frac{2}{n^{2}}} \xrightarrow[]{\label{satz-2.2}} \frac{5}{4}$ % todo: replace double frac with nice fracs
	\end{enumerate}
\end{beispiele*}


\begin{beweis}[von 2.2]
	\begin{enumerate}
		\item folgt aus der Definition der Konvergenz
		\item $\exists m \in \N: |a_{n} - a | \leq \alpha_{m} ~\forall n \geq m$. Sei $\epsilon > 0$
			$$
		 		\exists n_{1} \in \N: \alpha_{n} < \epsilon ~\forall n \geq n_{1}.
		 	$$
		 	$n_{0} \coloneqq \max \{ m , n_{1} \}$. Für $n \geq n_{0}$: $|a_{n} - a| \leq \alpha_{n} < \epsilon$
		\item \begin{enumerate}
				\item $| |a_{n}| - |a|| \leq_{\S 1} |a_{n} - a| ~\forall n \in \N \xRightarrow[a)]{b)} |a_{n}| \rightarrow |a|$
				\item Sei $\epsilon > 0$. $\exists n_{1}, n_{2} \in \N; |a_{n} - a| < \frac{\epsilon}{2} ~\forall n \geq n_{1}$, $|b_{n} - b| < \frac{\epsilon}{2} ~\forall n \geq n_{2}$ \\
					$n_{0} \coloneqq \max \{ n_{1}, n_{2} \}$. Für $n \geq n_{0}$:
					$$
						|a_{n} + b_{n} - (a + b)| = |a_{n} - a + b_{n} - b| \leq |a_{n} - a| + |b_{n} - b| < \frac{\epsilon}{2} + \frac{\epsilon}{2} = \epsilon
					$$
				\item Übung
				\item $c_{k} \coloneqq |a_{n} b_{n} - ab|$. z. z.: $c_{n} \rightarrow 0$
					\begin{align*}
						c_{n} & = |a_{n}b_{n} - a_{n}b + a_{n}b - ab| = |a{n}(b_{n} - b)+ (a_{n} - a)b| \\
							  & \leq |a_{n}||b_{n} - b| + |b||a_{n}-a|
					\end{align*}
					$\xRightarrow[]{\hyperref[satz-2.1]{2.1 b)}} \exists c \geq 0 : |a_{n}| \leq c ~\forall n \in \N$ und $c \geq |b|$. Dann:
					$$
						c_{n} \leq c(|b_{n}-b| + |a_{n}-a|) \eqqcolon \alpha_{n} \xRightarrow[c) (ii), c) (iii)]{a)} \alpha_{n} \rightarrow 0
					$$
					Also: $|c_{n} - 0| = c_{n} \leq \alpha_{n} ~\forall n \in \N$ und $\alpha_{n} \rightarrow 0 \xRightarrow[]{b)} c_{n} \rightarrow 0$.
				\item $\epsilon \coloneqq \frac{|a|}{2}$; (aus (i): $|a_{n}| \rightarrow |a| \Rightarrow \exists n \in N$:
					$$
						 % todo todo
					$$
					$\Rightarrow |a_{n}| > \frac{|a|}{2} > 0 ~\forall n \geq m \Rightarrow a_{n} \neq 0 ~\forall n \geq m$.
			  \end{enumerate}
	\end{enumerate}	
\end{beweis}
